\documentclass{article}

\usepackage[margin=1.6in]{geometry}

\title{Foundations of Algebraic Geometry}
\author{A. Grothendieck}
\date{1957--1962}

\usepackage{amssymb,amsmath,amscd}
\usepackage{hyperref}
\usepackage{xcolor}
\hypersetup{colorlinks,linkcolor={blue!50!black},citecolor={blue!50!black},urlcolor={blue!80!black}}
\usepackage{enumerate}
\usepackage{tikz-cd}
\usepackage{booktabs}
\usepackage{mathtools}

% for pandoc tables
\usepackage{calc,array}
\usepackage{longtable}
%

\usepackage{mathrsfs}
\usepackage{fouriernc}

\providecommand{\tightlist}{%
  \setlength{\itemsep}{0pt}\setlength{\parskip}{0pt}}

\let\oldhref\href 
\renewcommand{\href}[2]{#2\footnote{\url{#1}}}


%% Header and footer %%

\usepackage{fancyhdr}
\usepackage{lastpage}
\usepackage{xstring}
\pagestyle{fancy}
\fancypagestyle{plain}{}
\fancyhf{}
\lhead{\footnotesize\nouppercase\leftmark}
\cfoot{\small\thepage\ of \pageref*{LastPage}}


%% Theorem environments %%

\usepackage{amsthm}

\newenvironment{itenv}[1]
  {\phantomsection\par\smallskip\noindent\textbf{#1.}\itshape}
  {\par\smallskip}

\newenvironment{rmenv}[1]
  {\phantomsection\par\smallskip\noindent\textbf{#1.}\rmfamily}
  {\par\smallskip}

\newenvironment{titenv}[1]
  {\phantomsection\par\noindent\textbf{#1.}\itshape}
  {\par}

\newenvironment{trmenv}[1]
  {\phantomsection\par\noindent\textbf{#1.}\rmfamily}
  {\par}

\newenvironment{eqenv}
  {}
  {}


%% Shortcuts %%

\newcommand{\oldpage}[1]{\marginpar{\footnotesize$\Big\vert$ \textit{p.~#1}}}


%% Bibliography %%

\newlength{\cslhangindent}
\setlength{\cslhangindent}{1.5em}
\newlength{\csllabelwidth}
\setlength{\csllabelwidth}{3em}
\newenvironment{CSLReferences}[2] % #1 hanging-ident, #2 entry spacing
 {% don't indent paragraphs
  \setlength{\parindent}{0pt}
  % turn on hanging indent if param 1 is 1
  \ifodd #1 \everypar{\setlength{\hangindent}{\cslhangindent}}\ignorespaces\fi
  % set entry spacing
  %% WE MANUALLY SET THIS INSTEAD
  %\ifnum #2 > 0
  % \setlength{\parskip}{#2\baselineskip}
  %\fi
  \setlength{\parskip}{0.75\baselineskip}
 }%
 {}
\usepackage{calc} % for \widthof, \maxof
\newcommand{\CSLBlock}[1]{#1\hfill\break}
\newcommand{\CSLLeftMargin}[1]{\parbox[t]{\maxof{\widthof{#1}}{\csllabelwidth}}{#1}}
\newcommand{\CSLRightInline}[1]{\parbox[t]{\linewidth}{#1}}
\newcommand{\CSLIndent}[1]{\hspace{\cslhangindent}#1}


%% Document %%

\usepackage{amsthm}
\newtheorem{theorem}{Theorem}[section]
\newtheorem{lemma}{Lemma}[section]
\newtheorem{corollary}{Corollary}[section]
\newtheorem{proposition}{Proposition}[section]
\newtheorem{conjecture}{Conjecture}[section]
\theoremstyle{definition}
\newtheorem{definition}{Definition}[section]
\theoremstyle{definition}
\newtheorem{example}{Example}[section]
\theoremstyle{definition}
\newtheorem{exercise}{Exercise}[section]
\theoremstyle{definition}
\newtheorem{hypothesis}{Hypothesis}[section]
\theoremstyle{remark}
\newtheorem*{remark}{Remark}
\newtheorem*{solution}{Solution}
\begin{document}

\maketitle
\thispagestyle{fancy}

\renewcommand{\abstractname}{Translator's note}

\setcounter{footnote}{0}

\tableofcontents


%% Content %%

\leavevmode\vadjust pre{\hypertarget{translators-note}{}}%
\begin{abstract}
This is an English translation of Grothendieck's ``Fondements de la Géometrie Algébrique''.
The original (French) notes have been scanned and uploaded by the Grothendieck Circle \href{https://webusers.imj-prg.fr/~leila.schneps/grothendieckcircle/FGA.pdf}{here}.

Note that in this present version the volume numbering has changed: FGA 1 and FGA 2 appear as chapters 1 and 2 (respectively), and FGA 3.I to FGA 3.VI appear as chapters 3 to 8 (respectively).
Furthermore, the numbering of citations within the text has changed, since the eight individual original bibliographies have been merged into a single one.

\emph{The translator (\href{https://thosgood.com}{Tim Hosgood}) takes full responsibility for any errors introduced, and claims no rights to any of the mathematical content herein.}
\emph{Corrections and comments welcome.}

Version: \href{https://github.com/thosgood/fga/commit/fcf8fc2}{\texttt{fcf8fc2}}

\end{abstract}

\hypertarget{foreword}{%
\section*{Foreword}\label{foreword}}
\addcontentsline{toc}{section}{Foreword}

\oldpage{C-01}

In the \emph{Séminaire Bourbaki}, between 1957 and 1962, we gave eight talks on the foundations of algebraic geometry.
With the exception of the first, these talks are expressed in the language of \emph{schemes}.
All of the stated results will find their place in Jean Dieudonné and Alexander Grothendieck's ``Éléments de Géométrie Algébrique'' ({[}\protect\hyperlink{ref-GD1960}{10}{]}).
However, none of the essence of any of these talks is currently found in any of the chapters (neither published nor still in preparation) of the EGA, nor in any other book or article, and this will probably remain the case for a few more years still.
This is the main reason that persuaded us to combine these talks, giving readers access to a number of ideas and key results of the theory of schemes whilst awaiting a well-written summary.
Also, reading these chapters will allow the reader to quickly familiarise themselves with the aforementioned results and ideas, without being bothered by the necessarily cumbersome details of a systematic treatment, and also endow them with vital motivations for the study of such a treatment.

For the sake of the reader, we have assembled here some comments and errata, grouped by section, that, most notably, show the progress accomplished since the editing of this text, as well as indicating some supplementary references.\footnote{\emph{{[}Trans.{]}} Rather than translating the comments and errata here, we have inserted them throughout the text in the relevant places; we preface them with ``\emph{{[}Comp.{]}}'' (for ``Complément'') except for small corrections, which we have inserted silently.}

Many of the results appearing in these articles have been discussed in detail in the \emph{Séminaire de Géométrie Algébrique du Bois Marie}, as well as in the two subsequent seminars at Harvard, in 1961--62 (the first by myself, and the second by Mumford--Tate), the notes for which are currently in preparation by Lichtenbaum.\footnote{\emph{{[}Trans.{]}} These notes are referenced again multiple times throughout these talks, but they do not actually exist: they were never finished, and no draft of them was ever published.}

\hypertarget{fga-1}{%
\section{Duality theorems for coherent algebraic sheaves}\label{fga-1}}

\providecommand{\scr}[1]{{\mathscr{#1}}}
\renewcommand{\cal}[1]{{\mathcal{#1}}}
\renewcommand{\frak}[1]{{\mathfrak{#1}}}
\renewcommand{\geq}{\geqslant}
\renewcommand{\leq}{\leqslant}

\providecommand{\from}{\leftarrow}
\providecommand{\bb}{\mathbf}

\providecommand{\Ext}{\operatorname{Ext}}
\providecommand{\shExt}{\mathscr{E}\kern -.5pt xt}
\providecommand{\Hom}{\operatorname{Hom}}
\providecommand{\shHom}{\mathscr{H}\kern -.5pt om}
\providecommand{\Tor}{\operatorname{Tor}}
\providecommand{\RR}{\operatorname{R}}
\providecommand{\HH}{\operatorname{H}}

{[}FGA 1{]}
Grothendieck, A.
``Théorème de dualité pour les faisceaux algébriques cohérents''.
\emph{Séminaire Bourbaki} \textbf{9} (1956--57), Talk no. 149.

\oldpage{149-01}The results that follow, inspired by Serre's ``theorem of algebraic duality'', were discovered in the winter of 1955 and the winter of 1956.
They can be established very simply, thanks to reasonably elementary results on the cohomology of projective spaces {[}\protect\hyperlink{ref-Ser1955}{18}{]}. and an intensive use of Cartan--Eilenberg's homological algebra, in the form given in {[}\protect\hyperlink{ref-Gro1957}{6}{]}.

\hypertarget{fga-1-section-1}{%
\subsection{\texorpdfstring{\(\operatorname{Ext}\) of sheaves of modules}{\textbackslash operatorname\{Ext\} of sheaves of modules}}\label{fga-1-section-1}}

\emph{(cf. {[}\protect\hyperlink{ref-Gro1957}{6} and chap.~4{]})}

Let \(X\) be a topological space endowed with a sheaf \({\mathscr{O}}\) of unital (but not necessarily commutative) rings.
We consider the abelian category \({\mathcal{C}}^{\mathscr{O}}\) of sheaves of \({\mathscr{O}}\)-modules, which are also referred to as \({\mathscr{O}}\)-modules.
We know that every object of this category admits an injective resolution, which allows us to define the \(\operatorname{Ext}\) functors that have the well-known formal properties.
More precisely, to avoid confusion, we denote by \(\operatorname{Hom}_{\mathscr{O}}(X;{\mathscr{A}},{\mathscr{B}})\), or simply \(\operatorname{Hom}(X;{\mathscr{A}},{\mathscr{B}})\), the abelian \emph{groups} of \({\mathscr{O}}\)-homomorphisms from \({\mathscr{A}}\) to \({\mathscr{B}}\), whereas \(\mathscr{H}\kern -.5pt om_{\mathscr{O}}({\mathscr{A}},{\mathscr{B}})\) denotes the \emph{sheaf} of germs of homomorphisms from \({\mathscr{A}}\) to \({\mathscr{B}}\) (where \({\mathscr{A}},{\mathscr{B}}\in {\mathcal{C}}^{\mathscr{O}}\)).
We define, for fixed \({\mathscr{A}}\in {\mathcal{C}}^{\mathscr{O}}\), functors \(h_{\mathscr{A}}\) and \(\underline{h}_{\mathscr{A}}\), with values in the category \({\mathcal{C}}\) of abelian groups and the category \({\mathcal{C}}^Z\) of abelian sheaves on \(X\) (respectively), by the formulas:
\[
  \begin{aligned}
    h_{\mathscr{A}}({\mathscr{B}}) &= \operatorname{Hom}_{\mathscr{O}}(X;{\mathscr{A}},{\mathscr{B}})
  \\\underline{h}_{\mathscr{A}}({\mathscr{B}}) &= \mathscr{H}\kern -.5pt om_{\mathscr{O}}({\mathscr{A}},{\mathscr{B}}).
  \end{aligned}
\tag{1.1}
\]
The functors \(h_{\mathscr{A}}\) and \(\underline{h}_{\mathscr{A}}\) are left exact and covariant, and so we consider their right-derived functors, denoted by \(\operatorname{Ext}_{\mathscr{O}}^p(X;{\mathscr{A}},{\mathscr{B}})\) and \(\mathscr{E}\kern -.5pt xt_{\mathscr{O}}^p({\mathscr{A}},{\mathscr{B}})\) (respectively).
We then have, by definition,

\leavevmode\vadjust pre{\hypertarget{fga-1-equation-1.2}{}}%
\begin{eqenv}
\[
  \begin{gathered}
    \operatorname{Ext}_{\mathscr{O}}^p(X;{\mathscr{A}},{\mathscr{B}}) = (\operatorname{R}^p h_{\mathscr{A}})({\mathscr{B}}) = \operatorname{H}^p(\operatorname{Hom}_{\mathscr{O}}(X;{\mathscr{A}},C({\mathscr{B}})))
  \\\mathscr{E}\kern -.5pt xt_{\mathscr{O}}^p({\mathscr{A}},{\mathscr{B}}) = (\operatorname{R}^p \underline{h}_{\mathscr{A}})({\mathscr{B}}) = \operatorname{H}^p(\mathscr{H}\kern -.5pt om_{\mathscr{O}}({\mathscr{A}},C({\mathscr{B}})))
  \end{gathered}
\tag{1.2}
\]

\end{eqenv}

where \(\operatorname{R}^p\) denotes the passage to right-derived functors, and where \(C({\mathscr{B}})\) denotes an arbitrary injective resolution of \({\mathscr{B}}\) in \({\mathcal{C}}^{\mathscr{O}}\).
We denote by \(\Gamma\colon{\mathcal{C}}^Z\to{\mathcal{C}}\) the ``sections'' functor;
recall that its right-derived functors are denoted by \(B\mapsto\operatorname{H}^p(X,{\mathscr{B}})\):
\[
  \operatorname{H}^p(X,{\mathscr{B}}) = (\operatorname{R}^p\Gamma)({\mathscr{B}}) = \operatorname{H}^p(\Gamma(C({\mathscr{B}}))).
\tag{1.3}
\]

\oldpage{149-02}We evidently have \(h_{\mathscr{A}}=\Gamma\underline{h}_{\mathscr{A}}\);
we can also show that \(\underline{h}_{\mathscr{A}}\) sends injective objects to \(\Gamma\)-acyclic objects.
From this, it is a well-known result that:

\hypertarget{fga-1-proposition-1}{}
\begin{itenv}{Proposition 1}

For every \({\mathscr{O}}\)-module \({\mathscr{A}}\), there exists a cohomological spectral functor on \({\mathcal{C}}^{\mathscr{O}}\) that abuts to the graded functor \((\operatorname{Ext}_{\mathscr{O}}^\bullet(X;{\mathscr{A}},{\mathscr{B}}))\), and whose initial page is

\leavevmode\vadjust pre{\hypertarget{fga-1-equation-1.4}{}}%
\begin{eqenv}
\[
  E_2^{p,q}({\mathscr{A}},{\mathscr{B}}) = \operatorname{H}^p(X,\mathscr{E}\kern -.5pt xt_{\mathscr{O}}^q({\mathscr{A}},{\mathscr{B}})).
\tag{1.4}
\]

\end{eqenv}

\end{itenv}

From this, we obtain ``\emph{boundary homomorphisms}'', as well as a short exact sequence, which we will not write.

\leavevmode\vadjust pre{\hypertarget{fga-1-proposition-1-corollary-1}{}}%
\begin{itenv}{Corollary 1}
If \({\mathscr{A}}\) is locally isomorphic to \({\mathscr{O}}^n\), then we have canonical isomorphisms

\leavevmode\vadjust pre{\hypertarget{fga-1-equation-1.5}{}}%
\begin{eqenv}
\[
  \operatorname{Ext}_{\mathscr{O}}^p(X;{\mathscr{A}},{\mathscr{B}}) \xleftarrow{\sim} \operatorname{H}^p(x,\mathscr{H}\kern -.5pt om_{\mathscr{O}}({\mathscr{A}},{\mathscr{B}}))
\tag{1.5}
\]

\end{eqenv}

(given by the boundary homomorphisms of the spectral sequence).
In particular, we have a canonical isomorphism
\[
  \operatorname{Ext}_{\mathscr{O}}^p(X;{\mathscr{O}},{\mathscr{B}}) = \operatorname{H}^p(X,{\mathscr{B}}).
\tag{1.6}
\]

\end{itenv}

To use these results, we need to know how to explicitly describe the \(\operatorname{Ext}_{\mathscr{O}}^p({\mathscr{A}},{\mathscr{B}})\).
They are functors that we calculate locally, i.e.~if \(V\) is an open subset of \(X\), then
\[
  \mathscr{E}\kern -.5pt xt_{\mathscr{O}}^p({\mathscr{A}},{\mathscr{B}})|U = \mathscr{E}\kern -.5pt xt_{{\mathscr{O}}|U}^p({\mathscr{A}}|U,{\mathscr{B}}|U),
\]
as follows from the fact that the restriction to \(U\) of an injective \({\mathscr{O}}\)-module is an injective \(({\mathscr{O}}|U)\)-module.
Furthermore, for fixed \(x\in X\), we have functorial homomorphisms

\leavevmode\vadjust pre{\hypertarget{fga-1-equation-1.7}{}}%
\begin{eqenv}
\[
  \mathscr{H}\kern -.5pt om_{\mathscr{O}}({\mathscr{A}},{\mathscr{B}})_x \to \operatorname{Hom}_{{\mathscr{O}}_x}({\mathscr{A}}_x,{\mathscr{B}}_x)
\tag{1.7}
\]

\end{eqenv}

that uniquely extend to homomorphisms of cohomological functors (in \({\mathscr{B}}\)):

\leavevmode\vadjust pre{\hypertarget{fga-1-equation-1.8}{}}%
\begin{eqenv}
\[
  \mathscr{E}\kern -.5pt xt_{\mathscr{O}}^p({\mathscr{A}},{\mathscr{B}})_x \to \operatorname{Ext}_{{\mathscr{O}}_x}^p({\mathscr{A}}_x,{\mathscr{B}}_x).
\tag{1.8}
\]

\end{eqenv}

\leavevmode\vadjust pre{\hypertarget{fga-1-proposition-2}{}}%
\begin{itenv}{Proposition 2}
If \({\mathscr{A}}\) is isomorphic, in a neighbourhood of \(x\), to the cokernel of some homomorphism \({\mathscr{O}}^m\to{\mathscr{O}}^n\), then \protect\hyperlink{fga-1-equation-1.7}{(1.7)} is an isomorphism for all \(p\).
This is the case, in particular, if \({\mathscr{A}}\) is a coherent \({\mathscr{O}}\)-module {[}\protect\hyperlink{ref-Ser1955}{18}{]}.

\end{itenv}

\leavevmode\vadjust pre{\hypertarget{fga-1-proposition-3}{}}%
\begin{itenv}{Proposition 3}
\oldpage{149-03}Let \({\mathscr{L}}_\bullet=({\mathscr{L}}_i)\) be a left resolution of the \({\mathscr{O}}\)-module \({\mathscr{A}}\) by \({\mathscr{O}}\)-modules that are all locally isomorphic to some \({\mathscr{O}}^n\).
Then \(\mathscr{E}\kern -.5pt xt_{\mathscr{O}}({\mathscr{A}},{\mathscr{B}})\) can be identified with \(\operatorname{H}^\bullet(\mathscr{H}\kern -.5pt om_{\mathscr{O}}({\mathscr{L}}_\bullet,{\mathscr{B}}))\), and \(\operatorname{Ext}_{\mathscr{O}}(X;{\mathscr{A}},{\mathscr{B}})\) can be identified with the hypercohomology of \(X\) with respect to the complex \(\mathscr{H}\kern -.5pt om_{\mathscr{O}}({\mathscr{L}}_\bullet,{\mathscr{B}})\).

\end{itenv}

\begin{proof}
The proof is standard: we consider the bicomplex \(\mathscr{H}\kern -.5pt om_{\mathscr{O}}({\mathscr{L}}_\bullet,C({\mathscr{B}}))\), where \(C({\mathscr{B}})\) is an injective resolution of \({\mathscr{B}}\), as well as the natural homomorphisms into this bicomplex from \(\mathscr{H}\kern -.5pt om_{\mathscr{O}}({\mathscr{L}}_\bullet,{\mathscr{B}})\) and \(\mathscr{H}\kern -.5pt om_{\mathscr{O}}({\mathscr{A}},C({\mathscr{B}}))\).
\}
\end{proof}

To finish, we note that the two \(\operatorname{Ext}\) functors introduced in \protect\hyperlink{fga-1-equation-1.2}{(1.2)} are not only cohomological functors in \({\mathscr{B}}\), but in fact \emph{cohomological bifunctors}, covariant in \({\mathscr{B}}\), and contravariant in \({\mathscr{A}}\).

\hypertarget{fga-1-section-2}{%
\subsection{\texorpdfstring{The composition law in \(\operatorname{Ext}\)}{The composition law in \textbackslash operatorname\{Ext\}}}\label{fga-1-section-2}}

The results of this section are due, independently, to Cartier and Yoneda;
see a talk by Cartier {[}\protect\hyperlink{ref-Car1957}{2}{]} for more details.
Let \({\mathcal{C}}\) be an abelian category, and let \(K\) and \(L\) be two graded objects of \({\mathcal{C}}\).
We denote by \(\operatorname{Hom}(K,L)\) the graded abelian group whose degree-\(n\) component consists of homogeneous homomorphisms of degree \(n\) from \(K\) to \(L\) (i.e.~systems \((u_i)\) of homomorphisms \(K^i\to L^{i+n}\)).
If \(K\) and \(L\) are complexes (with differentials of degree \(+1\), to fix conventions), then we endow \(\operatorname{Hom}(K,L)\) with the differential operator given by
\[
  \delta(u) = \mathrm{d}u + (-1)^{n+1}u\mathrm{d}
  \quad\text{where }n=\deg(u)
\tag{2.1}
\]
which makes it a complex with a differential of degree \(+1\).
The cycles of degree \(n\) are the maps of degree \(n\) that anticommute with \(u\) (as homogeneous maps).
We can then consider \(\operatorname{H}^\bullet(\operatorname{Hom}(K,L))\), which is an invariant of the homotopy types of \(K\) and \(L\), and which we sometimes denote by \(\operatorname{H}^\bullet(K,L)\).
If we have a third complex \(M\), then the composition of homomorphisms defines a pairing \(\operatorname{Hom}(K,L)\times\operatorname{Hom}(L,M)\to\operatorname{Hom}(K,M)\) which is compatible with the differential maps, whence, by passing to the cohomology of pairings,

\leavevmode\vadjust pre{\hypertarget{fga-1-equation-2.2}{}}%
\begin{eqenv}
\[
  \operatorname{H}^\bullet(K,L)\times\operatorname{H}^\bullet(L,M) \to \operatorname{H}^\bullet(K,M)
\tag{2.2}
\]

\end{eqenv}

which we write as \((u,v)\mapsto vu\).
These pairings satisfy an evident associativity property;
in particular, \(\operatorname{H}^\bullet(K,K)\) is an associative graded unital ring, and \(\operatorname{H}^\bullet(K,L)\) (resp. \(\operatorname{H}^\bullet(L,K)\)) is a graded right (resp. left) module over this ring, etc.
In dimension \(0\), \protect\hyperlink{fga-1-equation-2.2}{(2.2)} reduces to the composition of permissible homomorphisms of complexes.
Finally, an exact sequence of complexes \(0\to K'\to K\to K''\to0\) such that, for all \(i\), \(K'^i\) can be identified with a direct factor of \(K^i\), gives rise to an exact sequence of complexes of groups \(\operatorname{Hom}(K'',L)\), etc., whence a coboundary map \(\operatorname{H}^i(K',L)\to\operatorname{H}^{i+1}(K'',L)\).
We similarly define the boundary maps relative to an exact sequence in \(L\).
The pairings in \protect\hyperlink{fga-1-equation-2.2}{(2.2)} are compatible, in the usual sense, with these coboundary maps.

\oldpage{149-04}Now suppose that \({\mathcal{C}}\) is a category such that every element \(A\) of \({\mathcal{C}}\) admits an injective resolution \(C(A)\).
We then note that, using one of the many variants of the theorem of bicomplexes,
\[
  \operatorname{H}^\bullet(C(A),C(B)) = \operatorname{H}^\bullet(\operatorname{Hom}(C(A),C(B)))
\]
is canonically isomorphic to
\[
  \operatorname{H}^\bullet(\operatorname{Hom}(A,C(B))) = \operatorname{Ext}^\bullet(A,B).
\]
The coboundary maps described above give coboundary maps of the \(\operatorname{Ext}\).
Furthermore, the pairings in \protect\hyperlink{fga-1-equation-2.2}{(2.2)} give associative pairings here:
\[
  \operatorname{Ext}^\bullet(A,B)\times\operatorname{Ext}^\bullet(B,C) \to \operatorname{Ext}^\bullet(A,C)
\tag{2.3}
\]
and these are compatible with the coboundary maps.
In particular, \(\operatorname{Ext}^\bullet(A,A)\) is an associative graded unital ring, etc.
(We can show in an analogous manner that the \(\operatorname{Ext}\) functors behave like derived functors of an arbitrary functor;
we do not make use of this fact here).

In the case where the category in question is the category \({\mathcal{C}}^{\mathscr{O}}\) of \({\mathscr{O}}\)-modules on \(X\), we then obtain pairings
\[
  \operatorname{Ext}_{\mathscr{O}}^p(X;{\mathscr{A}},{\mathscr{B}})\times\operatorname{Ext}_{\mathscr{O}}^q(X;{\mathscr{B}},{\mathscr{C}}) \to \operatorname{Ext}_{\mathscr{O}}^{p+q}(X;{\mathscr{A}},{\mathscr{C}})
\tag{2.4}
\]
that can be calculated as already described.
The same method, but replacing the category of abelian groups with the category of abelian sheaves on \(X\), and the \(\operatorname{Hom}\) functors by the \(\mathscr{H}\kern -.5pt om\) functors, again defines pairings, having the same formal properties, and of a ``local nature'' this time:
\[
  \mathscr{E}\kern -.5pt xt_{\mathscr{O}}^p({\mathscr{A}},{\mathscr{B}})\times\mathscr{E}\kern -.5pt xt_{\mathscr{O}}^q({\mathscr{B}},{\mathscr{C}}) \to \mathscr{E}\kern -.5pt xt_{\mathscr{O}}^{p+q}({\mathscr{A}},{\mathscr{C}}).
\tag{2.5}
\]
These can be understood by noting that the homomorphisms in \protect\hyperlink{fga-1-equation-1.8}{(1.8)} are compatible with the pairings between the \(\operatorname{Ext}\).

\oldpage{149-05}Finally, recall that we also have a multiplicative structure between functors \(\operatorname{H}^p(X,A)\), namely the cup product.
We note then that the spectral sequences of \protect\hyperlink{fga-1-proposition-1}{Proposition 1} are compatible with the multiplicative structures;
more precisely, we have a pairing from the spectral sequence \(E(A,B)\) with the spectral sequence \(E(B,C)\) to the spectral sequence \(E(A,C)\) that abuts to the pairing between the global \(\operatorname{Ext}\), and whose initial page comes from the cup product and the local \(\operatorname{Ext}\) pairings in the right-hand side of \protect\hyperlink{fga-1-equation-1.4}{(1.4)}.
It then follows, in particular, that the ``boundary homomorphisms''
\[
  \operatorname{Ext}_{\mathscr{O}}^n(X;{\mathscr{A}},{\mathscr{B}}) \to \operatorname{H}^0(X;\mathscr{E}\kern -.5pt xt_{\mathscr{O}}^n({\mathscr{A}},{\mathscr{B}}))
\tag{2.6}
\]
\[
  \operatorname{H}^n(X,\mathscr{H}\kern -.5pt om_{\mathscr{O}}({\mathscr{A}},{\mathscr{B}})) \to \operatorname{Ext}_{\mathscr{O}}^n(X;{\mathscr{A}},{\mathscr{B}})
\tag{2.7}
\]
are compatible with the multiplicative structures.
So, if we restrict to sheaves that are locally isomorphic to some \({\mathscr{O}}^m\), then this completely explains the composition of the global \(\operatorname{Ext}\) by means of the cup product, taking into account the isomorphisms of \protect\hyperlink{fga-1-equation-1.5}{(1.5)}.

\hypertarget{fga-1-section-3}{%
\subsection{Results concerning local cohomology}\label{fga-1-section-3}}

Let \(A\) be a unital commutative ring endowed with an ideal \({\mathfrak{J}}\).
We will define, for any \(A\)-module \(M\), functorial homomorphisms

\leavevmode\vadjust pre{\hypertarget{fga-1-equation-3.1}{}}%
\begin{eqenv}
\[
  \begin{aligned}
    \operatorname{Ext}_A^p(A/{\mathfrak{J}},M) &\to \operatorname{Hom}_A(\wedge^p{\mathfrak{J}}/{\mathfrak{J}}^2,M\otimes A/{\mathfrak{J}})
  \\\operatorname{Tor}_p^A(A/{\mathfrak{J}},M) &\leftarrow(\wedge^p{\mathfrak{J}}/{\mathfrak{J}}^2)\otimes\operatorname{Hom}_A(A/{\mathfrak{J}},M)
  \end{aligned}
\tag{3.1}
\]

\end{eqenv}

where the tensor and exterior products are taken over the ring \(A\);
note also that \({\mathfrak{J}}/{\mathfrak{J}}^2\) is in fact an \(A/{\mathfrak{J}}\)-module, and that its exterior powers as an \(A\)-module agree with its exterior powers as an \(A/{\mathfrak{J}}\)-module.
The definition of the homomorphisms in \protect\hyperlink{fga-1-equation-3.1}{(3.1)} come from the definition, for every system \(x=(x_1,\ldots,x_p)\) of points of \({\mathfrak{J}}\), of homomorphisms \(\varphi_x\) given by

\leavevmode\vadjust pre{\hypertarget{fga-1-equation-3.2}{}}%
\begin{eqenv}
\[
  \begin{aligned}
    \varphi_x\colon \operatorname{Ext}_A^p(A/{\mathfrak{J}},M) &\to M\otimes A/{\mathfrak{J}}
  \\\varphi_x\colon \operatorname{Hom}_A(A/{\mathfrak{J}},M) &\to \operatorname{Tor}_p^A(A/{\mathfrak{J}},M)
  \end{aligned}
\tag{3.2}
\]

\end{eqenv}

\oldpage{149-06}such that the following conditions are satisfied:

\begin{enumerate}
\def\labelenumi{\roman{enumi}.}
\item
  \(\varphi_{x_1,\ldots,x_p}\) depends on the system of the \(x_i\in{\mathfrak{J}}\) in an alternating \(A\)-multilinear way;
\item
  \(\varphi_{x_1,\ldots,x_p}\) is zero when any of the \(x_i\) is in \({\mathfrak{J}}^2\).
\end{enumerate}

In fact, ii. follows from i., since \(a\varphi_x=0\) for \(a\in{\mathfrak{J}}\), as we see by noting that all the modules in \protect\hyperlink{fga-1-equation-3.2}{(3.2)} are annihilated by \({\mathfrak{J}}\).

To define the \(\varphi_x\), we consider the complex \(K_x\) whose underlying \(A\)-modules are the \(\wedge A^p\), and whose differential is the interior product \(i_x\) by \(x\), considered as a linear form on \(A^p\) with components \(x_1,\ldots,x_p\).
The differential is of degree \(-1\), the degrees of the complex are positive, and the cohomology of this complex in dimension \(0\) is \(A/(x_1A+\ldots+x_pA)\).
Since the \(x_i\) are in \({\mathfrak{J}}\), we obtain an augmentation \(K_{x,0}\to A/{\mathfrak{J}}\).
Thus \(K_x\) is a \emph{free} augmented complex, with augmentation module \(A/{\mathfrak{J}}\).
We thus obtain known homomorphisms
\[
  \begin{aligned}
    \operatorname{Ext}_A^\bullet(\operatorname{H}_0(K_x),M) &\to \operatorname{H}^\bullet(\operatorname{Hom}_A(K_x,M))
  \\\operatorname{Tor}_\bullet^A(\operatorname{H}_0(K_x),M) &\leftarrow\operatorname{H}_\bullet(K_x\otimes M)
  \end{aligned}
\]
whence, by composing with the homomorphisms to the \(\operatorname{Ext}\) and the \(\operatorname{Tor}\) induced by the augmentation homomorphism \(\operatorname{H}_0(K_x)\to A/{\mathfrak{J}}\), we obtain homomorphisms

\leavevmode\vadjust pre{\hypertarget{fga-1-equation-3.3}{}}%
\begin{eqenv}
\[
  \begin{aligned}
    \psi_x\colon \operatorname{Ext}_A^\bullet(A/{\mathfrak{J}},M) &\to \operatorname{H}^\bullet(\operatorname{Hom}_A(K_x,M))
  \\\psi_x\colon \operatorname{Tor}_\bullet^A(A/{\mathfrak{J}},M) &\leftarrow\operatorname{H}_\bullet(K_x\otimes M).
  \end{aligned}
\tag{3.3}
\]

\end{eqenv}

But we immediately note that, in the maximal dimension \(p\), the cohomology of the right-hand side is \(M(x_1M+\ldots+x_pM)\) (resp. the set of elements of \(M\) that are annihilated by each of the \(x_i\)).
Since the \(x_i\) are in \({\mathfrak{J}}\), we thus obtain homomorphisms

\leavevmode\vadjust pre{\hypertarget{fga-1-equation-3.4}{}}%
\begin{eqenv}
\[
  \begin{aligned}
    \operatorname{H}^p(\operatorname{Hom}_A(K_x,M)) &\to M\otimes A/{\mathfrak{J}}
  \\\operatorname{H}_p(K_x\otimes M) &\leftarrow\operatorname{Hom}_A(A/{\mathfrak{J}},M).
  \end{aligned}
\tag{3.4}
\]

\end{eqenv}

By composing the homomorphisms in \protect\hyperlink{fga-1-equation-3.3}{(3.3)} and \protect\hyperlink{fga-1-equation-3.4}{(3.4)} we obtain the homomorphisms in \protect\hyperlink{fga-1-equation-3.2}{(3.2)} that we wanted to define.
The verification of i. is tedious, but does not present any difficulties.

\leavevmode\vadjust pre{\hypertarget{fga-1-proposition-4}{}}%
\begin{itenv}{Proposition 4}
\oldpage{149-07}Let \(A\) be a commutative unital ring, and let \((x_1,\ldots,x_p)\) be a sequence of elements of \(A\) such that, for \(1\leqslant i\leqslant p\), the image of \(x_i\) in the quotient of \(A\) by the ideal generated by \((x_1,\ldots,x_{i-1})\) is not a zero divisor.
Let \({\mathfrak{J}}\) be the ideal generated by the \(x_i\).
Then \({\mathfrak{J}}/{\mathfrak{J}}^2\) is a free \((A/{\mathfrak{J}})\)-module, with basis given by the canonical images of the \(x_i\);
the complex \(K_x\) is a free resolution of \(A/{\mathfrak{J}}\);
and, for every \(A\)-modules \(M\), the homomorphisms in \protect\hyperlink{fga-1-equation-3.1}{(3.1)} in dimension \(p\) are bijective.
The same is true for the analogous homomorphisms defined for arbitrary degree \(i\) as long as \({\mathfrak{J}}\cdot M=0\).

\end{itenv}

(The essential point, from which all others follow, is the acyclicity of \(K_x\), which is a well-known fact, under the conditions given).

\leavevmode\vadjust pre{\hypertarget{fga-1-proposition-4-corollary-1}{}}%
\begin{itenv}{Corollary 1}
With \(A\) and \({\mathfrak{J}}\) as above, suppose further that \(A\) is a regular affine algebra of \(\dim n\) over a perfect field \(k\), and that \(A/{\mathfrak{J}}\) is a regular affine algebra.
Denote by \(\Omega^i(A)\) and \(\Omega^i(A/{\mathfrak{J}})\) the modules of Kähler differentials.
Then we have a canonical isomorphism
\[
  \operatorname{Ext}_A^p(\Omega^{n-p}(A/{\mathfrak{J}}),\Omega^n(A)) = A/{\mathfrak{J}}.
\tag{3.5}
\]
which is compatible with localisation.

\end{itenv}

\begin{proof}
Indeed, \(\Omega^{n-p}(A/{\mathfrak{J}})\) is a free \((A/{\mathfrak{J}})\)-module of rank \(1\), and, similarly, \(\Omega^n(A)\) is a free \(A\)-module of rank \(n\), and so the left-hand side is equal to
\[
  \operatorname{Ext}_A^p(A/{\mathfrak{J}},A) \otimes \Omega^{n-p}(A/{\mathfrak{J}})' \otimes \Omega^n(A)
\]
(where the ``\('\)'' notation denotes the dual \((A/{\mathfrak{J}})\)-module).
The tensor product of these last two factors can be identified with \(\wedge^p({\mathfrak{J}}/{\mathfrak{J}}^2)\), and so the whole thing can be identified with \(\operatorname{Ext}_A^p(A/{\mathfrak{J}},\wedge^p({\mathfrak{J}}/{\mathfrak{J}}^2))\), and thus, by \protect\hyperlink{fga-1-proposition-4}{the proposition}, with
\[
  \operatorname{Hom}_A(\wedge^p {\mathfrak{J}}/{\mathfrak{J}}^2,\wedge^p {\mathfrak{J}}/{\mathfrak{J}}^2)
\]
i.e.~to \(A/{\mathfrak{J}}\).
\end{proof}

In particular, there is a distinguished element in \(\operatorname{Ext}_A^p(\Omega^{n-p}(A/{\mathfrak{J}}),\Omega^n(A))\), corresponding to the unit of \(A/{\mathfrak{J}}\), called the \emph{fundamental class} of the ideal \({\mathfrak{J}}\) in \(A\).
(In fact, it can be defined under rather more general conditions).
We can write \protect\hyperlink{fga-1-proposition-4-corollary-1}{Corollary 1} in a more geometric and global form:

\hypertarget{fga-1-proposition-4-corollary-2}{}
\begin{itenv}{Corollary 2}

Let \(X\) be a non-singular variety over an algebraically-closed field \(k\), \(Y\) a closed non-singular subvariety of \(X\), \({\mathscr{O}}_X\) the structure sheaf of \(X\), and \({\mathscr{O}}_Y\) the structure sheaf of \(Y\), considered as a quotient sheaf of \({\mathscr{O}}_X\).
Let \(n\) be the dimension of \(X\), and \(n-p\) the dimension of \(Y\).
Let \(\Omega_X\) (resp. \(\Omega_Y\)) be the sheaf of germs of regular differential forms on \(X\) (resp. \(Y\)).
Then we have canonical isomorphisms
\[
  \mathscr{E}\kern -.5pt xt_{{\mathscr{O}}_X}^p(\Omega_Y^{n-p},\Omega_X^n) = {\mathscr{O}}_Y
\tag{3.6}
\]
as well as
\oldpage{149-08}

\leavevmode\vadjust pre{\hypertarget{fga-1-equation-3.6bis}{}}%
\begin{eqenv}
\[
  \operatorname{Ext}_{{\mathscr{O}}_X}^p({\mathscr{O}}_Y,\Omega_X^n) = \Omega_Y^{n-p}.
\tag{3.6 bis}
\]

\end{eqenv}

\end{itenv}

Equation \protect\hyperlink{fga-1-equation-3.6bis}{(3.6 bis)} can serve as the \emph{definition} of \(\Omega_Y^{n-p}\) when \(Y\) is a singular variety.
More precisely:

\leavevmode\vadjust pre{\hypertarget{fga-1-proposition-5}{}}%
\begin{itenv}{Proposition 5}
Let \(X\) be a non-singular algebraic variety of dimension \(n\), and let \(Y\) be an algebraic subset of dimension \(q=n-p\) of \(X\).
Let \({\mathscr{F}}\) be a coherent algebraic sheaf on \(X\) with support contained in \(Y\), and let \({\mathscr{L}}\) be a locally-free algebraic sheaf on \(X\).
Then the sheaves \(\mathscr{E}\kern -.5pt xt_{{\mathscr{O}}_X}^i({\mathscr{F}},{\mathscr{L}})\) are zero for \(i<p\), and, when \(i=p\), there is a canonical isomorphism
\[
  \mathscr{E}\kern -.5pt xt_{{\mathscr{O}}_X}^p({\mathscr{F}},{\mathscr{L}}) = \mathscr{H}\kern -.5pt om_{{\mathscr{O}}_X}({\mathscr{F}},\mathscr{E}\kern -.5pt xt^p({\mathscr{O}}_X/{\mathfrak{J}},{\mathscr{L}}))
\tag{3.7}
\]
where \({\mathfrak{J}}\) denotes an arbitrary sheaf of ideals on \(X\) that annihilates \({\mathscr{F}}\) and has \(Y\) as its set of zeros.
In particular, if \({\mathscr{F}}\) is a coherent algebraic sheaf on \(Y\), then
\[
  \mathscr{E}\kern -.5pt xt_{{\mathscr{O}}_X}^p({\mathscr{F}},{\mathscr{L}}) = \mathscr{H}\kern -.5pt om_{{\mathscr{O}}_Y}({\mathscr{F}},\mathscr{E}\kern -.5pt xt^p({\mathscr{O}}_Y,{\mathscr{L}})).
\tag{3.7 bis}
\]
Finally, with \({\mathscr{F}}\) still a coherent algebraic sheaf on \(Y\), the sheaves \({\mathscr{E}}^i=\mathscr{E}\kern -.5pt xt_{{\mathscr{O}}_X}^{p+i}({\mathscr{F}},\Omega_X^n)\) do not depend on the choice of immersion of the algebraic space \(Y\) into the non-singular algebraic variety \(X\).

\end{itenv}

\begin{proof}
Since the question is local, we can assume that \(X\) is affine and that \({\mathscr{L}}={\mathscr{O}}_X\).
This then reduces to a problem of commutative algebra, and, more specifically, of local algebra:
if \(A\) is a regular locality, and \(M\) an \(A\)-module whose support is of dimension \(\leqslant q=n-p\), then we have to prove that \(\operatorname{Ext}_A^i(M,A)=0\) for \(i<p\) and that \(\operatorname{Ext}_A^p(M,A)=\operatorname{Hom}_A(M,\operatorname{Ext}^p(A/{\mathfrak{J}},A))\), where \({\mathfrak{J}}\) is an arbitrary ideal of ``dimension'' \(\leqslant q\) that annihilates \(M\).
For the first claim, we proceed by induction on \(q\):
an immediate \emph{dévissage} leads to the case where \(M\) is of the form \(A/{\mathfrak{J}}\), and thus leads, by replacing \({\mathfrak{J}}\) with a smaller ideal and using the induction hypothesis, as well as the exact sequence of the \(\operatorname{Ext}\), to the case where \({\mathfrak{J}}\) is generated by a ``system of parameters'', as in \protect\hyperlink{fga-1-proposition-4}{Proposition 4}, where the result is immediate.
The previous result implies that, if \({\mathfrak{J}}\) is a fixed ideal of ``dimension'' \(\leqslant q\), then the contravariant functor \(E(M)=\operatorname{Ext}_A^p(M,A)\) to the category of \((A/{\mathfrak{J}})\)-modules is left exact;
furthermore, it sends direct sums to direct products, from which it easily follows that \(E(M)=\operatorname{Hom}_A(M,E(A))\).
Finally, the last claim of \protect\hyperlink{fga-1-proposition-5}{Proposition 5} is more subtle, and follows from an intrinsic characterisation of the \(E^i(F)\) via a local duality theorem which cannot be stated here.
\end{proof}

\hypertarget{fga-1-proposition-5-corollary}{}
\begin{itenv}{Corollary}

\oldpage{149-09}Denote by \(\omega_Y^q\) the sheaf \(\operatorname{Ext}_{{\mathscr{O}}_X}^p({\mathscr{O}}_Y,\Omega_X^n)\).
Then there is a functorial isomorphism for coherent algebraic sheaves \({\mathscr{F}}\) on \(Y\):

\leavevmode\vadjust pre{\hypertarget{fga-1-equation-3.8}{}}%
\begin{eqenv}
\[
  \mathscr{E}\kern -.5pt xt_{{\mathscr{O}}_X}^p({\mathscr{F}},\Omega_X^n) = \mathscr{H}\kern -.5pt om_{{\mathscr{O}}_X}({\mathscr{F}},\omega_Y^q).
\tag{3.8}
\]

\end{eqenv}

\end{itenv}

\hypertarget{fga-1-section-4}{%
\subsection{Cohomology class associated to a subvariety}\label{fga-1-section-4}}

In all that follows, \(X\) denotes an algebraic set of dimension \(n\), defined over a field \(k\) that we assume, for simplicity, to be algebraically closed.
Except for in \protect\hyperlink{fga-1-section-6}{§6}, we assume that \(X\) is non-singular.
We denote by \({\mathscr{O}}_X\) the structure sheaf of \(X\), and by \(\Omega_X^\bullet=\bigcup_p\Omega_X^p\) the sheaf of germs of differential forms on \(X\).
If \(Y\) is a closed subset of \(X\), then we identify coherent algebraic sheaves on \(Y\) with coherent algebraic sheaves on \(X\) that are zero outside of \(Y\);
we do this, in particular, with \({\mathscr{O}}_Y\) and \(\Omega_Y\).

\leavevmode\vadjust pre{\hypertarget{fga-1-lemma-1}{}}%
\begin{itenv}{Lemma 1}
Let \({\mathscr{F}}\) be a coherent algebraic sheaf on \(X\) whose support is of dimension \(\leqslant n-p\), and let \({\mathscr{L}}\) be a coherent algebraic sheaf on \(X\) that is locally free.
Then \(\mathscr{E}\kern -.5pt xt_{{\mathscr{O}}_X}^i(X;{\mathscr{F}},{\mathscr{L}})\) is zero for \(i<p\), and there is a canonical isomorphism

\leavevmode\vadjust pre{\hypertarget{fga-1-equation-4.1}{}}%
\begin{eqenv}
\[
  \operatorname{Ext}_{\mathscr{O}}^p(X;{\mathscr{F}},{\mathscr{L}}) = \operatorname{H}^0(X,\mathscr{E}\kern -.5pt xt_{\mathscr{O}}^p({\mathscr{F}},{\mathscr{L}})).
\tag{4.1}
\]

\end{eqenv}

If \({\mathscr{F}}\) is a coherent algebraic sheaf on a closed subset \(W\) of \(X\) of dimension \(\leqslant n-p\), then we have a canonical isomorphism

\leavevmode\vadjust pre{\hypertarget{fga-1-equation-4.1bis}{}}%
\begin{eqenv}
\[
  \mathscr{E}\kern -.5pt xt_{{\mathscr{O}}_X}^p({\mathscr{F}},{\mathscr{L}}) = \mathscr{H}\kern -.5pt om_{{\mathscr{O}}_X}({\mathscr{F}}\otimes{\mathscr{L}}'\otimes\Omega_X^n,\omega_Y^{n-p})
\tag{4.1 bis}
\]

\end{eqenv}

where \(\omega_Y^{n-p}\) is the sheaf on \(Y\) defined in \protect\hyperlink{fga-1-proposition-5-corollary}{the corollary to Proposition 5} (which can be identified with \(\Omega_Y^{n-p}\) if \(Y\) is non-singular).

\end{itenv}

\begin{proof}
The formula in \protect\hyperlink{fga-1-equation-4.1}{(4.1)} is an immediate consequence of the spectral sequence from \protect\hyperlink{fga-1-proposition-1}{Proposition 1}, as well as \protect\hyperlink{fga-1-proposition-5}{Proposition 5};
by the formula in \protect\hyperlink{fga-1-equation-3.8}{(3.8)}, we can write
\[
  \begin{gathered}
    \mathscr{E}\kern -.5pt xt_{{\mathscr{O}}_X}^p({\mathscr{F}},{\mathscr{L}})
    = {\mathscr{L}}\otimes(\Omega_X^n)'\otimes\mathscr{E}\kern -.5pt xt_{{\mathscr{O}}_X}({\mathscr{F}},\Omega_X^n)
  \\= {\mathscr{L}}\otimes(\Omega_X^n)'\otimes\mathscr{H}\kern -.5pt om_{{\mathscr{O}}_X}({\mathscr{F}},\omega_Y^q)
  \\= \mathscr{H}\kern -.5pt om_{{\mathscr{O}}_X}({\mathscr{F}}\otimes{\mathscr{L}}'\otimes\Omega_X^n,\omega_Y^q)
  \end{gathered}
\]
where \(q=n-p\), whence the formula in \protect\hyperlink{fga-1-equation-4.1bis}{(4.1 bis)}.
\end{proof}

\oldpage{149-10}Setting, in particular, \({\mathscr{F}}={\mathscr{O}}_Y\) and \({\mathscr{L}}=\Omega_X^p\), we obtain (taking into account the fact that \(\Omega_X^n\otimes(\Omega_X^p)'=\Omega_X^{n-p}\)) a canonical isomorphism

\leavevmode\vadjust pre{\hypertarget{fga-1-equation-4.2}{}}%
\begin{eqenv}
\[
  \operatorname{Ext}_{{\mathscr{O}}_X}^p(X;{\mathscr{O}}_Y,\Omega_X^p) = \operatorname{Hom}_{{\mathscr{O}}_X}(X;\Omega_X^{n-p},\omega_Y^{n-p}).
\tag{4.2}
\]

\end{eqenv}

Now suppose, for simplicity, that \(Y\) is \emph{non-singular}, so that \(\omega_Y^{n-p}=\Omega_Y^{n-p}\).
There is a natural homomorphism from \(\Omega_X^{n-p}\) to \(\Omega_Y^{n-p}\), whence a canonical section \(s_Y\) of the sheaf \(\mathscr{E}\kern -.5pt xt_{{\mathscr{O}}_X}^p({\mathscr{O}}_Y,\Omega_X^p)\), that we call, if all the components of \(Y\) are of dimension \(n-p\), the \emph{fundamental section} of the sheaf \(\mathscr{E}\kern -.5pt xt_{{\mathscr{O}}_X}^p({\mathscr{O}}_Y,\Omega_X^p)\).
By \protect\hyperlink{fga-1-equation-4.1}{(4.1)}, this section defines an element of \(\operatorname{Ext}_{{\mathscr{O}}_X}^p(X;{\mathscr{O}}_Y,\Omega_X^p)\).
But the natural homomorphism \({\mathscr{O}}_X\to{\mathscr{O}}_Y\) defines a homomorphism
\[
  \operatorname{Ext}_{{\mathscr{O}}_X}^p(X;{\mathscr{O}}_Y,\Omega_X^p) \to \operatorname{Ext}_{{\mathscr{O}}_X}^p(X;{\mathscr{O}}_X,\Omega_X^p) = \operatorname{H}^p(X,\Omega_X^p).
\]
We thus obtain an element of \(\operatorname{H}^p(X,\Omega_X^p)\), denoted by \(P_X(Y)\), that we call the \emph{cohomology class of \(Y\) in \(X\)};
it is induced by the section \(s_Y\) of \(\mathscr{E}\kern -.5pt xt_{{\mathscr{O}}_X}^p({\mathscr{O}}_Y,\Omega_X^p)\) by the following diagram of homomorphisms:

\leavevmode\vadjust pre{\hypertarget{fga-1-equation-4.3}{}}%
\begin{eqenv}
\[
  \begin{CD}
    \operatorname{Ext}_{{\mathscr{O}}_X}^p(X;{\mathscr{O}}_Y,\Omega_X^p) @>\sim>> \operatorname{H}^0(X,\mathscr{E}\kern -.5pt xt_{{\mathscr{O}}_X}^p({\mathscr{O}}_Y,\Omega_X^p))
  \\@VVV
  \\\begin{gathered}
      \operatorname{Ext}_{{\mathscr{O}}_X}^p(X;{\mathscr{O}}_X,\Omega_X^y)
      \\=\operatorname{H}^p(X,\Omega_X^p)
    \end{gathered}
  \end{CD}
\tag{4.3}
\]

\end{eqenv}

We define a \emph{non-singular cycle} of dimension \(n-p\) to be any element of the free abelian group generated by the non-singular irreducible subvarieties of dimension \(n-p\) in \(X\).
Then the function \(Y\mapsto P(Y)\) can be extended to a homomorphism from the group of non-singular cycles of dimension \(n-p\) on \(X\) to the group \(\operatorname{H}^p*X,\Omega_X^p)\).

Let \(Z^{n-p}\) and \(Z'^{n-p'}\) be non-singular cycles of dimension \(n-p\) and \(n-p'\) (respectively);
we say that they \emph{intersect transversally} if every component of \(Z\) intersects transversally with every component of \(Z'\).
Then the cycle \(Z\cdot Z'\) is defined, and is a non-singular cycle of dimension \(n-p-p'\).
With this, we have:

\leavevmode\vadjust pre{\hypertarget{fga-1-theorem-1}{}}%
\begin{itenv}{Theorem 1}
\oldpage{149-11}If \(Z^{n-p}\) and \(Z'^{n-p'}\) are non-singular cycles that intersect transversally, then

\leavevmode\vadjust pre{\hypertarget{fga-1-equation-4.4}{}}%
\begin{eqenv}
\[
  P_X(Z\cdot Z') = P_X(Z)\cdot P_X(Z')
\tag{4.4}
\]

\end{eqenv}

where the product on the right-hand side is the cup product:
\[
  \operatorname{H}^p(X,\Omega_X^p)\times\operatorname{H}^{p'}(X,\Omega_X^{p'}) \to \operatorname{H}^{p+p'}(X,\Omega_X^{p+p'}).
\]
(We assume that \(X\) is isomorphic to a locally closed subset of a projective space).

\end{itenv}

This last hypothesis is used only to be able to conclude that every coherent algebraic sheaf on \(X\) is a quotient of a locally-free coherent algebraic sheaf (Serre) and thus admits a left resolution by locally-free sheaves.

\begin{proof}
To prove \protect\hyperlink{fga-1-theorem-1}{Theorem 1}, we can assume that \(Z\) and \(Z'\) are irreducible non-singular subvarieties \(Y\) and \(Y'\) that intersect transversally.
Let \({\mathscr{L}}_\bullet\) be a left resolution of \({\mathscr{O}}_Y\) by locally-free sheaves;
then, by \protect\hyperlink{fga-1-proposition-3}{Proposition 3}, the diagram of homomorphisms in \protect\hyperlink{fga-1-equation-4.3}{(4.3)} can be identified with the diagram
\[
  \begin{CD}
    (\underline{\operatorname{R}}^p\Gamma)\big(\mathscr{H}\kern -.5pt om_{{\mathscr{O}}_X}({\mathscr{L}}_\bullet,\Omega_X^p)\big) @>\beta>> \Gamma\big(\operatorname{H}^p\big(\mathscr{H}\kern -.5pt om_{{\mathscr{O}}_X}({\mathscr{L}}_\bullet,\Omega_X^p)\big)\big)
  \\@V\alpha VV .
  \\(\operatorname{R}^p\Gamma)(\Omega_X^p)
  \end{CD}
\]
where \(\beta\) is an isomorphism, and where \(\Gamma\) is the ``group of sections'' functor on the category of abelian sheaves on \(X\), \(\underline{\operatorname{R}}^p\Gamma\) is its hypercohomology in dimension \(p\), and \(\operatorname{R}^p\Gamma\) is its \(p\)-th derived functor.
For simplicity, we assume that \({\mathscr{L}}_0={\mathscr{O}}_X\), and that the augmentation \({\mathscr{L}}_0\to{\mathscr{O}}_Y\) is the natural homomorphism (which we can indeed safely assume);
then \(\alpha\) is induced by the homomorphism of complexes \({\mathscr{O}}_X\to{\mathscr{L}}\) (with \({\mathscr{O}}_X\) being thought of as a complex concentrated in degree \(0\)), taking into account the fact that \(\underline{\operatorname{R}}^p\Gamma({\mathscr{K}})=\operatorname{R}^p\Gamma({\mathscr{K}}_0)\) if \({\mathscr{K}}\) is a complex of sheaves concentrated in degree \(0\).
The homomorphism \(\beta\) is a well-known ``boundary map''.
Consider an analogous diagram, relative to a locally-free resolution \({\mathscr{L}}'_\bullet\) of \({\mathscr{O}}_Y\), and consider the commutative diagram of pairings:

\leavevmode\vadjust pre{\hypertarget{fga-1-equation-4.5}{}}%
\begin{eqenv}
\[
  \scriptsize
  \begin{CD}
    \operatorname{R}^p\Gamma(\Omega_X^p)
    @<<<
    \underline{\operatorname{R}}^p\Gamma\big(\mathscr{H}\kern -.5pt om_{{\mathscr{O}}_X}({\mathscr{L}}_\bullet,\Omega_X^p)\big)
    @>\sim>>
    \Gamma\big(\operatorname{H}^p\big(\mathscr{H}\kern -.5pt om_{{\mathscr{O}}_X}({\mathscr{L}}_\bullet,\Omega_X^p)\big)\big)
  \\@. @. @.
  \\\times @. \times @. \times
  \\@. @. @.
  \\\operatorname{R}^{p'}\Gamma(\Omega_X^{p'})
    @<<<
    \underline{\operatorname{R}}^{p'}\Gamma\big(\mathscr{H}\kern -.5pt om_{{\mathscr{O}}_X}({\mathscr{L}}'_\bullet,\Omega_X^{p'})\big)
    @>\sim>>
    \Gamma\big(\operatorname{H}^{p'}\big(\mathscr{H}\kern -.5pt om_{{\mathscr{O}}_X}({\mathscr{L}}'_\bullet,\Omega_X^{p'})\big)\big)
  \\@VVV @VVV @VVV
  \\\operatorname{R}^{p+p'}\Gamma(\Omega_X^{p+p'})
    @<<<
    \underline{\operatorname{R}}^{p+p'}\Gamma\big(\mathscr{H}\kern -.5pt om_{{\mathscr{O}}_X}({\mathscr{L}}_\bullet\otimes{\mathscr{L}}'_\bullet,\Omega_X^{p+p'})\big)
    @>\sim>>
    \Gamma\big(\operatorname{H}^{p+p'}\big(\mathscr{H}\kern -.5pt om_{{\mathscr{O}}_X}({\mathscr{L}}_\bullet\otimes{\mathscr{L}}'_\bullet,\Omega_X^{p+p'})\big)\big)
  \end{CD}
\tag{4.5}
\]

\end{eqenv}

\oldpage{149-12}The pairings in the two columns on the right are induced by the pairing of complexes of sheaves
\[
  \mathscr{H}\kern -.5pt om_{{\mathscr{O}}_X}({\mathscr{L}}_\bullet,\Omega_X^p) \times \mathscr{H}\kern -.5pt om_{{\mathscr{O}}_X}({\mathscr{L}}'_\bullet,\Omega_X^{p'}) \to \mathscr{H}\kern -.5pt om_{{\mathscr{O}}_X}({\mathscr{L}}_\bullet\otimes{\mathscr{L}}',\Omega_X^{p+p'})
\]
that we define by using the exterior product \(\Omega_X^p\times\Omega_X^{p'}\to\Omega_X^{p+p'}\);
the pairing in the first column is the cup product (relative to the exterior product).
I claim that the last line of \protect\hyperlink{fga-1-equation-4.5}{(4.5)} can be identified with the diagram of isomorphisms analogous to \protect\hyperlink{fga-1-equation-4.3}{(4.3)}, where \(Y\) is replaced by \(Y\cap Y'\) and \(p\) by \(p+p'\).
For this, it suffices to show that \({\mathscr{L}}\otimes{\mathscr{L}}'\) is a resolution (evidently locally-free) of \({\mathscr{O}}_{Y\cap Y'}\).
But then
\[
  \operatorname{H}_0({\mathscr{L}}\otimes{\mathscr{L}}') = {\mathscr{O}}_Y\otimes{\mathscr{O}}_{Y'} = {\mathscr{O}}_{Y\cap Y'}
\]
and
\[
  \operatorname{H}_i({\mathscr{L}}\otimes{\mathscr{L}}') = \operatorname{Tor}_i^{{\mathscr{O}}_X}({\mathscr{O}}_Y,{\mathscr{O}}_{Y'}) = 0
\]
for \(i>0\), from the fact that \(Y\) and \(Y'\) intersect transversally.
Then \protect\hyperlink{fga-1-theorem-1}{Theorem 1} follows from the formula:

\leavevmode\vadjust pre{\hypertarget{fga-1-equation-4.6}{}}%
\begin{eqenv}
\[
  s_{Y}\cdot s_{Y'} = s_{Y\cdot Y'}
\tag{4.6}
\]

\end{eqenv}

(where the product on the left-hand side is that from the last column of \protect\hyperlink{fga-1-equation-4.5}{(4.5)}).
This formula in \protect\hyperlink{fga-1-equation-4.6}{(4.6)}, which is of a purely local nature, can easily be proven by taking \({\mathscr{L}}_\bullet\) and \({\mathscr{L}}'_\bullet\) to be the resolutions described in \protect\hyperlink{fga-1-proposition-4}{Proposition 4}.
We can similarly prove (even more easily) that \(Z\mapsto P_X(Z)\) is compatible with the cartesian product:

\leavevmode\vadjust pre{\hypertarget{fga-1-equation-4.7}{}}%
\begin{eqenv}
\[
  P_{X\times X'}(Z\times Z')  = P_X(Z)\otimes P_{X'}(Z')
\tag{4.7}
\]

\end{eqenv}

(a formula which holds true if \(Z\) (resp. \(Z'\)) is a non-singular cycle on the non-singular variety \(X\) (resp. \(X'\)), with \(Z\times Z'\) being thought of as a non-singular cycle on \(X\times X'\)).
From \protect\hyperlink{fga-1-equation-4.4}{(4.4)} and \protect\hyperlink{fga-1-equation-4.7}{(4.7)}, it follows that \(P_X(Z)\) is also compatible with the operation given by taking the ``inverse image'' under a morphism \(f\colon X\to X'\) of non-singular varieties:

\leavevmode\vadjust pre{\hypertarget{fga-1-equation-4.8}{}}%
\begin{eqenv}
\[
  P_X(f^{-1}(Z')) = f^*(P_{X'}(Z))
\tag{4.8}
\]

\end{eqenv}

a formula which holds true if \(Z\) is a non-singular cycle on \(X'\) such that \(f\) is ``transversal'' to \(Z\), i.e.~such that the graph of \(f\) is transversal to the cycle \(X\times Z'\) in \(X\times X'\).
\end{proof}

\leavevmode\vadjust pre{\hypertarget{fga-1-theorem-1-corollary-1}{}}%
\begin{itenv}{Corollary 1}
\oldpage{149-13}Let \(X\) and \(X'\) be non-singular varieties that are locally-closed in a projective space, and suppose that \(X'\) is complete.
Let \(U\) be a non-singular cycle on \(X\times X'\), and let \(a\) and \(b\) be points of \(X'\) such that \(U\) intersects transversally with the cycles \(X\times(a)\) and \(X\times(b)\).
Let \(Z\) and \(Z'\) be non-singular cycles on \(X\) such that \(Z\times(a)=(X\times(a))\cdot U\) and \(Z\times(b)=(X\times(b))\cdot U\).
Then
\[
  P_X(Z) = P_X(Z').
\]

\end{itenv}

\begin{proof}
Let \(f_a\colon X\to X\times X'\) be defined by \(f_a(x)=(x,a)\).
Then, by \protect\hyperlink{fga-1-equation-4.8}{(4.8)}, we have \(P(Z)=f_a^*(P(U))\);
similarly, \(P(Z')=f_b^*(P(U))\).
But then, using the Künneth formula
\[
  \operatorname{H}^\bullet(X\times X',\Omega_{X\times X'}^\bullet)
  = \operatorname{H}^\bullet(X,\Omega_X^\bullet)\otimes\operatorname{H}^\bullet(X',\Omega_{X'}^\bullet)
\]
and the fact that \(\operatorname{H}^0(X',\Omega_{X'})\) is simply the scalars, we easily see that \(f_a^*=f_b^*\), whence the result.
\end{proof}

For all \(x\in X\), \((x)\) is a non-singular subvariety of \(X\) of codimension \(n\), and thus defines an element \(\varepsilon_x\) of \(\operatorname{H}^n(X,\Omega_X^n)\).
If \(X\) is a non-singular projective variety, then it follows from the above \protect\hyperlink{fga-1-theorem-1-corollary-1}{Corollary 1} that \(\varepsilon_x\) does not depend on the chosen point \(x\), and we thus denote it by \(\varepsilon_X\) and call it the \emph{fundamental class} of \(\operatorname{H}^n(X,\Omega_X^n)\).

\leavevmode\vadjust pre{\hypertarget{fga-1-section-4-remark}{}}%
\begin{rmenv}{Remark}
To have a satisfying theory, we must define \(P_X(Z)\) for arbitrary cycles \(Z\), and prove \protect\hyperlink{fga-1-theorem-1}{Theorem 1} for proper intersections of cycles.
(At the time of writing this talk, this has still not been done in full generality).
Assuming that we have done this, the above \protect\hyperlink{fga-1-theorem-1-corollary-1}{Corollary 1} becomes the following:
\emph{if \(Z\) and \(Z'\) are two algebraically-equivalent cycles, then \(P_X(Z)=P_X(Z')\)}
(a claim which does not seem to follow from the above, even if \(Z\) and \(Z'\) are non-singular).

\end{rmenv}

\begin{rmenv}{Remark}
\emph{{[}Comp.{]}}
As I pointed out in my conference at the international Congress of Mathematicians in 1958\footnote{Grothendieck, A. ``The cohomology theory of abstract algebraic varieties'', in \emph{Proceedings of the international Congress of Mathematicians {[}1958, Edinburgh{]}}, Cambridge University Press (1960), 103--118.}, the questions raised here are now completely resolved.

The reader will find more information on the duality of coherent sheaves in \emph{loc. cit.}, p.112--115, as well as in {[}\protect\hyperlink{ref-GD1960}{10}{]}, and in {[}\protect\hyperlink{ref-Gro1960b}{9}{]}.
A more systematic treatment can be found in a later chapter of {[}\protect\hyperlink{ref-GD1960}{10}{]} (chapter IX in the provisional plan).

\end{rmenv}

\hypertarget{fga-1-section-5}{%
\subsection{The duality theorem}\label{fga-1-section-5}}

In this section, \(X\) denotes a non-singular projective variety of dimension \(n\).

\leavevmode\vadjust pre{\hypertarget{fga-1-theorem-2}{}}%
\begin{itenv}{Theorem 2}
The fundamental class \(\varepsilon_X\) of \(\operatorname{H}^n(X,\Omega_X^n)\) is a basis of this vector space.

\end{itenv}

(The proof of this will be given later on).
With the above theorem, we can thus identify \(\operatorname{H}^n(X,\Omega_X^n)\) with the field \(k\).
\oldpage{149-14}We now consider the pairings described in \protect\hyperlink{fga-1-section-2}{§2}, which give, in particular, a pairing
\[
  \operatorname{Ext}_{{\mathscr{O}}_X}^p(X;{\mathscr{O}}_X,{\mathscr{F}})\times\operatorname{Ext}_{{\mathscr{O}}_X}^{n-p}(X;{\mathscr{F}},\Omega_X^n) \to \operatorname{Ext}_{{\mathscr{O}}_X}^n(X;{\mathscr{O}}_X,\Omega_X^n)
\]
i.e.

\leavevmode\vadjust pre{\hypertarget{fga-1-equation-5.1}{}}%
\begin{eqenv}
\[
  \operatorname{H}^p(X,{\mathscr{F}})\times\operatorname{Ext}_{{\mathscr{O}}_X}^{n-p}(X;{\mathscr{F}},\Omega_X^n) \to \operatorname{H}^n(X,\Omega_X^n).
\tag{5.1}
\]

\end{eqenv}

Taking \protect\hyperlink{fga-1-theorem-2}{Theorem 2} into account, this pairing defines a homomorphism

\leavevmode\vadjust pre{\hypertarget{fga-1-equation-5.2}{}}%
\begin{eqenv}
\[
  \operatorname{Ext}_{{\mathscr{O}}_X}^{n-p}(X;{\mathscr{F}},\Omega_X^n) \to (\operatorname{H}^p(X,{\mathscr{F}}))'.
\tag{5.2}
\]

\end{eqenv}

This homomorphism is functorial in \({\mathscr{F}}\), and commutes with the coboundary maps relative to the exact sequences \(0\to{\mathscr{F}}'\to{\mathscr{F}}\to{\mathscr{F}}''\to0\).

\leavevmode\vadjust pre{\hypertarget{fga-1-theorem-3}{}}%
\begin{itenv}{Theorem 3}
The homomorphism in \protect\hyperlink{fga-1-equation-5.2}{(5.2)} is an isomorphism.

\end{itenv}

In particular, we recover the following result of Serre:

\leavevmode\vadjust pre{\hypertarget{fga-1-theorem-3-corollary}{}}%
\begin{itenv}{Corollary}
Let \(E\) be an algebraic vector bundle on \(X\), and \({\mathscr{O}}_X(E)\) the sheaf of germs of regular sections of \(X\).
Then we have canonical isomorphisms\footnote{\emph{{[}Trans.{]}} This equation is labelled (5.3) in the original copy of the notes, but this seems to be a typo, since a later equation shares the same number, and any references to (5.3) seem to indeed point to the later equation instead of this one.}
\[
  (\operatorname{H}^p(X,{\mathscr{O}}_X(E)))' = \operatorname{H}^{n-p}(X,\Omega_X^n\otimes{\mathscr{O}}_X(E')).
\]

\end{itenv}

\begin{proof}
It suffices to apply \protect\hyperlink{fga-1-theorem-3}{Theorem 3} and \protect\hyperlink{fga-1-proposition-1-corollary-1}{Corollary 1 of Proposition 1}.
\end{proof}

\protect\hyperlink{fga-1-theorem-2}{Theorem 2} and \protect\hyperlink{fga-1-theorem-3}{Theorem 3} will follow from the following claim:

\leavevmode\vadjust pre{\hypertarget{fga-1-D}{}}%
\begin{itenv}{(D)}
The homomorphism

\leavevmode\vadjust pre{\hypertarget{fga-1-equation-5.2bis}{}}%
\begin{eqenv}
\[
  \operatorname{Ext}_{{\mathscr{O}}_X}^{n-p}(X;{\mathscr{F}},\Omega_X^n) \to (\operatorname{H}^p(X,{\mathscr{F}}))'\otimes{\mathscr{L}}
\tag{5.2 bis}
\]

\end{eqenv}

(where \({\mathscr{L}}=\operatorname{H}^n(X,\Omega_X^n)\)) induced by the pairing in \protect\hyperlink{fga-1-equation-5.1}{(5.1)} is an isomorphism.

\end{itenv}

We will show that \protect\hyperlink{fga-1-D}{(D)} implies \protect\hyperlink{fga-1-theorem-2}{Theorem 2}.
Let \(k_x={\mathscr{O}}_{(x)}\) be the structure sheaf of the variety consisting of a single point \(x\in X\), and consider the canonical homomorphism \({\mathscr{O}}_X\to k_x\), and the associated homomorphism

\leavevmode\vadjust pre{\hypertarget{fga-1-equation-5.3}{}}%
\begin{eqenv}
\[
  \operatorname{H}^0(X,{\mathscr{O}}_X) \to \operatorname{H}^0(X,k_x).
\tag{5.3}
\]

\end{eqenv}

Its transpose can be identified with the homomorphism

\leavevmode\vadjust pre{\hypertarget{fga-1-equation-5.4}{}}%
\begin{eqenv}
\[
  \operatorname{Ext}_{{\mathscr{O}}_X}^n(X;k_x,\Omega_X^n)\otimes{\mathscr{L}}' \to \operatorname{Ext}_{{\mathscr{O}}_X}^n(X;{\mathscr{O}}_X,\Omega_X^n)\otimes{\mathscr{L}}'
\tag{5.4}
\]

\end{eqenv}

induced by the homomorphism between the \(\operatorname{Ext}^n\) associated to \({\mathscr{O}}_X\to k_x\), i.e.

\leavevmode\vadjust pre{\hypertarget{fga-1-equation-5.5}{}}%
\begin{eqenv}
\[
  \operatorname{Ext}_{{\mathscr{O}}_X}^n(X;k_x,\Omega_X^n) \to \operatorname{Ext}_{{\mathscr{O}}_X}^n(X;{\mathscr{O}}_X,\Omega_X^n)
\tag{5.5}
\]

\end{eqenv}

\oldpage{149-15}Since \protect\hyperlink{fga-1-equation-5.3}{(5.3)} is an isomorphism, so too is \protect\hyperlink{fga-1-equation-5.4}{(5.4)}, and thus so too is \protect\hyperlink{fga-1-equation-5.5}{(5.5)}.
Since \(s_{(x)}\) is a basis of \(\operatorname{Ext}_{{\mathscr{O}}_X}^n(X;k_x,\Omega_X^n)\) by \protect\hyperlink{fga-1-equation-4.2}{(4.2)}, it indeed follows that its image \(\varepsilon_X\) is a basis of \(\operatorname{H}^n(X,\Omega_X^n)\).

It remains only to prove the statement of \protect\hyperlink{fga-1-D}{(D)}, which will follow in a purely formal way from some elementary facts summarised in the following lemmas.
Here we suppose that \(X\) is a closed subset (singular or not) of the projective space \(\mathbf{P}\) of dimension \(r\).
We use the notation \({\mathscr{O}}_\mathbf{P}(m)\) to denote the sheaf on \(\mathbf{P}\) denoted by \({\mathscr{O}}(m)\) in {[}\protect\hyperlink{ref-Ser1955}{18}{]}, and the notation \({\mathscr{O}}_X(m)\) for the analogous sheaf on \(X\).

\leavevmode\vadjust pre{\hypertarget{fga-1-lemma-2}{}}%
\begin{itenv}{Lemma 2}
The statement of \protect\hyperlink{fga-1-D}{(D)} is true if \(X=\mathbf{P}\) and \({\mathscr{F}}={\mathscr{O}}_\mathbf{P}(m)\).

\end{itenv}

\begin{proof}
This lemma can be proved by a direct calculation.
The explicit calculation of the \(\operatorname{H}^i(\mathbf{P},{\mathscr{O}}_\mathbf{P}(m))\) can be found in {[}\protect\hyperlink{ref-Ser1955}{18}{]}, but it also can be done in a simpler way.
Computing the cup product \(\operatorname{H}^i(\mathbf{P},{\mathscr{O}}_\mathbf{P}(m))\times\operatorname{H}^j(\mathbf{P},{\mathscr{O}}_\mathbf{P}(m)') \to \operatorname{H}^{i+j}(\mathbf{P},{\mathscr{O}}_\mathbf{P}(m+m'))\) (which is necessary to calculate the pairing in \protect\hyperlink{fga-1-equation-5.1}{(5.1)}) does not present any difficulty.
\end{proof}

\leavevmode\vadjust pre{\hypertarget{fga-1-lemma-3}{}}%
\begin{itenv}{Lemma 3}
Every coherent algebraic sheaf \({\mathscr{F}}\) on \(X\) is isomorphic to a sheaf that is some quotient of \({\mathscr{O}}_X(-m)^k\), and we can take \(m\) to be as large as we wish.

\end{itenv}

\begin{proof}
This follows from the fact that \({\mathscr{F}}\otimes{\mathscr{O}}_X(m)\) is ``generated by its sections'' for large enough \(m\);
see {[}\protect\hyperlink{ref-Ser1955}{18}{]}.
\end{proof}

\leavevmode\vadjust pre{\hypertarget{fga-1-lemma-4}{}}%
\begin{itenv}{Lemma 4}
Let \(i>0\).
Then \(\operatorname{H}^{r-i}(\mathbf{P},{\mathscr{O}}_\mathbf{P}(-m))=0\) for large enough \(m\);
and, for every coherent algebra sheaf \({\mathscr{B}}\) on \(X\), we have that \(\operatorname{Ext}_{{\mathscr{O}}_X}^i(X;{\mathscr{O}}_X(-m),{\mathscr{B}})=0\) for large enough \(m\).

\end{itenv}

\begin{proof}
The first claim follows from the explicit calculations mentioned above;
for the second, we note that we have an isomorphism
\[
  \operatorname{Ext}_{{\mathscr{O}}_X}^i(X;{\mathscr{O}}_X(-m),{\mathscr{B}}) = \operatorname{H}^i(X,{\mathscr{B}}\otimes{\mathscr{O}}(m))
\]
(\protect\hyperlink{fga-1-proposition-1-corollary-1}{Corollary 1 of Proposition 1}), whence the conclusion, by a well-known result of {[}\protect\hyperlink{ref-Ser1955}{18}{]}.
\end{proof}

Combining the previous two lemmas, we find:

\leavevmode\vadjust pre{\hypertarget{fga-1-lemma-3-and-lemma-4-corollary}{}}%
\begin{itenv}{Corollary}
Let \(i>0\).
Then the functor \({\mathscr{F}}\mapsto\operatorname{H}^{r-i}(\mathbf{P},{\mathscr{F}})\) on the category of coherent algebraic sheaves on \(\mathbf{P}\) is coeffaceable, and so too is the functor \(\operatorname{Ext}_{{\mathscr{O}}_X}^i(X;{\mathscr{F}},{\mathscr{B}})\) on the category of coherent algebraic sheaves on \(X\).

\end{itenv}

\leavevmode\vadjust pre{\hypertarget{fga-1-lemma-5}{}}%
\begin{itenv}{Lemma 5}
\oldpage{149-16}Let \({\mathscr{A}}\) and \({\mathscr{B}}\) be coherent algebraic sheaves on \(X\), and let \({\mathscr{A}}(m)={\mathscr{A}}\otimes{\mathscr{O}}_X(m)\).
Then, for large enough \(m\), the canonical homomorphism
\[
  \operatorname{Ext}_{{\mathscr{O}}_X}^i(X;{\mathscr{A}}(-m),{\mathscr{B}})
  \to \operatorname{H}^0(X,\mathscr{E}\kern -.5pt xt_{{\mathscr{O}}_X}^i({\mathscr{A}}(-m),{\mathscr{B}}))
  = \operatorname{H}^0(X,\mathscr{E}\kern -.5pt xt_{{\mathscr{O}}_X}^i({\mathscr{A}},{\mathscr{B}})(m))
\]
is an isomorphism.

\end{itenv}

\begin{proof}
This follows immediately from the spectral sequence in \protect\hyperlink{fga-1-proposition-1}{Proposition 1} applied to \({\mathscr{A}}(-m)\) and \({\mathscr{B}}\), since we then have that
\[
  E_2^{p,q}({\mathscr{A}}(-m),{\mathscr{B}})
  = \operatorname{H}^p(X,\mathscr{E}\kern -.5pt xt_{{\mathscr{O}}_X}^q({\mathscr{A}}(-m),{\mathscr{B}}))
  = \operatorname{H}^p(X,\mathscr{E}\kern -.5pt xt_{{\mathscr{O}}_X}^q({\mathscr{A}},{\mathscr{B}})(m))
\]
which is zero for \(p>0\) and large enough \(m\).
\end{proof}

We now prove \protect\hyperlink{fga-1-D}{(D)} in the case where \(X=\mathbf{P}\).
We will first prove that \protect\hyperlink{fga-1-equation-5.2bis}{(5.2 bis)} is an isomorphism for \(p=n\);
since both sides are then left-exact functors (since \(\operatorname{H}^{r+i}(\mathbf{P},{\mathscr{F}})=0\)), it follows from \protect\hyperlink{fga-1-lemma-3}{Lemma 3} that it suffices to prove the claim for \({\mathscr{F}}={\mathscr{O}}_\mathbf{P}(-m)\), but this is covered by \protect\hyperlink{fga-1-lemma-2}{Lemma 2}.
Since the homomorphisms in \protect\hyperlink{fga-1-equation-5.2bis}{(5.2 bis)} are functorial and compatible with the coboundary maps, and since, for \(p<n\), both sides of \protect\hyperlink{fga-1-equation-5.2bis}{(5.2 bis)} are coeffaceable functors in \({\mathscr{F}}\) (the \protect\hyperlink{fga-1-lemma-3-and-lemma-4-corollary}{corollary to Lemmas 3 and 4}), it follows, by a standard argument, that \protect\hyperlink{fga-1-equation-5.2bis}{(5.2 bis)} is an isomorphism for all \(p\).
This proves the duality theorem for the projective space.

Now suppose that \(X\) is arbitrary, but non-singular.
By the duality theorem for \(\mathbf{P}\), we have an isomorphism
\[
  \operatorname{H}^n(X,{\mathscr{F}})
  = \operatorname{H}^n(\mathbf{P},{\mathscr{F}})'
  = \operatorname{Ext}_{{\mathscr{O}}_\mathbf{P}}^{r-n}(\mathbf{P};{\mathscr{F}},\Omega_\mathbf{P}^r).
\]
By \protect\hyperlink{fga-1-lemma-1}{Lemma 1}, the far-right-hand side can be identified with
\[
  \operatorname{Hom}_{{\mathscr{O}}_\mathbf{P}}(\mathbf{P};{\mathscr{F}},\omega_X^n)
  = \operatorname{Hom}_{{\mathscr{O}}_X}(X;{\mathscr{F}},\Omega_X^n)
  = \operatorname{Ext}_{{\mathscr{O}}_X}^0(X;{\mathscr{F}},\Omega_X^n)
\]
whence we have an isomorphism

\leavevmode\vadjust pre{\hypertarget{fga-1-equation-5.6}{}}%
\begin{eqenv}
\[
  \operatorname{H}^n(X,{\mathscr{F}})'
  = \operatorname{Hom}_{{\mathscr{O}}_X}(X;{\mathscr{F}},\Omega_X^n)
  = \operatorname{Ext}_{{\mathscr{O}}_X}^0(X;{\mathscr{F}},\Omega_X^n).
\tag{5.6}
\]

\end{eqenv}

Taking \({\mathscr{F}}=\Omega_X^n\), we obtain an isomorphism

\leavevmode\vadjust pre{\hypertarget{fga-1-equation-5.7}{}}%
\begin{eqenv}
\[
  \eta\colon \operatorname{H}^n(X,\Omega_X^n) \xrightarrow{\sim} k.
\tag{5.7}
\]

\end{eqenv}

\oldpage{149-17}We can prove that the isomorphism in \protect\hyperlink{fga-1-equation-5.6}{(5.6)} is exactly \protect\hyperlink{fga-1-equation-5.2bis}{(5.2 bis)} with \(p=n\) and \({\mathscr{L}}=k\), by \protect\hyperlink{fga-1-equation-5.7}{(5.7)}.
Subsequently, \protect\hyperlink{fga-1-equation-5.2bis}{(5.2 bis)} is an isomorphism for \(p=n\).
To prove that it is an isomorphism for all \(p\), it again suffices to prove that, for \(p<n\), the two sides of \protect\hyperlink{fga-1-equation-5.2bis}{(5.2 bis)} are coeffaceable functors in \({\mathscr{F}}\), and, a fortiori (taking \protect\hyperlink{fga-1-lemma-3}{Lemma 3} into account), that the two sides are zero when we take \({\mathscr{F}}={\mathscr{O}}_X(-m)\) with large enough \(m\).
But, for the left-hand side, this is true by \protect\hyperlink{fga-1-lemma-4}{Lemma 4}, and for the right-hand side we can write, using the duality theorem for \(\mathbf{P}\),
\[
  \operatorname{H}^p(X,{\mathscr{O}}_X(-m))'
  = \operatorname{Ext}_{{\mathscr{O}}_\mathbf{P}}^{r-p}(\mathbf{P};{\mathscr{O}}_X(-m),\Omega_\mathbf{P}^r).
\]
The right-hand side is zero for \(p<n\) and large enough \(m\), as follows from \protect\hyperlink{fga-1-lemma-5}{Lemma 5} (where in fact \(X=\mathbf{P}\)) and from the fact that \({\mathscr{O}}_X\) is of cohomological dimension \(\leqslant r-n\) when thought of as a coherent algebraic sheaf on \(\mathbf{P}\) (since \(X\) is non-singular), whence
\[
  \mathscr{E}\kern -.5pt xt_{{\mathscr{O}}_\mathbf{P}}^{r-p}({\mathscr{O}}_X,\Omega_\mathbf{P}^r) = 0
  \quad\text{for }p<n.
\]

\hypertarget{fga-1-section-6}{%
\subsection{The duality theorem for singular varieties}\label{fga-1-section-6}}

Let \(X\) be a closed subset of dimension \(n\) of the projective space \(\mathbf{P}\) of dimension \(r\).
Equation \protect\hyperlink{fga-1-equation-5.6}{(5.6)} can then be written as

\leavevmode\vadjust pre{\hypertarget{fga-1-equation-6.1}{}}%
\begin{eqenv}
\[
  \operatorname{H}^n(X,{\mathscr{F}})'
  \simeq \operatorname{Hom}_{{\mathscr{O}}_X}(X;{\mathscr{F}},\omega_X^n)
  = \operatorname{Ext}_{{\mathscr{O}}_X}^0(X;{\mathscr{F}},\omega_X^n)
\tag{6.1}
\]

\end{eqenv}

where we set\footnote{\emph{{[}Trans.{]}} This equation is labelled (6.2) in the original, but this seems to be a typo, since a later equation shares the same number, and any references to (6.2) seem to indeed point to the later equation instead of this one.}
\[
  \omega_X^n = E^0({\mathscr{O}}_X) = \mathscr{E}\kern -.5pt xt_{{\mathscr{O}}_\mathbf{P}}^{r-n}({\mathscr{O}}_X,\Omega_\mathbf{P}^r).
\]
As mentioned in \protect\hyperlink{fga-1-proposition-5}{Proposition 5}, the sheaf thus defined does not depend on the chosen immersion of \(X\) into the non-singular variety \(\mathbf{P}\).
Taking \({\mathscr{F}}=\omega_X^n\) in \protect\hyperlink{fga-1-equation-6.1}{(6.1)}, we find that
\[
  \operatorname{H}^n(X,\omega_X^n)' \simeq \operatorname{Hom}_{{\mathscr{O}}_X}(X;\omega_X^n,\omega_X^n)
\tag{6.2}
\]
whence the existence of a distinguished element in \(\operatorname{H}^n(X,\omega_X^n)\), corresponding to the identity morphism from \(\omega_X^n\) to itself:

\leavevmode\vadjust pre{\hypertarget{fga-1-equation-6.3}{}}%
\begin{eqenv}
\[
  \eta\colon \operatorname{H}^n(X,\Omega_X^n) \to k.
\tag{6.3}
\]

\end{eqenv}

\oldpage{149-18}Then consider the pairings defined by the composition of the \(\operatorname{Ext}\):
\[
  \operatorname{H}^p(X,{\mathscr{F}}) \times \operatorname{Ext}_{{\mathscr{O}}_X}^{n-p}(X;{\mathscr{F}},\omega_X^n)
  \to \operatorname{H}^n(X,\omega_X^n)
\tag{6.4}
\]
and compose them with the homomorphism \(\eta\) in \protect\hyperlink{fga-1-equation-6.3}{(6.3)}; we thus obtain functorial homomorphisms

\leavevmode\vadjust pre{\hypertarget{fga-1-equation-6.5}{}}%
\begin{eqenv}
\[
  \operatorname{Ext}_{{\mathscr{O}}_X}^{n-p}(X;{\mathscr{F}},\omega_X^n) \to \operatorname{H}^p(X,{\mathscr{F}})'
\tag{6.5}
\]

\end{eqenv}

which are compatible with the boundary maps (generalising \protect\hyperlink{fga-1-equation-5.2}{(5.2)}).
We can prove that, for \(p=n\), we thus obtain the isomorphism in \protect\hyperlink{fga-1-equation-6.1}{(6.1)}.
With this, we have:

\hypertarget{fga-1-theorem-3bis}{}
\begin{itenv}{Theorem 3 bis}

For any given integer \(k\geqslant 0\), the four following conditions on \(X\) are equivalent:

\begin{enumerate}
\def\labelenumi{\roman{enumi}.}
\item
  The functorial homomorphism in \protect\hyperlink{fga-1-equation-6.5}{(6.5)} is an isomorphism for \(n-k\leqslant p\leqslant n\).
\item
  \(\operatorname{H}^p(X,{\mathscr{O}}_X(-m)) = 0\) for \(n-k\leqslant p<n\) and large enough \(m\).
\item
  The functor \(\operatorname{H}^p(X,{\mathscr{F}})\) on the category of coherent algebraic sheaves on \(X\) is coeffaceable for \(n-k\leqslant p<n\).
\item
  \(E^i({\mathscr{O}}_X) = \mathscr{E}\kern -.5pt xt_{{\mathscr{O}}_\mathbf{P}}^{r-n+i}({\mathscr{O}}_X,\omega_\mathbf{P}^r) = 0\) for \(0<i\leqslant k\).
\end{enumerate}

\end{itenv}

\begin{proof}
i.\(\implies\)ii. by \protect\hyperlink{fga-1-lemma-4}{Lemma 4};
ii.\(\implies\)iii. by \protect\hyperlink{fga-1-lemma-3}{Lemma 3};
iii.\(\implies\)i. by a well-known standard argument, taking into account the fact that the two sides of \protect\hyperlink{fga-1-equation-6.5}{(6.5)} are then coeffaceable functors for \(n-k\leqslant p< n\) (the first being so by \protect\hyperlink{fga-1-lemma-4}{Lemma 4});
finally, ii.\(\iff\)iv. follows from \protect\hyperlink{fga-1-proposition-6-corollary}{the corollary to the following proposition}.
\end{proof}

\leavevmode\vadjust pre{\hypertarget{fga-1-proposition-6}{}}%
\begin{itenv}{Proposition 6}
Let \({\mathscr{F}}\) be a coherent algebraic sheaf on \(X\), and let \(i\) be an integer.
Then, for large enough \(m\), we have an isomorphism

\leavevmode\vadjust pre{\hypertarget{fga-1-equation-6.6}{}}%
\begin{eqenv}
\[
  \operatorname{H}^i(X,{\mathscr{F}}(-m))' \simeq \operatorname{H}^0(X,E^{n-i}({\mathscr{F}})(m))
\tag{6.6}
\]

\end{eqenv}

where we set
\[
  E^j({\mathscr{F}}) = \mathscr{E}\kern -.5pt xt_{{\mathscr{O}}_\mathbf{P}}^{r-n+j}({\mathscr{F}},\Omega_\mathbf{P}^r)
\tag{6.7}
\]
(compare with \protect\hyperlink{fga-1-proposition-5}{Proposition 5} in \protect\hyperlink{fga-1-section-3}{§3}).

\end{itenv}

\begin{proof}
Indeed, by the duality theorem for \(\mathbf{P}\), the left-hand side of \protect\hyperlink{fga-1-equation-6.6}{(6.6)} is isomorphic to \(\operatorname{Ext}_{{\mathscr{O}}_\mathbf{P}}^{r-i}(\mathbf{P};{\mathscr{F}}(-m),\Omega_\mathbf{P}^r)\), and so \protect\hyperlink{fga-1-equation-6.6}{(6.6)} follows from \protect\hyperlink{fga-1-lemma-5}{Lemma 5}.
\end{proof}

\leavevmode\vadjust pre{\hypertarget{fga-1-proposition-6-corollary}{}}%
\begin{itenv}{Corollary}
\oldpage{149-19}For \(\operatorname{H}^i(X,{\mathscr{F}}(-m))\) to be zero for large enough \(m\), it is necessary and sufficient that \(E^{n-i}({\mathscr{F}})\) be zero.

\end{itenv}

Recall that the \(E^j({\mathscr{F}})\) do not depend on the projective immersion in question.
The condition of the corollary is purely local, and so, if it is satisfied for \({\mathscr{F}}\), then it is also satisfied for every sheaf that is locally isomorphic to some \({\mathscr{F}}^n\).
In particular, if this condition is satisfied for \({\mathscr{O}}_X\), then it is satisfied for every locally-free coherent algebraic sheaf.
This is the case, for example, for all \(i<n\) if \(X\) is non-singular;
and for \(i=0\) if no component of \(X\) consists of a single point;
and for \(i=0,1\) if \(S\) is normal and all its components are of dimension \(>1\) (see {[}\protect\hyperlink{ref-Ser1955}{18}{]}).
For it to be satisfied for \(i<k\), it is necessary and sufficient, by definition, that the local rings \({\mathscr{O}}_x\) (\(x\in X\)) be of ``homological codimension \(\geqslant k\)'' (see {[}\protect\hyperlink{ref-Ser1956a}{20}{]} for details on this notion).
If \(k=n\), then this implies, by \protect\hyperlink{fga-1-theorem-3bis}{Theorem 3 bis}, that the duality theorem is true for \(X\), i.e.~that \protect\hyperlink{fga-1-equation-6.5}{(6.5)} is an isomorphism for all \(p\) and for all \({\mathscr{F}}\).
We can give many equivalent conditions on the local rings \({\mathscr{O}}_x\) for this to be the case (Nagata);
for example, those that satisfy the Cohen-Macaulay equidimensionality theorem.
It is also the case, for example, if \(X\) is locally a ``complete intersection'' in a non-singular ambient variety.

\hypertarget{fga-1-section-7}{%
\subsection{Poincaré duality}\label{fga-1-section-7}}

Let \(X\) be a non-singular projective variety of dimension \(n\).
Then \(\operatorname{H}^\bullet(X)=\operatorname{H}^\bullet(X,\Omega_X^\bullet)\) is a finite-dimensional bigraded anticommutative algebra, that we grade by the total degree, so that \(\operatorname{H}^{p,q}(X)=\operatorname{H}^p(X,\Omega_X^q)\) is of degree \(p+q\);
the degrees of \(\operatorname{H}^\bullet(X)\) are concentrated between \(0\) and \(2n\).
By \protect\hyperlink{fga-1-theorem-2}{Theorem 2} and \protect\hyperlink{fga-1-theorem-3-corollary}{the corollary to Theorem 3}, \(\operatorname{H}^\bullet(X)\) is a ``Poincaré algebra'' of dimension \(2n\), i.e.~\(\operatorname{H}^{2n}(X)\) is endowed with an isomorphism to the base field \(k\), and the product \(\operatorname{H}^m(X)\times\operatorname{H}^{2n-m}(X)\to\operatorname{H}^{2n}(X)=k\) is a duality between \(\operatorname{H}^m(X)\) and \(\operatorname{H}^{2n-m}(X)\).
Furthermore, if \(Y\) is another non-singular projective variety, then the Künneth formula for coherent algebraic sheaves gives
\[
  \operatorname{H}^\bullet(X\times Y) = \operatorname{H}^\bullet(X)\otimes\operatorname{H}^\bullet(Y)
\tag{7.1}
\]
which is an isomorphism that is compatible with the Poincaré algebra structures.
Furthermore, \(\operatorname{H}^\bullet(X)\) is, as a commutative algebra, a covariant functor in \(X\), since a morphism \(f\colon Y\to X\) defines, in an evident way, a homomorphism of graded algebras
\[
  f^*\colon \operatorname{H}^\bullet(X)\to\operatorname{H}^\bullet(Y).
\tag{7.2}
\]
\oldpage{149-20}Since we are working with Poincaré algebras, we obtain, by transposition, a homomorphism of vector spaces

\leavevmode\vadjust pre{\hypertarget{fga-1-equation-7.3}{}}%
\begin{eqenv}
\[
  f_*\colon \operatorname{H}^\bullet(Y)\to\operatorname{H}^\bullet(X).
\tag{7.3}
\]

\end{eqenv}

We have seen in \protect\hyperlink{fga-1-section-4}{§4} that the effect of \(f^*\) on cohomology classes that correspond to non-singular cycles can be interpreted geometrically by taking the cohomology classes that correspond to their inverse images.
It is important, in our current study, to show that \protect\hyperlink{fga-1-equation-7.3}{(7.3)} corresponds similarly to the ``direct image'' operation on cycles.
This follows (under suitable non-singularity conditions, at least) from the following particular case:

\leavevmode\vadjust pre{\hypertarget{fga-1-theorem-4}{}}%
\begin{itenv}{Theorem 4}
If \(f\) is the identity map from a non-singular subvariety \(Y^m\) of \(X^n\) to \(X^n\), then, denoting by \(1_Y\) the unit element of \(\operatorname{H}(Y)\), we have
\[
  f_*(1_Y) = P_X(Y)
\tag{7.4}
\]
where the right-hand side is the cohomology class in \(X\) associated to \(Y\).

\end{itenv}

This formula is equivalent to
\[
  \langle \xi^{m,m}, P_X(Y) \rangle \varepsilon_Y
  = f_*(\xi^{m,m})
  \quad\text{where }\xi^{m,m}\in\operatorname{H}^m(X,\Omega_X^m)
\tag{7.4 bis}
\]
where \(\varepsilon_Y\) is the fundamental element of \(\operatorname{H}^m(Y,\Omega_Y^m)\), and this, in the case of non-singular projective varieties, gives a new definition of the cohomology class associated to \(Y\).

\begin{proof}
To prove \protect\hyperlink{fga-1-theorem-4}{Theorem 4}, we consider, by \protect\hyperlink{fga-1-theorem-3}{Theorem 3}, the transpose of the homomorphism
\[
  \operatorname{H}^m(X,\Omega_X^m) \to \operatorname{H}^m(Y,\Omega_Y^m) = \operatorname{H}^m(X,\Omega_Y^m)
\]
as the homomorphism
\[
  \begin{CD}
    @.
    \operatorname{Ext}_{{\mathscr{O}}_X}^{n-m}(X;\Omega_Y^m,\Omega_X^n)
    @>\sim>>
    \operatorname{Hom}_{{\mathscr{O}}_X}(X;\Omega_Y^m,\Omega_Y^m)
  \\@. @VVV @.
  \\\operatorname{H}^{n-m}(X,\Omega_X^{n-m})
    @>\sim>>
    \operatorname{Ext}_{{\mathscr{O}}_X}^{n-m}(X;\Omega_X^m,\Omega_X^n)
  \end{CD}
\tag{7.5}
\]
We can verify that the element \(1_Y\) of the dual of \(\operatorname{H}^m(Y,\Omega_Y^m)\) is identified with the element of the right-hand side corresponding to the identity endomorphism of \(\Omega_Y^m\), and also that the image of this element in \(\operatorname{H}^{n-m}(X,\Omega_X^{n-m})\) is indeed \(P_X(Y)\).
\end{proof}

\oldpage{149-21}These relations (which could have been given in \protect\hyperlink{fga-1-section-4}{§4}) can be stated, and are indeed true, for arbitrary non-singular varieties, with the second, for example, following from the commutativity of the following diagram of canonical endomorphisms:
\[
  \footnotesize
  \begin{CD}
    \operatorname{Ext}_{{\mathscr{O}}_X}^{n-m}(X;\Omega_X^m,\Omega_X^n)
    @<<<
    \operatorname{Ext}_{{\mathscr{O}}_X}^{n-m}(X;\Omega_Y^m,\Omega_X^n)
    @>\sim>>
    \operatorname{Hom}_{{\mathscr{O}}_X}(X;\Omega_Y^m,\Omega_Y^m)
  \\@VVV @. @VVV
  \\\operatorname{Ext}_{{\mathscr{O}}_X}^{n-m}(X;{\mathscr{O}}_X,\Omega_X^{n-m})
    @<<<
    \operatorname{Ext}_{{\mathscr{O}}_X}^{n-m}(X;{\mathscr{O}}_Y,\Omega_X^{n-m})
    @>\sim>>
    \operatorname{Hom}_{{\mathscr{O}}_X}(X;\Omega_X^m,\Omega_Y^m)
  \end{CD}
\tag{7.6}
\]

We thus obtain an exact equivalent of the formalism of Poincaré duality for compact oriented varieties.
In particular, \protect\hyperlink{fga-1-theorem-4}{Theorem 4} allows us to determine the cohomology class associated to the diagonal of \(X\times X\).
By a well-known argument, we thus deduce, for example, a \emph{Lefschetz formula}:

\leavevmode\vadjust pre{\hypertarget{fga-1-theorem-5}{}}%
\begin{itenv}{Theorem 5}
Let \(f\) be an endomorphism of a non-singular projective variety \(X\) such that the fixed points of \(f\) are of multiplicity \(1\).
Then the number of these fixed points is congruent, modulo the characteristic of \(k\), to the alternating sum of the traces of the endomorphisms of the \(\operatorname{H}^i(X)\) defined by \(f\).

\end{itenv}

The restriction on \(f\) that we have to make is related to the difficulties mentioned in \protect\hyperlink{fga-1-section-4-remark}{the remark in §4}.
We note, however, that the Lefschetz formula still holds true if \(f\) is ``homotopic'' to an endomorphism whose fixed points are all of multiplicity \(1\).

\hypertarget{fga-1-section-8}{%
\subsection{Generalisation of the duality theorem}\label{fga-1-section-8}}

Let \(X\) be a non-singular algebraic variety such that every coherent algebraic sheaf \({\mathscr{F}}\) on \(X\) is isomorphic to a locally-free coherent algebraic sheaf (which is the case if \(X\) is locally closed in some projective space).
Then every coherent algebraic sheaf \({\mathscr{F}}\) on \(X\) admits a finite resolution \({\mathscr{L}}\) by locally-free sheaves, and, for any two such resolutions, we can always find a third, along with homomorphisms, from it to the first two, that are compatible with the augmentations.
Similarly, if \({\mathscr{L}}\) is a finite locally-free resolution of \({\mathscr{F}}\), and if we have a homomorphism \({\mathscr{F}}'\to{\mathscr{F}}\), then there exists a finite locally-free resolution \({\mathscr{L}}'\) of \({\mathscr{F}}'\) along with a homomorphism \({\mathscr{L}}'\to{\mathscr{L}}\) that is compatible with \({\mathscr{F}}'\to{\mathscr{F}}\), that we can even assume to be surjective if \({\mathscr{F}}'\to{\mathscr{F}}\) is surjective.
This allows us to define, given integers \(r,s\geqslant 0\), two cohomological multifunctors, with arguments \({\mathscr{A}}_1,\ldots,{\mathscr{A}}_r;{\mathscr{B}}_1,\ldots,{\mathscr{B}}_s\) in the category of coherent algebraic sheaves on \(X\);
one takes values in the category of coherent algebraic sheaves on \(X\), and the other in the category of modules over \(\operatorname{H}^0(X,{\mathscr{O}}_X)\).
\oldpage{149-22}We define them by the formulas

\leavevmode\vadjust pre{\hypertarget{fga-1-equation-8.1}{}}%
\begin{eqenv}
\[
  \begin{aligned}
    &\underline{T}_r^{s\bullet}({\mathscr{A}}_1,\ldots,{\mathscr{A}}_r;{\mathscr{B}}_1,\ldots,{\mathscr{B}}_s)
  \\&= \operatorname{H}^\bullet(\mathscr{H}\kern -.5pt om_{{\mathscr{O}}_X}(\underline{L}({\mathscr{A}}_1)\otimes\ldots\otimes\underline{L}({\mathscr{A}}_r), \underline{L}({\mathscr{B}}_1)\otimes\ldots\otimes\underline{L}({\mathscr{B}}_s))),
  \\&T_r^{s\bullet}({\mathscr{A}}_1,\ldots,{\mathscr{A}}_r;{\mathscr{B}}_1,\ldots,{\mathscr{B}}_s)
  \\&= \underline{\operatorname{R}}^\bullet\Gamma(\mathscr{H}\kern -.5pt om_{{\mathscr{O}}_X}(\underline{L}({\mathscr{A}}_1)\otimes\ldots\otimes\underline{L}({\mathscr{A}}_r), \underline{L}({\mathscr{B}}_1)\otimes\ldots\otimes\underline{L}({\mathscr{B}}_s)))
  \end{aligned}
\tag{8.1}
\]

\end{eqenv}

where \(\underline{L}({\mathscr{F}})\) denotes a finite locally-free resolution of the coherent algebraic sheaf \({\mathscr{F}}\), and \(\underline{\operatorname{R}}^\bullet\Gamma({\mathscr{K}})\) denotes the hypercohomology of the space \(X\) with respect to the complex of sheaves \({\mathscr{K}}\).
If \(r\) (resp. \(s\)) is zero, then we replace the tensor product of the \(\underline{L}({\mathscr{A}}_i)\) (resp. of the \(\underline{L}({\mathscr{B}}_j)\)) by \({\mathscr{O}}_X\).
In particular, \(\underline{T}_0^0\) and \(T_0^0\) are graded functors with no arguments:
\(\underline{T}_0^0\) is concentrated in degree \(0\), where it is the sheaf \({\mathscr{O}}_X\);
and \(T_0^0\) is equal to \(\operatorname{H}^\bullet(X,{\mathscr{O}}_X)\).
The fact that the right-hand sides of \protect\hyperlink{fga-1-equation-8.1}{(8.1)} do not depend on the chosen resolutions is evident for \(\underline{T}\) (since the question is then local), and for \(T\) it follows from preceding general remarks, taking into account the spectral sequence for the hypercohomology of the complex of sheaves \({\mathscr{K}}=\mathscr{H}\kern -.5pt om_{{\mathscr{O}}_X}(\underline{L}({\mathscr{A}}_1)\otimes\ldots,\underline{L}({\mathscr{B}}_1)\otimes\ldots\) that abuts to the hypercohomology of \(X\) with respect to \({\mathscr{K}}\), and whose initial page is \(\operatorname{H}^p(X,\operatorname{H}^q({\mathscr{K}}))\), i.e.

\leavevmode\vadjust pre{\hypertarget{fga-1-equation-8.2}{}}%
\begin{eqenv}
\[
  E_2^{p,q} = \operatorname{H}^p(X,(\underline{T}_r^s)^{(q)}({\mathscr{A}}_1,\ldots,{\mathscr{A}}_r;{\mathscr{B}}_1,\ldots,{\mathscr{B}}_s)).
\tag{8.2}
\]

\end{eqenv}

We then see that this spectral sequence itself does not depend on the chosen resolutions, and its abutment is the left-hand side of \protect\hyperlink{fga-1-equation-8.1}{(8.1)}.
We can easily define the coboundary maps relative to miscellaneous arguments \({\mathscr{A}}_i,{\mathscr{B}}_j\) by noting that every exact sequence \(0\to{\mathscr{F}}'\to{\mathscr{F}}\to{\mathscr{F}}''\to0\) can be resolved by an exact sequence of finite locally-free complexes.

We define, on the system of functors \(\underline{T}_r^{s\bullet}\) (resp. \(T_r^{s\bullet}\)), operations that are analogous to those of tensor calculus, and whose definitions are immediate from the defining formulas in \protect\hyperlink{fga-1-equation-8.1}{(8.1)}.
We thus have a \emph{composition} (generalising that which was described in \protect\hyperlink{fga-1-section-2}{§2}):

\leavevmode\vadjust pre{\hypertarget{fga-1-equation-8.3}{}}%
\begin{eqenv}
\[
  \begin{gathered}
    T_r^{s\bullet}({\mathscr{A}}_1,\ldots,{\mathscr{A}}_r;{\mathscr{B}}_1,\ldots,{\mathscr{B}}_s)
    \times T_{r'}^{s'\bullet}({\mathscr{A}}'_1,\ldots,{\mathscr{A}}'_{r'};{\mathscr{B}}'_1,\ldots,{\mathscr{B}}'_{s'})
  \\\to T_{r+r'}^{(s+s')\bullet}({\mathscr{A}}_1,\ldots,{\mathscr{A}}_r,{\mathscr{A}}'_1,\ldots,{\mathscr{A}}'_{r'};{\mathscr{B}}_1,\ldots,{\mathscr{B}}_s,{\mathscr{B}}'_1,\ldots,{\mathscr{B}}'_{s'})
  \end{gathered}
\tag{8.3}
\]

\end{eqenv}

that satisfies the evident properties of associativity, compatibility with the functorial homomorphisms and the coboundary homomorphisms, and spectral sequences.
Similarly, we have symmetry operations, whose explicit descriptions we leave to the reader.
\oldpage{149-23}We further have a \emph{contraction} operation every time one of the arguments \({\mathscr{A}}_i\) is equal to one of the arguments \({\mathscr{B}}_j\):

\leavevmode\vadjust pre{\hypertarget{fga-1-equation-8.4}{}}%
\begin{eqenv}
\[
  \begin{gathered}
    T_r^{s\bullet}({\mathscr{A}}_1,\ldots,{\mathscr{A}}_{i-1},{\mathscr{C}},{\mathscr{A}}_{i+1},\ldots,{\mathscr{A}}_r;{\mathscr{B}}_1,\ldots,{\mathscr{B}}_{j-1},{\mathscr{C}},{\mathscr{B}}_{j+1},\ldots,{\mathscr{B}}_s)
  \\\to -T_{r-1}^{(s-1)\bullet}({\mathscr{A}}_1,\ldots,\widehat{{\mathscr{A}}_i},\ldots,{\mathscr{A}}_r;{\mathscr{B}}_1,\ldots,\widehat{{\mathscr{B}}_j},\ldots,{\mathscr{B}}_s).
  \end{gathered}
\tag{8.4}
\]

\end{eqenv}

Furthermore, if some argument \({\mathscr{A}}_i\) is a locally-free sheaf, then we can suppress it by replacing one of the \({\mathscr{A}}_j\) (for \(j\neq i\)) by \({\mathscr{A}}_j\otimes{\mathscr{A}}_i\), or one of the \({\mathscr{B}}_k\) by \({\mathscr{B}}_k\otimes{\mathscr{A}}'_i\) (where \({\mathscr{A}}'_i=\mathscr{H}\kern -.5pt om_{{\mathscr{O}}_X}({\mathscr{A}}_i,{\mathscr{O}}_X\));
we have an analogous rule for the case where one of the arguments \({\mathscr{B}}_j\) is locally free.
In particular, we can always suppress any argument that is equal to \({\mathscr{O}}_X\).
If all the arguments are locally free, except for at most one of the arguments \({\mathscr{B}}_i\), then the rule that we have just stated gives a functorial isomorphism
\[
  T_r^{s\bullet}({\mathscr{A}}_1,\ldots,{\mathscr{A}}_r;{\mathscr{B}}_1,\ldots,{\mathscr{B}}_s)
  = \operatorname{H}^\bullet(X,{\mathscr{A}}'_1\otimes\ldots\otimes{\mathscr{A}}'_{r}\otimes{\mathscr{B}}_1\otimes{\mathscr{B}}_s)
\tag{8.5}
\]
(since we can restrict to the case where \(r=0\) and \(s=1\), and there it is immediate;
we can also directly use the spectral sequence whose initial term is \protect\hyperlink{fga-1-equation-8.2}{(8.2)}).
The corresponding operations of all the above can also be defined for the \(\underline{T}_r^s\).
The relations between the various operations thus introduced are the same as for the analogous relations in tensor calculus.

Now let \(n\) be the dimension of \(X\).
By successively applying a tensor composition \protect\hyperlink{fga-1-equation-8.3}{(8.3)} and contractions \protect\hyperlink{fga-1-equation-8.4}{(8.4)} on repeated arguments, we obtain a pairing

\leavevmode\vadjust pre{\hypertarget{fga-1-equation-8.6}{}}%
\begin{eqenv}
\[
  \begin{gathered}
    (T_r^s)^p({\mathscr{A}}_1,\ldots;{\mathscr{B}}_1,\ldots)
    \times (T_r^s)^{n-p}({\mathscr{B}}_1,\ldots;{\mathscr{A}}_1,\ldots,{\mathscr{A}}_r\otimes\Omega_X^n)
  \\\longrightarrow\operatorname{H}^n(X,\Omega_X^n).
  \end{gathered}
\tag{8.6}
\]

\end{eqenv}

\leavevmode\vadjust pre{\hypertarget{fga-1-theorem-6}{}}%
\begin{itenv}{Theorem 6}
If \(X\) is a non-singular projective variety, then the pairings in \protect\hyperlink{fga-1-equation-8.6}{(8.6)} are dualities.

\end{itenv}

\begin{proof}
This follows in a purely formal way from the \protect\hyperlink{fga-1-theorem-3-corollary}{corollary of Theorem 3}.
In fact, it easily follows from this corollary that, if \({\mathscr{K}}\) is a complex of \emph{locally-free} coherent algebraic sheaves, then the hypercohomology of \(X\) with respect to \({\mathscr{K}}\) is in duality with the hypercohomology of \(X\) with respect to \({\mathscr{K}}'\otimes\Omega_X^n\) via the natural pairings
\[
  \underline{\operatorname{R}}^p\Gamma({\mathscr{K}})
  \times \underline{\operatorname{R}}^{n-p}\Gamma({\mathscr{K}}'\otimes\Omega_X^n)
  \to \underline{\operatorname{R}}^n\Gamma(\Omega_X^n)
  = \operatorname{H}^n(X,\Omega_X^n).
\tag{8.7}
\]
We can see this by using the spectral sequence with initial page \(\operatorname{H}^p(\operatorname{H}^q(X,{\mathscr{K}}))\) and the analogous spectral sequence for \({\mathscr{K}}'\otimes\Omega_X^n\).
From the above result, \protect\hyperlink{fga-1-theorem-6}{Theorem 6} can be deduced from the definition \protect\hyperlink{fga-1-equation-8.1}{(8.1)}.
\end{proof}

\begin{rmenv}{Remarks}

\oldpage{149-24}---

\begin{enumerate}
\def\labelenumi{\arabic{enumi}.}
\item
  For the definitions preceding \protect\hyperlink{fga-1-theorem-6}{Theorem 6}, it was not necessary for \(X\) to be non-singular, since it was not necessary to work with only \emph{finite} resolutions.
  But, if \(X\) is singular, then we can no long be sure, a priori, that the \((\underline{T}_r^s)^p({\mathscr{A}}_1,\ldots;{\mathscr{B}}_1,\ldots)\) are \emph{coherent} sheaves, since, in the complex of sheaves
  \[
   \mathscr{H}\kern -.5pt om_{{\mathscr{O}}_X}(\underline{L}({\mathscr{A}}_1)\otimes\ldots,\underline{L}({\mathscr{B}}_1)\otimes\ldots)
    \]
  there will be an infinite number of components of any given total degree.
\item
  We can easily verify that, in the formulas in \protect\hyperlink{fga-1-equation-8.1}{(8.1)}, we can replace \emph{one} of the \(\underline{L}({\mathscr{B}}_i)\) with \({\mathscr{B}}_i\).
  Taking \protect\hyperlink{fga-1-proposition-3}{Proposition 3} into account, this shows that we have
  \[
   \begin{aligned}
     \underline{T}_1^{1\bullet}({\mathscr{A}};{\mathscr{B}})
     &= \mathscr{E}\kern -.5pt xt_{{\mathscr{O}}_X}^\bullet({\mathscr{A}},{\mathscr{B}})
   \\T_1^{1\bullet}({\mathscr{A}};{\mathscr{B}})
     &= \operatorname{Ext}_{{\mathscr{O}}_X}^\bullet(X;{\mathscr{A}},{\mathscr{B}}).
   \end{aligned}
    \tag{8.8}
    \]
  In particular, taking \(r=s=1\) and \({\mathscr{A}}_1={\mathscr{O}}_X\) in \protect\hyperlink{fga-1-equation-8.6}{(8.6)}, we recover \protect\hyperlink{fga-1-theorem-3}{Theorem 3}.
  Equation (8.8) also implies that \(T_0^{1\bullet}({\mathscr{B}})=\operatorname{H}^\bullet(X,{\mathscr{B}})\), and that \(T_1^{0\bullet}({\mathscr{A}})=\operatorname{Ext}_{{\mathscr{O}}_X}^\bullet(X;{\mathscr{A}},{\mathscr{O}}_X)\).
\item
  We see, in \protect\hyperlink{fga-1-equation-8.1}{(8.1)}, that the functors \(\underline{T}_r^{s\bullet}\) and \(T_r^{s\bullet}\) have, in general, components in positive \emph{and} negative degrees.
  Using the above remark, we see that, if the dimension of \(X\) is \(n\), then the non-zero components of \(\underline{T}_r^{s\bullet}\) are concentrated between degrees \(-(s-1)n\) and \(rn\) if \(s>0\), and between degrees \(0\) and \(rn\) if \(s=0\); the non-zero components of \(T_r^{s\bullet}\) are concentrated between degrees \(-(s-1)n\) and \((r+1)n\) if \(s>0\), and between degrees \(0\) and \((r+1)n\) if \(s=0\) (and, unless I am mistaken, if \(r>0\), between degrees \(-(s-1)n\) and \(rn\), resp. \(0\) and \(rn\)).
\end{enumerate}

\end{rmenv}

\hypertarget{fga-2}{%
\section{Formal geometry and algebraic geometry}\label{fga-2}}

\providecommand{\scr}[1]{{\mathscr{#1}}}
\renewcommand{\cal}[1]{{\mathcal{#1}}}
\renewcommand{\frak}[1]{{\mathfrak{#1}}}
\renewcommand{\geq}{\geqslant}
\renewcommand{\leq}{\leqslant}

\providecommand{\PP}{\mathbb{P}}
\providecommand{\ZZ}{\mathbb{Z}}
\providecommand{\red}{\mathrm{red}}

\providecommand{\Spec}{\operatorname{Spec}}
\providecommand{\supp}{\operatorname{supp}}
\providecommand{\RR}{\operatorname{R}}
\providecommand{\Hom}{\operatorname{Hom}}
\providecommand{\shHom}{\mathscr{H}\kern -.5pt om}
\providecommand{\Aut}{\operatorname{Aut}}
\providecommand{\shAut}{\mathscr{A}\kern -.5pt ut}
\providecommand{\gr}{\operatorname{gr}}

{[}FGA 2{]}
Grothendieck, A.
``Géométrie formelle et géométrie algébrique''.
\emph{Séminaire Bourbaki} \textbf{11} (1958--59), Talk no. 182.

\oldpage{C-03}The substance of \protect\hyperlink{fga-2-section-1}{§1} to \protect\hyperlink{fga-2-section-5}{§5} is contained in the published part of {[}\protect\hyperlink{ref-GD1960}{10}, III{]}; that of \protect\hyperlink{fga-2-section-6}{§6} and \protect\hyperlink{fga-2-section-7}{§7} is contained in {[}\protect\hyperlink{ref-Gro1960b}{9}, III{]}.
For the study of the fundamental group, see {[}\protect\hyperlink{ref-Gro1960b}{9}, V, IX, X, and XI{]}, as well as {[}\protect\hyperlink{ref-Gro1960b}{9}, X, XII, and XIII{]} for the Lefschetz-type theorems and numerous open questions.
Only the theory of moderately ramified coverings (cf.~\protect\hyperlink{fga-2-theorem-14}{Theorem 14}) has not yet been the subject of a dedicated talk.
The \protect\hyperlink{fga-2-theorem-14-corollary-1}{corollary to Theorem 14}, which completely determines Galois coverings of order coprime to the characteristic of an algebraic curve over an algebraically closed field, has been used in an essential manner on three separate occasions:

\begin{enumerate}
\def\labelenumi{\arabic{enumi}.}
\item
  in the proof by Igusa of the Picard inequality for non-singular projective surfaces in arbitrary characteristic;
\item
  in the study (developed independently by Ogg and Šafarevič) of the group of homogeneous principal bundles over an abelian variety defined over a function field in one variable, in arbitrary characteristic; and
\item
  in the recent proof, by Artin, of certain key theorems concerning the ``Weil cohomology'' of algebraic varieties.
\end{enumerate}

\hypertarget{fga-2-section-1}{%
\subsection{Schemes}\label{fga-2-section-1}}

\oldpage{182-01}We know that an affine algebraic space defined over a field \(k\) is essentially determined by its affine algebra \(A\) (the ring of regular functions defined over \(k\)), and the morphisms \(X\to Y\) of algebraic spaces correspond bijectively to homomorphisms \(A(Y)\to A(X)\) of \(k\)-algebras.
The affine algebra corresponding to an algebraic space is a \(k\)-algebra of finite type, and, from the ``classical'' point of view, it has no nilpotent elements;
conversely, every such algebra is obtained as the affine algebra of an algebraic space defined over \(k\).
There is thus a known dictionary that allows us to interpret situations concerning affine algebraic spaces in terms of commutative algebra.
We have long since noted that we thus obtain more general statements, since it was not generally necessary to suppose that the rings in play were of the form just described, with the Noetherian hypothesis being sufficient the most of the time.
In particular, whether or not a base field were given, it was not necessary to exclude the case where these rings contained nilpotent elements.
Up until now, geometers had refused to take into account this information, and were obstinate in restricting to the consideration of affine algebra without nilpotent elements, i.e.~algebraic spaces in whose structure sheaves there are no nilpotent elements (and even, most of the time, ``absolutely irreducible'' algebraic spaces).
The speaker thinks that this state of mind has been a serious obstacle to the development of truly natural methods in algebraic geometry.

Let \(A\) be a commutative ring.
It is well known that the set \(X=\operatorname{Spec}(A)\) of prime ideals of \(A\) is endowed with a natural topology: the ``\emph{Zariski topology}'', or the spectral topology.
Also, there is a sheaf of commutative rings \({\mathscr{O}}_X\) on \(X\), whose fibre at \({\mathfrak{p}}\in X\) is the localised ring \(A_{\mathfrak{p}}\), and whose ring of sections can be identified with \(A\).
Thus \(X\) becomes a \emph{ringed space}, and is called the \emph{prime spectrum} of \(A\).
A ring homomorphism \(f\colon A\to B\) defines a morphism \(f'\colon\operatorname{Spec}(B)\to\operatorname{Spec}(A)\) of ringed spaces, with the underlying map of sets being exactly \({\mathfrak{p}}\mapsto f^{-1}({\mathfrak{p}})\).
The homomorphisms \(\operatorname{Spec}(B)\to\operatorname{Spec}(A)\) of ringed spaces obtained in this manner are exactly those for which the homomorphisms \({\mathscr{O}}_x\to{\mathscr{O}}_y\) (where \(x=f'(y)\)) are local (i.e.~the inverse image of a maximal ideal is a maximal ideal).

\oldpage{182-02}We define an \emph{affine scheme} to be a ringed space that is isomorphic to some \(\operatorname{Spec}(A)\), and a \emph{prescheme} to be a locally-affine ringed space, i.e.~such that every point has an open neighbourhood that is an affine scheme for the induced structure.
We define, in an evident way, \emph{morphisms} of preschemes;
locally, they correspond to ring homomorphisms.

When we fix a prescheme \(S\), and we look at morphisms \(X\to S\) of preschemes, then \(S\) plays the role of a base field or base ring (or, even better, of a base space in a fibration).
We then say that \(X\) is an \emph{\(S\)-prescheme};
if \(S=\operatorname{Spec}(A)\), then this also implies that \({\mathscr{O}}_X\) is a sheaf of \emph{\(A\)-algebras}.
So every prescheme can be regarded in a unique way as a \(\mathbb{Z}\)-prescheme.
Of course, \(S\)-preschemes form a category, and we can further show that, in this category, the product of two objects \(X\) and \(Y\) always exists;
it is denoted by \(X\times_S Y\).
This notion of product allows us to define the \emph{change of base} of an \(S\)-prescheme, corresponding to a morphism \(S'\to S\), since \(X\times_S S'\) can be considered as an \(S'\)-prescheme.

We say that \(X\) is \emph{separated} over \(S\) if the diagonal of \(X\times_S X\) is closed.
We define a \emph{scheme} to be a prescheme that is separated over \(\mathbb{Z}\);
it is then separated over anything.
For simplicity, we will now only speak of schemes, which we will further suppose to be \emph{Noetherian}, i.e.~finite unions of affine opens that are spectra of Noetherian rings.
We say that \(X\) is \emph{of finite type} over \(S\) if, for every affine open subset \(U\) of \(S\), its inverse image in \(X\) is a finite union of affine opens whose rings are algebras of finite type over the ring of \(U\).
It is such \(S\)-schemes that lend themselves to a properly geometry study.
In particular, for every \(s\in S\), the fibre \(f^{-1}(s)\) of \(X\) over \(s\) is an algebraic scheme over the residue field \(k(s)\) of the local ring \({\mathscr{O}}_s\) of \(s\) in \(S\).
Thus \(X\) can be, to a certain extent, considered as a family of ``algebraic spaces'' \(f^{-1}(s)\), with the parameter \(s\) running over \(S\) (i.e., from the local point of view, the set of prime ideals of a given ring).
Of course, the \(k(s)\) can have different characteristics.
If \(S=\operatorname{Spec}(k)\), where \(k\) is a field, then we essentially recover the usual notion of ``algebraic space'', with the only difference being that now the structure sheaf can have nilpotent elements.

\oldpage{182-03}Inspired by well-known ideas, we can define the notion of a \emph{projective morphism}, and, more generally, of a \emph{proper morphism}.
Such a morphism is of finite type, and further sends closed subsets to closed subsets, and retains this property under an arbitrary change of base.

With \(X\) being a (Noetherian, as always) scheme, the sheaf \({\mathscr{O}}_X\) is a \emph{coherent sheaf of rings} in the sense of {[}\protect\hyperlink{ref-Gro1960}{5}{]}.
The coherent sheaves of modules on \(X\) are thus also the sheaves which are locally isomorphic to a cokernel of some morphism \({\mathscr{O}}_X^m\to{\mathscr{O}}_X^n\).

\hypertarget{fga-2-section-2}{%
\subsection{Formal schemes}\label{fga-2-section-2}}

Let \(X\) be a scheme, and \(X'\) a closed subset of \(X\).
Then there exists a coherent subsheaf \({\mathscr{J}}\) of \({\mathscr{O}}_X\) such that \(X'=\operatorname{supp}{\mathscr{O}}_X/{\mathscr{J}}\) (and there even exists a largest such one).
Endowing \(X'\) with the sheaf \({\mathscr{O}}_X/{\mathscr{J}}\) makes \(X'\) a scheme, denoted \(X_0\);
such a scheme is called a \emph{closed subscheme of \(X\)}.
We can also, for any \(n\), consider \(X'\) endowed with \({\mathscr{O}}_X/{\mathscr{J}}^{n+1}\), denoted \(X_n\), which is a closed subprescheme of \(X\) whose underlying set is again \(X'\), but with a different structure sheaf, namely \({\mathscr{O}}_{X_n}={\mathscr{O}}_X/{\mathscr{J}}^{n+1}\).
Clearly the \({\mathscr{O}}_{X_n}\) form a projective system of sheaves of rings on \(X\), whose projective limit \(\overline{{\mathscr{O}}_X}\) is called the \emph{formal completion of \({\mathscr{O}}_X\) along \(X'\)}.
Endowed with this sheaf of rings, \(X'\) is called the \emph{formal completion of \(X\) along \(X'\)}, and is thus a ringed space, but not a scheme in general.
For every coherent sheaf \({\mathscr{F}}\) on \(X\), we can similarly consider the formal completion \(\overline{{\mathscr{F}}}=\varprojlim_n{\mathscr{F}}_n\) of \({\mathscr{F}}\) along \(X'\) (where \({\mathscr{F}}_n={\mathscr{F}}\otimes_{{\mathscr{O}}_X}{\mathscr{O}}_X/{\mathscr{J}}^{n+1}\)), which is a sheaf of modules on \(\overline{X}\).
Its sections are called \emph{formal sections of \({\mathscr{F}}\) along \(X\)}, and can be identified with elements of \(\varprojlim_n\Gamma(X',{\mathscr{F}}_n)\).
For \({\mathscr{F}}={\mathscr{O}}_X\), we recover the ``holomorphic functions'' of \(X\) along \(X'\), in the sense of Zariski (whose terminology we will not follow, due to its interferences with classical terminology).

We define a \emph{formal scheme} (implicitly assumed to be Noetherian) to be a topological space \({\mathfrak{X}}\) endowed with a sheaf of topological rings \({\mathscr{O}}_{{\mathfrak{X}}}\) satisfying the following condition:
there is an isomorphism of sheaves of topological rings \({\mathscr{O}}_{{\mathfrak{X}}}=\varprojlim_n{\mathscr{O}}_n\), where the \({\mathscr{O}}_n\) form a projective system of sheaves of rings on \({\mathfrak{X}}\), with each one making \({\mathfrak{X}}\) into a scheme \({\mathfrak{X}}_n\), and such that, for \(m\geqslant n\), the homomorphism \({\mathscr{O}}_m\to{\mathscr{O}}_n\) is surjective and has \({\mathscr{J}}_m^{n+1}\) as its kernel, where \({\mathscr{J}}_m\) is the kernel of \({\mathscr{O}}_m\to{\mathscr{O}}_0\).
We will show that \({\mathscr{O}}_{{\mathfrak{X}}}\) is a \emph{coherent} sheaf of \emph{local Noetherian} rings.

\oldpage{182-04}By the definitions, a formal completion \(\overline{X}\) as above is a formal scheme, and, conversely, every formal scheme is \emph{locally} of this type.
In fact, the data of a formal \emph{affine} scheme (i.e.~such that \({\mathfrak{X}}_0\) is affine, which implies that all the \({\mathfrak{X}}_n\) are affine) is equivalent to the data of a separated complete \({\mathscr{J}}\)-adic Noetherian topological ring.

The usual definitions (morphism, morphism of finite type, proper morphism, etc.) for ordinary schemes generalise without problem to formal schemes.

\hypertarget{fga-2-section-3}{%
\subsection{The three fundamental theorems}\label{fga-2-section-3}}

Let \(f\colon X\to Y\) be a proper morphism of schemes (Noetherian, as always), and let \(Y'\) be a closed subset of \(Y'\), with \(X'\) its inverse image in \(X\), and consider the corresponding formal completions \(\overline{Y}\) and \(\overline{X}\).
Then \(f\) induces a morphism \(\overline{f}\colon\overline{X}\to\overline{Y}\) of formal schemes, which is also proper.
Let \({\mathscr{F}}\) be a coherent sheaf on \(X\), then \(\overline{{\mathscr{F}}}\) is a coherent sheaf on \(\overline{X}\).
In \protect\hyperlink{fga-2-theorem-1}{Theorem 1}, we forget \(X\), \(Y\), and \({\mathscr{F}}\), and consider only the proper morphism \(\overline{f}\) of formal schemes, along with the coherent sheaf \(\overline{{\mathscr{F}}}\) on \(\overline{X}\).
(However, the speaker has only written the complete proof in the case where we start with some \(X\), \(Y\), \(f\), and \(F\)).

\hypertarget{fga-2-theorem-1}{}
\begin{itenv}{Theorem 1}

\emph{Finiteness theorem.} ---

\begin{enumerate}
\def\labelenumi{\roman{enumi}.}
\item
  The \(\operatorname{R}^q\overline{f}_*(\overline{{\mathscr{F}}})\) are coherent sheaves on \(\overline{Y}\).
\item
  The natural homomorphisms
  \[
  \operatorname{R}^q\overline{f}_*(\overline{{\mathscr{F}}}) \to \varprojlim_n\operatorname{R}^q (f_n)_*({\mathscr{F}}_n)
    \]
  are isomorphisms.
\end{enumerate}

\end{itenv}

In this statement, we suppose that we already have some coherent subsheaf \({\mathscr{J}}\) of \({\mathscr{O}}_Y\) that defines \(Y'\), whence, by taking the inverse image, a coherent subsheaf of \({\mathscr{O}}_X\) that defines \(X'\), whence, by definition, \({\mathscr{F}}_n\), \(X_n\), \(Y_n\), and \(f_n\colon X_n\to Y_n\) as in \protect\hyperlink{fga-2-section-2}{§2}.
The minor changes that need to be made to the notation in the explanation if we started with an arbitrary proper morphism between two formal schemes are evident.

\protect\hyperlink{fga-2-theorem-1}{Theorem 1} deals only with ``formal cohomology'';
the following theorem relates it with ``algebraic cohomology'', and resembles a well-known theorem of Serre {[}\protect\hyperlink{ref-Ser1956}{19}{]} on the comparison between algebraic cohomology and analytic cohomology.

\leavevmode\vadjust pre{\hypertarget{fga-2-theorem-2}{}}%
\begin{itenv}{Theorem 2}
\oldpage{182-05}\emph{First comparison theorem.} ---
The \(\operatorname{R}^q f_*({\mathscr{F}})\) are coherent sheaves on \(Y\) (which is a particular case of \protect\hyperlink{fga-2-theorem-1}{Theorem 1}), and the natural homomorphisms
\[
  \overline{\operatorname{R}^q f_*({\mathscr{F}})} \to \varprojlim_n \operatorname{R}^q (f_n)_*({\mathscr{F}}_n)
\]
are isomorphisms.

\end{itenv}

\leavevmode\vadjust pre{\hypertarget{fga-2-theorem-2-corollary-1}{}}%
\begin{itenv}{Corollary 1}
There are canonical isomorphisms \(\overline{\operatorname{R}^q f_*({\mathscr{F}})} = \operatorname{R}^q\overline{f}_*(\overline{{\mathscr{F}}})\).

\end{itenv}

This corollary is, for \(q=0\), a generalisation of Zariski's ``fundamental theorem of holomorphic functions'', from which we will deduce a generalisation of Zariski's ``connection theorem''.
We also note that, while \protect\hyperlink{fga-2-theorem-1}{Theorem 1} (ii) is trivial for \(q=0\), this is not at all the case for \protect\hyperlink{fga-2-theorem-2}{Theorem 2} nor for its equivalent formulation (\protect\hyperlink{fga-2-theorem-2-corollary-1}{Corollary 1}).
In fact, the proof proceeds by decreasing induction on \(q\) (being trivial for large \(q\), since then both sides of the equation are zero), and the case \(q=0\) thus appears as the last induction step, and so could be called the ``most difficult'' case.

\leavevmode\vadjust pre{\hypertarget{fga-2-theorem-2-corollary-2}{}}%
\begin{itenv}{Corollary 2}
Let \(Y=\operatorname{Spec}(A)\), and let \(Y'\) be defined by an ideal \({\mathscr{J}}\) of \(A\).
Then, for every coherent sheaf \({\mathscr{F}}\) on \(X\), the \(H^q(X,{\mathscr{F}})\) are \(A\)-modules of finite type, whose \({\mathscr{J}}\)-adic completions are the \(H^q(\overline{X},\overline{{\mathscr{F}}})\).

\end{itenv}

Finally, applying this corollary to \(H=\mathscr{H}\kern -.5pt om_{{\mathscr{O}}_X}({\mathscr{F}},{\mathscr{G}})\), we obtain:

\leavevmode\vadjust pre{\hypertarget{fga-2-theorem-2-corollary-3}{}}%
\begin{itenv}{Corollary 3}
Let \(Y=\operatorname{Spec}(A)\), and let \(Y'\) be defined by an ideal \({\mathscr{J}}\) of \(A\).
Let \({\mathscr{F}}\) and \({\mathscr{G}}\) be coherent sheaves on \(X\).
Then \(\operatorname{Hom}({\mathscr{F}},{\mathscr{G}})\) is an \(A\)-module of finite type, whose \({\mathscr{J}}\)-adic completion can be identified with \(\operatorname{Hom}(\overline{{\mathscr{F}}},\overline{{\mathscr{G}}})\).

\end{itenv}

Of course, the natural map \(\operatorname{Hom}({\mathscr{F}},{\mathscr{G}})\to\operatorname{Hom}(\overline{{\mathscr{F}}},\overline{{\mathscr{G}}})\) is that which sends a homomorphism \(u\colon{\mathscr{F}}\to{\mathscr{G}}\) to its extension ``by continuity'' \(\overline{u}\colon\overline{{\mathscr{F}}}\to\overline{{\mathscr{G}}}\) (so that \(\overline{{\mathscr{F}}}\) becomes a functor in \({\mathscr{F}}\)).

Now suppose that \(A\) is separated and complete for its \({\mathscr{J}}\)-adic topology.
Then \protect\hyperlink{fga-2-theorem-2-corollary-2}{Corollary 2} and \protect\hyperlink{fga-2-theorem-2-corollary-3}{Corollary 3} above give:
\[
  \begin{aligned}
    H^q(X,{\mathscr{F}}) &= H^q(\overline{X},\overline{{\mathscr{F}}}),
  \\\operatorname{Hom}({\mathscr{F}},{\mathscr{G}}) &= \operatorname{Hom}(\overline{{\mathscr{F}}},\overline{{\mathscr{G}}}).
  \end{aligned}
\]

\oldpage{182-06}The latter identity shows that the category of coherent sheaves on \(X\) can be identified with a \emph{subcategory} (with morphisms being the induced morphisms) of the category of coherent sheaves on \(\overline{X}\).
In fact, we even have:

\leavevmode\vadjust pre{\hypertarget{fga-2-theorem-3}{}}%
\begin{itenv}{Theorem 3}
For a sheaf of modules on \(\overline{X}\) to be coherent, it is necessary and sufficient that it be isomorphic to a sheaf of the form \(\overline{{\mathscr{F}}}\), where \({\mathscr{F}}\) is a coherent sheaf on \(X\) (determined up to canonical isomorphism, by \protect\hyperlink{fga-2-theorem-2-corollary-3}{Corollary 3 of Theorem 2}).
{[}We recall that now \(Y=\operatorname{Spec}(A)\), with \(A\) being a complete and separated \({\mathscr{J}}\)-adic Noetherian topological ring{]}.

\end{itenv}

\leavevmode\vadjust pre{\hypertarget{fga-2-theorem-3-corollary-1}{}}%
\begin{itenv}{Corollary 1}
The closed subschemes of \(X\) are in bijective correspondence with the closed formal subschemes of \(\overline{X}\).

\end{itenv}

Indeed, they correspond to coherent subsheaves of \({\mathscr{O}}_X\) (resp. of \({\mathscr{O}}_{\overline{X}}\)).
Considering the graphs of morphisms as closed subschemes, \protect\hyperlink{fga-2-theorem-3-corollary-1}{Corollary 1} implies:

\leavevmode\vadjust pre{\hypertarget{fga-2-theorem-3-corollary-2}{}}%
\begin{itenv}{Corollary 2}
Let \(X\) and \(Z\) be proper schemes over \(A\) (which is a separated complete \({\mathscr{J}}\)-adic Noetherian ring).
Then the map \(g\mapsto\overline{g}\) defines a bijective correspondence between \(Y\)-morphism from \(X\) to \(Z\) and \(\overline{Y}\)-morphisms from \(\overline{X}\) to \(\overline{Z}\).

\end{itenv}

In other words, proper algebraic schemes over \(A\) give a subcategory (with the morphisms being the induced morphisms) of the category of proper formal schemes over \(\overline{Y}\).
We note, however, that \emph{there exist proper formal schemes over \(\overline{Y}\) that are not ``algebraisable''}, i.e.~isomorphic to some \(\overline{X}\), where \(X\) is proper over \(A\) (just as there exist compact complex-analytic varieties that do not come from algebraic varieties defined over the field of complex numbers).
Such formal schemes naturally appear in ``module theory''.
We note, however, an interesting case where a formal scheme is algebraisable:

\leavevmode\vadjust pre{\hypertarget{fga-2-theorem-4}{}}%
\begin{itenv}{Theorem 4}
Let \(A\) be a complete local Noetherian ring, with residue field \(k\), and let \({\mathfrak{X}}\) be a proper formal scheme over \(A\) (endowed with its \({\mathfrak{r}}(A)\)-adic topology).
We suppose that

\begin{enumerate}
\def\labelenumi{\roman{enumi}.}
\item
  the local rings of \({\mathscr{O}}_{{\mathfrak{X}}}\) are \emph{flat} \(A\)-modules, or, equivalently, that, if we endow \({\mathscr{O}}_{{\mathfrak{X}}}\) and \(A\) with the filtration given by powers of the maximal ideal of \(A\), then the associated graded algebras satisfy
  \[
   \operatorname{gr}({\mathscr{O}}_{{\mathfrak{X}}}) \simeq \operatorname{gr}^0({\mathscr{O}}_{{\mathfrak{X}}})\otimes_k\operatorname{gr}(A);
    \]
\item
  \(H^2({\mathfrak{X}}_0,{\mathscr{O}}_{{\mathfrak{X}}_0})=0\), where we consider \({\mathfrak{X}}_0={\mathfrak{X}}\otimes_Ak\) as an algebraic scheme over \(k\);
\item
  \({\mathfrak{X}}_0\) is projective.
\end{enumerate}

\oldpage{182-07}Then, under these conditions, \({\mathfrak{X}}\) is algebraisable, and, more precisely, is isomorphic to \(\overline{X}\), where \(X\) is some projective \(A\)-scheme.

\end{itenv}

Conditions (ii) and (iii) will be satisfied if, in particular, \({\mathfrak{X}}_0\) is a \emph{simple curve} over \(k\), and \protect\hyperlink{fga-2-theorem-4}{Theorem 4} can be applied, in particular, in the ``module theory'' of curves of a given genus\ldots{}
We will give here a hint on how to prove \protect\hyperlink{fga-2-theorem-4}{Theorem 4}:
we can show (cf.~\protect\hyperlink{fga-2-proposition-3}{Proposition 3} below) that (i) and (ii) imply that every coherent sheaf on \({\mathfrak{X}}_0\) that is locally isomorphic to a fundamental sheaf can be obtained by reduction starting from a sheaf of the same nature on \({\mathfrak{X}}\).
So, starting with an ``ample'' sheaf on \({\mathfrak{X}}_0\) (which, by (iii), exists), we lift it to obtain an invertible sheaf on \({\mathfrak{X}}\), and, using \protect\hyperlink{fga-2-theorem-1}{Theorem 1}, we prove that a multiple of this invertible sheaf defined an immersion of \(X\) into the formal completion of a scheme \(\mathbb{P}_A^r\) (``projective type'' of dimension \(r\) over \(A\)).

For the proof of \protect\hyperlink{fga-2-theorem-1}{Theorem 1}, \protect\hyperlink{fga-2-theorem-2}{Theorem 2}, and \protect\hyperlink{fga-2-theorem-3}{Theorem 3}, we refer the reader to {[}\protect\hyperlink{ref-GD1960}{10}, I{]}.

\hypertarget{fga-2-section-4}{%
\subsection{Applications to Zariski's connection theorem and ``main theorem''}\label{fga-2-section-4}}

Let \(f\colon X\to Y\) be a proper morphism of schemes.
Then, by the finiteness theorem (\protect\hyperlink{fga-2-theorem-1}{Theorem 1}), \(f_*({\mathscr{O}}_X)=\underline{A}\) is a coherent sheaf on \(Y\), and is also a sheaf of commutative algebras, and thus corresponds to a \(Y\)-scheme \(g\colon Y'\to Y\) that is finite over \(Y\) (defined by the condition of being affine over \(Y\), i.e.~the inverse image of an affine open is affine, and \(g_*({\mathscr{O}}_{Y'})=\underline{A}\)).
It is immediate that \(f\) then canonically factors as \(f=gf'\), where \(f'\colon X\to Y'\) is a morphism from \(X\) to \(Y\) that is now such that \(f'_*({\mathscr{O}}_X)={\mathscr{O}}_{Y'}\).
This factorisation of \(f\) is called \emph{the Stein factorisation} of \(f\).
Applying the first comparison theorem (\protect\hyperlink{fga-2-theorem-2}{Theorem 2}) and its \protect\hyperlink{fga-2-theorem-2-corollary-1}{Corollary 1} to \(f'\) and the subset \(Y'\) consisting of a single point \(y'\), we see that \((f')^{-1}(y')=X'\) is connected (or, in other words, the formal sections of \(X\) along \(X'\) do not form a local ring, but the completion \(f'_*({\mathscr{O}}_X)_{y'}={\mathscr{O}}_{y'}\) is local!)
We have proven:

\leavevmode\vadjust pre{\hypertarget{fga-2-theorem-5}{}}%
\begin{itenv}{Theorem 5}
\emph{Zariski's ``connection theorem''} ---
Let \(f\colon X\to Y\) be a proper morphism.
Then \(f\) factors uniquely (up to isomorphism) as \(f=gf'\), where \(g\colon Y'\to Y\) is finite, and \(f'\colon X\to Y'\) is such that \(f'_*({\mathscr{O}}_X)={\mathscr{O}}_{Y'}\) (whence \(g_*({\mathscr{O}}_{Y'})=f_*({\mathscr{O}}_X)\)).
Also, the fibres of \(f'\) are connected, i.e.~the set of connected components of a fibre \(f^{-1}(y)\) of \(f\) is in bijective correspondence with the set of points of \(Y'\) over \(y\), i.e.~the set of maximal ideals in \(f_*({\mathscr{O}}_X)_y\).

\end{itenv}

\oldpage{182-06}From this, we immediately deduce the usual variants of the connection theorem.
We state here only the following:

\leavevmode\vadjust pre{\hypertarget{fga-2-theorem-5-corollary-1}{}}%
\begin{itenv}{Corollary 1}
For a point \(x\) of \(X\) to be isolated in its fibre \(f^{-1}(y)\), it is necessary and sufficient that the fibre \((f')^{-1}(y')\) (where \(y'=f'(x)\)) consist of a single point \(x\), or that \(f'\) induce an isomorphism from a neighbourhood of \(x\) to a neighbourhood of \(y'\).
The set of these points is an open subset \(U\), and \(f'\) induces an isomorphism from \(U\) to an open subset of \(Y'\).

\end{itenv}

To show that \(f'\) is a local isomorphism at \(x\), we note that \(f'\) induces an \emph{isomorphism} \({\mathscr{O}}_{y'}\to{\mathscr{O}}_x\), as we see thanks to \(f'({\mathscr{O}}_X)={\mathscr{O}}_{Y'}\);
we also note that the \((f')^{-1}(V)\) give a fundamental system of neighbourhoods of \(x\) when \(V\) runs over a fundamental system of neighbourhoods of \(y'\) (since \(f'\) is a closed map whose fibre at \(y'\) consists of the single point \(x\)).
We thus immediately deduce the following result, due to Chevalley in the ``geometric'' case:

\leavevmode\vadjust pre{\hypertarget{fga-2-theorem-5-corollary-2}{}}%
\begin{itenv}{Corollary 2}
For \(f\) to be a finite morphism, it is necessary and sufficient that it be proper with finite fibres.

\end{itenv}

If this is so, then \(f'\) is effectively an isomorphism, by the above.

Let \(f\colon X\to Y\) be a morphism that is not necessarily proper, but suppose that \(X\) is contained in some proper \(Y\)-scheme \(\overline{f}\colon\overline{X}\to Y\) as an open subset (which is the case if, in particular, \(\overline{f}\) is quasi-projective).
Applying \protect\hyperlink{fga-2-theorem-5-corollary-1}{Corollary 1}, we see that \(\overline{f'}\) induces an isomorphism from the set \(U\) of points of \(X\) that are isolated in their fibre to an open subset of \(Y'\) (and that \(U\) is indeed an open subset).
We thus deduce the following global version of Zariski's ``main theorem'':

\leavevmode\vadjust pre{\hypertarget{fga-2-theorem-6}{}}%
\begin{itenv}{Theorem 6}
Let \(f\colon X\to Y\) be a morphism of finite type.
Then the set \(U\) of points of \(X\) that are isolated in their fibre is open, and, if \(f\) is quasi-projective\footnote{\emph{{[}Comp.{]}} This hypothesis can be replaced by the weaker hypothesis ``if \(f\) is separated'', by means of the following result (see {[}\protect\hyperlink{ref-Gro1960b}{9}, VIII, 6.2{]}): every morphism \(f\colon X\to Y\) which is quasi-finite and separated is also projective.}, then \(U\) is \(Y\)-isomorphic to an open subset of some scheme \(Y'\) that is finite over \(Y\).

\end{itenv}

Since a morphism of finite type is locally affine, and \emph{a fortiori} locally quasi-projective, we immediately deduce from \protect\hyperlink{fga-2-theorem-6}{Theorem 6} the usual local variants of the main theorem.

\hypertarget{fga-2-section-5}{%
\subsection{Application to the cohomological study of proper and flat morphisms}\label{fga-2-section-5}}

Let \(f\colon X\to Y\) be a proper morphism, and \({\mathscr{F}}\) a coherent sheaf on \(X\), with \({\mathscr{F}}\) assumed to be \(Y\)-flat, i.e.~the \({\mathscr{F}}_x\) are flat modules over the rings \({\mathscr{O}}_y\) (where \(y=f(x)\)).
\oldpage{182-09}This also implies that, for every \(y\in Y\), if we filter \({\mathscr{F}}\) along the fibre \(f^{-1}(y)\) by the \({\mathfrak{m}}_y^n{\mathscr{F}}\) (where \({\mathfrak{m}}_y\) is the maximal ideal of \({\mathscr{O}}_y\)), then the associated graded algebra is isomorphic to \(({\mathscr{F}}/{\mathfrak{m}}_y{\mathscr{F}})\otimes_{k(y)}\operatorname{gr}({\mathscr{O}}_y)\);
in other words, we have that
\[
  {\mathfrak{m}}_y^n{\mathscr{F}}/{\mathfrak{m}}_y^{n+1}
  = {\mathscr{F}}_y\otimes_{k(y)}({\mathfrak{m}}_y^n/{\mathfrak{m}}_y^{n+1})
\]
for every integer \(n\), where \({\mathscr{F}}_y\) denotes the sheaf \({\mathscr{F}}/{\mathfrak{m}}_y{\mathscr{F}}\) induced by \({\mathscr{F}}\) on \(X_y\) (with \(X_y\) denoting the fibre \(f^{-1}(y)\) considered as a proper scheme over the residue field \(k(y)\) of \(y\)).
Taking this isomorphism, as well as \protect\hyperlink{fga-2-theorem-2}{Theorem 2}, into account, we obtain augmentations, and sometimes computations, of the \(\operatorname{R}^q f_*({\mathscr{F}})\) in a neighbourhood of \(y\), knowing the cohomology of \(X_y\) with coefficients in \({\mathscr{F}}_y\).
Here \protect\hyperlink{fga-2-theorem-2}{Theorem 2} takes the form
\[
  \overline{\operatorname{R}^q f_*({\mathscr{F}})} = \varprojlim_n H^q({\mathscr{F}}_y,{\mathscr{F}}/{\mathfrak{m}}_y^n{\mathscr{F}}).
\]
We will mention here only the following consequence:

\leavevmode\vadjust pre{\hypertarget{fga-2-proposition-1}{}}%
\begin{itenv}{Proposition 1}
Let \(f\colon X\to Y\) be a proper morphism, and \({\mathscr{F}}\) a coherent \(Y\)-flat sheaf on \(X\).
Let \(y\in Y\), let \(q\) be an integer, and suppose that \(H^q(X_y,{\mathscr{F}}_y)=0\).
Then \(\operatorname{R}^q f_*({\mathscr{F}})\) is zero on a a neighbourhood of \(y\), and, for every \(n\), the natural homomorphism
\[
  \operatorname{R}^{q-1}f_*({\mathscr{F}})_y \to H^{q-1}(X_y,{\mathscr{F}}_y/{\mathfrak{m}}_y^n{\mathscr{F}}_y)
\]
is surjective.

\end{itenv}

In particular, if \(f\) is a flat morphism (i.e.~if \({\mathscr{O}}_X\) is \(Y\)-flat), then every locally free coherent sheaf \({\mathscr{F}}\) on \(X\) is \(Y\)-flat.
Let \({\mathscr{F}}\) and \({\mathscr{G}}\) be two such sheaves, and apply \protect\hyperlink{fga-2-proposition-1}{Proposition 1} to \(\mathscr{H}\kern -.5pt om_{{\mathscr{O}}_X}({\mathscr{F}},{\mathscr{G}})\) and \(q=1\) to obtain:

\leavevmode\vadjust pre{\hypertarget{fga-2-theorem-7}{}}%
\begin{itenv}{Theorem 7}
Let \(f\) be a flat proper morphism, \({\mathscr{F}}\) and \({\mathscr{G}}\) locally free coherent sheaves on \(X\), and \(y\in Y\);
suppose that \(H^1(X_y,\mathscr{H}\kern -.5pt om_{{\mathscr{O}}_X}({\mathscr{F}}_y,{\mathscr{G}}))=0\).
Then every homomorphism \(u_0\colon{\mathscr{F}}_y\to{\mathscr{G}}_y\) is induced by a homomorphism \(u\colon{\mathscr{F}}|V\to{\mathscr{G}}|V\), where \(V=f^{-1}(U)\) is the inverse image of a neighbourhood \(U\) of \(y\).

\end{itenv}

\leavevmode\vadjust pre{\hypertarget{fga-2-theorem-7-corollary-1}{}}%
\begin{itenv}{Corollary 1}
If \(u_0\) is an isomorphism (resp. a monomorphism, resp. an epimorphism), then so too is \(u\), for small enough \(U\).

\end{itenv}

In particular:

\leavevmode\vadjust pre{\hypertarget{fga-2-theorem-7-corollary-2}{}}%
\begin{itenv}{Corollary 2}
\oldpage{182-10}Let \(E_0\) be a locally free coherent sheaf on \(X_y\) such that \(H^1(X_y;\mathscr{H}\kern -.5pt om_{{\mathscr{O}}_X}(E_0,E_0))=0\).
Then any two locally free sheaves whose restrictions to \(X_y\) are isomorphic to \(E_0\) are themselves isomorphic to one another in a neighbourhood of \(X_y\).

\end{itenv}

Thus:

\leavevmode\vadjust pre{\hypertarget{fga-2-theorem-7-corollary-3}{}}%
\begin{itenv}{Corollary 3}
Suppose that \(H^1(X_y,{\mathscr{O}}_{X_y})=0\).
Then any two invertible sheaves on \(X\) (i.e.~locally isomorphic to \({\mathscr{O}}_X\)) whose restrictions to \(X_y\) are isomorphic are themselves isomorphic to one another.

\end{itenv}

It thus follows that:

\leavevmode\vadjust pre{\hypertarget{fga-2-proposition-2}{}}%
\begin{itenv}{Proposition 2}
Let \(Y\) be a connected scheme, and \(E\) a locally free coherent sheaf on \(Y\).
Consider the bundle of projective spaces \(X=\mathbb{P}(E)\) associated to \(E\), endowed with its well-known invertible sheaf \({\mathscr{O}}_X(1)\).
Then every invertible sheaf \({\mathscr{L}}\) on \(X\) is isomorphic to a sheaf of the form \(f^*({\mathscr{L}}')\otimes{\mathscr{O}}_X(n)\), where \({\mathscr{L}}'\) is an invertible sheaf on \(Y\), and \(n\) is an integer.
Further, \(n\) is uniquely determined, and \({\mathscr{L}}'\) is determined up to isomorphism.

\end{itenv}

\protect\hyperlink{fga-2-theorem-7-corollary-3}{Corollary 3} above proves that \({\mathscr{L}}\) is isomorphic to an \({\mathscr{O}}_X(n)\)-module on a neighbourhood of each fibre.
The rest is more or less formal.

\protect\hyperlink{fga-2-proposition-2}{Proposition 2} allows us to determine the \(Y\)-morphisms from \(X=\mathbb{P}(E)\) to another projective bundle.
We see, in particular:

\leavevmode\vadjust pre{\hypertarget{fga-2-proposition-2-corollary-1}{}}%
\begin{itenv}{Corollary 1}
Let \(u\) be an automorphism of \(X=\mathbb{P}(E)\).
Then there exists an invertible sheaf \({\mathscr{L}}'\) on \(Y\), and an isomorphism \(v\) from \(E\) to \(E\otimes{\mathscr{L}}'\) such that \(u\) is the isomorphism corresponding to \(\mathbb{P}(E)\xrightarrow{\sim}\mathbb{P}(E\otimes{\mathscr{L}}')=\mathbb{P}(E)\);
the pair \((v,{\mathscr{L}}')\) is determined up to isomorphism.

\end{itenv}

Let \(\Gamma\) be the set of classes of invertible bundles \({\mathscr{L}}'\) on \(Y\) such that \(E\otimes{\mathscr{L}}'\) is isomorphic to \(E\).
Its elements are torsion, since, if \(n\) is the rank of \(E\), then (by taking \(n\)-th exterior powers) we must have that \(({\mathscr{L}}')^{\otimes n}\xrightarrow{\sim}{\mathscr{O}}_Y\).
The above corollary can then be expressed by saying that we have an exact sequence of groups:
\[
  e \to \operatorname{Aut}(E)/\Gamma(Y,{\mathscr{O}}_Y^*) \to \operatorname{Aut}_Y(X) \to \Gamma \to e
\]
\oldpage{182-11}(which can also be deduced from the exact sequence in cohomology induced by the exact sequence of \emph{sheaves} of groups
\[
  e \to {\mathscr{O}}_X^\times \to \mathscr{A}\kern -.5pt ut\to \mathscr{A}\kern -.5pt ut_Y(X) \to e,
\]
where \({\mathscr{O}}_X^\times\) is the sheaf of ``units'' of \({\mathscr{O}}_X\), identified with the centre of \(\mathscr{A}\kern -.5pt ut(E)\).)

\hypertarget{fga-2-section-6}{%
\subsection{\texorpdfstring{Application to existence and uniqueness theorems for sheaves and schemes over a complete \(\mathscr{J}\)-adic ring}{Application to existence and uniqueness theorems for sheaves and schemes over a complete \textbackslash mathscr\{J\}-adic ring}}\label{fga-2-section-6}}

\protect\hyperlink{fga-2-theorem-7}{Theorem 7} gave a uniqueness result for locally free coherent sheaves, by using \protect\hyperlink{fga-2-theorem-1}{Theorem 1} and \protect\hyperlink{fga-2-theorem-2}{Theorem 2}.
Using \protect\hyperlink{fga-2-theorem-3}{Theorem 3}, we now obtain \emph{existence} theorems for sheaves, for morphisms of schemes, or for schemes.
In the following, \(A\) denotes a local Noetherian ring, assumed to be separated and \emph{complete}.
The general method still consists of making \emph{formal} construction, which consists essentially of doing \emph{algebraic geometry over an Artinian ring}, and deducing conclusions from this that are ``algebraic'' in nature, by using the three fundamental theorems.

\leavevmode\vadjust pre{\hypertarget{fga-2-proposition-3}{}}%
\begin{itenv}{Proposition 3}
Let \({\mathfrak{X}}\) be a proper formal scheme that is flat over \(A\), and let \({\mathscr{F}}_0\) be a locally free sheaf on \(X_0\) such that \(H^2(X_0,\mathscr{H}\kern -.5pt om_{{\mathscr{O}}_{X_0}}({\mathscr{F}}_0,{\mathscr{F}}_0))=0\).
Then there exists a locally free sheaf \({\mathscr{F}}\) on \({\mathfrak{X}}\) that induces, on \(X_0\), a sheaf isomorphic to \({\mathscr{F}}_0\).
(This \({\mathscr{F}}\) is also unique up to isomorphism if \(H^1(X_0,\mathscr{H}\kern -.5pt om_{{\mathscr{O}}_{X_0}}({\mathscr{F}}_0,{\mathscr{F}}_0))=0\)).

\end{itenv}

We construct, step by step, locally free sheaves \({\mathscr{F}}_n\) on the \(X_n\) that induce one another.
The construction of \({\mathscr{F}}_0\) is met with an obstruction living in \(H^2(X_0,\mathscr{H}\kern -.5pt om_{{\mathscr{O}}_{X_0}}({\mathscr{F}}_0,{\mathscr{F}}_0))\otimes_{A/{\mathscr{J}}}({\mathscr{J}}^n/{\mathscr{J}}^{n+1})\), but this is zero, by hypothesis.
Now, using \protect\hyperlink{fga-2-theorem-3}{Theorem 3}, we obtain:

\leavevmode\vadjust pre{\hypertarget{fga-2-proposition-3-corollary-1}{}}%
\begin{itenv}{Corollary 1}
Let \(X\) be a proper scheme that is flat over \(A\), and let \({\mathscr{F}}_0\) be as above.
Then there exists a locally free sheaf \({\mathscr{F}}\) on \(X\) that induces, on \(X_0\), a sheaf that is isomorphic to \({\mathscr{F}}_0\).
This \({\mathscr{F}}\) is also unique up to isomorphism if \(H^1(X_0,\mathscr{H}\kern -.5pt om_{{\mathscr{O}}_{X_0}}({\mathscr{F}}_0,{\mathscr{F}}_0))=0\).

\end{itenv}

Let \(X_0\) be a scheme of finite type over the field \(k\), and suppose that \(X_0\) is \emph{simple} (by which we mean \emph{absolutely} simple) over \(k\), but not necessarily proper over \(k\).
Let \(A\) be a local Artinian ring with residue field \(k\).
We are interested in finding schemes \(X\) that are flat over \(A\), and such that \(X\otimes_A k=X_0\) (this is the starting point of the ``\emph{theory of modules}'', or of ``structure variations'' of \(X_0\)).
\oldpage{182-12}It is equivalent to give either such an \(X\) or, on the topological space \(X_0\), a sheaf \({\mathscr{O}}_X\) endowed with the following structures:

\begin{enumerate}
\def\labelenumi{\roman{enumi}.}
\item
  \({\mathscr{O}}_X\) is a sheaf of \(A\)-algebras;
\item
  \({\mathscr{O}}_X\) is endowed with an augmentation homomorphism \({\mathscr{O}}_X\to{\mathscr{O}}_{X_0}\) (that is compatible with the \(A\)-algebra structures);
\end{enumerate}

and with the above data being subject to the following conditions: the augmentation induces an isomorphism \({\mathscr{O}}_X\otimes_A k\xrightarrow{\sim}{\mathscr{O}}_{X_0}\);
\({\mathscr{O}}_X\) is flat over \(A\), i.e.~the graded algebra associated to \({\mathscr{O}}_X\) filtered by the powers of the maximal ideal \({\mathfrak{m}}\) of \(A\) is isomorphic to \(\operatorname{gr}^0({\mathscr{O}}_X)\otimes_k\operatorname{gr}(A)\), i.e.~we have isomorphisms \({\mathfrak{m}}^n{\mathscr{O}}_X/{\mathfrak{m}}^{n+1}{\mathscr{O}}_X = {\mathscr{O}}_{X_0}\otimes_k({\mathfrak{m}}^n/{\mathfrak{m}}^{n+1})\).
The fundamental fact is the following:

\leavevmode\vadjust pre{\hypertarget{fga-2-theorem-8}{}}%
\begin{itenv}{Theorem 8}
Let \(X_0\) be a simple scheme of finite type over the field \(k\), and assume \(X_0\) to be affine.
Let \(A\) be a local Artinian ring of residue field \(k\).
Then there exists an \(A\)-scheme \(X\) that is flat over \(A\) and such that \(X\otimes_A k=X_0\).
Further, any two such schemes are necessarily isomorphic.

\end{itenv}

Note that the isomorphic in question is not canonical, since \(X\) will have, in general, non-trivial \(A\)-automorphisms that induce the identity on \(X_0\).
Furthermore, there is not, in general, a ``canonical'' choice of \(X\) satisfying the given conditions, except in the case where \(A\) is a \(k\)-algebra (the case of \emph{equal characteristics}), where we can take \(X=X_0\otimes_k A\), i.e.~\({\mathscr{O}}_X={\mathscr{O}}_{X_0}\otimes_k A\) (whether or not \(X_0\) is affine, in fact).
In the case of unequal characteristics, I do not know in general, when \(X_0\) is not affine, if we can ``lift'' \(X_0\) to an \(X\) defined over \(A\).
However, for any integer \(n>0\), let \(A_{n-1}=A/{\mathfrak{m}}^n\), and suppose that we have lifted \(X_0\) to a flat \(A_{n-1}\)-scheme \(X_{n-1}\);
we intend to lift \(X_{n-1}\) to a flat \(A_n\)-scheme \(X_n\).
We already know, by \protect\hyperlink{fga-2-theorem-8}{Theorem 8}, that this is possible locally;
we can also easily verify that, if \(U_n\) lifts an open subset \(U_{n-1}\) of \(X_{n-1}\), then the sheaf of groups of automorphisms of \(U_n\) (that induce the identity on \(U_{n-1}\)) is canonically isomorphic to \({\mathfrak{G}}_{X_0/k}\otimes_k{\mathfrak{m}}^n/{\mathfrak{m}}^{n+1}\) restricted to \(U_{n-1}\), and thus, in particular, commutative (where \({\mathfrak{G}}_{X_0/k}\) denotes the sheaf of germs of \(k\)-derivations on \(X_0\)).
It follows easily that we have an \emph{obstruction of constructing \(X_n\) lifting \(X_{n-1}\)}, which lives in \(H^2(X_0,{\mathfrak{G}}_{X_0/k})\otimes{\mathfrak{m}}^n/{\mathfrak{m}}^{n+1}\).
Then:

\leavevmode\vadjust pre{\hypertarget{fga-2-theorem-8-corollary-1}{}}%
\begin{itenv}{Corollary 1}
Let \(X_0\) be a simple scheme of finite type over \(k\), and suppose that
\[H^2(X_0,{\mathfrak{G}}_{X_0/k})=0.\]
\oldpage{182-13}Then, for every local Artinian ring \(A\) with residue field \(k\), there exists a flat \(A\)-scheme \(X\) such that \(X\otimes_A k=X_0\).

\end{itenv}

Also, if we can find \emph{one} \(X\) that is flat over \(A\) and that lifts \(X_0\), then, by \protect\hyperlink{fga-2-theorem-8}{Theorem 8}, the set of classes (up to isomorphism) of flat \(A\)-schemes that lift \(X_0\) can be identified with \(H^1(X_0,\mathscr{A}\kern -.5pt ut(X))\), where \(\mathscr{A}\kern -.5pt ut(X)\) denotes the sheaf of germs of automorphisms of the sheaf \({\mathscr{O_X}}\) of \(A\)-algebras \emph{that are compatible with the augmentation}.
The filtration of \({\mathscr{O}}_X\) defines a filtration of \(\mathscr{A}\kern -.5pt ut(X)\), with the quotient of this sheaf by the \(n\)-th subgroup of the filtration being \(\mathscr{A}\kern -.5pt ut(X_n)\);
the graded algebra associated to this filtration is commutative, and can be identified with \({\mathfrak{G}}_{X_0/k}\otimes_k\operatorname{gr}(A)\).
In particular, if \({\mathfrak{m}}^{n+1}\) is the first power of \({\mathfrak{m}}\) that is not zero, then \(F^n(\mathscr{A}\kern -.5pt ut(X))\) (the last stage of the filtration) is in the centre of \(\mathscr{A}\kern -.5pt ut(X)\), and is isomorphic to \({\mathfrak{G}}_{X_0/k}\otimes_k{\mathfrak{m}}^n\);
it is also the sheaf of germs of automorphisms of \(X\) that induce the identity on \(X_{n-1}=X\otimes_A A/{\mathfrak{m}}^n\).
Using these results, we immediately obtain the following statements:

\leavevmode\vadjust pre{\hypertarget{fga-2-theorem-8-corollary-2}{}}%
\begin{itenv}{Corollary 2}
Let \(X_0\) be a simple scheme of finite type over \(k\), and let \(A\) be a local Artinian ring with residue field \(k\) and maximal ideal \({\mathfrak{m}}\).
Suppose that \({\mathfrak{m}}^{n+1}=0\).
Let \(A_{n-1}=A/{\mathfrak{m}}^n\), and let \(X_{n-1}\) be a flat \(A_{n-1}\)-scheme such that \(X_{n-1}\otimes_Ak=X_0\).
Then the set of classes (up to an isomorphism that induces the identity on \(X_{n-1}\)) of flat \(A\)-schemes \(X_n\) such that \(X\otimes_AA_{n-1}=X_{n-1}\) is either empty, or a homogeneous principal space under \(H^1(X_0,{\mathfrak{G}}_{X_0/k})\otimes_k{\mathfrak{m}}^n\).

\end{itenv}

(Note that, in general, there is no privileged choice of origin in the latter homogeneous principal space, since there is no privileged way of lifting \(X_{n-1}\) to \(X_n\)).

\leavevmode\vadjust pre{\hypertarget{fga-2-theorem-8-corollary-3}{}}%
\begin{itenv}{Corollary 3}
Let \(X_0\) be a simple scheme of finite type over \(k\), and suppose that \(H^1(X_0,{\mathfrak{G}}_{X_0/k})=0\).
Then, for every local Artinian ring \(A\) with residue field \(k\), \emph{there exists at most one flat \(A\)-scheme \(X\)} (up to isomorphism) such that \(X\otimes_Ak=X_0\).

\end{itenv}

\protect\hyperlink{fga-2-theorem-8-corollary-1}{Corollary 1} and \protect\hyperlink{fga-2-theorem-8-corollary-3}{Corollary 3} immediately imply the claims, which seem more general, obtained by supposing only that \(A\) is a \emph{complete local Noetherian ring with residue field \(k\)}, provided that we introduce \(X\) as a formal scheme over \(A\):

\hypertarget{fga-2-theorem-9}{}
\begin{itenv}{Theorem 9}

Let \(k\) be a field, and \(X_0\) a \emph{simple} scheme of finite type over \(k\).
For every complete locally Noetherian ring \(A\) with residue field \(k\), let \(F(A)\) be the set of classes (up to an isomorphism that induces the identity on \(X_0\)) of formal schemes \({\mathfrak{X}}\) over \(A\), of finite type, and flat over \(A\), such that \(X\otimes_Ak=X_0\).
\oldpage{182-14}With this notation:
for all \(A\),

\begin{enumerate}
\def\labelenumi{\roman{enumi}.}
\tightlist
\item
  if \(H^1(X_0,{\mathfrak{G}}_{X_0/k})=0\) then \(F(A)\) has at most one element;
\item
  if \(H^2(X_0,{\mathfrak{G}}_{X_0/k})=0\) then \(F(A)\) has at least one element.
\end{enumerate}

\end{itenv}

\leavevmode\vadjust pre{\hypertarget{fga-2-theorem-9-corollary-1}{}}%
\begin{itenv}{Corollary 1}
Suppose that \(X_0\) is proper over \(k\).
Under condition (i) of \protect\hyperlink{fga-2-theorem-9}{Theorem 9}, for all \(A\), there exists at most (up to an isomorphism that induces the identity on \(X_0\)) one scheme \(X\) that is proper, flat over \(A\), and such that \(X\otimes_Ak=X_0\).

\end{itenv}

We can use \protect\hyperlink{fga-2-theorem-3-corollary-2}{Corollary 2 of Theorem 3}.
For example:

\leavevmode\vadjust pre{\hypertarget{fga-2-theorem-9-corollary-2}{}}%
\begin{itenv}{Corollary 2}
If \(X\) is a proper flat \(A\)-scheme such that \(X\otimes_Ak\) is isomorphic to the projective-type scheme \({\mathfrak{P}}_k^r\) of dimension \(r\) over \(k\), then \(X\) is isomorphic to \({\mathfrak{P}}_k^r\).

\end{itenv}

(We can also deduce this result from \protect\hyperlink{fga-2-proposition-3-corollary-1}{Corollary 1 of Proposition 3}).

\leavevmode\vadjust pre{\hypertarget{fga-2-theorem-9-corollary-3}{}}%
\begin{itenv}{Corollary 3}
Let \(X_0\) be a simple projective scheme over \(k\), and suppose that
\[
  H^2(X_0,{\mathscr{O}}_{X_0}) = H^2(X_0,{\mathfrak{G}}_{X_0/k}) = 0.
\]
Then, for all \(A\), there exists a flat projective \(A\)-scheme such that \(X\otimes_Ak=X_0\).

\end{itenv}

We can combine \protect\hyperlink{fga-2-theorem-9}{Theorem 9} (ii) with \protect\hyperlink{fga-2-theorem-4}{Theorem 4}.
In particular:

\leavevmode\vadjust pre{\hypertarget{fga-2-theorem-9-corollary-4}{}}%
\begin{itenv}{Corollary 4}
Let \(X_0\) be the scheme of a complete simple algebraic curve over \(k\).
Then, for every complete local Noetherian ring \(A\) with residue field \(k\), there exists a ``simple curve scheme'' \(X\) over \(A\), such that \(X\otimes_Ak=X_0\).

\end{itenv}

\begin{rmenv}{Remarks}

---

\begin{enumerate}
\def\labelenumi{\arabic{enumi}.}
\item
  \protect\hyperlink{fga-2-theorem-9-corollary-3}{Corollary 3} and \protect\hyperlink{fga-2-theorem-9-corollary-4}{Corollary 4} are above all interesting if \(k\) is of characteristic \(p\neq0\), taking \(A\) to be a discrete valuation ring \emph{of characteristic \(0\)}, with residue field \(k\);
  for example, the ``smallest possible \(A\)'', i.e.~that for which \(p\) generates the maximal ideal.
  (In fact, by theorems of Cohen, it suffices to have \protect\hyperlink{fga-2-theorem-9-corollary-3}{Corollary 3} and \protect\hyperlink{fga-2-theorem-9-corollary-4}{Corollary 4} for such a ring \(A\)).
  We note that, concerning this point, according to the specialists, we do not know if there exist schemes over a field \(k\) that are not reductions \(\mod p\) of a flat scheme defined over such a ring \(A\).
  At the least, the results of this section give a way of systematically investigating this question.
  We must start by seeing if the first obstruction that we have, in \(H^2(X_0,{\mathfrak{G}}_{X_0/k})\), is necessarily zero.
\item
  \oldpage{182-15}We note that \protect\hyperlink{fga-2-theorem-3}{Theorem 3}, and the corresponding technique, only works for a \emph{complete} (local, for simplicity) base ring.
\end{enumerate}

In order to go from known results concerning the completion of a local ring to the corresponding results for the local ring itself, we would need a fourth ``fundamental theorem'', whose precise statement still needs to be found.

\begin{enumerate}
\def\labelenumi{\arabic{enumi}.}
\setcounter{enumi}{2}
\tightlist
\item
  We will compare the results from this section (mainly the above \protect\hyperlink{fga-2-theorem-9-corollary-1}{Corollary 1} and \protect\hyperlink{fga-2-theorem-9-corollary-2}{Corollary 2}), as well as those from the following, with the results of Kodaira--Spencer on the variation of complex structures.
  Using the conjectural theorem to which we have just alluded, we should be able to conclude, under the conditions of \protect\hyperlink{fga-2-theorem-9-corollary-1}{Corollary 1}, but where \(A\) is no longer assumed to be complete, that there exists a ring \(A'\) that contains \(A\), and that is finite and unramified over \(A\), such that \(X\otimes_AA'\) and \(X'\otimes_AA'\) are \(A'\)-isomorphic (where \(X\) and \(X'\) are given, and are proper flat \(A\)-schemes such that \(X\otimes_Ak=X'\otimes_Ak=X_0\)).
  This is what we can prove, at least, when \(X_0={\mathfrak{P}}_k^r\), by using \protect\hyperlink{fga-2-proposition-2-corollary-1}{Corollary 1 of Proposition 2}.
  In any case, if \(H^1(X_0,{\mathfrak{G}}_{X_0/k})=0\), then we can prove that the fibres of \(X\) and \(X'\) over any point \(y\) of \(Y=\operatorname{Spec}(A)\) are isomorphic, or at least when we pass to the algebraic closure of the residue field \(k(y)\).
  (We have a local result, seemingly stronger, when we don't suppose that \(A\) is necessarily local).
  As for ``structure variations'' of the projective space, we again point out the following question, suggested by a corresponding problem of Kodaira--Spencer.
  Let \(X\) be a proper flat scheme, over a local integral ring \(A\) with field of fractions \(K\) and residue field \(k\), and suppose that \(X\otimes_AK\) is isomorphic to \({\mathfrak{P}}_K^r\).
  Is it then true that \(X\otimes_Ak=X_0\) is isomorphic to \({\mathfrak{P}}_k^R\) (or at least, over the algebraic closure of \(k\))?
  In this question, we can assume that \(A\) is a complete discrete valuation ring.
  There is an analogous question when \(X_0\) is an abelian variety.
\end{enumerate}

\end{rmenv}

\begin{rmenv}{Remark}
\emph{{[}Comp.{]}}
\emph{(Concerning Remark 1 above).}
We note that J.-P. Serre has constructed in {[}\protect\hyperlink{ref-Ser1961}{24}{]} a non-singular projective variety, of dimension \(3\), over an algebraically closed field \(k\), of characteristic \(p>0\), which does not come from reduction of a proper scheme over a local integral ring with residue field \(k\), and having a field of fractions of characteristic \(0\).
Mumford would have found an analogous result, with a non-singular projective \emph{surface}.

\end{rmenv}

\begin{rmenv}{Remark}
\emph{{[}Comp.{]}}
\emph{(Concerning Remarks 2 and 3 above.)}
I am now less optimistic concerning the results conjectured here.
However, the question concerning structure variations for projective space, mentioned at the end of Remark 3 above, has been positively resolved by Hironaka, and the analogous question for abelian varieties has been resolved by Koizumi.

\end{rmenv}

\hypertarget{fga-2-section-7}{%
\subsection{Application to the ``theory of modules''}\label{fga-2-section-7}}

Since the speaker has only recently encountered this theory himself, we will be obliged to limit ourselves to just cursory remarks.
For simplicity, we work over a \emph{field} \(k\), i.e.~we work in \emph{equal characteristic}, even though \protect\hyperlink{fga-2-theorem-8}{Theorem 8} allows us to also discuss the more general case, without any fundamental changes, so it seems.
We have not yet gotten past the ``formal'' stage, but the speaker still hopes to be able to construct true schemes of modules in certain cases from this, and, in particular, construct, for every integer \(g\), a scheme over the integers that plays the role of universal scheme of modules for the simple curves of genus \(g\).

\begin{rmenv}{Remark}
\emph{{[}Comp.{]}}
\oldpage{182-16}We note that Mumford has recently constructed schemes of modules for the curves of genus \(g\) (cf.~\emph{Mumford--Tate seminar}, Harvard University, 1961--62).
\protect\hyperlink{fga-2-theorem-10}{Theorem 10} also proves that the ``level \(n\) Jacobi schemes'' from the theory of modules are non-singular (and even simple over \(\underline{\mathbb{Z}}\)).

\end{rmenv}

We continue to use the setting and notation of \protect\hyperlink{fga-2-theorem-9}{Theorem 9}, and now suppose that \(A\) is a local algebra of finite rank over \(k\), which is assumed to be algebraically closed, for simplicity.
Then \(F(A)\) can be thought of as a covariant functor in \(A\), with values in the category of sets, with a homomorphism \(A\to B\) of \(k\)-algebras defining a map \(F(A)\to F(B)\), since every flat \(A\)-scheme \(X\) with \(X\otimes_Ak=X_0\) gives rise to a \(B\)-scheme \(X\otimes_AB\) with the same properties.
\emph{Suppose} that we can find a complete local Noetherian \(k\)-algebra \({\mathcal{O}}\), as well as a functorial isomorphism
\[
  \operatorname{Hom}({\mathcal{O}},A) \xrightarrow{\sim} F(A)
  \tag{$*$}
\]
(where the left-hand side denotes homomorphisms of \(k\)-algebras).
We can easily see that such an \({\mathcal{O}}\) is determined up to canonical isomorphism, and so we call the formal spectrum \({\mathfrak{Y}}\) of \({\mathcal{O}}\) (i.e.~the topological space consisting of a single point, endowed with a sheaf of topological rings consisting of just \({\mathcal{O}}\)) the \emph{formal scheme of modules for \(X_0\)}.
(Note that it does not necessarily exist).
Let \({\mathfrak{r}}\) be the maximal ideal of \({\mathcal{O}}\), and, for all \(n\), let \({\mathscr{O}}_n={\mathcal{O}}/{\mathfrak{r}}^{n+1}\) (so that \({\mathscr{O}}_0=k\)).
Then the canonical homomorphism \({\mathcal{O}}\to{\mathscr{O}}_n\) is an element of \(\operatorname{Hom}({\mathcal{O}},{\mathscr{O}}_n)\), and thus defines an element of \(F({\mathscr{O}}_n)\), i.e.~a flat \({\mathscr{O}}_n\)-scheme \(X_n\) whose restriction \(\mod{\mathfrak{r}}\) is \(X_0\).
These \(X_n\) are induced from one another by extension of scalars (i.e.~here by reductions), whence it follows that they come from a formal scheme \({\mathfrak{X}}\) that is well determined by the formal scheme of modules \({\mathfrak{Y}}\);
further, \({\mathfrak{X}}\) is flat over \({\mathfrak{Y}}\), and \({\mathfrak{X}}_0=X_0\).
The isomorphism (\(*\)) is then given, as we can immediately see, by associating to each homomorphism \({\mathcal{O}}\to A\) of \(k\)-algebras the class of the \(A\)-scheme \({\mathfrak{X}}\otimes_{{\mathcal{O}}} A\) (i.e.~to every morphism \({\mathfrak{Y}}'=\operatorname{Spec}(A)\to{\mathfrak{Y}}\) of \(k\)-schemes, we associate the \({\mathfrak{Y}}'\)-scheme \({\mathfrak{X}}\otimes_{{\mathfrak{Y}}}{\mathfrak{Y}}'\) given by base change).
Furthermore, we see that the isomorphism (\(*\)) and its above description still hold even if we only suppose that \(A\) is a complete local Noetherian \(k\)-algebra (not necessarily Artinian).
Of course, as always, \({\mathcal{O}}\) can indeed \emph{a priori} have nilpotent elements, and it seems likely that there should exist cases where \({\mathcal{O}}\) is itself Artinian, without being identical to \(k\).
This tells us at which point the point of view of Kodaira--Spencer (restricting to considering the \(A\) that are \emph{regular} rings) is \emph{a priori} inadequate in the general case.

\oldpage{182-17}It remains to give sufficient conditions for there to exist a formal scheme of modules for \(X_0\), assumed to be proper over \(k\).
Generally, it is easy to give simple necessary and sufficient conditions on a functor \(A\to F(A)\) (from local \(k\)-algebras of finite rank to sets) in order for it to be of the form \(\operatorname{Hom}({\mathcal{O}},A)\) for some suitable \({\mathcal{O}}\).
We do not give the details here.
We point out only that, in the case which we are studying, these conditions impose non-trivial conditions of a cohomological nature on \(X_0\), and it seems unlikely that they will always be satisfied, even though the speaker has not constructed any counterexamples.
It seems plausible, however, that the condition \(H^0(X,{\mathfrak{G}}_{X_0/k})=0\) is \emph{sufficient} (even if not at all necessary) in order to guarantee the existence of a formal scheme of modules.
We restrict ourselves to stating here a theorem that deals with a particularly simple case (whose analogue in the theory of analytic spaces is well known, cf.~Kodaira--Spencer), which can easily be proven using the results from the previous section:

\leavevmode\vadjust pre{\hypertarget{fga-2-theorem-10}{}}%
\begin{itenv}{Theorem 10}
Let \(X_0\) be a simple proper scheme over the field \(k\) such that
\[
  H^0(X_0,{\mathfrak{G}}_{X_0/k}) = H^2(X_0,{\mathfrak{G}}_{X_0/k}) = 0.
\]
Then there exists a formal scheme of modules for \(X_0\), corresponding to a local regular ring \({\mathcal{O}}\) (i.e.~an algebra of formal series over \(k\)).

\end{itenv}

As we have already pointed out, it is not true in general that the formal scheme \({\mathfrak{X}}\) over \({\mathcal{O}}\) is algebraisable;
but we know that this is true, however, when \(X_0\) is projective and \(H^2(X_0,{\mathscr{O}}_{X_0})=0\) (\protect\hyperlink{fga-2-theorem-4}{Theorem 4}), such as when \(X_0\) is of dimension \(1\).
This is what gives some hope of constructing a scheme of modules over the integers for curves of a given genus\ldots{}

Note also that methods such as those described in this section can be applied in the construction and study of Picard varieties, as well as in many other constructions.
We will return to this soon.

\hypertarget{fga-2-section-8}{%
\subsection{Application to the fundamental group}\label{fga-2-section-8}}

The techniques described allows us to tackle the system study of the fundamental group, using the example of topological theory.
The first two theorems stated in this section are generalisations of results in a recent work by Lang--Serre.

Let \(X\) be a scheme.
\oldpage{182-18}Then an \(X\)-scheme \(X'\) is said to be an \emph{unramified covering of \(X\)} if

\begin{enumerate}
\def\labelenumi{\roman{enumi}.}
\item
  \(X'\) is finite over \(X\), i.e.~it is defined by a coherent sheaf of algebras \({\mathscr{A}}={\mathscr{A}}(X')\) on \(X\);
\item
  \({\mathscr{A}}\) is a locally free sheaf on \(X\);
\item
  for all \(x\in X\), the quotient \({\mathscr{A}}_x/{\mathfrak{m}}_x{\mathscr{A}}_x = {\mathscr{A}}_x\otimes_{{\mathscr{O}}_X}\pi(x)\) is a separable algebra over \(k(x)\).
\end{enumerate}

This notion of unramified covering (due to Serre and the speaker) posses all the elementary properties for which we can reasonably hope, and which we will not list.
We restrict ourselves to saying that it gives rise to a \emph{Galois theory} modelled on classical Galois theory (and containing it; the proofs being overall simpler than the proofs generally seen for the latter) and the Galois theory of topological coverings.
More precisely, we define a \emph{geometric point} of a scheme \(X\) to be a morphism \(a\) from the spectrum \(\xi\) of an algebraically closed field \(\Omega\) to \(X\), i.e.~the data of an algebraically closed extension of the residue field \(k(x)\) of a point \(x=|a|\) of \(X\) (called the \emph{locality} of the geometric point \(a\)).
If \(X'\) is an unramified covering of \(X\), then we can associate to it the set \(E_a(X')\) of ``geometric points of \(X'\) over \(a\)'', i.e.~the set of pairs consisting of an \(x'\in X'\) over \(x\) and a \(k(x)\)-homomorphism to \(\Omega\).
We thus obtain (for fixed \((X,a)\)) a functor \(F(X,a)\) from the category \(R(X)\) of unramified coverings \(X'\) of \(X\) to the category of finite sets.
If \(X\) is connected, then the pair given by \(R(X)\) and \(F(X,a)\) has all the formal properties necessary in order to be isomorphic to the analogous pair defined by a suitable totally disconnected compact topological group \(\pi\) (i.e.~a projective limit of finite groups): we take the category \({\mathcal{C}}(\pi)\) of finite sets on which \(\pi\) acts continuously, and the identity functor \(F(\pi)(E)=E\) from this category to the category of finite sets.
The group \(\pi\) is also determined up to canonical isomorphic by the condition that \(({\mathcal{C}}(\pi),F(\pi))\) is isomorphic to a given pair.
To be precise, \(\pi\) is called the \emph{fundamental group of the connected scheme \(X\) at the geometric point \(a\)}, and we denote it by \(\pi_1(X,a)\).
If \(X\) is not connected, then we can replace it by the connected component containing \(x=|a|\).
If, however, \(X\) is connected, then the groups \(\pi_1(X,a')\) and \(\pi_1(X,a')\) are isomorphic for any two geometric points \(a'\) and \(a''\) of \(X\) (with the isomorphism being determined up to inner automorphism), and thus we can, as per usual, choose the most suitable \(a\) for our purposes, such as the generic point of \(X\) that is assumed to be irreducible.
\oldpage{182-19}Of course, \(\pi_1(X,a)\) is a \emph{covariant functor in the pointed scheme \((X,a)\)}.
Every statement concerning the classification of inseparable coverings can then be translated into the language of group theory, following the well-known dictionary (except that we must take into account the fact that here we have \emph{topological groups}).

Our goal is to develop an analogue of the homotopy exact sequence of fibre bundles, relative to a proper morphism \(f\colon X\to Y\).
Clearly, since we don't know what the higher homotopy groups are, we will only have necessarily incomplete results.
In order to be able to apply the fundamental theorems from \protect\hyperlink{fga-2-section-3}{§3}, we must first explain certain elementary lemmas concerning schemes over Artinian rings or fields (following the general procedure!).

\leavevmode\vadjust pre{\hypertarget{fga-2-lemma-1}{}}%
\begin{itenv}{Lemma 1}
Let \((X',a')\) be a pointed unramified covering associated to a pointed representation of \(\pi_1(X,a)\) in a finite set \(E\) (endowed with a marked point \(e\)),
Then the canonical morphism \(\pi_1(X',a')\to\pi_1(X,a)\) identifies the domain with the stabiliser of \(e\) in \(\pi_1(X,a)\) (and is thus injective).

\end{itenv}

\leavevmode\vadjust pre{\hypertarget{fga-2-lemma-2}{}}%
\begin{itenv}{Lemma 2}
Let \(X\) be an algebraic scheme over the field \(k\), and let \(k'\) be a radicial extension of \(k\).
Then every unramified covering of \(X\otimes_kk'\) is given by the inverse image (i.e.~extension of scalars) of an unramified covering of \(X\), determined up to isomorphism.

\end{itenv}

It follows, in particular, from these two lemmas that, for \emph{every} algebraic extension \(K\) of \(k\), and every geometric point \(a'\) of \(X'=X\otimes_kK\) that projects to the geometric point \(a\) of \(X\), that the functorial homomorphism \(\pi_1(X',a')\to\pi_1(X,a)\) is injective.

\leavevmode\vadjust pre{\hypertarget{fga-2-lemma-3}{}}%
\begin{itenv}{Lemma 3}
Let \(X\) be a complete scheme over a local Artinian ring \(A\), such that \(H^0(X,{\mathscr{O}}_X)=A\).
Let \(X'\) be an unramified covering of \(X\), and let \(A'=H^0(X',{\mathscr{O}}_{X'})\), which is thus a ring that is finite over \(A\) (and which may a priori be ramified over \(A\)).
Let \(X_0\) and \(X'_0\) be the reduced subschemes associated to \(X\) and \(X'\), respectively (obtained by splitting by the sheaves of nilpotent elements in \({\mathscr{O}}_X\) and \({\mathscr{O}}_{X'}\), respectively).
Let \(k\) be a subfield of \(A/{\mathfrak{r}}(A)\) over which \(A/{\mathfrak{r}}(A)\) is finite (so \(X_0\) is a complete algebraic scheme over \(k\), and \(X'_0\) is an unramified covering).
Finally, let \(\Omega\) be an algebraically closed extension of \(k\), and consider the unramified covering \(X'_0\otimes_k\Omega\) of \(X_0\otimes_k\Omega\).
\oldpage{182-20}Then the following two conditions are equivalent:

\begin{enumerate}
\def\labelenumi{\roman{enumi}.}
\tightlist
\item
  \(X'_0\otimes_k\Omega\) is completely decomposed over \(X_0\otimes_k\Omega\);
\item
  the natural morphism \(X'\to X\otimes_AA'\) is an isomorphism.
\end{enumerate}

Under these conditions, \(A'\) is an \emph{unramified} extension of \(A\).
Finally, if \(X'\) is connected, then condition (i) is equivalent to the following, seemingly weaker, condition:

i bis. \(X'_0\otimes_k\Omega\) admits a regular section over \(X_0\otimes_k\Omega\).

\end{itenv}

When condition (ii) is satisfied, we say that the unramified covering \(X'\) of \(X\) is \emph{geometrically trivial}.

\leavevmode\vadjust pre{\hypertarget{fga-2-lemma-4}{}}%
\begin{itenv}{Lemma 4}
Let \(f\colon X\to Y\) be a proper morphism such that \(f_*({\mathscr{O}}_X)={\mathscr{O}}_Y\).
Let \(a\) be a geometric point of \(X\), and \(b\) its projection over \(Y\).
Then \(\pi_1(X,a)\to\pi_1(Y,b)\) is \emph{surjective}.

\end{itenv}

What we need to show is effectively the following: if an unramified covering \(Y'\) of \(Y\) (corresponding to a locally free sheaf of algebras \({\mathscr{A}}\)) is such that \(X\otimes_YY'\) is disconnected, then \(Y'\) is also disconnected.
In fact, \({\mathscr{A}}\) is then the direct sum of two non-zero sheaves of rings, and thus so too is its direct image, which is exactly \({\mathscr{A}}\otimes f_*({\mathscr{O}}_X)={\mathscr{A}}\).

\leavevmode\vadjust pre{\hypertarget{fga-2-lemma-5}{}}%
\begin{itenv}{Lemma 5}
Let \(X\) be a complete scheme over a field \(k\), and suppose that \(H^0(X,{\mathscr{O}}_X)\) is a local ring \(A\), and that \(A/{\mathfrak{r}}(A)\) is radicial over \(k\).
Let \(\Omega\) be an algebraic closure of \(k\), and let \(\overline{X}=X\otimes_k\Omega\) (which is connected).
Pick a geometric point \(\overline{a}\) of \(\overline{X}\) that projects to the geometric point \(a\) of \(X\).
Then we have an exact sequence
\[
  e
  \to \pi_1(\overline{X},\overline{e})
  \to \pi_1(X,a)
  \to \pi_1(k,b)
  \to e
\]
(where \(\pi_1(k,b)\) is the Galois group of \(\Omega\) over \(k\)).

\end{itenv}

The fact that the first homomorphism is injective has already been shown with \protect\hyperlink{fga-2-lemma-1}{Lemma 1} and \protect\hyperlink{fga-2-lemma-2}{Lemma 2};
the exactness in the middle follows from \protect\hyperlink{fga-2-lemma-3}{Lemma 3};
finally, the surjectivity of the last homomorphism (which is the only thing to rely on the fact that \(A/{\mathfrak{r}}(A)\) is radicial) follows from \protect\hyperlink{fga-2-lemma-4}{Lemma 4}.

\leavevmode\vadjust pre{\hypertarget{fga-2-proposition-4}{}}%
\begin{itenv}{Proposition 4}
Let \(f\colon X\to Y\) be a proper flat morphism such that, for all \(y\in Y\), the algebra \(H^0(f^{-1}(y),{\mathscr{O}}_{f^{-1}(y)})\) is separable over the residue field \(k(y)\) (which is the case, for example, if \(f^{-1}(y)\) is a \emph{separable scheme} over \(k(y)\), i.e.~reduced and such that the fields corresponding to its irreducible components are separable extensions of \(k(y)\)).
\oldpage{182-21}Then the covering \(Y'\) of \(Y\) associated to \(f_*({\mathscr{O}}_X)\) is unramified.

\end{itenv}

The proof is easy, thanks to \protect\hyperlink{fga-2-theorem-2}{Theorem 2}.

This proposition, combined with \protect\hyperlink{fga-2-lemma-1}{Lemma 1}, practically reduces the homotopical study of proper and flat morphisms (with separable fibres) to the case where \(f_*({\mathscr{O}}_X)={\mathscr{O}}_Y\) (since, using Stein factorisation, we can replace \(Y\) by \(Y'\)).

\begin{rmenv}{Remark}
A flat morphism of finite type whose fibres are separable (resp. simple) schemes is said to be \emph{separable} (resp. \emph{simple}).
We show that, if \(f\) is flat and if \(f^{-1}(y)\) is separable (resp. simple) then there exists a neighbourhood of \(f^{-1}(y)\) on which \(f\) is separable (resp. simple).
The same result holds true for ``absolutely normal'' (this is \emph{Bertini's theorem}).

\end{rmenv}

Let \(f\colon X\to Y\) be a proper morphism such that

\leavevmode\vadjust pre{\hypertarget{fga-2-equation-i}{}}%
\[
  f_*({\mathscr{O}}_X) = {\mathscr{O}}_Y
\tag{i}
\]

and let \(X'\) be a finite scheme over \(X\).
Let \(Y'\) be the covering of \(Y\) corresponding to the Stein factorisation of \(X'\to Y\) (cf.~\protect\hyperlink{fga-2-theorem-5}{Theorem 5}).
Let \(y\in Y\), so that the set of connected components of the fibre \(F'\) of \(X'\) over \(y\) can be identified with the set of points \(y'\in Y'\) over \(y\) (\protect\hyperlink{fga-2-theorem-5}{Theorem 5}).
Consider the evident morphism

\leavevmode\vadjust pre{\hypertarget{fga-2-equation-asterisk}{}}%
\[
  X'\to X\times_Y Y'
\tag{$*$}
\]

induced by the natural morphisms \(X'\to X\) and \(X\to Y'\);
this will be an isomorphism whenever \(X'\) is of the form \(X\times_Y Y''\), where \(Y''\) is an unramified covering of \(Y\), and then \(Y'\) will be exactly \(Y''\), and \protect\hyperlink{fga-2-equation-asterisk}{Equation (\(*\))} will be the identity.
We wish to precisely give the conditions for which \(X'\) is of the form that we have just indicated, i.e.~such that \(Y'\) is unramified and \protect\hyperlink{fga-2-equation-asterisk}{Equation (\(*\))} is an isomorphism.
For this, we introduce the fibre \(F\) of \(X\) at \(y\), which is a proper scheme over \(k(y)\), for which \(F'\) is a cover (an unramified one if \(X\) is).
Let \(F'_1\) be a connected component of \(F'\) corresponding to a point \(y'_1\) of \(Y'\) over \(y\).
Suppose further that

\begin{enumerate}
\def\labelenumi{\roman{enumi}.}
\setcounter{enumi}{1}
\tightlist
\item
  \(X'\) is unramified over \(X\) at the points of \(F'_1\) (and thus \(F'_1\) is an unramified cover of \(F\)), and
\item
  \(F'_1\) is a geometrically trivial covering of \(F\) (cf.~\protect\hyperlink{fga-2-lemma-3}{Lemma 3}).
\end{enumerate}

\leavevmode\vadjust pre{\hypertarget{fga-2-theorem-11}{}}%
\begin{itenv}{Theorem 11}
Under these conditions, there exists an open neighbourhood \(U'\) of \(y'_1\) in \(Y'\) such that \protect\hyperlink{fga-2-equation-asterisk}{Equation (\(*\))} is an isomorphism over \(U'\).
\oldpage{182-22}Furthermore, \(Y'\) is unramified at \(y'_1\) over \(Y\) (but can be ramified at other points \(y'\) of \(Y'\) over \(y\)).

\end{itenv}

Of course, conditions (ii) and (iii) are also necessary for the conclusion of the theorem.
The proof of the theorem is easy, thanks to \protect\hyperlink{fga-2-lemma-3}{Lemma 3} and \protect\hyperlink{fga-2-theorem-2}{Theorem 2}.

\leavevmode\vadjust pre{\hypertarget{fga-2-theorem-11-corollary-1}{}}%
\begin{itenv}{Corollary 1}
Suppose that (i) is still satisfied.
For an unramified covering over \(X\) to be isomorphic to the inverse image of an unramified covering \(Y'\) of \(Y\), it is necessary and sufficient that \(X'\) induce, on each fibre \(f^{-1}(y)\), a geometrically trivial covering.

\end{itenv}

By \protect\hyperlink{fga-2-theorem-11}{Theorem 11}, the set of points of \(Y\) for which this condition is satisfied is open, and so it suffices to verify it at the points \(y\) which are closed\ldots{}
Note that the following statement is equivalent to \protect\hyperlink{fga-2-theorem-11-corollary-1}{Corollary 1}:

\emph{The kernel of the homomorphism \(\pi_1(X)\to\pi_1(Y)\) (which is surjective, by \protect\hyperlink{fga-2-lemma-4}{Lemma 4}) is the closed invariant subgroup generated by the images in \(\pi_1(X)\) of \(\pi_1(f^{-1}(y))\), where \(f^{-1}(y)\) denotes the scheme \(f^{-1}(y)\otimes_{k(y)}\overline{k(y)}\) (where \(\overline{k(y)}\) denotes an algebraic closure of \(k(y)\)).}

We note that, since we cannot \emph{choose} the same base point for all the fibres, the homomorphisms \(\pi_1(f^{-1}(y))\to\pi_1(X)\) are determined (after having picked a base point for \(X\), and then for \(Y\)) only up to composition with an inner automorphism of \(\pi_1(X)\).

\leavevmode\vadjust pre{\hypertarget{fga-2-theorem-11-corollary-2}{}}%
\begin{itenv}{Corollary 2}
Under the general conditions of \protect\hyperlink{fga-2-theorem-11}{Theorem 11}, suppose further that \(Y\), \(X\), and \(X'\) are integral, and let \(K\), \(L\), and \(L'\) be their fields (respectively).
Then there exists a separable sub-extension \(K'\) of \(K\) in \(L'\), linearly disjoint from \(L\), such that \(L'=LK'\) (whence \(L'=L\otimes_KK'\)).

\end{itenv}

(We apply the last part of \protect\hyperlink{fga-2-lemma-3}{Lemma 3} to the generic fibre of \(X\)).
The most interesting case in which we can apply \protect\hyperlink{fga-2-theorem-11}{Theorem 11} is when \(f\) is a \emph{separable} morphism.
Then \(X'\) is also separable over \(Y\), and so, by \protect\hyperlink{fga-2-proposition-4}{Proposition 4}, \(Y'\) is unramified over \(Y\), and so the right-hand side \(X\times_YY'\) in \protect\hyperlink{fga-2-equation-asterisk}{Equation (\(*\))} is unramified over \(X\).
From this, we easily conclude:

\leavevmode\vadjust pre{\hypertarget{fga-2-theorem-11-corollary-3}{}}%
\begin{itenv}{Corollary 3}
Suppose, in addition to (i), that \(f\) is separable.
Let \(X'\) be a connected unramified covering of \(X\).
For \(X\) to be the inverse image of an unramified covering \(Y'\) of \(Y\), it is necessary and sufficient that the induced covering \(\overline{F}'\) of a geometric fibre \(\overline{F}=\overline{f^{-1}(y)}\) admit a regular section.

\end{itenv}

Note that it was not necessary to suppose that \(\overline{F}'\) be \emph{geometrically} trivial over \(\overline{F}\) (which will be true \emph{a posteriori}, even though \emph{a priori} this condition is a lot stronger).
\oldpage{182-23}\protect\hyperlink{fga-2-theorem-11-corollary-3}{Corollary 3} is equivalent to the following statement:

\leavevmode\vadjust pre{\hypertarget{fga-2-theorem-11-corollary-4}{}}%
\begin{itenv}{Corollary 4}
Let \(f\colon X\to Y\) be a proper and separable morphism such that \(f_*({\mathscr{O}}_X)={\mathscr{O}}_Y\).
Let \(\overline{F}\) be the geometric fibre of a point \(y\in Y\), and pick a geometric point in \(\overline{F}\), which, by the morphisms \(\overline{F}\to X\to Y\), gives geometric points in \(X\) and \(Y\); we take these three points as base points for the fundamental groups of \(\overline{F}\), \(X\), and \(Y\), respectively.
Under these conditions, we have the exact sequence
\[
  \boxed{\pi_1(\overline{F}) \to \pi_1(X) \to \pi_1(Y) \to 0.}
\]

\end{itenv}

From this, we easily deduce the two following statements of Serre--Lang, with all normality hypotheses removed:

\leavevmode\vadjust pre{\hypertarget{fga-2-theorem-11-corollary-5}{}}%
\begin{itenv}{Corollary 5}
Let \(X\) and \(Y\) be connected schemes over a field \(k\), with \(X\) or \(Y\) proper over \(k\), and suppose that the reduced scheme \(X_\mathrm{red}\) is separable over \(k\) (which is automatically true if \(k\) is perfect) and complete.
Pick a geometric point \(a\) (resp. \(b\)) in \(X\) (resp. \(Y\)); this gives a geometric point \(c=(a,b)\) in \(X\times_kY\), and a natural morphism
\[
  \pi_1(X\times_kY,c) \to \pi_1(X,a)\times\pi_1(Y,b)
\]
(induced by the functorial morphisms from \(\pi_1(X\times Y,c)\) to \(\pi_1(X,a)\) and \(\pi_1(Y,b)\)).
This morphism is injective, and further bijective if \(k\) is algebraically closed.

\end{itenv}

(The surjectivity in the above claim is almost trivial).
We thus deduce, with Serre--Lang:

\leavevmode\vadjust pre{\hypertarget{fga-2-theorem-11-corollary-6}{}}%
\begin{itenv}{Corollary 6}
Let \(X\) be a connected algebraic scheme over an algebraically closed field \(k\), and let \(K\) be an algebraically closed extension of \(k\).
Then the fundamental groups of \(X\) and \(X\times_kK\) are the same, i.e.~every unramified covering of the latter scheme is given by extension of scalars of an unramified covering (which is unique up to isomorphism) of \(X\).

\end{itenv}

\begin{rmenv}{Remarks}

---

\begin{enumerate}
\def\labelenumi{\arabic{enumi}.}
\item
  Using \protect\hyperlink{fga-2-proposition-4}{Proposition 4}, we see that the hypothesis that \(f_*({\mathscr{O}}_X)={\mathscr{O}}_Y\) in \protect\hyperlink{fga-2-theorem-11-corollary-4}{Corollary 4} is not essential.
  In the general case, instead of putting the trivial group \(e\) after \(\pi_1(Y)\), one must continue by \(\pi_0(\overline{F})\to\pi_0(X)\to\pi_0(Y)\to e\),
  as in algebraic topology.
\item
  \oldpage{182-24}In general, we cannot say anything at the moment about the kernel of \(\pi_1(\overline{F})\to\pi_1(X)\), although it should involve a \(\pi_2(Y)\).
  It seems, however, that we should be able to prove that \(\pi_1(\overline{F})\to\pi_1(X)\) is \emph{injective} if \(Y\) is the spectrum of a local ring \(A\), by appealing to \protect\hyperlink{fga-2-theorem-12}{Theorem 12} below (which shows that this is the case if \(A\) is \emph{complete}).
\end{enumerate}

\end{rmenv}

\protect\hyperlink{fga-2-theorem-11}{Theorem 11} used only \protect\hyperlink{fga-2-theorem-1}{Theorem 1} and \protect\hyperlink{fga-2-theorem-2}{Theorem 2};
we will now use \protect\hyperlink{fga-2-theorem-3}{Theorem 3}, along with the following elementary lemma:

\leavevmode\vadjust pre{\hypertarget{fga-2-lemma-6}{}}%
\begin{itenv}{Lemma 6}
Let \(X\) be a scheme, and \(X_0\) the corresponding reduced scheme (i.e.~where we have killed all the nilpotent elements).
Then every unramified covering \(X'_0\) of \(X_0\) is induced by an unramified covering \(X'\) of \(X\), determined up to isomorphism.

\end{itenv}

This lemma, which is of a purely local nature, plays a role analogous to that of \protect\hyperlink{fga-2-theorem-8}{Theorem 8} here, in the theory of modules.
Combining it with the existence theorem (\protect\hyperlink{fga-2-theorem-3}{Theorem 3}), we obtain:

\leavevmode\vadjust pre{\hypertarget{fga-2-theorem-12}{}}%
\begin{itenv}{Theorem 12}
Let \(A\) be a complete local Noetherian ring with residue field \(k\).
Let \(X\) be a proper scheme over \(A\).
Then every unramified covering \(X'_0\) of \(X_0=X\otimes_Ak\) is induced by an unramified covering \(X'\) of \(X\), unique up to isomorphism.

\end{itenv}

In other words:

\leavevmode\vadjust pre{\hypertarget{fga-2-theorem-12-corollary-1}{}}%
\begin{itenv}{Corollary 1}
Pick a geometric point in \(X_0\) as the base point for the fundamental groups of \(X_0\) and \(X\).
Then the canonical homomorphism \(\pi_1(X_0)\to\pi_1(X)\) is an \emph{isomorphism}.

\end{itenv}

Applying \protect\hyperlink{fga-2-lemma-5}{Lemma 5} to \(X_0\) (supposing that \(H^0(X_0,{\mathscr{O}}_{X_0})=k\), for simplicity), and noting that, since \(A\) is complete, the unramified extensions of \(A\) correspond to unramified extensions of its residue field, i.e.~\(\pi_1(Y)=\pi_1(k)\) (where \(Y=\operatorname{Spec}(A)\)).
We obtain the exact sequence:
\[
  e \to \pi_1(\overline{X_0}) \to \pi_1(X) \to \pi_1(Y) \to e.
\]

\leavevmode\vadjust pre{\hypertarget{fga-2-theorem-12-corollary-2}{}}%
\begin{itenv}{Corollary 2}
Let \(f\colon X\to Y\) be a proper flat morphism, and let \(y_1\) be a point of \(Y\), and \(y_0\) a specialisation of \(y_1\).
Consider the corresponding ``geometric'' fibres \(\overline{X_1}\) and \(\overline{X_0}\), and suppose that \(\overline{X_0}\) is separable and connected (which implies that \(\overline{X_1}\) satisfies the same conditions).
Then we can find a group homomorphism \(\pi_1(X_1)\to\pi_1(X_0)\), defined up to inner automorphism.
Further, this homomorphism is \emph{surjective}.

\end{itenv}

\oldpage{182-25}We might hope that this homomorphism is always \emph{bijective}.
Unfortunately, this is not the case in general if \(k(y_0)\) is of characteristic \(>0\).
We will, however, obtain below a group containing the kernel of this homomorphism (at least in the case where \(\overline{X_0}\) is simple), implying that, if \(k(y_0)\) is of characteristic \(0\), then the above homomorphism is bijective (which is a result that we can also prove by transcendentality).
At the very least, we already have, in any case, a group containing \(\pi_1\), given by a special fibre, using the one given by a generic fibre.
Using, for example, the fact that an algebraic curve in characteristic \(p\) lifts to a curve in characteristic \(0\) (\protect\hyperlink{fga-2-theorem-9-corollary-4}{Corollary 4 of Theorem 9}), we obtain, by transcendentality:

\leavevmode\vadjust pre{\hypertarget{fga-2-theorem-12-corollary-3}{}}%
\begin{itenv}{Corollary 3}
Let \(X_0\) be the scheme of complete simple curve over an algebraically closed field of arbitrary characteristic, and let \(g\) be the genus of \(X_0\).
Then \(\pi_1(X_0)\) admits \(2g\) topological generators, related by the well-known relation.

\end{itenv}

We thus deduce, by a well-known technique using hyperplane sections:

\leavevmode\vadjust pre{\hypertarget{fga-2-theorem-12-corollary-4}{}}%
\begin{itenv}{Corollary 4}
Let \(X\) be a simple projective scheme over an algebraically closed field of arbitrary characteristic.
Then \(\pi_1(X)\) admits a finite number of topological generators.

\end{itenv}

We wish to describe the kernel of the homomorphism \(\pi_1(\overline{X_1})\to\pi_1(\overline{X_0})\).
For this, we can suppose that \(Y\) is the spectrum of a discrete complete valuation ring \(V={\mathscr{O}}_y\) (where \(y=y_0\)).
The question is the equivalent to the following:
given an unramified covering \(\overline{X'_1}\) of \(\overline{X_1}\) (which as can suppose to be Galois, if we wish), under which conditions must it come from an unramified covering of \(X_0\)?
A priori, the given covering comes, by extension of scalars, from an unramified covering \(X'_1\) of \(X_1\otimes_KK'\), where \(K'\) is a finite extension of the algebraic closure \(\overline{K}\) of the field of fractions \(K\) of \(V\);
if \(X'_1\) were Galois, of group \(G\), then we could choose \(X'_1\) to also be Galois of group \(G\).
Thus: \emph{for \(X'_1\otimes_{K'}\overline{K}=\overline{X'_1}\) to come from an unramified covering of \(X_0\), it is necessary and sufficient that there exist a finite extension \(K''\) of \(K'\) in \(\overline{K}\) such that \(X''_1=X'_1\otimes_{K'}K''\) is of the form \(X''\otimes_{V''}K''\), where \(V''\) is the normal closure of \(V\) in \(K''\), and where \(X''\) is an unramified covering of \(X\otimes_V V''\).}
Suppose, for example, that \(X_0\) is absolutely normal, whence \(X\otimes_V V''\) is normal (since it is flat over \(V''\) and has normal special fibre), and its field of functions is identical to \(K''(X_1)\), which is the field of functions of
\[
  X_1\otimes_K K'' = (X\otimes_VK)\otimes_KK'' = X\otimes_VK'' = (X\otimes_VV'')\otimes_{V''}K''.
\]
\oldpage{182-26}Let \(L''=K''(X'_1)\) be the field of functions of \(X'_1\otimes_{K'}K''\), which is a separable finite extension of \(K''(X_1)\), and the above condition also implies that \emph{\(L''\) is an unramified extension of the field of functions of \(X\otimes_VV''\)} (i.e.~the normalisation of \(X\otimes_V V''\) in \(L''\) is unramified over \(X\otimes_VV''\)).
It suffices to show that \(L''\) is unramified at the points of the special fibre of \(X\otimes_VV''\) (since it is unramified over the generic fibre \(X_1\otimes_KK''\)).
If \(X_0\) is now \emph{simple}, then it follows from the ``\emph{purity theorem}'' of Nagata--Zariski that it even suffices to show that \(L''\) is unramified over the local ring \({\mathscr{O}}''\) of the generic point of the normalisation of \(X\otimes_VV''\), which is a discrete valuation ring, equal to the normalisation in \(K''(X_1)\) of the local ring \({\mathscr{O}}\subset K(X_1)\) of the generic point of the special fibre of \(X\).
We thus obtain:

\leavevmode\vadjust pre{\hypertarget{fga-2-theorem-12-corollary-5}{}}%
\begin{itenv}{Corollary 5}
Under the above conditions, and with the above notation, for the unramified covering \(X'_1\otimes_{K'}\overline{K}\) of \(\overline{X_1}=X_1\otimes_K\overline{K}\) to come from an unramified covering of \(\overline{X_0}\), it is necessary and sufficient that there exist a finite sub-extension \(K''\) of \(\overline{K}/K'\) such that \(K''(X'_1)\) is unramified over the discrete valuation ring \({\mathscr{O}}''\subset K''(X_1)\).

\end{itenv}

Now note that \({\mathscr{O}}''\) is the normalisation in \(K''(X_1)\) of the discrete valuation ring \({\mathscr{O}}'\subset K'(X_1)\) (which is the normalisation of \({\mathscr{O}}\) in \(K'(X_1)\)), and that \({\mathscr{O}}'\) contains the normalisation \(V'\) of \(V\) in \(K'\), with a uniformiser \(u\) of \(V'\) being also a uniformiser of \({\mathscr{O}}'\).
Suppose now that \(X'_1\) is Galois, with Galois group \(G\) \emph{of order \(n\), coprime to the characteristic \(p\) of \(k(y_0)\)} (which is also the characteristic of the residue field of \({\mathscr{O}}'\)).
Then \(K'(X_1')\) is ``tamely ramified'' over \({\mathscr{O}}'\), from which it easily follows (via ``\emph{Abhyankar's lemma}'') that, if we adjoin an \(n\)-th root \(v\) of a uniformiser of \({\mathscr{O}}'\), then it becomes unramified over the normalisation of \({\mathscr{O}}'\) in \(K'(X_1)(v)\).
But we can take \(v\) to be an \(n\)-th root of a uniformiser of \(V'\), which shows that the condition of \protect\hyperlink{fga-2-theorem-12-corollary-5}{Corollary 5} is satisfied.
(This idea of using Abhyankar's lemma and the purity theorem was given to me by Serre).
To express the result we thus obtain, we introduce, for every totally disconnected compact group \(\pi\), the quotient \(\overline{\pi}\) of \(\pi\) by the closed subgroup generated by its Sylow \(p\)-sub-groups, i.e.~the projective limit of the discrete quotient groups of \(\pi\) that are of order coprime to \(p\).
With this notation, we obtain:

\leavevmode\vadjust pre{\hypertarget{fga-2-theorem-13}{}}%
\begin{itenv}{Theorem 13}
Let \(f\colon X\to Y\) be a proper flat morphism, \(y_1\) a point of \(Y\), and \(y_0\) a specialisation of \(y_1\).
Suppose that \(\overline{X_0}\) is connected and simple.
\oldpage{182-27}Then the homomorphism \(\overline{\pi_1}(\overline{X_1})\to\overline{\pi_1}(\overline{X_0})\) induced by the
surjective homomorphism from \protect\hyperlink{fga-2-theorem-12-corollary-2}{Corollary 2 of Theorem 12} is an \emph{isomorphism}.

\end{itenv}

In other words:

\leavevmode\vadjust pre{\hypertarget{fga-2-theorem-13-corollary-1}{}}%
\begin{itenv}{Corollary 1}
The classification of unramified Galois coverings, of Galois group of order coprime to the characteristic \(p\) of \(k(y_0)\), is the same for \(\overline{X_0}\) and for \(\overline{X_1}\).

\end{itenv}

In particular, if \(k(y_0)\) is of characteristic \(0\), then we see, algebraically, that \(\pi_1(\overline{X_1})\to\pi_1(\overline{X_0})\) is bijective.

We finally point out that the techniques utilised also give the following result, which is more general than \protect\hyperlink{fga-2-theorem-13}{Theorem 13}:

\leavevmode\vadjust pre{\hypertarget{fga-2-theorem-14}{}}%
\begin{itenv}{Theorem 14}
Let \(f\colon X\to Y\) be a proper simple morphism, and let \(D\) be a closed subscheme of \(X\) that is simple over \(Y\), and of codimension \(1\) at all points.
Given a fibre \(Z=f^{-1}(z)\) of \(f\), let \(Z'=Z\setminus Z\cap D\), and let \(\pi_1^\mathrm{t}(\overline{Z'})\) be the quotient of the fundamental group \(\pi_1(Z')\) that classifies the unramified coverings of \(\overline{Z'}\) that are ``tamely ramified'' over \(\overline{Z\cap D}\).
Let \(y_0\) and \(y_1\) be as in \protect\hyperlink{fga-2-theorem-13}{Theorem 13}.
Then there exists a \emph{surjective} homomorphism (defined up to inner automorphism) \(\pi_1^\mathrm{t}(\overline{X'_1})\to\pi_1^\mathrm{t}(\overline{X'_0})\), and the corresponding homomorphism \(\overline{\pi_1^\mathrm{t}}(\overline{X'_1})\to\overline{\pi_1^\mathrm{t}}(\overline{X'_0})\) is an isomorphism.

\end{itenv}

From this we obtain corresponding variants of the corollaries of \protect\hyperlink{fga-2-theorem-13}{Theorem 13}, and of \protect\hyperlink{fga-2-theorem-12-corollary-4}{Corollary 4 of Theorem 12}.
Similarly, using \protect\hyperlink{fga-2-theorem-9-corollary-3}{Corollary 3 of Theorem 9}, we obtain, transcendentally:

\leavevmode\vadjust pre{\hypertarget{fga-2-theorem-14-corollary-1}{}}%
\begin{itenv}{Corollary 1}
Let \(X_0\) be the scheme of a complete simple curve over an algebraically closed field of arbitrary characteristic, and let \(S=(s_i)_{1\leqslant i\leqslant n}\) be a finite subset of \(X_0\) with \(n\) elements.
Then \(\pi_1^\mathrm{t}(X_0\setminus S)\) admits \(2g+n\) topological generators, \(x_i,y_i,\sigma_j\) (for \(1\leqslant i\leqslant g\) and \(1\leqslant j\leqslant n\)), satisfying the relation
\[
  \left(\prod_i x_iy_ix_i^{-1}y_i^{-1}\right)\sigma_1\ldots\sigma_n = 1,
\]
where the \(\sigma_j\) are generators of the inertia groups corresponding to the \(s_j\).
For every finite group \(G\) \emph{of order coprime to the characteristic} that is generated by elements \(\overline{x_i},\overline{y_i},\overline{\sigma_j}\) satisfying the above relation, there exists an unramified Galois covering of \(X_0\setminus S\), of group \(G\), with inertia groups at the points \(s_j\) generated by the \(\overline{\sigma_j}\).

\end{itenv}

If \(X_0\) is of genus \(0\), and \(n=3\), then we have a solution to the ``\emph{three point problem}'', at least for Galois coverings of order coprime to the characteristic.
\oldpage{182-28}(Here, \protect\hyperlink{fga-2-theorem-9}{Theorem 9} is actually useless, and it seems that we can deduce the above corollary from the particular case in question in the three point problem).

\begin{rmenv}{Remarks}

---

\begin{enumerate}
\def\labelenumi{\arabic{enumi}.}
\item
  A more complete study, probably involving generalised Galois coverings of \(X\), \(X_0\), and \(X_1\) (of eventually infinitesimal Galois group), should allow one to recover the kernel in \protect\hyperlink{fga-2-theorem-12-corollary-2}{Corollary 2 of Theorem 12}.
  However, a study of coverings admitting ramifications that are not ``tame'' seems much more difficult.
\item
  \protect\hyperlink{fga-2-lemma-6}{Lemma 6}, combined with a result of Grauert concerning the formal completion of a non-singular projective scheme along a hyperplane section (or with the theorem, as yet unproven, mentioned in Remark 2 after \protect\hyperlink{fga-2-theorem-11}{Theorem 11}), would also allow us to prove, in ``abstract'' algebraic geometry, the classical \emph{Lefschetz theorem} on the fundamental group.
\end{enumerate}

\end{rmenv}

\hypertarget{part-descent-and-existence-theorems-in-algebraic-geometry}{%
\part*{Descent, and existence theorems in algebraic geometry}\label{part-descent-and-existence-theorems-in-algebraic-geometry}}
\addcontentsline{toc}{part}{Descent, and existence theorems in algebraic geometry}

\hypertarget{fga-3.i}{%
\section{Generalities, and descent by faithfully flat morphisms}\label{fga-3.i}}

\providecommand{\scr}[1]{{\mathscr{#1}}}
\renewcommand{\cal}[1]{{\mathcal{#1}}}
\renewcommand{\frak}[1]{{\mathfrak{#1}}}
\renewcommand{\geq}{\geqslant}
\renewcommand{\leq}{\leqslant}

\providecommand{\ourrar}[2]{\overset{#1}{\underset{#2}{\rightrightarrows}}}
\providecommand{\oullar}[2]{\overset{#1}{\underset{#2}{\leftleftarrows}}}
\providecommand{\id}{\operatorname{id}}
\providecommand{\Hom}{\operatorname{Hom}}
\providecommand{\shHom}{\mathscr{H}\kern -.5pt om}
\providecommand{\Aut}{\operatorname{Aut}}
\providecommand{\HH}{\operatorname{H}}
\providecommand{\RR}{\operatorname{R}}
\providecommand{\GL}{\operatorname{GL}}
\providecommand{\Ga}{\operatorname{G_a}}
\providecommand{\Gm}{\operatorname{G_m}}
\providecommand{\SL}{\operatorname{SL}}
\providecommand{\Sp}{\operatorname{Sp}}
\providecommand{\Spec}{\operatorname{Spec}}

{[}FGA 3.I{]}
Grothendieck, A.
``Technique de descente et théorèmes d'existence en géométrie algébrique, I: Généralités. Descente par morphismes fidèlement plats''.
\emph{Séminaire Bourbaki} \textbf{12} (1959--60), Talk no. 190.

\emph{{[}Trans.{]}}
Sections 3.1 to 3.4 were numbered A.1 to A.4 in the original;
sections 3.5 to 3.9 were numbered B.1 to B.5.

\begin{rmenv}{Remark}
\emph{{[}Comp.{]}}
For various details concerning the theory of descent, see also {[}\protect\hyperlink{ref-Gro1960b}{9}, VI, VII, and VIII{]}.

\end{rmenv}

\oldpage{190-01}From a technical point of view, the current article, and those that will follow, can be considered as variations on Hilbert's celebrated ``Theorem 90''.
The introduction of the method of descent in algebraic geometry seems to be due to A. Weil, under the name of ``descent of the base field''.
Weil considered only the case of separable finite field extensions.
The case of radicial extensions of height \(1\) was then studied by P. Cartier.
Lacking the language of schemes, and, more particularly, lacking nilpotent elements in the rings that were under consideration, the essential identity between these two cases could not have been formulated by Cartier.

Currently, it seems that the general technique of descent that will be explained (combined with, when necessary, the fundamental theorems of ``formal geometry'', cf. {[}\protect\hyperlink{ref-Gro1958a}{7}{]}) is at the base of the majority of existence theorems in algebraic geometry.\footnote{\emph{{[}Comp.{]}} It now seems excessive to say that the technique of descent is ``at the base of the majority of existence theorems in algebraic geometry''. This is true to a large extent for the non-projective techniques that are the object of study of the first two articles of this current series (i.e.~``Techniques of descent and existence theorems in algebraic geometry''), but not for the projective techniques (articles IV, V, and VI).}
It is worth noting as well that this aforementioned technique of descent can certainly be transported to ``analytic geometry'', and we can hope that, in the not-too-distant future, the specialists will know how to prove the ``analytic'' analogues of the existence theorems in formal geometry that will be given in article II.
In any case, the recent work of Kodaira--Spencer, whose methods seem unfit for defining and studying ``varieties of modules'' in the neighbourhood of their singular points, show reasonably clearly the necessity of methods that are closer to the theory of schemes (which should naturally complement transcendental techniques).

In the present article (namely article I) we will discuss the most elementary case of descent (the one indicated in the title).
The applications of \protect\hyperlink{fga-3-i-section-B.1-theorem-1}{Theorem 1}, \protect\hyperlink{fga-3-i-section-B.1-theorem-2}{Theorem 2}, and \protect\hyperlink{fga-3-i-section-B.1-theorem-3}{Theorem 3} below (in \protect\hyperlink{fga-3-i-section-B.1}{§B.1}) are, however, already vast in number.
We will restrict ourselves to giving only some of them as examples, without aiming for the maximum generality possible.

We will freely use the language of schemes, for which we refer to the already cited article, as well as {[}\protect\hyperlink{ref-GR1958}{4}{]}.
We make clear to point out, however, that the preschemes considered in this current article are not necessarily Noetherian, and that the morphisms are not necessarily of finite type.
\oldpage{190-02}So, if \(A\) is a local Noetherian ring, with completion \(\overline{A}\), then we will need to consider the non-Noetherian ring \(\overline{\overline{A}}\otimes_A\overline{A}\), as well as the morphisms of affine schemes that correspond to the inclusions between the rings in question.

\hypertarget{a.-preliminaries-on-categories}{%
\subsection*{\texorpdfstring{\textbf{A.} Preliminaries on categories}{A. Preliminaries on categories}}\label{a.-preliminaries-on-categories}}
\addcontentsline{toc}{subsection}{\textbf{A.} Preliminaries on categories}

\hypertarget{fga-3-i-section-A.1}{%
\subsection{\texorpdfstring{(A.1) Fibred categories, descent data, \({\mathcal{F}}\)-descent morphisms}{(A.1) Fibred categories, descent data, \{\textbackslash mathcal\{F\}\}-descent morphisms}}\label{fga-3-i-section-A.1}}

\hypertarget{fga-3-i-section-A.1.a}{%
\subsubsection{(a)}\label{fga-3-i-section-A.1.a}}

\hypertarget{fga-3-i-section-A.1-definition-1.1}{}
\begin{rmenv}{Definition 1.1}

A \emph{fibred category \({\mathcal{F}}\) with base \({\mathcal{C}}\)} (or \emph{over \({\mathcal{C}}\)}) consists of

\begin{itemize}
\item
  a category \({\mathcal{C}}\)
\item
  for every \(X\in{\mathcal{C}}\), a category \({\mathcal{F}}_X\)
\item
  for every \({\mathcal{C}}\)-morphism \(f\colon X\to Y\), a functor \(f^*\colon{\mathcal{F}}_Y\to{\mathcal{F}}_X\), which we also write as
  \[
    f^*(\xi) = \xi \times_Y X
  \]
  for \(\xi\in{\mathcal{F}}_Y\) (with \(X\) being thought of as an ``object of \({\mathcal{C}}\) over \(Y\)'', i.e.~as being endowed with the morphism \(f\))
\item
  for any two composible morphisms \(X\xrightarrow{f}Y\xrightarrow{g}Z\), an isomorphism of functors
  \[
    c_{f,g}\colon (gf)^* \to f^*g^*
  \]
\end{itemize}

with the above data being subject to the conditions that

\begin{enumerate}
\def\labelenumi{\roman{enumi}.}
\item
  \(\operatorname{id}^*=\operatorname{id}\)
\item
  \(c_{f,g}\) is the identity isomorphism if \(f\) or \(g\) is an identity isomorphism
\item
  for any three composible morphisms \(X\xrightarrow{f}Y\xrightarrow{g}Z\xrightarrow{h}T\), the following diagram, given by using the isomorphisms of the form \(c_{u,v}\), commutes:
  \[
      \begin{CD}
   (h(gf))^* @= ((hg)f)^*
      \\@VVV @VVV
      \\(gf)^*h^* @. f^*(hg)^*
      \\@VVV @VVV
      \\(f^*g^*)h^* @= f^*(g^*h^*)
      \end{CD}
    \]
\end{enumerate}

\end{rmenv}

\leavevmode\vadjust pre{\hypertarget{fga-3-i-section-A.1-example-1}{}}%
\begin{rmenv}{Example 1}
Let \({\mathcal{C}}\) be a category where all fibre products exist.
We then define a fibred category \({\mathcal{F}}\) with base \({\mathcal{C}}\) by setting \({\mathcal{F}}_X\) to be the category of objects of \({\mathcal{C}}\) over \(X\), and the functor \(f^*\colon{\mathcal{F}}_Y\to{\mathcal{F}}_X\) corresponding to a morphism \(f\colon X\to Y\) being defined by the \emph{fibre product} \(Z\rightsquigarrow Z\times_Y X\).

\end{rmenv}

\leavevmode\vadjust pre{\hypertarget{fga-3-i-section-A.1-example-2}{}}%
\begin{rmenv}{Example 2}
Let \({\mathcal{C}}\) be the category of preschemes, and, for \(X\in{\mathcal{C}}\), let \({\mathcal{F}}_X\) be the category of quasi-coherent sheaves of modules on \(X\).
If \(f\colon X\to Y\) is a morphism of preschemes, then \(f^*\colon{\mathcal{F}}_Y\to{\mathcal{F}}_X\) is the \emph{inverse image of sheaves of modules} functor.
\oldpage{190-03}We thus obtain a category fibred over \({\mathcal{C}}\).

\end{rmenv}

\hypertarget{fga-3-i-section-A.1.b}{%
\subsubsection{(b)}\label{fga-3-i-section-A.1.b}}

\leavevmode\vadjust pre{\hypertarget{fga-3-i-section-A.1-definition-1.2}{}}%
\begin{rmenv}{Definition 1.2}
A diagram of maps of sets
\[
  E \xrightarrow{u}
  E' \overset{v_1}{\underset{v_2}{\rightrightarrows}}
  E''
\]
is said to be \emph{exact} if \(u\) is a bijection from \(E\) to the subset of \(E'\) consisting of the \(x'\in E'\) such that \(v_1(x')=v_2(x')\).

\end{rmenv}

\leavevmode\vadjust pre{\hypertarget{fga-3-i-section-A.1-definition-1.3}{}}%
\begin{rmenv}{Definition 1.3}
Let \({\mathcal{F}}\) be a fibred category with base \({\mathcal{C}}\), and consider a diagram of morphisms in \({\mathcal{C}}\)
\[
  S \xleftarrow{\alpha}
  S' \overset{\beta_1}{\underset{\beta_2}{\leftleftarrows}}
  S''
\]
such that \(\alpha\beta_1=\alpha\beta_2\);
this diagram is said to be \emph{\({\mathcal{F}}\)-exact} if, for every pair \((\xi,\eta)\) of elements of \({\mathcal{F}}_S\), the diagram of sets

\leavevmode\vadjust pre{\hypertarget{fga-3-i-section-A.1-definition-1.3-equation}{}}%
\[
  \operatorname{Hom}(\xi,\eta) \xrightarrow{\alpha^*}
  \operatorname{Hom}(\alpha^*(\xi),\alpha^*(\eta)) \overset{\beta_1^*}{\underset{\beta_2^*}{\rightrightarrows}}
  \operatorname{Hom}(\gamma^*(\xi),\gamma^*(\eta))
\tag{+}
\]

(where \(\gamma=\alpha\beta_1=\alpha\beta_2\)) is exact.

In this above diagram, for simplicity, we have identified \(\beta_i^*\alpha^*\) with \((\alpha\beta_i)^*=\gamma^*\), using \(c_{\beta_i,\alpha}\).

\end{rmenv}

\leavevmode\vadjust pre{\hypertarget{fga-3-i-section-A.1-definition-1.4}{}}%
\begin{rmenv}{Definition 1.4}
Let \({\mathcal{F}}\) be a fibred category with base \({\mathcal{C}}\), and consider morphisms \(\beta_1,\beta_2\colon S''\to S'\) in \({\mathcal{C}}\).
Let \(\xi'\in{\mathcal{F}}_{S'}\).
We define a \emph{gluing data} on \(\xi'\) (with respect to the pair \((\beta_1,\beta_2)\)) to be an isomorphism from \(\beta_1^*(\xi')\) to \(\beta_2^*(\xi')\).
If \(\xi',\eta'\in{\mathcal{F}}_{S'}\) are both endowed with gluing data, then a morphism \(u\colon\xi'\to\eta'\) in \({\mathcal{F}}_{S'}\) is said to be \emph{compatible with the gluing data} if the following diagram commutes:
\[
  \begin{CD}
    \beta_1^*(\xi') @>>> \beta_2^*(\xi')
  \\@VVV @VVV
  \\\beta_1^*(\eta') @>>> \beta_2^*(\eta').
  \end{CD}
\]

\end{rmenv}

With this definition, the objects of \({\mathcal{F}}_{S'}\) that are endowed with gluing data (with respect to \(\beta_1\) and \(\beta_2\)) then form a \emph{category}.
\oldpage{190-04}If \(\alpha\colon S'\to S\) is a morphism such that \(\alpha\beta_1=\alpha\beta_2\), then, for every \(\xi\in{\mathcal{F}}_{S'}\), the object \(\xi'=\alpha^*(\xi)\) of \({\mathcal{F}}_{S'}\) is endowed with a canonical gluing data, since
\[
  \beta_i^*\alpha^*(\xi)
  \simeq (\alpha\beta_i)^*(\xi)
  = \gamma^*(\xi),
\]
where we again set \(\gamma=\alpha\beta_1=\alpha\beta_2\);
furthermore, if \(u\colon\xi\to\eta\) is a morphism in \({\mathcal{F}}_s\), then
\[
  \alpha^*(u)\colon \alpha^*(\xi) \to \alpha^*(\eta)
\]
is a morphism in \({\mathcal{F}}_{S'}\) that is compatible with the canonical gluing data.
We thus obtain a \emph{canonical functor} from the category \({\mathcal{F}}_S\) to the category of objects of \({\mathcal{F}}_{S'}\) endowed with gluing data with respect to the pair \((\beta_1,\beta_2)\).
With this, we can also rephrase \protect\hyperlink{fga-3-i-section-A.1-definition-1.3}{Definition 1.3} by saying that the diagram \protect\hyperlink{fga-3-i-section-A.1-definition-1.3-equation}{(+)} is \emph{\({\mathcal{F}}\)-exact} if the above functor is \emph{fully faithful}, i.e.~if the above functor defines an equivalence between the category \({\mathcal{F}}_S\) and a subcategory of the category of objects of \({\mathcal{F}}_S\) endowed with gluing data with respect to \((\beta_1,\beta_2)\).

\leavevmode\vadjust pre{\hypertarget{fga-3-i-section-A.1-definition-1.5}{}}%
\begin{rmenv}{Definition 1.5}
We say that a gluing data on \(\xi'\in{\mathcal{F}}_{S'}\) is \emph{effective} (with respect to \(\alpha\)) if \(\xi'\), endowed with this data, is isomorphic to \(\alpha^*(\xi)\) for some \(\xi\in{\mathcal{F}}_S\).

\end{rmenv}

In the case where the diagram \protect\hyperlink{fga-3-i-section-A.1-definition-1.3-equation}{(+)} is \({\mathcal{F}}\)-exact, the object \(\xi\) in \protect\hyperlink{fga-3-i-section-A.1-definition-1.5}{Definition 1.5} is then determined up to unique isomorphism, and \emph{the category \({\mathcal{F}}_S\) is equivalent to the category of objects of \({\mathcal{F}}_{S'}\) endowed with effective gluing data}.

\hypertarget{fga-3-i-section-A.1.c}{%
\subsubsection{(c)}\label{fga-3-i-section-A.1.c}}

The most important case is that where
\[
  S'' = S' \times_S S',
\]
with the \(\beta_i\) being the two projections \(p_1\) and \(p_2\) from \(S'\times_S S'\) to its two factors (where we now suppose that \({\mathcal{C}}\) has all fibre products).
We then have two immediate necessary conditions for a gluing data \(\varphi\colon p_1^*(\xi')\to p_2^*(\xi')\) on some \(\xi'\in{\mathcal{F}}_S\) to be effective:

\begin{enumerate}
\def\labelenumi{\roman{enumi}.}
\item
  \(\Delta^*(\varphi) = \operatorname{id}_\xi\), where \(\Delta\colon S'\to S'\times_S S'\) denotes the diagonal morphism, and where we identify \(\Delta^* p_i^*(\xi')\) with \((p_i\Delta)^*(\xi')=\xi'\).
\item
  \(p_{31}^*(\varphi) = p_{32}^*(\varphi)p_{21}^*(\varphi)\), where \(p_{ij}\) denotes the canonical projection from \(S'\times_S S'\times_S S'\) to the partial product of its \(i\)th and \(j\)th factors.
\end{enumerate}

\leavevmode\vadjust pre{\hypertarget{fga-3-i-section-A.1-definition-1.6}{}}%
\begin{rmenv}{Definition 1.6}
\oldpage{190-05}We define \emph{descent data} on \(\xi'\in{\mathcal{F}}_{S'}\), with respect to the morphism \(\alpha\colon S'\to S\), to be a gluing data on \(\xi'\) with respect to the pair \((p_1,p_2)\) of canonical projections \(S'\times_S S'\to S'\) that satisfies conditions (i) and (ii) above.

\end{rmenv}

\leavevmode\vadjust pre{\hypertarget{fga-3-i-section-A.1-definition-1.7}{}}%
\begin{rmenv}{Definition 1.7}
A morphism \(\alpha\colon S'\to S\) is said to be an \emph{\({\mathcal{F}}\)-descent morphism} if the diagram of morphisms
\[
  S \xleftarrow{\alpha}
  S' \overset{p_1}{\underset{p_2}{\leftleftarrows}}
  S'\times_S S'
\]
is \({\mathcal{F}}\)-exact (\protect\hyperlink{fga-3-i-section-A.1-definition-1.3}{Definition 1.3});
we say that \(\alpha\) is a \emph{strict \({\mathcal{F}}\)-descent morphism} if, further, every descent data (\protect\hyperlink{fga-3-i-section-A.1-definition-1.6}{Definition 1.6}) on any object of \({\mathcal{F}}_{S'}\) is effective.

This latter condition (of strictness) can also be stated in a more evocative way:
``giving an object of \({\mathcal{F}}_S\) is equivalent to giving an object of \({\mathcal{F}}_{S'}\) endowed with a descent data''.

\end{rmenv}

Note that, if an \({\mathcal{F}}\)-descent morphism\footnote{\emph{{[}Comp.{]}} It is useless to assume here that \(\alpha\) is an \({\mathcal{F}}\)-descent morphism.} \(\alpha\colon S'\to S\) admits a \emph{section} \(s\colon S\to S'\) (i.e.~a morphism \(s\) such that \(\alpha s=\operatorname{id}_S\)), then it is a strict \({\mathcal{F}}\)-descent morphism:
if \(\xi'\in{\mathcal{F}}_{S'}\) is endowed with descent data with respect to \(\alpha\), then ``it comes from'' \(\xi=s^*(\xi')\).

\hypertarget{fga-3-i-section-A.1.d}{%
\subsubsection{(d)}\label{fga-3-i-section-A.1.d}}

We can present the above notions in a more intuitive manner, by introducing, for an object \(T\) of \({\mathcal{C}}\) over \(S\), the set
\[
  \operatorname{Hom}_S(T,S') = S'(T),
\]
where elements will be denoted by \(t\), \(t'\), etc.
Given an object \(\xi'\in{\mathcal{F}}_{S'}\), there then corresponds, to every \(t\in S'(T)\), an object \(t^*(\xi')\) of \({\mathcal{F}}_T\), which will also be denoted by \(\xi'_t\).
A gluing data on \(\xi'\) (with respect to \((p_1,p_2)\)) is then defined by the data, for every \(T\) over \(S\), and every pair of points \(t,t'\in S'(T)\), of an isomorphism
\[
  \varphi_{t',t}\colon \xi'_t \to \xi'_{t'}
\]
(satisfying the evident conditions of functoriality in \(T\)).
Conditions (i) and (ii) of \protect\hyperlink{fga-3-i-section-A.1.c}{§A.1.c} can then be written as

i bis. \oldpage{190-06}\(\varphi_{t,t}=\operatorname{id}_{\xi'_t}\), for all \(T\) and all \(t\in S'(T)\).

ii bis. \(\varphi_{t,t''}=\varphi_{t,t'}\varphi_{t',t''}\), for all \(T\) and all \(t,t',t''\in S'(T)\).

We can show that (ii bis) implies that \(\varphi_{t,t}^2=\varphi_{t,t}\), by taking \(t=t'=t''\), and thus, since \(\varphi_{t,t}\) is an isomorphism by hypothesis, implies (i bis), which is thus a consequence of (ii bis) (and so (i) is also a consequence of (ii)).
But if we no longer suppose a priori that the \(\varphi_{t,t}\) are isomorphisms (i.e.~that \(\varphi\colon p_1^*(\xi')\to p_2^*(\xi')\) is an isomorphism), then (ii bis) no longer necessarily implies (i bis);
the combination of (ii bis) and (i bis), however, does imply that the \(\varphi_{t,t'}\) are isomorphisms (since we then have \(\varphi_{t,t'}\varphi_{t',t}=\varphi_{t,t}=\operatorname{id}_{\xi'_t}\)).

\hypertarget{fga-3-i-section-A.2}{%
\subsection{(A.2) Exact diagrams and strict epimorphisms, descent morphisms, and examples}\label{fga-3-i-section-A.2}}

\hypertarget{fga-3-i-section-A.2.a}{%
\subsubsection{(a)}\label{fga-3-i-section-A.2.a}}

\leavevmode\vadjust pre{\hypertarget{fga-3-i-section-A.2-definition-2.1}{}}%
\begin{rmenv}{Definition 2.1}
Let \({\mathcal{C}}\) be a category.
A diagram of morphisms
\[
  T \xrightarrow{\alpha}
  T' \overset{\beta_1}{\underset{\beta_2}{\rightrightarrows}}
  T''
\]
is said to be \emph{exact} if, for all \(Z\in{\mathcal{C}}\), the corresponding diagram of sets
\[
  \operatorname{Hom}(Z,T) \to
  \operatorname{Hom}(Z,T') \rightrightarrows
  \operatorname{Hom}(Z,T'')
\]
is exact (\protect\hyperlink{fga-3-i-section-A.1-definition-1.2}{Definition 1.2}).
We then say that \((T,\alpha)\) (or, by an abuse of language, \(T\)) is a \emph{kernel} of the pair \((\beta_1,\beta_2)\) of morphisms.

\end{rmenv}

This kernel is evidently determined up to unique isomorphism.
If \({\mathcal{C}}\) is the category of sets, then the above definition is compatible with \protect\hyperlink{fga-3-i-section-A.1-definition-1.2}{Definition 1.2}.
Dually, we define the exactness of a diagram of morphisms in \({\mathcal{C}}\)
\[
  S \xleftarrow{\alpha}
  S' \overset{\beta_1}{\underset{\beta_2}{\leftleftarrows}}
  S''
\]
and then say that \((S,\alpha)\) is a \emph{cokernel} of the pair \((\beta_1,\beta_2)\) morphisms.

\leavevmode\vadjust pre{\hypertarget{fga-3-i-section-A.2-definition-2.2}{}}%
\begin{rmenv}{Definition 2.2}
A morphism \(\alpha\colon S'\to S\) is said to be a \emph{strict epimorphism} if it is an epimorphism and, for every morphism \(u\colon S'\to Z\), the following necessary condition is also sufficient for \(u\) to factor as \(S'\to S\to Z\):
for every \(S''\in{\mathcal{C}}\) and every pair \(\beta_1,\beta_2\colon S''\to S\) of morphisms such that \(\alpha\beta_1=\alpha\beta_2\), we also have that \(u\beta_1=u\beta_2\).

\end{rmenv}

If the fibre product \(S'\times_S S'\) exists, then it is equivalent to say that the diagram
\[
  S \xleftarrow{\alpha}
  S' \overset{p_1}{\underset{p_2}{\leftleftarrows}}
  S'\times_S S'
\]
\oldpage{190-07}is exact, i.e.~that \(S\) is a cokernel of the pair \((p_1,p_2)\).
In any case, a cokernel morphism is a strict epimorphism.
Note also that, if a strict epimorphism is also a monomorphism, then it is an isomorphism.
We leave to the reader the task of developing the dual notion of a \emph{strict monomorphism}.

To make the relation between the ideas of \({\mathcal{F}}\)-descent morphisms and strict epimorphisms more precise, we introduce the following definitions:

\leavevmode\vadjust pre{\hypertarget{fga-3-i-section-A.2-definition-2.3}{}}%
\begin{rmenv}{Definition 2.3}
A morphism \(\alpha\colon S'\to S\) is said to be a \emph{universal epimorphism} (resp. a \emph{strict universal epimorphism}) if, for every \(T\) over \(S\), the fibre product \(T'=S'\times_S T\) exists, and the projection \(T'\to T\) is an epimorphism (resp. a strict epimorphism).

\end{rmenv}

In very nice categories (such as the category of sets, the category of sets over a topological space, abelian categories, etc.), the four notions of ``epijectivity'' introduced above all coincide;
they are, however, all distinct in a category such as the category of preschemes, or the category of preschemes over a given non-empty prescheme \(S\), even if we restrict to \(S\)-schemes that are finite over \(S\).

\leavevmode\vadjust pre{\hypertarget{fga-3-i-section-A.2-definition-2.4}{}}%
\begin{rmenv}{Definition 2.4}
A morphism \(\alpha\colon S'\to S\) is said to be a \emph{descent morphism} (resp. a \emph{strict descent morphism}) if it is an \({\mathcal{F}}\)-descent morphism (resp. a strict \({\mathcal{F}}\)-descent morphism) (cf.~\protect\hyperlink{fga-3-i-section-A.1-definition-1.7}{Definition 1.7}), where \({\mathcal{F}}\) denotes the fibred category over \({\mathcal{C}}\) of objects of \({\mathcal{C}}\) over objects of \({\mathcal{C}}\) (cf.~\protect\hyperlink{fga-3-i-section-A.1-example-1}{Example 1}).

\end{rmenv}

\leavevmode\vadjust pre{\hypertarget{fga-3-i-section-A.2-proposition-2.1}{}}%
\begin{rmenv}{Proposition 2.1}
If \({\mathcal{C}}\) has all finite products and (finite) fibre products, then there is an identity between descent morphisms in \({\mathcal{C}}\) and strict universal epimorphisms in \({\mathcal{C}}\).

\end{rmenv}

\hypertarget{fga-3-i-section-A.2.b}{%
\subsubsection{(b)}\label{fga-3-i-section-A.2.b}}

\begin{rmenv}{Example}
Let \({\mathcal{C}}\) be the category of preschemes.
Let \(S\in{\mathcal{C}}\), and let \(S'\) and \(S''\) be preschemes that are \emph{finite} over \(S\), i.e.~that correspond to sheaves of algebras \({\mathscr{A}}'\) and \({\mathscr{A}}''\) over \(S\) that are quasi-coherent (as sheaves of modules) and of finite type (i.e.~coherent if \(S\) is locally Noetherian).
Let \(\alpha\colon S'\to S\) be the structure morphism of \(S'\), and let \(\beta_1\) and \(\beta_2\) be \(S\)-morphisms from \(S''\) to \(S'\), defined by algebra homomorphisms \({\mathscr{A}}'\to{\mathscr{A}}''\), which we also denote by \(\beta_1\) and \(\beta_2\).
Using the fact that a finite morphism is closed (the first Cohen--Seidenberg theorem), we can easily prove that the diagram in \({\mathcal{C}}\)

\leavevmode\vadjust pre{\hypertarget{fga-3-i-section-A.2.b-equation}{}}%
\[
  S \xleftarrow{\alpha}
  S' \overset{\beta_1}{\underset{\beta_2}{\leftleftarrows}}
  S''
\tag{+}
\]

\oldpage{190-08}is exact if and only if the diagram of sheaves on \(S\)
\[
  {\mathscr{O}}_S = {\mathscr{A}} \xrightarrow{\alpha}
  {\mathscr{A}}' \overset{\beta_1}{\underset{\beta_2}{\rightrightarrows}}
  {\mathscr{A}}''
\]
is exact.
In particular, if \(\alpha\colon S'\to S\) is a finite morphism corresponding to a sheaf \({\mathscr{A}}'\) of algebras on \(S\), then \(\alpha\) is a strict epimorphism if and only if the diagram of sheaves
\[
  {\mathscr{O}}_S = {\mathscr{A}} \to
  {\mathscr{A}}' \overset{p_1}{\underset{p_2}{\rightrightarrows}}
  {\mathscr{A}}'\otimes_{{\mathscr{A}}}{\mathscr{A}}'
\]
is exact (it is an epimorphism if and only if \({\mathscr{A}}\to{\mathscr{A}}'\) is injective).
If \(S\) is affine of ring \(A\), then \(S'\) is affine of ring \(A'\), with \(A'\) finite over \(A\), and so \(S'\to S\) is a strict epimorphism if and only if \(A\to A'\) is an isomorphism from \(A\) to the subring of \(A'\) consisting of the \(x'\in A'\) such that
\[
  1_{A'}\otimes_A x' - x'\otimes_A 1_{A'} = 0
\]
(it is an epimorphism if and only if \(A\to A'\) is injective).
As we have already mentioned, even if \(S\) is the scheme of a local Artinian ring, then a finite morphism \(S'\to S\) that is an epimorphism is not necessarily a strict epimorphism.
However, we can prove that, \emph{if \(S\) is a Noetherian prescheme, then every finite morphism \(S'\to S\) that is an epimorphism is the composition of a finite sequence of strict epimorphisms} (also finite).
This also shows that the composition of two strict epimorphisms is not necessarily a strict epimorphism.

\end{rmenv}

\hypertarget{fga-3-i-section-A.2.c}{%
\subsubsection{(c)}\label{fga-3-i-section-A.2.c}}

If \protect\hyperlink{fga-3-i-section-A.2.b-equation}{(+)} is an exact diagram of finite morphisms, then, for every \emph{flat} morphism \(T\to S\) of prescheme, the diagram induced from \protect\hyperlink{fga-3-i-section-A.2.b-equation}{(+)} by a change of base \(T\to S\) is again exact.
It thus follows that, if \(X\) and \(Y\) are \(S\)-preschemes, with \(X\) \emph{flat} over \(S\), then the following diagram of maps (where \(X'\) and \(Y'\) are the inverse images of \(X\) and \(Y\) over \(S'\), and \(X''\) and \(Y''\) are their inverse images over \(S''\)) is exact:
\[
  \operatorname{Hom}_S(X,Y) \to
  \operatorname{Hom}_{S'}(X',Y') \rightrightarrows
  \operatorname{Hom}_{S''}(X'',Y'').
\]
In particular, if \({\mathcal{F}}\) denotes the fibred category (over the category \({\mathcal{C}}\) of preschemes) such that, for \(X\in{\mathcal{C}}\), \({\mathcal{F}}_X\) is the category of \emph{flat} \(X\)-preschemes, then the diagram \protect\hyperlink{fga-3-i-section-A.2.b-equation}{(+)} is \({\mathcal{F}}\)-exact.
(This result becomes false if we do not impose the flatness hypothesis; in particular, a finite strict epimorphism is not necessarily a descent morphism).
We similarly see that \protect\hyperlink{fga-3-i-section-A.2.b-equation}{(+)} is \({\mathcal{F}}\)-exact if \({\mathcal{F}}\) denotes the fibred category for which \({\mathcal{F}}_X\) is the category of \emph{flat} quasi-coherent sheaves on the prescheme \(X\) (here, again, the flatness hypothesis is essential).
\oldpage{190-09}In either case, the question of \emph{effectiveness} of a gluing data (and, more specifically, of a descent data, when \(S''=S'\times_S S'\)) on a flat object over \(S'\) is delicate (and its answer in many particular cases in one of the principal objects of these current articles).
The speaker does not know if, for every finite strict epimorphism \(S'\to S\), every descent data on a flat quasi-coherent sheaf on \(S'\) is effective (even if we suppose that \(S\) is the spectrum of a local Artinian ring, and we restrict to locally free sheaves of rank \(1\)).
More generally, let \(A\) be a ring, and \(A'\) an \(A\)-algebra (where everything is commutative) such that the diagram
\[
  A \to
  A' \rightrightarrows
  A'\otimes_A A'
\]
is exact, which is equivalent to saying that the corresponding morphism \(S'\to S\) between the spectra of \(A'\) and \(A\) is an \({\mathcal{F}}\)-descent morphism, where \({\mathcal{F}}\) is the fibred category of flat quasi-coherent sheaves.
Let \(M'\) be a flat \(A'\)-module endowed with a descent data to \(A\), i.e.~with an isomorphism
\[
  \varphi\colon M'\otimes_A A' \xrightarrow{\sim} A'\otimes_A M'
\]
of \((A'\otimes_A A')\)-modules that satisfies conditions (i) and (ii) of \protect\hyperlink{fga-3-i-section-A.1.c}{§A.1.c} (which we leave to the reader to write out in terms of modules).
Is this data effective (relative to the fibred category of flat quasi-coherent sheaves)?
Let \(M\) be the subset of \(M'\) consisting of the \(x'\in M'\) such that
\[
  \varphi(x'\otimes_A 1_{A'}) = 1_{A'}\otimes_A x',
\]
which is a sub-\(A\)-module of \(M'\).
The canonical injection \(M\to M'\) defines a homomorphism of \(A'\)-modules \(M\otimes_A A'\to M'\).
\emph{The effectiveness of \(\varphi\) then implies the following: \(M\) is a flat \(A\)-module, and the above homomorphism is an isomorphism.}

\begin{rmenv}{Remark}
In the above, we have imposed no flatness hypotheses on the morphisms of the diagram \protect\hyperlink{fga-3-i-section-A.2.b-equation}{(+)}, and this obliges us, in order to have a technique of descent, to impose flatness hypotheses on the objects over \(S\) and \(S'\) that we consider.
In \protect\hyperlink{fga-3-i-section-B.2}{§B.2}, we will impose a flatness hypothesis on \(\alpha\colon S'\to S\), which will allow us to have a technique of descent for objects over \(S\) and \(S'\) that are no longer under any flatness hypotheses.
In any case, there is a flatness hypothesis involved somewhere.
This is one of the main reasons for the importance of the notion of flatness in algebraic geometry (whose role could not be visible when we restricted to base \emph{fields}, over which everything, in fact, is flat!).

\end{rmenv}

\hypertarget{fga-3-i-section-A.3}{%
\subsection{(A.3) Application to étalements}\label{fga-3-i-section-A.3}}

Let \(A\) be a local ring, and \(B\) a local algebra over \(A\) whose maximal ideal induces that of \(A\).
\oldpage{190-10}We say that \(B\) is \emph{étalé} over \(A\) (instead of ``unramified'', as used elsewhere) if it satisfies the following conditions:

\begin{enumerate}
\def\labelenumi{\roman{enumi}.}
\tightlist
\item
  \(B\) is flat over \(A\); and
\item
  \(B/{\mathfrak{m}}B\) is a separable finite extension of \(A/{\mathfrak{m}}=k\) (where \({\mathfrak{m}}\) denotes the maximal ideal of \(A\)).
\end{enumerate}

If \(A\) and \(B\) are Noetherian, and \(k\) is algebraically closed, then this implies that the homomorphism \(\overline{A}\to\overline{B}\) between the completions that extends \(A\to B\) is an isomorphism.
A morphism \(f\colon T\to S\) of finite type is said to be \emph{étale at \(x\in T\)}, or \(T\) is said to be \emph{étalé over \(S\) at \(x\)}, if \({\mathscr{O}}_x\) is étalé over \({\mathscr{O}}_{f(x)}\);
\(f\) is said to be \emph{étale}, or an \emph{étalement}, or \(T\) is said to be \emph{étalé over \(S\)}, if \(f\) is étale at all \(x\in T\).
Note that, if \(S\) is locally Noetherian, then the set of points of \(T\) where \(f\) is étale is open, and Zariski's ``main theorem'' allows us to precisely state the structure of \(T/S\) in a neighbourhood around such a point (by an equation of well-known type).

If \(S\) is a scheme of finite type over the field of complex numbers, then there exists a corresponding analytic space \(\overline{S}\) (in the sense of Serre {[}\protect\hyperlink{ref-Ser1956}{19}{]}), except for the fact that \(\overline{S}\) can have nilpotent elements in its structure sheaf, which changes nothing essential in {[}\protect\hyperlink{ref-Ser1956}{19}{]}.
We then easily see that \(f\) is an étalement if and only if \(\overline{f}\colon\overline{T}\to\overline{S}\) is an étalement, i.e.~if every point of \(\overline{T}\) admits a neighbourhood on which \(\overline{f}\) induces an isomorphism onto an open subset of \(\overline{S}\).
In particular, to every \emph{étale covering} \(T\) of \(S\) (i.e.~every finite étale morphism \(f\colon T\to S\)), there is a corresponding étale covering \(\overline{T}\) of \(\overline{S}\), which is connected if and only if \(T\) is connected {[}\protect\hyperlink{ref-Ser1956}{19}{]}.
We can also easily see that, if \(T\) and \(T'\) are étale schemes over \(S\), then the natural map
\[
  \operatorname{Hom}_S(T,T') \to \operatorname{Hom}_{\overline{S}}(\overline{T},\overline{T}'')
\]
is bijective, i.e.~the functor \(T\mapsto\overline{T}\) from the category of étale schemes over \(S\) to the category of analytic spaces over \(S\) is ``fully faithful'', and thus defines an equivalence between the first category and a subcategory of the second.
A theorem of Grauert--Remmert {[}\protect\hyperlink{ref-GR1958}{4}{]} implies that, if \(S\) is normal, then we thus obtain an equivalence between the category of \emph{étale coverings} of \(S\) and the category of (\emph{finite}) étale coverings of \(S\), i.e.~every étale covering \({\mathscr{C}}\) of \(\overline{S}\) is \(\overline{S}\)-isomorphic to some \(\overline{T}\), where \(T\) is an étale covering of \(S\).
We will show that \emph{the Grauert--Remmert theorem remains true without any normality hypotheses on \(S\)}.
\oldpage{190-11}First let \(S'\to S\) be a finite strict epimorphism, and suppose that the theorem has been proven for \(S'\); we will show that it holds for \(S\).
Let \({\mathscr{C}}\) be an étale covering of \(\overline{S}\), and consider its inverse image \({\mathscr{C}}'\) over \(S'\), which corresponds to a coherent analytic sheaf \({\mathfrak{A}}'\) of algebras on \(S'\) that is the inverse image of a sheaf of algebras \({\mathfrak{A}}\) on \(\overline{S}\) defining \({\mathscr{C}}\).
By hypothesis, on \(S'\), \({\mathscr{C}}'\) comes from an étale covering \(T'\) of \(S'\), i.e.~\({\mathfrak{A}}'\) comes from a coherent sheaf of algebras \({\mathscr{A}}'\) on \(S'\).
Also, \({\mathfrak{A}}'\) is endowed with a canonical descent data with respect to \(\overline{S}'\to\overline{S}\), i.e.~with an isomorphism between its two inverse images on \(\overline{S}'\times_{\overline{S}}\overline{S}'=\overline{S'\times_SS'}\) (satisfying conditions (i) and (ii)), and this isomorphism comes from, by what has already been said, an isomorphism between the corresponding algebraic sheaves, i.e.~from a descent data on \({\mathscr{A}}'\) with respect to \(S'\to S\).
We can easily show that the latter is effective (since it is effective on \({\mathfrak{A}}'\), and the effectiveness of a descent data, as described explicitly in the previous section, is something that can be checked locally on the \emph{completions} of the modules that are involved).
From this, we obtain a coherent sheaf of algebras \({\mathscr{A}}\) on \(S\) that defines a covering \(T\) of \(S\), and this is the desired covering.
The above result then obviously holds true if \(S'\to S\) is just a composition of a finite number of finite strict epimorphisms, i.e.~is just an arbitrary finite epimorphism (by the factorisation result stated in \protect\hyperlink{fga-3-i-section-A.2}{§A.2}.
It thus follows that the Grauert--Remmert theorem holds true if \(S\) is a \emph{reduced} scheme, i.e.~such that \({\mathscr{O}}_S\) has no nilpotent elements, as we can see by introducing its normalisation \(S'\).
We can easily pass to the general case.

A completely analogous argument, again using the factorisation result for finite strict epimorphisms, and the ``formal'' nature of the effectiveness of descent data, allows us to prove the following result:
let \(S\) be a locally Noetherian prescheme, and let \(S'\to S\) be a finite, surjective, and radicial morphism (or, equivalently, a morphism of finite type such that, for every \(T\) over \(S\), the morphism \(T'=S'\times_S T\to T\) is a homeomorphism, which we can also express by saying that \(S'\to S\) is a \emph{universal homeomorphism}).
For every \(T\) étalé over \(S\), consider its inverse image \(T'=T\times_S S'\), which is étalé over \(S'\).
\emph{Then the functor \(T\mapsto T'\) is an equivalence between the category of preschemes \(T\) that are étalé over \(S\) and the category of preschemes \(T'\) that are étalé over \(S'\).}
(We use the bijectivity of
\[
  \operatorname{Hom}_S(T_1,T_2) \to \operatorname{Hom}_{S'}(T'_1,T'_2)
\]
for preschemes \(T_1\) and \(T_2\) that are étalé over \(S\), which can be proven directly without difficulty. We also use the fact that the stated theorem is true if \(S'=(S,{\mathscr{O}}_S/{\mathscr{J}})\), where \({\mathscr{J}}\) is a nilpotent coherent sheaf of ideals of \({\mathscr{O}}_S\), cf. {[}\protect\hyperlink{ref-Mur1958}{16}, Lemma 6{]}).
\oldpage{190-12}Note also that we do not suppose here that the \(T\) in question are finite over \(S\).
This result implies, in particular, that the morphism \(S'\to S\) induces an isomorphism between the fundamental group of \(S'\) and the fundamental group of \(S\) (``\emph{topological invariance of the fundamental group of a prescheme}'').

\hypertarget{fga-3-i-section-A.4}{%
\subsection{\texorpdfstring{(A.4) Relations to \(1\)-cohomology}{(A.4) Relations to 1-cohomology}}\label{fga-3-i-section-A.4}}

\hypertarget{fga-3-i-section-A.4.a}{%
\subsubsection{(a)}\label{fga-3-i-section-A.4.a}}

Let \({\mathcal{C}}\) be a category such that the product of any two objects always exists, and let \(T\in{\mathcal{C}}\).
For every finite set \(I\neq\varnothing\), we can consider \(T^I\), and so we obtain a covariant functor from the category of non-empty finite sets to the category \({\mathcal{C}}\), i.e.~what we can call a \emph{simplicial object} of \({\mathcal{C}}\), denoted by \(K_T\).
This object depends covariantly on \(T\);
also, \emph{if \(u\) and \(v\) are morphisms \(T\to T'\), then the corresponding morphisms \(K_T\to K_{T}\) are homotopic}.
We say that \(T\) \emph{dominates} \(T'\) if \(\operatorname{Hom}(T,T')\neq\varnothing\), and this gives an (upward) directed preorder on \({\mathcal{C}}\).
It follows from the above that, if \(T\) dominates \(T'\), then there exists a canonical class (up to homotopy) of homomorphisms of simplicial objects \(K_T\to K_{T'}\);
in particular, if \(K_T\) and \(K_{T'}\) are such that \(T\) and \(T'\) dominate one another, then \(K_T\) and \(K_{T'}\) are homotopically equivalent.
Now let \(F\) be a (contravariant, to be clear) functor from \({\mathcal{C}}\) to an \emph{abelian} category \({\mathcal{C}}'\).
Then
\[
  C^\bullet(T,F) = F(K_T)
\]
is a cosimplicial object of \({\mathcal{C}}'\), and thus defines, in a well-known way, a (cochain) complex in \({\mathcal{C}}'\), of which we can take the cohomology:
\[
  \operatorname{H}^\bullet(T,F)
  = \operatorname{H}^\bullet(C^\bullet(T,F))
  = \operatorname{H}^\bullet(F(K_T))
\]
(we may write a subscript ``\({\mathcal{C}}\)'' on the \(\operatorname{H}^\bullet\) if there is any possibility for confusion).
This is a cohomological functor in \(F\), of which the variance for \(T\) varying follows from what has already been said about the \(K_T\);
more precisely, for fixed \(F\) and varying \(T\) in \({\mathcal{C}}\) (preordered by the domination relation), the \(\operatorname{H}^\bullet(T,F)\) form an inductive system of graded objects of \({\mathcal{C}}'\);
in particular, if \(T\) and \(T'\) are such that each one dominates the other, then \(\operatorname{H}^\bullet(T,F)\) and \(\operatorname{H}^\bullet(T',F)\) are canonically isomorphic.

Suppose that \({\mathcal{C}}\) has all fibre products.
Then we can, for fixed \(S\in{\mathcal{C}}\), apply the above to the category \({\mathcal{C}}_S\) of objects of \({\mathcal{C}}\) over \(S\);
we then write \(C^\bullet(T/S,F)\) and \(\operatorname{H}^\bullet(T/S,F)\) instead of \(C^\bullet(T,F)\) and \(\operatorname{H}^\bullet(T,F)\) if we wish to make clear that we are working in the category \({\mathcal{C}}_S\);
\oldpage{190-13}then \(C^\bullet(T/S,F)\) is a cochain complex in \({\mathcal{C}}'\) that, in degree \(n\), is equal to \(F(T\times_S T\times_S\ldots\times_S T)\) (where there are \(n+1\) factors \(T\)).

Note that, as per usual, we can define \(\operatorname{H}^0(T/S,F)\) without assuming the category \({\mathcal{C}}'\) to be abelian:
it is the kernel (\protect\hyperlink{fga-3-i-section-A.2-definition-2.1}{Definition 2.1}), if it exists, of the pair \((F(p_1),F(p_2))\) of morphisms
\[
  F(T) \to F(T\times_S T)
\]
corresponding to the two projections \(p_1,p_2\colon T\times_S T\to T\).
In particular, we then have the natural morphism (called the \emph{augmentation})
\[
  F(S) \to \operatorname{H}^0(T/S,F)
\]
which is an isomorphism in nice cases (in particular, in the case where \(T\to S\) is a strict epimorphism and \(F\) is ``left exact'').
Similarly, if \(F\) takes values in the category of groups in a category \({\mathcal{C}}''\), then we can also define \(\operatorname{H}^1(T/S,F)\);
in the case where \({\mathcal{C}}''\) is the category of sets (i.e.~when \(F\) takes values in the category of non-necessarily-commutative groups), \(\operatorname{H}^1(T,F)\) is the quotient of the subgroup \(Z^1(T/S,F)\) of \(C^1(T/S,F) = F(T\times_S T)\) consisting of the \(g\) such that
\[
  F(p_{31})(g) = F(p_{32})(g) F(p_{21})(g)
\]
by the group with operators \(F(T)\) acting on \(C^1(T/S,F)\), and thus, in particular, on the subset \(Z^1(T/S,F)\), by
\[
  \rho(g')\cdot g = F(p_2)(g') g F(p_1)(g')^{-1}.
\]

\hypertarget{fga-3-i-section-A.4.b}{%
\subsubsection{(b)}\label{fga-3-i-section-A.4.b}}

For example, let \({\mathcal{F}}\) be a fibred category with base \({\mathcal{C}}\).
Let \(\xi,\eta\in{\mathcal{F}}_S\), and, for all \(S'\) over \(S\), let
\[
  F_{\xi,\eta}(S') = \operatorname{Hom}(\xi\times_S S', \eta\times_S S').
\]
Then \(F_{\xi,\eta}\) is a contravariant functor from \({\mathcal{C}}_S\) to the category of sets.
With this setup, \emph{saying that the augmentation morphism}
\[
  F_{\xi,\eta}(S) \to \operatorname{H}^0(S'/S,F_{\xi,\eta})
\]
\emph{is an isomorphism for every pair of elements \(\xi,\eta\in{\mathcal{F}}_S\) implies that \(\alpha\colon S'\to S\) is an \({\mathcal{F}}\)-descent morphism} (\protect\hyperlink{fga-3-i-section-A.1-definition-1.7}{Definition 1.7}).

\hypertarget{fga-3-i-section-A.4.c}{%
\subsubsection{(c)}\label{fga-3-i-section-A.4.c}}

Similarly, for \(\xi\in{\mathcal{F}}_S\) and any object \(S'\) of \({\mathcal{C}}\) over
\[
  G_\xi(S') = \operatorname{Aut}(\xi\times_S S'),
\]
we thus define a contravariant functor \(G_\xi\) from \({\mathcal{C}}_S\) to the category of groups.
\oldpage{190-14}With this setup, we claim that \emph{\(Z^1(S'/S,G)\) is canonically identified with the set of descent data on \(\xi'=\xi\times_S S'\) with respect to \(S'\to S\)} (\protect\hyperlink{fga-3-i-section-A.1-definition-1.6}{Definition 1.6}), and that \emph{\(\operatorname{H}^1(S'/S,G)\) can be identified with the set of isomorphism classes of objects of \({\mathcal{F}}_{S'}\) endowed with a descent data relative to \(\alpha\colon S'\to S\) that are isomorphic, as objects of \({\mathcal{F}}_{S'}\), to \(\xi'=\xi\times_S S'\)}.
Then, \emph{if \(\alpha\colon S'\to S\) is an \({\mathcal{F}}\)-descent morphism} (cf.~\protect\hyperlink{fga-3-i-section-A.4.b}{§A.4.b}), \emph{then \(\operatorname{H}^1(S'/S,G)\) contains as a subset the set of isomorphism classes of objects \(\eta\) of \({\mathcal{F}}_S\) such that \(\eta\times_S S'\) is isomorphic (in \({\mathcal{F}}_{S'}\)) to \(\xi\times_S S'\)};
further, \emph{this inclusion is the identity if and only if every descent data on \(\xi'=\xi\times_S S'\) with respect to \(\alpha\colon S'\to S\) is effective}.
(This will be the case, in particular, if \(\alpha\colon S'\to S\) is a strict \(S\)-descent morphism).

\begin{rmenv}{Remark}
The cochain complexes of the form \(C^\bullet(T/S,F)\) contain, as particular cases, the majority of standard known complexes (that of Čech cohomology, of group cohomology, etc.), and play an important role in algebraic geometry (notably in the ``Weil cohomology'' of preschemes).

\end{rmenv}

\hypertarget{fga-3-i-section-A.4.d}{%
\subsubsection{(d)}\label{fga-3-i-section-A.4.d}}

\leavevmode\vadjust pre{\hypertarget{fga-3-i-section-A.4-example-1}{}}%
\begin{rmenv}{Example 1}
Let \(S'\) be an object over \(S\in{\mathcal{C}}\), and let \(\Gamma\) be a group of automorphisms of \(S'\) such that \(S'\) is ``formally \(\Gamma\)-principal over \(S\)'', i.e.~such that the natural morphism
\[
  \Gamma\times S' \to S'\times_S S'
\]
(where \(\Gamma\times S'\) denotes the direct sum of \(\Gamma\) copies of \(S'\)) is an isomorphism.
(We suppose that all the necessary direct sums exist in \({\mathcal{C}}\)).
Let \(F\) be a contravariant functor from \({\mathcal{C}}\) to the category of abelian groups.
Then \emph{\(C^\bullet(S'/S,F)\) is canonically isomorphic to the simplicial group \(C^\bullet(\Gamma,F(S'))\) of standard homogeneous cochains, and so \(\operatorname{H}^\bullet(S'/S,F)\) is canonically isomorphic to \(\operatorname{H}^\bullet(\Gamma,F(S')\)}.

\end{rmenv}

\hypertarget{fga-3-i-section-A.4.e}{%
\subsubsection{(e)}\label{fga-3-i-section-A.4.e}}

\leavevmode\vadjust pre{\hypertarget{fga-3-i-section-A.4-example-2}{}}%
\begin{rmenv}{Example 2}
Let \({\mathcal{C}}\) be the category of preschemes.
We denote by \(\operatorname{G_a}\) (for ``additive group'') the contravariant functor from \({\mathcal{C}}\) to the category of abelian groups, defined by
\[
  \operatorname{G_a}(X) = \operatorname{H}^0(X,{\mathscr{O}}_X).
\]
We similarly define the functor \(\operatorname{G_m}\) (for ``multiplicative group'') by
\[
  \operatorname{G_m}(X) = \operatorname{H}^0(X,{\mathscr{O}}_X)^\times
\]
\oldpage{190-15}(i.e.~the group of invertible elements of the ring \(\operatorname{H}^0(X,{\mathscr{O}}_X)\)), and, more generally, the functor \(\operatorname{GL}(n)\) (for ``linear group of order \(n\)'') by
\[
  \operatorname{GL}(n)(X) = \operatorname{GL}(n,\operatorname{H}^0(X,{\mathscr{O}}_X)),
\]
which is a functor from \({\mathcal{C}}\) to the category of (not-necessary-commutative, if \(n>1\)) groups;
for \(n=1\) we recover \(\operatorname{G_m}\).
We can also think of \(\operatorname{GL}(n)\) as an automorphism functor (cf.~\protect\hyperlink{fga-3-i-section-A.4.c}{§A.4.c}) by considering the fibred category \({\mathcal{F}}\) with base \({\mathcal{C}}\) such that \({\mathcal{F}}_X\) is the category of locally free sheaves on \(X\), for \(X\in{\mathcal{C}}\), since then \(\operatorname{GL}(n)(X)=\operatorname{Aut}_{{\mathcal{F}}_X}({\mathscr{O}}_X^n)\).
By \protect\hyperlink{fga-3-i-section-A.4.b}{§A.4.b}, it follows that, if \(\alpha\colon S'\to S\) is an \({\mathcal{F}}\)-descent morphism (cf.~\protect\hyperlink{fga-3-i-section-A.2.c}{§A.2.c}), then \(\operatorname{H}^1(S'/S,\operatorname{GL}(n))\) contains the set of isomorphism classes of locally free sheaves on \(S\) whose inverse image on \(S'\) is isomorphic to \({\mathscr{O}}_{S'}^n\), and this inclusion is an equality if and only if every descent data on \({\mathscr{O}}_{S'}^n\) (with respect to \(\alpha\colon S'\to S\)) is effective.
If \(S\) is the spectrum of a local ring, then this implies that \(\operatorname{H}^1(S'/S,\operatorname{GL}(n))=(e)\), since every locally free sheaf on \(S\) is then trivial.

We note that the following conditions concerning a morphism \(\alpha\colon S'\to S\) are equivalent:

\begin{enumerate}
\def\labelenumi{\roman{enumi}.}
\tightlist
\item
  The augmentation homomorphism \(\operatorname{H}^0(S,{\mathscr{O}}_S) = \operatorname{G_a}(S)\to\operatorname{H}^0(S'/S,\operatorname{G_a})\) is an isomorphism.
\item
  \(\alpha\colon S'\to S\) is an \({\mathcal{F}}\)-descent morphism (where \({\mathcal{F}}\) is the fibred category over \({\mathcal{C}}\) described above).
\item
  \(\alpha\colon S'\to S\) is a strict epimorphism (cf.~\protect\hyperlink{fga-3-i-section-A.2.c}{§A.2.c}).
\end{enumerate}

Now suppose that \(S=\operatorname{Spec}(A)\) and \(S'=\operatorname{Spec}(A')\);
then
\[
  C^n(S'/S,\operatorname{G_a})
  = C^n(A'/A,\operatorname{G_a})
  = \underbrace{A'\otimes_A A'\otimes_A\ldots\otimes_A A'}_{n+1\text{ copies of }A'}
\]
with the coboundary operator \(C^n(A'/A,\operatorname{G_a})\to C^{n+1}(A'/A,\operatorname{G_a})\) being the alternating sum of the face operators
\[
  \partial_i(x_0\otimes x_1\otimes\ldots\otimes x_n)
  = x_0\otimes\ldots\otimes x_{i-1}\otimes1_{A'}\otimes x_i\otimes\ldots\otimes x_n.
\]
Similarly, \(C^n(S'/S,\operatorname{G_m})=C^n(A'/A,\operatorname{G_m})\) can be identified with \((\bigotimes_A^{n+1}A')^\times\), with the simplicial operations for \(C^\bullet(A'/A,\operatorname{G_m})\) being induced by those in \(C^\bullet(S'/S,\operatorname{G_a})\).
We can write down the simplicial operations for \(C^\bullet(A'/A,\operatorname{GL}(n))\) in the same explicit manner.
\emph{In all the cases known to the speaker, \(\operatorname{H}^i(A'/A,\operatorname{G_a})=0\) for \(i>0\), and, if \(A\) is local, then \(\operatorname{H}^1(A'/A,\operatorname{G_m})=0\), and, more generally, \(\operatorname{H}^1(A'/A,\operatorname{GL}(n))=(e)\)} (if \(S'\to S\) is an \({\mathcal{F}}\)-descent morphisms, i.e.~if the diagram \(A\to A'\rightrightarrows A'\otimes_A A'\) is exact, then compare this with \protect\hyperlink{fga-3-i-section-A.2.c}{§A.2.c}).
\oldpage{190-16}We note that \emph{Hilbert's ``Theorem 90'' is exactly the fact that \(\operatorname{H}^1(S'/S,\operatorname{G_m})=0\) if \(A\) is a field and \(A'\) is a finite Galois extension of \(A\)} (cf.~\protect\hyperlink{fga-3-i-section-A.4-example-1}{Example 1}), \emph{and we can also express it by saying that, in the case in question, \(S'\to S\) is a strict descent morphisms for the fibred category of locally free sheaves of rank \(1\).}
This latter statement is the one that is most suitable to generalise Hilbert's theorem, by varying the hypotheses both on the morphism \(S'\to S\) and on the quasi-coherent sheaves in question.

Finally, we note that, for a local \emph{Artinian} \(A\) with maximal ideal \({\mathfrak{m}}\), and an \(A\)-algebra \(A'\) (where we denote, for any integer \(k>0\), the ring \(A/{\mathfrak{m}}^{k+1}\) (resp. \(A'/{\mathfrak{m}}^{k+1}A'\)) by \(A_k\) (resp. \(A'_k\))), the following properties are equivalent:

\begin{enumerate}
\def\labelenumi{\roman{enumi}.}
\tightlist
\item
  \(\operatorname{H}^1(A'_k/A_k,\operatorname{G_a})=0\) for all \(k\).
\item
  \(\operatorname{H}^1(A'_k/A_k,\operatorname{G_m})=0\) for all \(k\).
\item
  \(\operatorname{H}^1(A'_k/A_k,\operatorname{GL}(n))=(e)\) for all \(k\) and all \(n\).
\end{enumerate}

If \(S'\to S\) is a strict epimorphism, then the above conditions imply that it is a \emph{strict} descent morphism for free modules (not necessarily of finite type) over \(A'\).

\end{rmenv}

\begin{rmenv}{Remark}
The definition of the groups \(\operatorname{H}^i(S'/S,\operatorname{G_m})\) in the case where \(S\) (resp. \(S'\)) is a scheme over the field \(A\) (resp. \(A'\)) is due to Amitsur.
The group \(\operatorname{H}^2(S'/S,\operatorname{G_m})\) is particularly interesting as a ``global'' variant of the Brauer group, for which we can refer to {[}\protect\hyperlink{ref-GD1960}{10}, VII{]}.

\end{rmenv}

\hypertarget{b.-descent-by-faithfully-flat-morphisms}{%
\subsection*{\texorpdfstring{\textbf{B.} Descent by faithfully flat morphisms}{B. Descent by faithfully flat morphisms}}\label{b.-descent-by-faithfully-flat-morphisms}}
\addcontentsline{toc}{subsection}{\textbf{B.} Descent by faithfully flat morphisms}

\hypertarget{fga-3-i-section-B.1}{%
\subsection{(B.1) Statement of the descent theorems}\label{fga-3-i-section-B.1}}

\leavevmode\vadjust pre{\hypertarget{fga-3-i-section-B.1-definition-1.1}{}}%
\begin{rmenv}{Definition 1.1}
A morphism \(\alpha\colon S'\to S\) of prescheme is said to be \emph{flat} if \({\mathscr{O}}_{x'}\) is a flat module over the ring \({\mathscr{O}}_{\alpha(x')}\) for all \(x'\in S'\) (i.e.~if \({\mathscr{O}}_{x'}\otimes_{{\mathscr{O}}_{\alpha(x')}}M\) is an exact functor in the \({\mathscr{O}}_{\alpha(x')}\)-module \(M\)).
A morphism is said to be \emph{faithfully flat} if it is flat and surjective.

\end{rmenv}

For example, if \(S=\operatorname{Spec}(A)\) and \(S'=\operatorname{Spec}(A')\), then \(S'\) is flat over \(S\) if and only if \(A'\) is a flat \(A\)-module, and \(S'\) is faithfully flat over \(S\) if and only if \(A'\) is a faithfully flat \(A\)-module (i.e.~if and only if \(A'\otimes_A M\) is an \emph{exact} and \emph{faithful} functor in the \(A\)-module \(M\));
this also implies, in the terminology of Serre {[}\protect\hyperlink{ref-Ser1956}{19}{]}, that the pair \((A,A')\) is flat.
If \(S'\) is faithfully flat over \(S\), then the inverse image functor of quasi-coherent sheaves on \(S\) is exact and faithful;
\oldpage{190-17}in other words, for a sequence of homomorphisms of quasi-coherent sheaves on \(S\) to be exact, it is necessary and sufficient that its inverse image on \(S'\) be exact (in particular, for a homomorphism of quasi-coherent sheaves on \(S\) to be a monomorphism (resp. an epimorphism, resp. an isomorphism), it is necessary and sufficient that its inverse image on \(S'\) be a monomorphism (resp. an epimorphism, resp. an isomorphism)).
This property holds true if we restrict to an arbitrary open subset of \(S'\), and then characterise faithfully flat morphisms in this form.

\leavevmode\vadjust pre{\hypertarget{fga-3-i-section-B.1-definition-1.2}{}}%
\begin{rmenv}{Definition 1.2}
A morphism \(\alpha\colon S'\to S\) is said to be \emph{quasi-compact} if the inverse image of every quasi-compact open subset \(U\) of \(S\) is quasi-compact (i.e.~a \emph{finite} union of affine open subsets).

\end{rmenv}

It evidently suffices to verify this property for the \emph{affine} open subsets of \(S\).
For example, an affine morphism (i.e.~a morphism such that the inverse image of an affine open subset is affine) is quasi-compact.

The class of flat (resp. faithfully flat, resp. quasi-compact) morphisms is stable under composition and by ``base extension'', and of course contains all isomorphisms.

\leavevmode\vadjust pre{\hypertarget{fga-3-i-section-B.1-theorem-1}{}}%
\begin{itenv}{Theorem 1}
Let \(\alpha\colon S'\to S\) be a morphism of preschemes that is \emph{faithfully flat} and \emph{quasi-compact}.
Then \(\alpha\) is a \emph{strict descent morphism} (cf.~\protect\hyperlink{fga-3-i-section-A.1-definition-1.7}{Definition 1.7} for the fibred category \({\mathcal{F}}\) of quasi-coherent sheaves (cf.~\protect\hyperlink{fga-3-i-section-A.1-example-2}{§A, Example 2}).

\end{itenv}

This statement implies two things:

\begin{enumerate}
\def\labelenumi{\roman{enumi}.}
\item
  If \({\mathcal{F}}\) and \({\mathscr{G}}\) are quasi-coherent sheaves on \(S\), and \({\mathcal{F}}'\) and \({\mathscr{G}}'\) their inverse images on \(S'\), then the natural homomorphism
  \[
   \operatorname{Hom}({\mathcal{F}},{\mathscr{G}}) \to \operatorname{Hom}({\mathcal{F}}',{\mathscr{G}}')
    \]
  is a bijection from the left-hand side to the subgroup of the right-hand side consisting of homomorphisms \({\mathcal{F}}'\to{\mathscr{G}}'\) that are compatible with the canonical descent data on these sheaves, i.e.~whose inverse images under the two projections of \(S''=S'\times_S S'\) to \(S'\) give the same homomorphism \({\mathcal{F}}''\to{\mathscr{G}}''\).
\item
  Every quasi-coherent sheaf \({\mathcal{F}}'\) on \(S'\) endowed with a descent data with respect to the morphism \(\alpha\colon S'\to S\) (cf.~\protect\hyperlink{fga-3-i-section-A.1-definition-1.6}{Definition 1.6} is isomorphic (endowed with this data) to the inverse image of a quasi-coherent sheaf \({\mathcal{F}}\) on \(S\).
\end{enumerate}

Setting \({\mathcal{F}}={\mathscr{O}}_S\) in (i), we obtain:

\leavevmode\vadjust pre{\hypertarget{fga-3-i-section-B.1-corollary-1}{}}%
\begin{rmenv}{Corollary 1}
Let \({\mathscr{G}}\) be a quasi-coherent sheaf on \(S\), with \({\mathscr{G}}'\) and \({\mathscr{G}}''\) denoting its inverse images on \(S'\) and \(S''=S'\times_S S'\) (respectively), and let \(p_1\) and \(p_2\) be the two projections from \(S''\) to \(S\).
\oldpage{190-18}Then the diagram of maps of sets
\[
  \Gamma({\mathscr{G}}) \xrightarrow{\alpha^*}
  \Gamma({\mathscr{G}}') \overset{p_1^*}{\underset{p_2^*}{\rightrightarrows}}
  \Gamma({\mathscr{G}}'')
\]
is \emph{exact} (cf.~\protect\hyperlink{fga-3-i-section-A.1-definition-1.1}{Definition 1.1}.

\end{rmenv}

Also, the combination of (i) and (ii) with \protect\hyperlink{fga-3-i-section-A.1-definition-1.1}{Definition 1.1} gives:

\leavevmode\vadjust pre{\hypertarget{fga-3-i-section-B.1-corollary-2}{}}%
\begin{rmenv}{Corollary 2}
Let \({\mathscr{G}}\) be as in \protect\hyperlink{fga-3-i-section-B.1-corollary-1}{Corollary 1}.
Then there is a bijective correspondence between quasi-coherent subsheaves of \({\mathscr{G}}\) and quasi-coherent subsheaves of \({\mathscr{G}}'\) whose inverse images on \(S''\) under the two projections \(p_1\) and \(p_2\) give the same subsheaf of \({\mathscr{G}}\).

\end{rmenv}

Of course, we have an equivalent statement in terms of quotient sheaves.
As we have already seen in \protect\hyperlink{fga-3-i-section-A.4.e}{§A.4.e}, \protect\hyperlink{fga-3-i-section-B.1-theorem-1}{Theorem 1} should be thought of as a generalisation of Hilbert's ``Theorem 90'', and implies, as particular cases, various formulations in terms of \(1\)-cohomology.
For the proof, we can easily reduce to the case where \(S=\operatorname{Spec}(A)\) and \(S'=\operatorname{Spec}(A')\), and, for (i), we can easily restrict to proving \protect\hyperlink{fga-3-i-section-B.1-corollary-1}{Corollary 1}, i.e.~the exactness of the diagram
\[
  M = A\otimes_A M \to
  A'\otimes_A M \rightrightarrows
  A'\otimes_A A'\otimes_A M
\]
for every \(A\)-module \(M\), which follows from the more general lemma:

\leavevmode\vadjust pre{\hypertarget{fga-3-i-section-B.1-lemma-1.1}{}}%
\begin{itenv}{Lemma 1.1}
Let \(A'\) be a faithfully flat \(A\)-algebra.
Then, for every \(A\)-module \(M\), the \(M\)-augmented complex \(C^\bullet(A'/A,\operatorname{G_a})\otimes_A M\) (cf.~\protect\hyperlink{fga-3-i-section-A.4.e}{§A.4.e}) is a \emph{resolution} of \(M\).

\end{itenv}

\begin{proof}
It suffices to prove that the augmented complex induced from the above by extension of the base \(A\) to \(A'\) satisfies the same conclusions.
This leads to proving the statement when we replace \(A\) by \(A'\), and \(A'\) by \(A'\otimes_A A'\), and so we can restrict to the case where there exists an \(A\)-algebra homomorphism \(A'\to A\) (or, in geometric terms, the case where \(S'\) over \(S\) admits a section).
In this case, the claim follows from the generalities of \protect\hyperlink{fga-3-i-section-A.4.a}{§A.4.a}.
\end{proof}

We note, in passing, the following corollary, which generalises a well-known statement in Galois cohomology (compare with \protect\hyperlink{fga-3-i-section-A.4.e}{§A.4.e}):

\begin{itenv}{Corollary}
If \(A'\) is faithfully flat over \(A\), then \(\operatorname{H}^0(A'/A,\operatorname{G_a})=A\), and \(\operatorname{H}^i(A'/A,\operatorname{G_a})=0\) for \(i\geqslant 1\).

\end{itenv}

\begin{proof}
\emph{Part (ii).}
To prove part (ii) of \protect\hyperlink{fga-3-i-section-B.1-theorem-1}{Theorem 1}, we proceed, as for (i), by restricting to the case where \(S'\) over \(S\) admits a section, where the result then follows from (i) (cf.~\protect\hyperlink{fga-3-i-section-A.1.c}{§A.1.c}).
\end{proof}

We can evidently vary \protect\hyperlink{fga-3-i-section-B.1-theorem-1}{Theorem 1} and its corollaries \emph{ad libitum} by introducing various additional structures on the quasi-coherent sheaves (or systems of sheaves) in question.
\oldpage{190-19}For example, the data on \(S\) of a quasi-coherent sheaf of commutative algebras ``is equivalent to'' the data on \(S'\) of such a sheaf endowed with a descent data relative to \(\alpha\colon S'\to S\).
Taking into account the functorial correspondence between such quasi-coherent sheaves on \(S\) and affine preschemes over \(S\), we obtain the second claim of the following theorem:

\leavevmode\vadjust pre{\hypertarget{fga-3-i-section-B.1-theorem-2}{}}%
\begin{itenv}{Theorem 2}
Let \(\alpha\colon S'\to S\) be as in \protect\hyperlink{fga-3-i-section-B.1-theorem-1}{Theorem 1}.
Then \(\alpha\) is a (non-strict, in general) \emph{descent morphism} (cf.~\protect\hyperlink{fga-3-i-section-A-definition-2.4}{§A, Definition 2.4}), and it is a \emph{strict descent morphism} for the fibred category of affine schemes over preschemes (cf.~\protect\hyperlink{fga-3-i-section-A.1-definition-1.7}{§A, Definition 1.7}.

\end{itenv}

The first claim of the theorem implies this:
let \(X\) and \(Y\) be preschemes over \(S\), with \(X'\) and \(Y'\) their inverse images over \(S\), and \(X''\) and \(Y''\) their inverse images over \(S''=S'\times_S S'\);
then the diagram of natural maps
\[
  \operatorname{Hom}_S(X,Y) \xrightarrow{\alpha^*}
  \operatorname{Hom}_{S'}(X',Y') \overset{p_1^*}{\underset{p_2^*}{\rightrightarrows}}
  \operatorname{Hom}_{S''}(X'',Y'')
\]
is \emph{exact}, i.e.~\(\alpha^*\) is a bijection from \(\operatorname{Hom}_S(X,Y)\) to the subset of \(\operatorname{Hom}_{S'}(X',Y')\) consisting of homomorphisms that are compatible with the canonical descent data on \(X'\) and \(Y'\) (i.e.~whose inverse images under the two projections from \(S''\) to \(S'\) are equal).
This follows easily from \protect\hyperlink{fga-3-i-section-B.1-theorem-1}{Theorem 1} and \protect\hyperlink{fga-3-i-section-B.1-corollary-1}{Corollary 1}, if we restrict to \(Y\) being affine over \(S\);
in the general case, we need to combine \protect\hyperlink{fga-3-i-section-B.1-theorem-1}{Theorem 1} with the following result:

\leavevmode\vadjust pre{\hypertarget{fga-3-i-section-B.1-lemma-1.2}{}}%
\begin{itenv}{Lemma 1.2}
Let \(\alpha\colon S'\to S\) be a faithfully flat and quasi-compact morphism.
Then \(S\) can be identified with a \emph{topological quotient space of \(S'\)}, i.e.~every subset \(U\) of \(S\) such that \(\alpha^{-1}(U)\) is open, is open.

\end{itenv}

To complete \protect\hyperlink{fga-3-i-section-B.1-theorem-2}{Theorem 2}, we must give effectiveness criteria for a descent data on an \(S'\)-prescheme \(X'\) (in the case where \(X'\) is not assumed to be affine over \(S'\)).
Note first of all that \emph{such a descent data is not necessarily effective}, even if \(S\) is the spectrum of a field \(k\), \(S'\) the spectrum of a quadratic extension \(k'\) of \(k\), and \(S''\) a proper algebraic scheme of dimension \(2\) over \(S'\) (as we can see, due to Serre, by using the non-projective surface of Nagata).
\emph{For a descent data on \(X'/S'\) with respect to \(\alpha\colon S'\to S\) (assumed to be faithfully flat and quasi-compact) to be effective, it is necessary and sufficient that \(X'\) be a union of open subsets \(X'_i\) that are affine over \(S'\) and ``stable'' under the descent data on \(X'\).}
This is certainly the case (for any \(X'/S'\) and any descent data on \(X'\)) if the morphism \(\alpha\colon S'\to S\) is \emph{radicial} (i.e.~injective, and with radicial residual extensions).
\oldpage{190-20}We can also show that this is the case if \(\alpha\colon S'\to S\) is \emph{finite}, and every finite subset of \(X'\) that is contained in a fibre of \(X'\) over \(S\) is also contained in an open subset of \(X'\) that is affine over \(S\) (this is the \emph{Weil criterion}).
It is, in particular, the case if \(X'/S'\) is quasi-projective, and, in this case, we can show that the ``descended'' prescheme \(X/S\) is also quasi-projective (and projective if \(X'/S'\) is projective).
In summary:

\leavevmode\vadjust pre{\hypertarget{fga-3-i-section-B.1-theorem-3}{}}%
\begin{itenv}{Theorem 3}
Let \(\alpha\colon S'\to S\) be faithfully flat and quasi-compact morphism of preschemes.
If \(\alpha\) is \emph{radicial}, then it is a \emph{strict descent morphism}.
If \(\alpha\) is finite, then it is a strict descent morphism with respect to the fibred category of quasi-projective (or projective) preschemes over preschemes.

\end{itenv}

\begin{rmenv}{Remark}
I do not know if, in the second claim above, the hypothesis that \(\alpha\) be \emph{finite} is indeed necessary;
we can prove that, in any case, we can ``formally'' replace it by the following, seemingly weaker, hypothesis:
\emph{for every point of \(S\) there exists an open neighbourhood \(U\), a finite and faithfully flat \(U'\) over \(U\), and an \(S\)-morphism from \(U'\) to \(S'\)}.
A type of case that is not covered by the above is that where \(S=\operatorname{Spec}(A)\) and \(S'=\operatorname{Spec}(\overline{A})\), with \(A\) a local Noetherian ring and \(\overline{A}\) its completion;
or even that where \(S'\) is quasi-finite over \(S\) (i.e.~locally isomorphic to an open subset of a finite \(S\)-scheme) but not finite.
In these two cases, the speaker also does not know the answer to the following question:
let \(X\) be an \(S\)-scheme such that \(X'=X\times_S S'\) is projective over \(S'\);
is it then true that \(X\) is projective over \(S\)?

\emph{{[}Comp.{]}}
A morphism \(S'\to S\) that is quasi-finite, étale, surjective, or a morphism of the form \(\operatorname{Spec}(\overline{A})\to\operatorname{Spec}(A)\), is not always a strict descent morphism, even if \(A\) is the local ring of an algebraic curve over an algebraically closed field \(k\) and \(S=\operatorname{Spec}(A)\).
We can thus find a proper simple morphism \(f\colon X\to S\) that makes \(X\) into a principal \(E\)-bundle over \(S\), with \(E\) an elliptic curve, such that \(f'\colon X'\to S'\) is projective, but \(f\) is not projective.
So this is also an example of a homogeneous principal bundle that is \emph{non-isotrivial} under an abelian scheme.

\end{rmenv}

\hypertarget{fga-3-i-section-B.2}{%
\subsection{(B.2) Application to the descent of certain properties of morphisms}\label{fga-3-i-section-B.2}}

Let \(P\) be a class of morphisms of preschemes.
Let \(\alpha\colon S\to S'\) be a morphism of preschemes, and let \(f\colon X\to Y\) be a morphism of \(S\)-preschemes, with \(f'\colon X'\to Y'\) the inverse image of \(f\) under \(\alpha\).
We can then ask if the relation ``\(f'\in P\)'' implies that ``\(f\in P\)''.
It appears that the answer is affirmative in many important cases, if we suppose that \(\alpha\) is \emph{faithfully flat} and \emph{quasi-compact} (this latter hypothesis being sometimes overly strong).
We can see this directly without difficulty if \(P\) is the class of surjective (resp. radicial) morphisms (with these two cases following from the surjectivity of \(\alpha\)), or flat (resp. faithfully flat, resp. simple) morphisms (with these three cases following from the faithful flatness of \(\alpha\)), or morphisms of finite type.
Using \protect\hyperlink{fga-3-i-section-B.1-theorem-1}{Theorem 1}, \protect\hyperlink{fga-3-i-section-B.1-theorem-2}{Theorem 2}, and \protect\hyperlink{fga-3-i-section-B.1-lemma-1.2}{Lemma 1.2}, we see that it is also true if \(P\) is one of the following classes:
isomorphisms, open immersions, closed immersions, immersions (if \(f\) is of finite type, and \(Y\) is locally Noetherian), affine morphisms, finite morphisms, quasi-finite morphisms, open morphisms, closed morphisms, homeomorphisms, separated morphisms, or proper morphisms.
\oldpage{190-21}The only important case not covered here is that of projective or quasi-projective morphisms, which has already been discussed in the remark in \protect\hyperlink{fga-3-i-section-B.1}{§B.1}.

\hypertarget{fga-3-i-section-B.3}{%
\subsection{(B.3) Decent by finite faithfully flat morphisms}\label{fga-3-i-section-B.3}}

Let \(\alpha\colon S'\to S\) be a \emph{finite} morphism, corresponding to a sheaf of algebras \({\mathscr{A}}'\) on \(S\) that is \emph{locally free} and of finite type as a sheaf of modules, and everywhere non-zero.
Then \(\alpha\) is a faithfully flat and quasi-compact morphism, to which we can thus apply the above results.
The data of a quasi-coherent sheaf \({\mathcal{F}}'\) on \(S'\) is equivalent to the data of the quasi-coherent sheaf \(\alpha_*({\mathcal{F}}')\) on \(S\) endowed with its \({\mathscr{A}}'\)-modules structure (noting that \({\mathscr{A}}'=\alpha_*({\mathscr{O}}_{S'})\)).
For simplicity, we also denote this sheaf on \(S\) by \({\mathcal{F}}'\).
The two inverse images \(p_i^*({\mathcal{F}}')\) of \({\mathcal{F}}'\) on \(S'\times_S S'\) similarly correspond to the quasi-coherent sheaves of \(({\mathscr{A}}'\otimes_{{\mathscr{O}}_S}{\mathscr{A}}')\)-modules \({\mathcal{F}}'\otimes_{{\mathscr{O}}_S}{\mathscr{A}}'\) and \({\mathscr{A}}'\otimes_{{\mathscr{O}}_S}{\mathcal{F}}'\).
The data of an \((S'\times_S S')\)-homomorphism from the former to the latter is equivalent to the data of a homomorphism of \(({\mathscr{A}}'\otimes{\mathscr{A}}')\)-modules, and, taking into account the fact that \({\mathscr{A}}'\) is locally free, this is equivalent to the data of a homomorphism of \(({\mathscr{A}}'\otimes{\mathscr{A}}')\)-modules:
\[
  {\mathscr{U}}
  = \mathscr{H}\kern -.5pt om_{{\mathscr{O}}_S}({\mathscr{A}}',{\mathscr{A}}')
  = {\mathscr{A}}'\otimes\check{{\mathscr{A}}}'
  \to \mathscr{H}\kern -.5pt om_{{\mathscr{O}}_S}({\mathcal{F}}',{\mathcal{F}}')
\]
i.e.~to the data, for every section \(\xi\) of \({\mathscr{U}}\) over an open subset \(V\), of a homomorphism of \({\mathscr{O}}_S\)-modules \(T_\xi\colon{\mathcal{F}}'|V\to{\mathcal{F}}'|V\) that satisfies the conditions

\leavevmode\vadjust pre{\hypertarget{fga-3-i-section-B.3-equation-3.1}{}}%
\[
  \begin{aligned}
    T_{f\xi}(x) &= fT_\xi(x),
  \\T_{\xi f}(x) &= T_\xi(fx),
  \end{aligned}
\tag{3.1}
\]

where \(f\) and \(x\) are (respectively) sections of \({\mathscr{A}}'\) and \({\mathcal{F}}'\) over an open subset of \(S\) that is contained inside \(V\).
Conditions (i) and (ii) of a descent data (cf.~\protect\hyperlink{fga-3-i-section-A.1.c}{§A.1.c}) can then be written (respectively) as

\leavevmode\vadjust pre{\hypertarget{fga-3-i-section-B.3-equation-3.2}{}}%
\[
  T_{1_U}(x) = x,
  \qquad\text{i.e. }T_{1_U}=\operatorname{id}_{{\mathcal{F}}'}
\tag{3.2}
\]

\leavevmode\vadjust pre{\hypertarget{fga-3-i-section-B.3-equation-3.3}{}}%
\[
  T_{\xi\eta} = T_\xi T_\eta.
\tag{3.3}
\]

In other words, \emph{a descent data on \({\mathcal{F}}'\) is equivalent to a representation of the sheaf \({\mathscr{U}}=\mathscr{H}\kern -.5pt om_{{\mathscr{O}}_S}({\mathscr{A}}',{\mathscr{A}}')\) of \({\mathscr{O}}_S\)-algebras in the sheaf \(\mathscr{H}\kern -.5pt om_{{\mathscr{O}}_S}({\mathcal{F}}',{\mathcal{F}}')\) of \({\mathscr{O}}_S\)-algebras that satisfies the two linearity conditions \protect\hyperlink{fga-3-i-section-B.3-equation-3.1}{(3.1)}}.
If we have a pairing of quasi-coherent sheaves on \(S'\):
\[
  {\mathcal{F}}'_1\times{\mathcal{F}}'_2 \to {\mathcal{F}}'_3
\]
\oldpage{190-22}(that we can think of as a pairing of sheaves of \({\mathscr{A}}'\)-modules on \(S\)), and gluing data on the \({\mathcal{F}}'_i\) defined by homomorphisms \(T_i\colon{\mathscr{U}}\to\mathscr{H}\kern -.5pt om_{{\mathscr{O}}_S}({\mathcal{F}}'_i,{\mathcal{F}}'_i)\) (for \(i=1,2,3\)), then these data are \emph{equivalent to the given pairing}, in the evident meaning of the phrase, if and only if the following condition is satisfied:

For every section \(\xi\) of \({\mathscr{U}}\) over an open subset, and denoting by \(\Delta\xi=\sum\xi'_i\otimes_{{\mathscr{A}}'}\xi''_i\) the section of \({\mathscr{U}}\otimes_{{\mathscr{A}}'}{\mathscr{U}}\) (where \({\mathscr{U}}\) is considered as an \({\mathscr{A}}'\)-module with its left structure) defined by the formula
\[
  \xi\cdot(fg) = \sum_i\xi'_i(f)\xi''_i(g)
\]
(where \(f\) and \(g\) are sections of \({\mathscr{A}}'\) over a smaller open subset), we have the formula

\leavevmode\vadjust pre{\hypertarget{fga-3-i-section-B.3-equation-3.4}{}}%
\[
  T_\xi^{(3)}(x\cdot y) = \sum_i T_{\xi'_i}^{(1)}x\cdot T_{\xi''_i}^{(2)}y
\tag{3.4}
\]

for every pair \((x,y)\) of sections of \({\mathscr{A}}'\) over a smaller subset.
(We can express this property by saying that the homomorphisms \(T^{(i)}\) are \emph{compatible with the diagonal map of \({\mathscr{U}}\)}, with respect to the given pair).
In particular, equations \protect\hyperlink{fga-3-i-section-B.3-equation-3.1}{(3.1)} to \protect\hyperlink{fga-3-i-section-B.3-equation-3.4}{(3.4)} allow us to understand, in terms of representations of algebras via diagonal maps, the descent data on a quasi-coherent sheaf of \emph{algebras} on \(S'\), and thus also (by restricting to commutative algebras) the descent data on an affine \(S'\)-scheme.

From here, we obtain an analogous interpretation of descent data on an arbitrary \(S'\)-prescheme \(X'\):
the data of such an \(X'\) is equivalent to the data of a prescheme \(X'\) \emph{over \(S\)} endowed with a homomorphism of \({\mathscr{O}}_S\)-algebras
\[
  {\mathscr{A}}'\to{\mathscr{O}}_{X'},
\]
and a descent data on \(X'\) is equivalent to the data of a sheaf homomorphism
\[
  {\mathscr{U}}
  \to \mathscr{H}\kern -.5pt om_{h^{-1}({\mathscr{O}}_S)}({\mathscr{O}}_{X'},{\mathscr{O}}_{X'})
\]
that is compatible with the morphism \(h\colon X'\to S'\) and that satisfies the conditions analogous to equations \protect\hyperlink{fga-3-i-section-B.3-equation-3.1}{(3.1)} to \protect\hyperlink{fga-3-i-section-B.3-equation-3.4}{(3.4)} above.

\leavevmode\vadjust pre{\hypertarget{fga-3-i-section-B.3-example-1}{}}%
\begin{rmenv}{Example 1}
(\emph{Weil}).
Suppose that \(S'/S\) is a \emph{Galois étale covering} with Galois group \(\Gamma\) (cf.~\protect\hyperlink{fga-3-i-section-A.3}{§A.3} and \protect\hyperlink{fga-3-i-section-A.4}{§A.4.d}.
Then a descent data on a quasi-coherent sheaf \({\mathcal{F}}'\) on \(S'\) (resp. on an \(S'\)-prescheme \(X'\)) is equivalent to the data of a representation of \(\Gamma\) by automorphisms of \((S',{\mathcal{F}}')\) (resp. of \((S',X')\)) that is compatible with the action of \(\Gamma\) on \(S'\).
\oldpage{190-23}This result is ``formal'', i.e.~it can be proven in terms of categories, but, from the point of view of this section, we also obtain the explicit structure of \({\mathscr{U}}\) (endowed with its ring structure, the ring homomorphism \({\mathscr{A}}'\to{\mathscr{U}}\), and the diagonal map), which is completely known thanks to the following result:
\emph{\({\mathscr{U}}\) admits, as a left \(A'\)-module, a basis given by the sections of \({\mathscr{U}}\) that correspond to elements of \(\Gamma\)}.

\end{rmenv}

\hypertarget{fga-3-i-section-B.3-example-2}{}
\begin{rmenv}{Example 2}

(\emph{Cartier}).
Let \(p\) be a prime number, and suppose that \(p{\mathscr{O}}_S=0\) (i.e.~that \({\mathscr{O}}_S\) is of \emph{characteristic \(p\)}), that \(({\mathscr{A}}')^p\subset{\mathscr{O}}_S={\mathscr{A}}\) (i.e.~that \(S'/S\) is \emph{radicial of height \(1\)}), and that the sheaf of algebras \({\mathscr{A}}'\) over \({\mathscr{A}}\) \emph{locally admits a \(p\)-basis} (i.e.~a family \((x_i)\) of sections such that \({\mathscr{A}}'\) is generated as an algebra by the \(x_i\) under the sole condition that \(x_i^p=0\)).
We suppose that the set of the \(i\) is finite, of cardinality \(n\).
Let \({\mathfrak{C}}\) be the sheaf of \(A\)-derivations of \(A'\), which is a locally free sheaf of rank \(n\) of \(A'\)-modules, and, furthermore, a sheaf of Lie \(p\)-algebras over \({\mathscr{A}}\) (but not over \({\mathscr{A}}'\)) that satisfies the following condition:

\leavevmode\vadjust pre{\hypertarget{fga-3-i-section-B.3-equation-3.5}{}}%
\[
  [X,fY] = X(f)Y + f[X,Y].
\tag{3.5}
\]

\end{rmenv}

\begin{itenv}{Lemma}

\({\mathscr{U}}=\mathscr{H}\kern -.5pt om_{{\mathscr{O}}_S}({\mathscr{A}}',{\mathscr{A}}')\) is generated, as an \({\mathscr{O}}_S\)-algebra endowed with an algebra homomorphism \({\mathscr{A}}'\to{\mathscr{U}}\), by the sub-left-\(A'\)-module \({\mathfrak{C}}\), with the following additional relations:

\leavevmode\vadjust pre{\hypertarget{fga-3-i-section-B.3-equation-3.6}{}}%
\[
  \begin{cases}
    Xf-fX &= X(f)
  \\XY-YX &= [X,Y]
  \\X^p &= X^{(p)}.
  \end{cases}
\tag{3.6}
\]

\end{itenv}

It follows from the above lemma that a descent data on the quasi-coherent sheaf \({\mathcal{F}}'\) on \(S'\) is equivalent to the data, for all \(X\in{\mathfrak{C}}\), of an \({\mathscr{O}}_S\)-endomorphism \(\overline{X}\) of \({\mathcal{F}}'\) that satisfies the following conditions:

\leavevmode\vadjust pre{\hypertarget{fga-3-i-section-B.3-equation-3.7}{}}%
\[
  \overline{fX} = f\overline{X}
\tag{3.7}
\]

\leavevmode\vadjust pre{\hypertarget{fga-3-i-section-B.3-equation-3.8}{}}%
\[
  \overline{X}(fx) = X(f)x + f\overline{X}(x)
\tag{3.8}
\]

\leavevmode\vadjust pre{\hypertarget{fga-3-i-section-B.3-equation-3.9}{}}%
\[
  \overline{[X,Y]} = [\overline{X},\overline{Y}]
\tag{3.9}
\]

\leavevmode\vadjust pre{\hypertarget{fga-3-i-section-B.3-equation-3.10}{}}%
\[
  \overline{X^{(p)}} = \overline{X}^p.
\tag{3.10}
\]

(This is what we may call a \emph{linear connection on \({\mathcal{F}}\)}, which is further \emph{flat} and \emph{compatible with the \(p\)-th powers}).
We can similarly write down the notion of a descent data on an \(S'\)-prescheme \(X'\), with equation \protect\hyperlink{fga-3-i-section-B.3-equation-3.4}{(3.4)} being replaced by the condition that the \(\overline{X}\) are \emph{derivations} of \({\mathscr{O}}_{X'}\).
\oldpage{190-24}Since the morphism \(S'\to S\) is radicial, \protect\hyperlink{fga-3-i-section-B.1-theorem-3}{Theorem 3} ensures that every such descent data is effective, and thus defines an \(S\)-prescheme \(X\).

Note that we have not needed to impose any flatness, non-singular, or finiteness hypotheses on \({\mathcal{F}}'\) or \(X'\).

\hypertarget{fga-3-i-section-B.4}{%
\subsection{(B.4) Application to rationality criteria}\label{fga-3-i-section-B.4}}

Let \(X\) be an \(S\)-prescheme such that the direct image of \({\mathscr{O}}_X\) on \(S\) is \({\mathscr{O}}_S\);
this property remains true for any flat base extension \(S'\to S\).
If \({\mathcal{F}}\) is an \emph{invertible sheaf} (i.e.~locally free of rank \(1\)) on \(X\), then there is a bijective correspondence between automorphisms of \({\mathcal{F}}\) (identified with the invertible sections of \({\mathscr{O}}_X\)) and invertible sections of \({\mathscr{O}}_S\).
So let \(s\) be a section of \(X\) over \(S\);
we define a \emph{section of \({\mathcal{F}}\) over \(s\)} to be a section of the invertible sheaf \(s^*({\mathcal{F}})\) on \(S\).
It follows from the above that, if \({\mathcal{F}}_i\) (for \(i=1,2\)) are invertible sheaves on \(X\), each endowed with a section over \(s\), and \emph{if \({\mathcal{F}}_1\) and \({\mathcal{F}}_2\) are isomorphic, then there exists exactly one isomorphism from \({\mathcal{F}}_1\) to \({\mathcal{F}}_2\) that is compatible with the sections in question} (i.e.~sending the first to the second).
We also, independently of the section \(s\), regard two invertible sheaves \({\mathcal{F}}_1\) and \({\mathcal{F}}_2\) on \(X\) as \emph{equivalent} if every point of \(S\) has an open neighbourhood \(U\) such that the restrictions of \({\mathcal{F}}_1\) and \({\mathcal{F}}_2\) to \(X|U\) are isomorphic.
Then \emph{every invertible sheaf \({\mathcal{F}}\) on \(X\) is equivalent to an invertible sheaf \({\mathcal{F}}_1\) endowed with a marked section over \(s\)} (we take \({\mathcal{F}}_1=Fs^*({\mathcal{F}})^{-1}\)), \emph{and \({\mathcal{F}}_1\) is determined up to isomorphism}.
In other words, the classification \emph{up to equivalence} of invertible sheaves on \(X\) is the same as the classification \emph{up to isomorphism} of invertible sheaves endowed with a marked section.

Since these properties remain true under flat extensions \(\alpha\colon S'\to S\) of the base (by replacing the section \(s\) with its inverse image \(s'\) under \(\alpha\)), we thus conclude, taking \protect\hyperlink{fga-3-i-section-B.1-theorem-1}{Theorem 1} into account:

\emph{With the prescheme \(X/S\) being as above, and admitting a section \(s\), let \(\alpha\colon S'\to S\) be a faithfully flat and quasi-compact morphism; let \({\mathcal{F}}'\) be an invertible sheaf on \(X'=X\times_S S'\).}
\emph{For \({\mathcal{F}}'\) to be equivalent to the inverse image on \(X'\) of an invertible sheaf \({\mathcal{F}}'\) on \(X\), it is necessary and sufficient that its inverse images \(p_1^*({\mathcal{F}}')\) and \(p_2^*({\mathcal{F}}')\) on \(X'\times_X X'=X\times_S(S'\times_S S')\) be equivalent.}
\emph{If this is the case, then \({\mathcal{F}}\) is determined up to equivalence.}
(We then say that \({\mathcal{F}}'\) is \emph{rational} on \(S\)).

Considering this principle in the case where \(\alpha\colon S'\to S\) is as in \protect\hyperlink{fga-3-i-section-B.3-example-1}{Example 1} and \protect\hyperlink{fga-3-i-section-B.3-example-2}{Example 2} in the previous section, we recover the \emph{rationality criteria of Weil and of Cartier}.
\oldpage{190-25}(We note that the authors restrict to the case where \(S\) and \(S'\) are spectra of fields;
a fortiori, \(S\) is then the spectrum of a local ring, and the equivalence relation introduced above is exactly the relation of being isomorphic).
The the first case, \({\mathcal{F}}'\) is rational on \(S\) if and only if its images under \(\Gamma\) are all equivalent to \({\mathcal{F}}'\).
To express the rationality criterion in the second case, we consider, more generally, the diagonal morphism \(X'\to X''=X'\times_X X'\) of \(X'/X\), with the corresponding sheaf of ideals \({\mathscr{I}}\) on \(X'\times_X X'\), and the sheaf \({\mathscr{I}}/{\mathscr{I}}^2\), which can be identified with its inverse image \(\Omega_{X'/X}^1\) on \(X\) (the \emph{sheaf of \(1\)-differentials of \(X'\) with respect to \(X\)}).
Since the restrictions of the \({\mathcal{F}}''_i=p_i({\mathcal{F}}')\) (for \(i=1,2\)) to the diagonal are isomorphic (since they are both isomorphic to \({\mathcal{F}}'\)), i.e.~\({\mathcal{F}}''_1({\mathcal{F}}''_2)^{-1}={\mathcal{F}}''\) has a restriction to the diagonal which is trivial, it follows that the restriction of \({\mathcal{F}}''\) to \((X'',{\mathscr{O}}_{X''}/{\mathscr{I}}^2)\) is given, up to isomorphism, by a well-defined element \(\xi\) of
\[
  \operatorname{H}^1(X'',{\mathscr{I}}/{\mathscr{I}}^2) = \operatorname{H}^1(X',\Omega_{X'/X}^1).
\]
Also, being precise, we have \(\Omega_{X'/X}^1=\Omega_{S'/S}^1\otimes_{{\mathscr{O}}_S}{\mathscr{O}}_X\), and thus, \emph{if \(\Omega_{S'/S}^1\) is locally free on \(S\)} (as in the Cartier case), \emph{then \(\xi\) defines a section of \(\operatorname{R}^1f'({\mathscr{O}}_{X'})\otimes\Omega_{S'/S}^1\) on \(S'\)} (called the \emph{Atiyah--Cartier class of the invertible sheaf \({\mathcal{F}}\) on \(X'/S\)}) \emph{whose vanishing is necessary and sufficient for the inverse images of \({\mathcal{F}}'\) under the two projections of}
\[
  (X'',{\mathscr{O}}_{X''}/{\mathscr{I}}^2) = X\times_S(S'',{\mathscr{O}}_{S''}/{\mathscr{J}}^2)
\]
\emph{to \(X'\) to be equivalent} (where \({\mathscr{J}}\) is the sheaf of ideals on \(S''=S'\times_S S'\) defined by the diagonal morphism \(S'\to S'\times_S S'\)).
This vanishing is thus trivially \emph{necessary} for the inverse images of \({\mathcal{F}}'\) on \(X''=X\times_S S''\) itself to be equivalent, and thus also for \({\mathcal{F}}\) to be equivalent to the inverse image of an invertible sheaf \({\mathcal{F}}\) on \(X\).
The Atiyah--Cartier class can also be understood as the obstruction to the existence, locally over \(S'\), of a \emph{connection} of \({\mathcal{F}}'\) relative to the derivations of \(X'/X\), with such a connection further being determined when we know the derivations of \({\mathcal{F}}'\) corresponding to the natural extensions of derivations of \(S'/S\) to \(X'\).
From this, and the results of the previous section, we easily conclude that, in the case of the aforementioned \protect\hyperlink{fga-3-i-section-B.3-example-1}{Example 2}, and when \(X/S\) admits a section, the vanishing of the Atiyah--Cartier class is also sufficient for \({\mathcal{F}}'\) to be rational on \(S\).

\hypertarget{fga-3-i-section-B.5}{%
\subsection{(B.5) Application to the restriction of the base scheme to an abelian scheme}\label{fga-3-i-section-B.5}}

Let \(S\) be a prescheme.
\oldpage{190-26}We define an \emph{abelian scheme} over \(S\) to be a simple proper scheme \(X\) over \(S\) whose fibres at the points \(x\in S\) are schemes of abelian varieties over the \(k(x)\).
Suppose that \(S\) is Noetherian and \emph{regular} (i.e.~that its local rings are regular), then we can show, using the \emph{connection theorem} of Murre {[}\protect\hyperlink{ref-Mur1958}{16}{]}
(at least in the case ``of equal characteristics'', where the cited theorem is currently proven) that \emph{every rational section of \(X\) over \(S\) is everywhere defined} (i.e.~is a section) (which generalises a classical theorem of Weil).
It then follows, more generally, that, if \(X'\) is a simple scheme over \(S\), then every rational \(S\)-map from \(X'\) to \(X\) is everywhere defined.
From this, we obtain the following, which generalises a result of Chow--Lang:
\emph{with \(S\) Noetherian and regular, and \(K\) denoting its ring of rational functions} (a direct sum of fields), \emph{let \(X\) be an abelian scheme over \(K\); if \(X\) is isomorphic to a \(K\)-scheme of the form \(X_0\times_S\operatorname{Spec}(K)\), where \(X_0\) is an abelian scheme over \(S\), then \(X_0\) is determined up to unique isomorphism.}

Using the above uniqueness result, we see that the question of restriction of the base to \(X\) is local on \(S\) (and thus that it suffices to know how to do the restriction to \(\operatorname{Spec}({\mathscr{O}}_x)\), where \(x\in S\)).
In the same way, we see that, if \(S'\to S\) is a \emph{simple} morphism of finite type, if \(Y'\) is the ring of rational functions of \(S'\), and if \(X\otimes_K K'\) is of the form \(X'_0\times_{S'}\operatorname{Spec}(K')\), \emph{then \(X'_0\) is endowed with a canonical descent data with respect to \(\alpha\)}.
Taking \protect\hyperlink{fga-3-i-section-B.1-theorem-3}{Theorem 3} into account, we thus conclude:

\leavevmode\vadjust pre{\hypertarget{fga-3-i-section-B.5-proposition-5.1}{}}%
\begin{itenv}{Proposition 5.1}
Let \(S\) be an irreducible regular Noetherian prescheme, with field of rational functions \(Y\), let \(K'\) be a finite extension of \(K\) that is \emph{unramified over \(S\)}, let \(S'\) be the normalisation of \(S\) in \(K'\) (which is thus an étale cover of \(S\)), and let \(X\) be an abelian scheme over \(K\) such that \(X\otimes_K K'\) is of the form \(X'_0\times_{S'}\operatorname{Spec}(K')\), where \(X'_0\) is a projective abelian scheme over \(S'\).
Then \(X\) is of the form \(X_0\times_S\operatorname{Spec}(K)\), where \(X_0\) is a projective abelian scheme over \(S\).

\end{itenv}

\begin{rmenv}{Remark}
The speaker does not know if we can replace the hypothesis that \(S'\to S\) be a surjective étale cover (which allows us to apply \protect\hyperlink{fga-3-i-section-B.1-theorem-3}{Theorem 3}) with the hypothesis that it is instead a \emph{simple} and \emph{surjective} morphism of finite type (not even if we suppose that it is an étalement), or if the proposition still holds true without supposing that \(X'_0\) is projective over \(S'\) (a condition which could be automatically satisfied).

\end{rmenv}

\hypertarget{fga-3-i-section-B.6}{%
\subsection{(B.6) Application to local triviality and isotriviality criteria}\label{fga-3-i-section-B.6}}

Let \(S\) be a prescheme, \(G\) a ``\emph{prescheme of groups}'' over \(S\), \(P\) a prescheme over \(S\) on which ``\emph{\(G\) acts}'' (on the right).
We say that \(P\) is \emph{formally principal homogeneous} for \(G\) if the well-known morphism
\[
  G\times_S P \to P\times_S P
\]
\oldpage{190-27}(induced from the actions of \(G\) on \(P\)) is an \emph{isomorphism}.
From now on, we assume \(G\) to be \emph{flat} over \(S\) (and thus faithfully flat over \(S\)), and we reserve the name of \emph{principal homogeneous bundle} for \(G\) for a formally principal homogeneous bundle \(P\) that is \emph{faithfully flat} and \emph{quasi-compact} over \(S\).
It is immediate that this is equivalent to being able to find a \emph{faithfully flat} and \emph{quasi-compact extension} \(S'\to S\) of the base \(S\) such that the formally principal homogeneous bundle \(P'=P\times_S S'\) for \(G'=G\times_S S'\) is \emph{trivial}, i.e.~isomorphic to \(G'\) (i.e.~admitting a section);
we can take, in particular, \(S'=P\).
Note also that, if \(S\) is locally Noetherian, then the faithfully-flat hypothesis on \(P\) is equivalent to the hypothesis that \(\overline{P}_S=P\times_S\operatorname{Spec}(\overline{{\mathscr{O}}}_s)\) be faithfully flat over \(\overline{{\mathscr{O}}}_s\) for all \(s\in S\) (where \(\overline{{\mathscr{O}}}_s\) denotes the completion of the local ring \({\mathscr{O}}_s\)), as follows from the fact that \(\overline{{\mathscr{O}}}_s\) is faithfully flat over \({\mathscr{O}}_s\).
Also, if \(P\) is of finite type over \(S\), and \(S\) is locally Noetherian, then the set of points \(s\) satisfying the above condition is constructible, and so, if \(S\) is a ``Jacobson prescheme'' (for example, a scheme of finite type over a field, or, more generally, over a Jacobson ring), then it suffices to verify the condition in question for the \emph{closed} points of \(S\).
This leads us to the case where the base is the spectrum of a complete local ring \(A\).
If \(S=\operatorname{Spec}(A)\) (with \(A\) a complete Noetherian local ring), and if \(P\) is of finite type over \(S\), then the faithful flatness of \(P/S\) is also equivalent to the existence of an \(S'\) that is \emph{finite and flat} over \(S\) such that \(P'\) is trivial, and, if, further, \(G\) is \emph{simple} over \(S\), then we can suppose \(S'\) to be \emph{étale} over \(S\).
Then, if, further, the residue field of \(A\) is algebraically closed (the ``\emph{geometric case}''), then \(P\) is faithfully flat over \(A\) if and only if it is trivial.
Thus, if \(S\) is an algebraic prescheme over an algebraically closed field, and if \(G\) is simple and of finite type over \(S\), then we see that the faithfully-flat condition on \(S\) is equivalent to the condition of being analytically trivial (SLF) of Serre {[}\protect\hyperlink{ref-Ser1958a}{22}{]}.

We can consider other, stronger, types of conditions on \(P\), that have a ``local triviality'' nature.
In particular, we say that \(P\) is \emph{isotrivial} (resp. \emph{strictly isotrivial}) if, for all \(s\in S\), there exists an open neighbourhood \(U\) of \(S\), and a \emph{finite and faithfully flat} morphism (resp. a \emph{surjective étale covering}) \(U'\to U\) such that \(P'=P\times_S U'\) is trivial.
(We stray from the terminology of Serre {[}\protect\hyperlink{ref-GD1960}{10}{]}, which uses ``locally isotrivial'' for what we call ``strictly isotrivial'').
Strict isotriviality is mainly useful if \(G\) is simple over \(S\), but is, however, an inadequate notion in other cases.

If \(G\) is \emph{affine} over \(S\), then every principal homogeneous bundle \(P\) for \(G\) is affine, by \protect\hyperlink{fga-3-i-section-B.2}{§B.2}, whence the possibility, thanks to \protect\hyperlink{fga-3-i-section-B.1-theorem-2}{Theorem 2}, to ``descend'' from such bundles by faithfully flat and quasi-compact morphisms.
\oldpage{190-28}Taking, in particular, \(G=\operatorname{GL}(n)_S\), defined by the condition that the functor \(S'\mapsto\operatorname{Hom}(S',G)\) of \(S\)-preschemes (with values in the category of groups) can be identified with the functor \(\operatorname{GL}(n)(S')=\operatorname{GL}(n,\operatorname{H}^0(S',{\mathscr{O}}_{S'}))\) described in \protect\hyperlink{fga-3-i-section-A.4}{§A.4.e}.
Using the facts

\begin{enumerate}
\def\labelenumi{\roman{enumi}.}
\item
  that every principal homogeneous bundle for \(G\) (resp. every locally free sheaf of rank \(n\) on \(S\)) becomes isomorphic to the ``trivial'' object \(G\) (resp. \({\mathscr{O}}_S^n\)) under a suitable faithfully flat and quasi-compact extension of \(S\);
\item
  that we can descend the type of objects in question (principal homogeneous bundles for \(G\), resp. locally free sheaves of rank \(n\)) by such morphisms; and, finally
\item
  that the automorphism group of the trivial bundle on any \(S'/S\) is functorially isomorphic to the automorphism group of the trivial locally free sheaf of rank \(n\) on \(S'\),
\end{enumerate}

we ``formally'' conclude that it is ``equivalent'' to give, on \(S\) (or on some \(S'/S\)) a principal homogeneous bundle for the group \(G\), or to give a locally free sheaf of rank \(n\).
(More precisely, we have an \emph{equivalence of fibred categories}).
We thus conclude, in particular:

\leavevmode\vadjust pre{\hypertarget{fga-3-i-section-B.6-proposition-6.1}{}}%
\begin{itenv}{Proposition 6.1}
Every principal homogeneous bundle for the group \(\operatorname{GL}(n)_S\) is locally trivial.

\end{itenv}

By known arguments, we thus conclude the same result for others structure groups such as \(\operatorname{SL}(n)_S\), \(\operatorname{Sp}(n)_S\), and products of such groups.
We thus also conclude that, if \(F\) is a closed subgroup of \(G=\operatorname{GL}(n)_S\) that is flat over \(S\), and such that the quotient \(G/F\) exists, and such that \(G\) is an isotrivial (resp. strictly isotrivial) principal homogeneous bundle on \(G/F\), of structure group \(F\times_S(G/F)\), then \emph{every} principal homogeneous bundle of structure group \(F\) is isotrivial (resp. strictly isotrivial).
This applies to all the ``linear groups'' on \(S\) that have been used up until now, and, in particular, to the case where \(G=S\times_k\Gamma\), with \(S\) a prescheme over the field \(k\), and \(\Gamma\) a linear group (in the classical sense) over \(k\) (and thus in particular simple).
This thus answers, for such groups, a question of Serre (\emph{loc. cit.}).

We also point out that, for most groups (linear or not) that are simple over \(S\) that we know of, and certainly for all those of the form \(S\times_k\Gamma\) as above, we can show that every isotrivial principal homogeneous bundle is strictly isotrivial, which answers, in particular, another question of Serre (\emph{loc. cit.} 1--14), taking into account the fact that a homogeneous principal bundle obtained by a descent \emph{à la} Cartier (cf.\protect\hyperlink{fga-3-i-section-B.3-example-2}{Example 2}) is obviously isotrivial.

\begin{rmenv}{Remark}
One of the essential difficulties in these questions (setting aside the question of the existence of quotient schemes) is the lack of effectiveness criteria for a descent data along a faithfully flat \emph{non-finite} morphism.

\end{rmenv}

\hypertarget{fga-3.ii}{%
\section{The existence theorem and the formal theory of modules}\label{fga-3.ii}}

\providecommand{\scr}[1]{{\mathscr{#1}}}
\renewcommand{\cal}[1]{{\mathcal{#1}}}
\renewcommand{\frak}[1]{{\mathfrak{#1}}}
\renewcommand{\geq}{\geqslant}
\renewcommand{\leq}{\leqslant}

\providecommand{\simto}{\xrightarrow{\sim}}
\providecommand{\simfrom}{\xleftarrow{\sim}}
\providecommand{\Set}{\mathtt{Set}}

\providecommand{\id}{\operatorname{id}}
\providecommand{\Hom}{\operatorname{Hom}}
\providecommand{\repHom}{\underline{\Hom}}
\providecommand{\Aut}{\operatorname{Aut}}
\providecommand{\repAut}{\underline{\Aut}}
\providecommand{\HH}{\operatorname{H}}
\providecommand{\RR}{\operatorname{R}}
\providecommand{\GL}{\operatorname{GL}}
\providecommand{\Ga}{\operatorname{G_a}}
\providecommand{\Gm}{\operatorname{G_m}}
\providecommand{\SL}{\operatorname{SL}}
\providecommand{\Sp}{\operatorname{Sp}}
\providecommand{\Spec}{\operatorname{Spec}}
\providecommand{\Pro}{\operatorname{Pro}}
\providecommand{\Ext}{\operatorname{Ext}}
\providecommand{\op}{\circ}

{[}FGA 3.II{]}
Grothendieck, A.
``Technique de descente et théorèmes d'existence en géométrie algébrique, II: Le théorème d'existence et théorie formelle des modules''.
\emph{Séminaire Bourbaki} \textbf{12} (1959--60), Talk no. 195.

\emph{{[}Trans.{]}}
Sections 4.1 to 4.5 were numbered A.1 to A.5 in the original;
sections 4.6 to 4.10 were numbered C.1 to C.5.

\hypertarget{a.-representable-and-pro-representable-functors}{%
\subsection*{\texorpdfstring{\textbf{A.} Representable and pro-representable functors}{A. Representable and pro-representable functors}}\label{a.-representable-and-pro-representable-functors}}
\addcontentsline{toc}{subsection}{\textbf{A.} Representable and pro-representable functors}

\hypertarget{fga-3-ii-section-A.1}{%
\subsection{(A.1) Representable functors}\label{fga-3-ii-section-A.1}}

Let \({\mathcal{C}}\) be a category.
\oldpage{195-01}For all \(X\in{\mathcal{C}}\), let \(h_X\) be the contravariant functor from \({\mathcal{C}}\) to the category \(\mathtt{Set}\) of sets given by
\[
  \begin{aligned}
    h_X\colon {\mathcal{C}} &\to \mathtt{Set}
  \\Y&\mapsto \operatorname{Hom}(Y,X).
  \end{aligned}
\]
If we have a morphism \(X\to X'\) in \({\mathcal{C}}\), then this evidently induces a morphism \(h_X\to h_{X'}\) of functors;
\(h_X\) is a covariant functor in \(X\), i.e.~we have defined a \emph{canonical covariant functor}
\[
  h\colon {\mathcal{C}} \to \operatorname{Hom}({\mathcal{C}}^\circ,\mathtt{Set})
\]
from \({\mathcal{C}}\) to the category of covariant functors from the dual \({\mathcal{C}}^\circ\) of \({\mathcal{C}}\) to the category of sets.
We then recall:

\leavevmode\vadjust pre{\hypertarget{fga-3-ii-section-A.1-proposition-1.1}{}}%
\begin{itenv}{Proposition 1.1}
This functor \(h\) is \emph{faithfully flat};
in other words, for every pair \(X,X'\) of objects of \({\mathcal{C}}\), the natural map
\[
\operatorname{Hom}(X,X') \to \operatorname{Hom}(h_X,h_{X'})
\]
is \emph{bijective}.

\end{itenv}

In particular, if a functor \(F\in\operatorname{Hom}({\mathcal{C}}^\circ,\mathtt{Set})\) is isomorphic to a functor of the form \(h_X\), then \emph{\(X\) is determined up to unique isomorphism}.
We then say that the functor \(F\) is \emph{representable}.
The above proposition then implies that the canonical functor \(h\) defines an \emph{equivalence} between the category \({\mathcal{C}}\) and the full subcategory of \(\operatorname{Hom}({\mathcal{C}}^\circ,\mathtt{Set})\) consisting of representable functors.
This fact is the basis of \emph{the idea of a ``solution of a universal problem''}, with such a problem always consisting of examining if a given (contravariant, as here, or covariant, in the dual case) functor from \({\mathcal{C}}\) to \(\mathtt{Set}\) is representable.
\oldpage{195-02}Note further that, even by the definition of products in a category {[}\protect\hyperlink{ref-Gro1957}{6}{]}, the functor \(h\colon X\mapsto h_X\) commutes with products whenever they exist (and, more generally, with finite or infinite projective limits, and, in particular, with fibred products, taking ``kernels'' {[}\protect\hyperlink{ref-Gro1958a}{7}{]}, etc., whenever such things exist): we have an isomorphism of functors
\[
  h_{X\times X'} \xrightarrow{\sim}h_X\times h_{X'}
\]
whenever \(X\times X'\) exists, i.e.~we have functorial (in \(Y\)) bijections
\[
  h_{X\times X'} \xrightarrow{\sim}h_X(Y)\times h_{X'}(Y).
\]
In particular, the data of a morphism
\[
  X\times X' \to X''
\]
in \({\mathcal{C}}\) (i.e.~of a ``\emph{composition law}'' in \({\mathcal{C}}\) between \(X\), \(X'\), and \(X''\)) is equivalent to the data of a morphism \(h_{X\times X'}=h_X\times h_{X'}\to h_{X''}\), i.e.~to the data, for all \(Y\in{\mathcal{C}}\), of a composition law of \emph{sets}
\[
  h_X(Y)\times h_{X'}(Y) \to h_{X''}(Y)
\]
such that, for every morphism \(Y\to Y'\) in \({\mathcal{C}}\), the system of set maps
\[
  h_{X^{(i)}}(Y) \to h_{X^{(i)}}(Y')
  \qquad\text{(for }i=0,1,2\text{)}
\]
is a morphism for the two composition laws, with respect to \(Y\) and \(Y'\).
In this way, we see that the idea of a ``\({\mathcal{C}}-group\)'' structure, or a ``\({\mathcal{C}}\)-ring'' structure, etc. on an object \(X\) of \({\mathcal{C}}\) can be expressed in the most manageable way (in theory as much as in practice) by saying that, for every \(Y\in{\mathcal{C}}\), we have a group law (resp. ring law, etc.) in the usual sense on the set \(h_X(Y)\), with the maps \(h_X(Y)\to h_X(Y')\) corresponding to morphisms \(Y\to Y'\) that should be group homomorphisms (resp. ring homomorphisms, etc.).
This is the most intuitive and manageable way of defining, for example, the various classical groups \(\operatorname{G_a}\), \(\operatorname{G_m}\), \(\operatorname{GL}(n)\), etc. on a prescheme \(S\) over an arbitrary base, and of writing the classical relations between these groups, or of placing a ``vector bundle'' structure on the affine scheme \(V({\mathscr{F}})\) over \(S\) defined by a quasi-coherent sheaf \({\mathscr{F}}\), and of defining and studying the many associated flag varieties (Grassmannians, projective bundles), etc.;
\emph{the general yoga is purely and simply identifying, using the canonical functor \(h\), the objects of \({\mathcal{C}}\) with particular contravariant functors (namely, representable functors) from \({\mathcal{C}}\) to the category of sets}.

\oldpage{195-03}The usual procedure of reversing the arrows that is necessary, for example, in the case of affine schemes in order to pass from the geometric language to the language of commutative algebra, leads us to dualise the above considerations, and, in particular, to also introduce \emph{covariant representable functors \({\mathcal{C}}\to\mathtt{Set}\)}, i.e.~those of the form \(Y\mapsto\operatorname{Hom}(X,Y)=h'_X(Y)\).

\hypertarget{fga-3-ii-section-A.2}{%
\subsection{(A.2) Pro-representable functors, pro-objects}\label{fga-3-ii-section-A.2}}

Let \({\mathcal{X}}=(X_i)_{i\in I}\) be a projective system of objects of \({\mathcal{C}}\);
there is a corresponding covariant functor
\[
  h'_{{\mathcal{X}}} = \varinjlim_i h'_{X_i}
\]
which can be written more explicitly as
\[
  h'_{{\mathcal{X}}}(Y) = \varinjlim_i h'_{X_i}(Y) = \varinjlim_i\operatorname{Hom}(X_i,Y)
\]
which is a functor from \({\mathcal{C}}\) to \(\mathtt{Set}\).
A functor from \({\mathcal{C}}\) to \(\mathtt{Set}\) that is isomorphic to a functor of this type \emph{with \(I\) filtered} is said to be \emph{pro-representable}.
By the previous section, these are exactly the functors that are isomorphic to \emph{filtered inductive limits of representable functors}.
Let \({\mathcal{X}}'=(X_j)_{j\in J}\) be another filtered projective system in \({\mathcal{C}}\) (indexed by another filtered preordered set of indices \(J\)).
Then we can easily show that we have a canonical bijection
\[
  \operatorname{Hom}(h_{{\mathcal{X}}'},h_{{\mathcal{X}}}) = \varprojlim_j\varinjlim_i\operatorname{Hom}(X_i,X'_j)
\]
(generalising \protect\hyperlink{fga-3-ii-section-A.1-proposition-1.1}{§A, Proposition 1.1}).
This leads to introducing the \emph{category \(\operatorname{Pro}({\mathcal{C}})\) of pro-objects of \({\mathcal{C}}\)}, whose objects are projective systems of objects of \({\mathcal{C}}\) (indexed by arbitrary filtered preordered sets of indices), and whose morphisms between objects \({\mathcal{X}}=(X_i)_{i\in I}\) and \({\mathcal{X}}'=(X_j)_{j\in J}\) are given by
\[
  \operatorname{Pro}\operatorname{Hom}({\mathcal{X}},{\mathcal{X}}') = \varprojlim_j\varinjlim_i\operatorname{Hom}(X_i,X'_j),
\]
with the composition of pro-homomorphisms being evident.
By the very construction itself, \({\mathcal{X}}\mapsto h'_{{\mathcal{X}}}\) can be considered as a contravariant functor in \({\mathcal{X}}\), establishing an \emph{equivalence between the dual category of the category \(\operatorname{Pro}({\mathcal{C}})\) of pro-objects of \({\mathcal{C}}\) and the category of pro-representable covariant functors from \({\mathcal{C}}\) to \(\mathtt{Set}\)}.
\oldpage{195-04}Of course, an object \(X\) of \({\mathcal{C}}\) canonically defines a pro-object, denoted again by \(X\), so that \emph{\({\mathcal{C}}\) is equivalent to a full subcategory of \(\operatorname{Pro}({\mathcal{C}})\)}.
Then, if \({\mathcal{X}}=(X_i)_{i\in I}\) is an arbitrary pro-object of \({\mathcal{C}}\), then (with the above identification) we have that
\[
  {\mathcal{X}} = \varprojlim_i X_i
\]
with the projective limit being \emph{taken in \(\operatorname{Pro}({\mathcal{C}})\)} (since \(h_{{\mathcal{X}}}=\varinjlim_i h_{X_i}\)).

We draw attention to the fact that, even if the projective limit of the \(X_i\) \emph{exists in \({\mathcal{C}}\)}, it will generally \emph{not} be isomorphic to the projective limit \({\mathcal{X}}\) in \(\operatorname{Pro}({\mathcal{C}})\), as is already evident in the case where \({\mathcal{C}}\) is the category of sets.
We note that, by the definition itself, \(\varprojlim{}_{{\mathcal{C}}}X_i=L\) is defined by the condition that the functor
\[
  \varprojlim_i\operatorname{Hom}_{{\mathcal{C}}}(Y,X_i)=\operatorname{Hom}_{\operatorname{Pro}({\mathcal{C}})}(Y,{\mathcal{X}})
\]
in \(Y\in{\mathcal{C}}\) and with values in \(\mathtt{Set}\) be representable via \({\mathcal{L}}\), i.e.~that it be isomorphic to \(\operatorname{Hom}_{{\mathcal{C}}}(Y,{\mathcal{L}})\);
then \emph{\(\lim{}_{{\mathcal{C}}}X_i\) is already defined in terms of the \emph{pro-object} \({\mathcal{X}}\)}, and, in a precise way, depends functorially on the pro-object \({\mathcal{X}}\) whenever it is defined;
there is therefore no problem with denoting it by \(\lim{}_{{\mathcal{C}}}({\mathcal{X}})\).
If projective limits in \({\mathcal{C}}\) always exist, then \(\lim{}_{{\mathcal{C}}}({\mathcal{X}})\) is a functor from \(\operatorname{Pro}({\mathcal{C}})\) to \({\mathcal{C}}\), and there is a canonical homomorphism of functors \(\lim_{\mathcal{C}}({\mathcal{X}})\to{\mathcal{X}}\).
Since every (covariant, say, for simplicity) functor
\[
  F\colon {\mathcal{C}} \to {\mathcal{C}}'
\]
can be extended in an obvious way to a functor
\[
  \operatorname{Pro}(F)\colon \operatorname{Pro}({\mathcal{C}}) \to \operatorname{Pro}({\mathcal{C}}'),
\]
it follows that, if projective limits always exist in \({\mathcal{C}}'\), then \(F\) also canonically defines a composite functor
\[
  \overline{F} = \varprojlim{}_{{\mathcal{C}}'}\colon \operatorname{Pro}({\mathcal{C}}) \to {\mathcal{C}}'
\]
sending \({\mathcal{X}}=(X_i)_{i\in I}\) to \(\varprojlim{}_{{\mathcal{C}}'}F(X_i)\).

A pro-object \({\mathcal{X}}\) is said to be a \emph{strict pro-object} if it is isomorphic to a pro-object \((X_i)_{i\in I}\), where the transition morphisms \(X_i\to X_j\) are \emph{epimorphisms};
a functor defined by such an object is said to be \emph{strictly pro-representable}.
We can thus further demand that \(I\) be a filtered \emph{ordered} set, and that every epimorphism \(X_i\to X'\) be equivalent to an epimorphism \(X_i\to X_j\) for some suitable \(j\in I\) (uniquely determined by this condition).
\oldpage{195-05}Under these conditions, the projective system \((X_i)_{i\in I}\) is determined \emph{up to unique isomorphism} (in the usual sense of isomorphisms of projective systems).
It thus follows that \emph{the projective limit of a projective system \({\mathcal{X}}^{(\alpha)}\) of strict pro-objects always exists in \(\operatorname{Pro}({\mathcal{C}})\)}, and that, with the above notation of \(F\) and \(\overline{F}\), we have that
\[
  \overline{F}\varprojlim_\alpha{\mathcal{X}}^{(\alpha)} = \varprojlim_\alpha{}_{{\mathcal{C}}'}F(X^{(\alpha)}).
\]
In particular, if every pro-object of \({\mathcal{C}}\) is strict (cf.~the previous section), then the extended functor \(\overline{F}\) commutes with projective limits.

\hypertarget{fga-3-ii-section-A.3}{%
\subsection{(A.3) Characterisation of pro-representable functors}\label{fga-3-ii-section-A.3}}

Let \({\mathcal{C}}\) and \({\mathcal{C}}'\) be categories in which all finite projective limits (i.e.~limits over finite, not necessarily filtered, preordered sets) exist, or, equivalently, in which finite products and finite fibred products exist (which implies, in particular, the exists of a ``right-unit object'' \(e\) such that \(\operatorname{Hom}(X,e)\) consists of only on element for all \(X\)).
Let \(F\) be a covariant functor from \({\mathcal{C}}\) to \({\mathcal{C}}'\).
Then the following conditions are equivalent:

\begin{enumerate}
\def\labelenumi{\roman{enumi}.}
\tightlist
\item
  \(F\) commutes with finite projective limits;
\item
  \(F\) commutes with finite products and with finite fibred products;
\item
  \(F\) commutes with finite products, and, for every exact diagram
  \[
      X\to X'\rightrightarrows X''
    \]
  in \({\mathcal{C}}\) (cf.~\protect\hyperlink{fga-3-i-section-A.2-definition-2.1}{FGA 3.I, A, Definition 2.1}), the image of the diagram under \(F\)
  \[
      F(X)\to F(X')\rightrightarrows F(X'')
    \]
  is exact.
\end{enumerate}

We then say that \(F\) is \emph{left exact}.

In what follows, we assume that finite projective limits always exist in \({\mathcal{C}}\).
It is then immediate from the definitions that a representable functor is left exact, and, by taking the limit, that \emph{a pro-representable functor is left exact}.

To obtain a converse, let
\[
  F\colon {\mathcal{C}} \to \mathtt{Set}
\]
be a covariant functor, and let \(X\in{\mathcal{C}}\) and \(\xi\in F(X)\).
\oldpage{195-06}We say that \(\xi\) (or the pair \((X,\xi)\)) is \emph{minimal} if, for all \(X'\in{\mathcal{C}}\) and all \(\xi'\in F(X')\), and for every strict monomorphism (cf.~\protect\hyperlink{fga-3-i-section-A.2}{FGA 3.I, A, §2}) \(u\colon X'\to X\) such that \(\xi=F(u)(\xi')\), \(u\) is an isomorphism.
We also say that a pair \((X,\xi)\) \emph{dominates} \((X'',\xi'')\) (where \(xi\in F(X)\) and \(\xi''\in F(X'')\)) if there exists a morphism \(v\colon X\to X''\) such that \(\xi''=F(v)(\xi)\);
\emph{if \(\xi\) is minimal, and if \(F\) is left exact, then this morphism \(v\) is unique};
\emph{if \(\xi''\) is minimal, then \(v\) is surjective}.
From this we easily deduce the following proposition:

\hypertarget{fga-3-ii-section-A.3-proposition-3.1}{}
\begin{itenv}{Proposition 3.1}

For \(F\) to be strictly pro-representable, it is necessary and sufficient that it satisfy the following two conditions:

\begin{enumerate}
\def\labelenumi{\roman{enumi}.}
\tightlist
\item
  \(F\) is left exact; and
\item
  every pair \((X,\xi)\), with \(\xi\in F(X)\), is dominated by some \emph{minimal} pair.
\end{enumerate}

\end{itenv}

This second condition is trivial if every object of \({\mathcal{C}}\) is Artinian (by taking a sub-object \(X'\) of \(X\) that is minimal amongst those for which there exists some \(\xi'\in F(X')\) such that \(\xi\) is the image of \(\xi'\)).
Whence:

\begin{itenv}{Corollary}
Let \({\mathcal{C}}\) be a category whose objects are all Artinian and in which all finite projective limits exist.
Then the pro-representable functors from \({\mathcal{C}}\) to \(\mathtt{Set}\) are exactly the left exact functors, and they are in fact strictly pro-representable.

\end{itenv}

This last fact also implies that \emph{every pro-object of \({\mathcal{C}}\) is then strict}.

\hypertarget{fga-3-ii-section-A.4}{%
\subsection{(A.4) Example: groups of Galois type, pro-algebraic groups}\label{fga-3-ii-section-A.4}}

If \({\mathcal{C}}\) is the category of ordinary finite groups, then \(\operatorname{Pro}({\mathcal{C}})\) is equivalent to the category of totally disconnected compact topological\footnote{\emph{{[}Trans.{]}} Here the word ``Hausdorff'' is implicit.} groups.
It is groups of this type, and their generalisations, obtained by replacing ordinary finite groups with schemes of finite groups over a given base prescheme (for example, finite algebraic groups over a field \(k\)), that serve as fundamental groups, homotopy groups, and absolute and relative homology groups for preschemes.
In all these examples, the corollary to \protect\hyperlink{fga-3-ii-section-A.3-proposition-3.1}{Proposition 3.1} applies, and it is indeed by the associated functor that the \(\pi_1\) should be defined {[}\protect\hyperlink{ref-Gro1958a}{7}{]}.
It is the same if we start with the category of algebraic or quasi-algebraic groups over a field (or, more generally, over a Noetherian prescheme): we recover the ``\emph{pro-algebraic groups}'' of Serre {[}\protect\hyperlink{ref-Ser1958}{21}{]}.

\hypertarget{fga-3-ii-section-A.5}{%
\subsection{(A.5) Example: ``formal varieties''}\label{fga-3-ii-section-A.5}}

Let \(\Lambda\) be a Noetherian ring, \({\mathcal{C}}\) the category of \(\Lambda\)-algebras that are Artinian modules of finite type over \(\Lambda\) (or, more concisely, \emph{Artinian \(\Lambda\)-algebras}).
\oldpage{195-07}The conditions of the corollary to \protect\hyperlink{fga-3-ii-section-A.3-proposition-3.1}{Proposition 3.1} are then satisfied.
Here, the category \(\operatorname{Pro}({\mathcal{C}})\) is equivalent to the category of \emph{topological algebras} \(O\) over \(\Lambda\) that are isomorphic to topological projective limits
\[
  O = \varprojlim O_i
\]
of algebras \(O_i\in{\mathcal{C}}\), i.e.~those whose topology is \emph{linear}, \emph{separated}, and \emph{complete}, and such that, for every open ideal \({\mathfrak{J}}_i\) of \(O\), the algebra \(O/{\mathfrak{J}}_i\) is an \emph{Artinian} algebra over \(\Lambda\).
The functor \({\mathcal{C}}\to\mathtt{Set}\) associated to such an algebra is exactly
\[
  \begin{aligned}
    F(A)
    &= h'_{O}(A)
  \\&= \{\text{continuous homomorphisms of topological }\Lambda\text{-algebras }O\to A\}
  \\&= \varinjlim_i \operatorname{Hom}_{\Lambda\text{-algebras}}(O_i,A).
  \end{aligned}
\]
Note also that the category \({\mathcal{C}}\) is essentially the product of analogous categories, corresponding to the local rings that are the completions of the \(\Lambda_{{\mathfrak{m}}}\) for the maximal ideals \({\mathfrak{m}}\) of \(\Lambda\);
we can thus, if so desired, restrict to the case where \(A\) is such a complete local ring.
In any case, \(O\) decomposes canonically as the topological product of its \emph{local components}, which correspond to the ``points'' of the \emph{formal scheme} {[}\protect\hyperlink{ref-Gro1958a}{7}{]} defined by \(O\).
Such a point is defined by an object \(\xi\) of some \(F(K)\), where \(K\in{\mathcal{C}}\) is a \emph{field} (for example, the residue field of the local component in question), and where two pairs \((\xi,K)\) and \((\xi',K')\) define the same point if and only if they are both dominated by the same \((\xi'',K'')\), or if they both dominate the same \((\xi''',K''')\).
(If the \(\Lambda/{\mathfrak{m}}\) are algebraically closed, then it suffices to take the set given by the sum of the \(F(\Lambda/{\mathfrak{m}})\)).

It is important to give conditions that ensure that the local component \(O_\xi\) of \(O\) corresponding to some \(\xi\in F(K)\) be a \emph{Noetherian} ring.
If \(\Lambda\) is a complete local ring (Noetherian, we recall), then it is equivalent to say that \(O_\xi\) is isomorphic to a \emph{quotient ring of a formal series ring \(\Lambda[[t_1,\ldots,t_n]]\)}.
To give such a criteria, we introduce (for every ring \(A\)) the \(A\)-algebra \(I_A\) of ``dual numbers'' of \(A\), defined by
\[
  I_A = A[t]/t^2A[t].
\]
Let \(\varepsilon\colon I_A\to A\) be the augmentation homomorphism, which defines (if \(A\in{\mathcal{C}}\)) a map
\[
  F(\varepsilon)\colon F(I_A) \to F(A).
\]
\oldpage{195-08}Using the fact that \(F\) is left exact, we intrinsically define the structure of an \(A\)-module on the subset
\[
  F(I_A,\xi) = F(\xi)^{-1}(\xi) \subset F(I_A)
\]
consisting of the \(\xi'\in F(I_A)\) that are ``reducible along \(\xi\)'';
using the explicit form of \(F\) in terms of \(O\), we find that this \(K\)-module can be identified with \(\operatorname{Hom}_\Lambda({\mathfrak{m}}_\xi/{\mathfrak{m}}_\xi^2,A\), where \(m_\xi\) is the kernel of the homomorphism \(\xi\colon O\to A\), i.e.~if \(A\) is a field, then the maximal ideal of the local component \(O_\xi\) of \(O\).
From this, we immediately deduce the following proposition:

\leavevmode\vadjust pre{\hypertarget{fga-3-ii-section-A.5-proposition-5.1}{}}%
\begin{itenv}{Proposition 5.1}
Let \(\xi\in F(K)\), where \(K\in{\mathcal{C}}\) is a field.
For the corresponding local component \(O_\xi\) of \(O\) to be a \emph{Noetherian} ring, it is necessary and sufficient that the set \(F(I_K,\xi)\) of elements of \(F(I_K)\) that are reducible along \(\xi\) be a vector space of \emph{finite dimension} over \(K\).
Under these conditions, we have a canonical isomorphism
\[
  F(I_K,\xi) = \operatorname{Hom}({\mathfrak{m}}_\xi/{\mathfrak{m}}_\xi^2+{\mathfrak{n}}_\xi{\mathscr{O}}_\xi, K)
\]
(where \({\mathfrak{n}}_\xi\) is the maximal ideal of \(\Lambda\) given by the kernel of the homomorphism \(\Lambda\to K\)), and so, in particular, the dimension of the \(K\)-vector space \(F(I_K,\xi)\) is equal to the dimension of the vector space \({\mathfrak{m}}_\xi/{\mathfrak{m}}_\xi^2\) over the field \(O_\xi/{\mathfrak{m}}_\xi=K(\xi)\).

\emph{{[}Comp.{]}} The formula given above is only correct when \(\Lambda\) is a field; in the general case, we must replace \({\mathfrak{m}}_\xi/{\mathfrak{m}}_\xi^2\) with the quotient of this space by the image of \({\mathfrak{n}}_\xi/{\mathfrak{n}}_\xi^2\), where \({\mathfrak{n}}\) is the maximal ideal of \(\Lambda\).

\end{itenv}

Suppose that \(O_\xi\) is Noetherian, and suppose, for notational simplicity, that \(\Lambda\) is complete and local, and that \(O=O_\xi\).
(\emph{{[}Comp.{]}} The following definition is correct only when the residue extension \(k'/k\) is \emph{separable}; for the general case, see {[}\protect\hyperlink{ref-Gro1960b}{9}, III, 1.1{]}.)
We say that \emph{\(O\) is simple over \(\Lambda\)} if \(O\) is a finite and étale algebra over the completion algebra of the localisation of \(\Lambda[t_1,\ldots,t_n]\) at one of its maximal ideals that induces the maximal ideal of \(\Lambda\);
if the residue extension of \(O\) over \(\Lambda\) is trivial (for example, if the residue field of \(\Lambda\) is algebraically closed), then this is equivalent to saying that \(O\) itself is isomorphic to such a formal series algebra.
Finally, if we no longer necessarily suppose that \(O\) is Noetherian, then we again say that \emph{\(O\) is simple over \(\Lambda\)} if \(O\) is isomorphic to a topological projective limit of quotient \(\Lambda\)-algebras that are Noetherian and \(\Lambda\)-simple in the above sense.
We can immediately generalise to the case where \(\Lambda\) and \(O\) are no longer assumed to be local.
With this, we have the following proposition:

\leavevmode\vadjust pre{\hypertarget{fga-3-ii-section-A.5-proposition-5.2}{}}%
\begin{itenv}{Proposition 5.2}
For \(O\) to be simple over \(\Lambda\), it is necessary and sufficient that the associated functor \(F\) send epimorphisms to epimorphisms.

\end{itenv}

If this is the case, then this implies that, for every \emph{surjective} homomorphism \(A\to A'\) in \({\mathcal{C}}\), the morphism \(F(A)\to F(A')\) is also \emph{surjective}.
Of course, it suffices to verify this condition in the case where \(A\) is \emph{local}, and (proceeding step-by-step) where the ideal of \(A\) given by the kernel of \(A\to A'\) is annihilated by the maximal ideal of \(A\).
This leads, in practice, to verifying that a certain obstruction, linked to \emph{infinitesimal} invariants of the situation that give us a functor \(F\), is null;
this is a problem of a \emph{cohomological} nature.

\oldpage{195-09}To finish, we say some words, in the above context, about \emph{rings of definition}.
Let \(F\) still be a functor from \({\mathcal{C}}\) to \(\mathtt{Set}\), assumed to be pro-representable via a topological \(\Lambda\)-algebra \(O\).
Then, for every \(A\in{\mathcal{C}}\) and every \(\xi\in F(A)\), there exists a \emph{smallest} subring \(A'\) of \(A\) such that \(\xi\) is the image of an element \(\xi'\) of \(F(A')\) (which is then uniquely determined):
indeed, it suffices to think of \(\xi\) as a homomorphism from \(O\) to \(A\), and to take \(A'\) to be the image of \(O\) under this \(\xi\).
We then say that \(A'\) is the \emph{ring of definition of the object \(\xi\in F(A)\)}.
If \(u\colon A\to B\) is an algebra homomorphism, and if \(\eta=F(u)(\xi)\), then the ring of definition of \(\eta\) is the image under \(u\) of the ring of definition of \(\xi\).
If we start with a functor \(F\) from \({\mathcal{C}}\) to \(\mathtt{Set}\), then the existence of rings of definition, along with their properties that we have just discussed, is more or less \emph{equivalent} to the condition that \(F\) be pro-representable;
that is, they are usually far from being trivial.

\hypertarget{fga-3-ii-section-B}{%
\subsection*{\texorpdfstring{\textbf{B.} The two existence theorems}{B. The two existence theorems}}\label{fga-3-ii-section-B}}
\addcontentsline{toc}{subsection}{\textbf{B.} The two existence theorems}

Keeping the notation of \protect\hyperlink{fga-3-ii-section-A.5}{§A.5}, and, given a covariant functor
\[
  F\colon {\mathcal{C}} \to \mathtt{Set},
\]
we wish to find manageable criteria for \(F\) to be pro-representable, i.e.~expressible via a \(\Lambda\)-algebra \(O\) as above.
By the corollary of \protect\hyperlink{fga-3-ii-section-A.3-proposition-3.1}{Proposition 3.1}, to ensure this, it is necessary and sufficient that \(F\) be \emph{left exact}.
In the current state of the technique of descent (cf.~the questions asked in \protect\hyperlink{fga-3-i-section-A.2.c}{FGA 3.I, p.9}), this criterion is not directly verifiable, in this form, in the most important cases, and we need criteria that seem less demanding.

\hypertarget{fga-3-ii-section-B-theorem-1}{}
\begin{itenv}{Theorem 1}

For the functor \(F\) to be pro-representable, it is necessary and sufficient that it satisfy the two following conditions:

\begin{enumerate}
\def\labelenumi{\roman{enumi}.}
\tightlist
\item
  \(F\) commutes with finite products;
\item
  for every algebra \(A\in{\mathcal{C}}\) and every homomorphism \(A\to A'\) in \({\mathcal{C}}\) such that the diagram
  \[
  A \to A' \rightrightarrows A'\otimes_A A'
    \]
  is exact (cf.~\protect\hyperlink{fga-3-i-section-A.1-definition-1.2}{FGA 3.I, A, Definition 1.2}), the diagram
  \[
  F(A) \to F(A') \rightrightarrows F(A'\otimes_A A')
    \]
  is also exact.
\end{enumerate}

\oldpage{195-10}Furthermore, it suffices to verify condition (ii) in the case where \(A\) is local, and when, further, we are in one of the two following cases:

\begin{enumerate}
\def\labelenumi{\alph{enumi}.}
\tightlist
\item
  \(A\) is a \emph{free} module over \(A\);
\item
  the quotient module \(A'/A\) is an \(A\)-module \emph{of length \(1\)}.
\end{enumerate}

\end{itenv}

The proof of this theorem is rather delicate, and cannot be sketched here.
We content ourselves with pointing out that it relies essentially on a study of \emph{equivalence relations} (in the sense of categories) in \emph{the spectrum of an Artinian algebra} (the study of which poses even more problems, whose solutions seems essential for the further development of the theory).

In applications, the verification of condition (i) is always trivial.
The verification of condition (ii) splits into two cases: case (a), where \(A'\) is a free \(A\)-module, can be dealt with using the \emph{technique of descent by flat morphisms} (cf.~\protect\hyperlink{fga-1}{FGA 1, Theorems 1, 2, and 3}), which offers no difficulty;
to deal with case (b), we will use the following result:

\leavevmode\vadjust pre{\hypertarget{fga-3-ii-section-B-theorem-2}{}}%
\begin{itenv}{Theorem 2}
Let \(A\) be a local Artinian ring with maximal ideal \({\mathfrak{m}}\), and let \(A'\) be an \(A\)-algebra containing \(A\), and such that \({\mathfrak{m}}A'\subset A\), and \(A\to A'\rightrightarrows A'\otimes_A A'\) is \emph{exact} (which is the case, in particular, if \(A'/A\) is an \(A\)-module of length \(1\)).
Let \({\mathcal{F}}\) be the fibred category (cf.~\protect\hyperlink{fga-3-i-section-A.1-definition-1.1}{FGA 3.I, A, Definition 1.1}) of quasi-coherent sheaves that are flat over varying preschemes.
Then the morphism \(\operatorname{Spec}(A')\to\operatorname{Spec}(A)\) is a \emph{strict \({\mathcal{F}}\)-descent morphism} (cf.~\protect\hyperlink{fga-3-i-section-A.1-definition-1.7}{FGA 3.I, A, Definition 1.7}).

\end{itenv}

In other words, the data of a flat \(A\)-module \(M\) is completely equivalent to the data of a flat \(A'\)-module \(M'\) endowed with an \((A'\otimes_A A')\)-isomorphism from \(M'\otimes_A A'\) to \(A'\otimes_A M'\) satisfying the usual transitivity condition for a descent data (\emph{loc. cit.}).

We prove \protect\hyperlink{fga-3-ii-section-B-theorem-2}{Theorem 2} by first proving that
\[
  \operatorname{H}^i(A'/A,\operatorname{G_a}) = 0
  \qquad\text{(for }i\geqslant 1\text{)}
\]
(cf.~\protect\hyperlink{fga-3-i-section-A.4.e}{FGA 3.I, A.4.e}), with the hypothesis that \({\mathfrak{m}}A'\subset A\) allowing us to easily reduce to the case where \(A\) is a field (namely \(A/{\mathfrak{m}}\)).
We can then apply the equivalences described in \protect\hyperlink{fga-3-i-section-A.4.e}{FGA 3.I, p.16}.

\hypertarget{c.-applications-to-some-particular-cases}{%
\subsection*{\texorpdfstring{\textbf{C.} Applications to some particular cases}{C. Applications to some particular cases}}\label{c.-applications-to-some-particular-cases}}
\addcontentsline{toc}{subsection}{\textbf{C.} Applications to some particular cases}

\hypertarget{fga-3-ii-section-C.1}{%
\subsection{(C.1) General remarks on functors represented by preschemes}\label{fga-3-ii-section-C.1}}

Let \(S\) be a locally Noetherian prescheme.
\oldpage{195-11}A prescheme \(X\) over \(S\) is said to be \emph{locally of finite type} over \(S\) if, for all \(x\in X\) that project to \(y\in Y\), there exists an affine neighbourhood of \(y\) of ring \(A\), and an affine neighbourhood of \(x\) (over the aforementioned affine neighbourhood of \(y\)) of ring \(B\), such that \(B\) is an \(A\)-algebra of finite type.
There are many important examples of preschemes locally of finite type over \(S\), that are not of finite type over \(S\), given by solutions of classical universal problems;
thus it is important to be able to consider the Picard scheme of a curve as a union of infinitely-many connected components (that we must avoid confusing with the connected component of the identity element, i.e.~the ``Picard variety'').
It is thus sometimes useful to place ourselves in the category \({\mathcal{C}}\) of preschemes locally of finite type over \(S\), in order to study the question of representability of a contravariant functor \(F\).
\emph{The main goal of these articles is to develop a general technique that allows us to recognise when such a functor \(F\) is representable, and to study the properties of the corresponding \(S\)-prescheme \(X\) by means of the properties of \(F\).}
We note in passing that, in this study, we find non-pathological examples of preschemes over \(S\) that are not separated over \(S\), notably as ``Picard preschemes'' of excellent \(S\)-schemes;
we must thus refrain from banishing preschemes that are not schemes from algebraic geometry.

Let \(X\) be a prescheme locally of finite type over \(S\), and let
\[
  F\colon Y \mapsto \operatorname{Hom}_S(Y,X)
\]
be the associated contravariant functor.
We can consider the restriction \(F_0\) of \(F\) to the subcategory \({\mathcal{C}}_0\) of \({\mathcal{C}}\) consisting of preschemes \(Y\) over \(S\) that are Artinian and finite over \(S\):
if \(S=\operatorname{Spec}(\Lambda)\), then \({\mathcal{C}}_0\) is the category dual to the category of Artinian \(\Gamma\)-algebras considered in \protect\hyperlink{fga-3-ii-section-B}{§B}.
If \(Y=\operatorname{Spec}(A)\), where \(A\) is a \emph{local} Artinian ring, then \(Y\) consists of a single point \(y\) living above a closed point \(s\) of \(S\), and an \(S\)-homomorphism from \(Y\) to \(X\) (i.e.~an element of \(F(Y)\)) is defined by the data of a point \(x\in X\) over \(s\), along with an \({\mathscr{O}}_s\)-homomorphism from \({\mathscr{O}}_x\) to \(A\).
If there exists such a homomorphism, then \(x\) is necessarily a closed point of \(X\) (since its residue field is algebraic over the residue field of \(s\)).
This thus shows that \emph{the restriction \(F_0\) of \(F\) to ``Artinian \(Y\)-algebras'' is pro-representable, and is represented by the topological \(Y\)-algebra whose local components are the completions \(\widehat{{\mathscr{O}}_x}\) of the local rings of \(X\) at the points \(x\) of \(X\) that are closed and live above closed points of \(Y\)}.
This shows that only knowing \(F_0\) gives precise information about the structure of \(X\) (that is, the structure of the completions of its local rings at the aforementioned points).
\oldpage{195-12}We note that, even in the case where \(S\) is the spectrum of an algebraically closed field, it is only thanks to the systematic consideration of ``varieties'' \(Y\) such that \({\mathscr{O}}_Y\) may admit nilpotent elements (and, in particular, working with the spectra of local Artinian rings) that we can arrive at the ``good formulation'' of classical universal problems, and understand the ``infinitesimal'' aspect.

If we start with a given functor \(F\), and we want to know whether or not it is representable, then studying the functor \(F_0\) (using \protect\hyperlink{fga-3-ii-section-B-theorem-1}{Theorem 1} and \protect\hyperlink{fga-3-ii-section-B-theorem-2}{Theorem 2}) will give quasi-complete hints;
either, as is often the case (by simply testing, for example, the nature of the sets \(F(I_K,\xi)\) and their functorial behaviour, cf.~\protect\hyperlink{fga-3-ii-section-A}{§A}), \(F_0\) is already not pro-representable (which explains the failure of attempts made up until now to define varieties of modules in a reasonably natural way for the classification of vector bundles of rank \(>1\));
or we might be able to show that \(F_0\) is indeed representable, but that that vector spaces \(F(I_K,\xi)\) are not of finite dimension, in which case we must be content with the ``formal'' solution;
or it could be the case that \(F_0\) is indeed representable by a product of complete Noetherian local rings, which gives very strong assumptions for \(F\) itself to be representable, and, combined with the analogous properties (but of a more global nature) that we will later develop, will in all likelihood suffice to imply that it is indeed so.
Finally, we come across interesting geometric problems (see sections \protect\hyperlink{fga-3-ii-section-C.4}{§C.4} and \protect\hyperlink{fga-3-ii-section-C.5}{§C.5} below) where we have only the functor \(F_0\) (not coming from any ``global'' functor \(F\)), and where we will consider ourselves content if we can associate to it a ``formal scheme of modules''.

To finish these generalities, we note how the theory of schemes explains some apparent anomalies, such as the Igusa surface \(V\) whose ``Picard variety'' \(P\) consists of a single point, and for which, however, \(\operatorname{H}^1(V,{\mathscr{O}}_V)\neq0\);
in this case, \(P\) is a non-trivial ``purely infinitesimal'' group, i.e.~defined by a local algebra \({\mathscr{O}}\) of finite rank over the base field \(k\) and endowed with a diagonal map corresponding to the multiplicative structure of \(P\);
if \({\mathfrak{m}}\) is the maximal ideal of \({\mathscr{O}}\), then the dual of \({\mathfrak{m}}/{\mathfrak{m}}^2\) is canonically isomorphic to \(\operatorname{H}^1(V,{\mathscr{O}}_V)\) (cf.~section \protect\hyperlink{fga-3-ii-section-C.3}{§C.3} below).
It is only when the Picard group is an algebraic group in the classical sense (i.e.~simple over the base field \(k\)) that the dimension of \(\operatorname{H}^1(V,{\mathscr{O}}_V)\) (which is always equal to that of \({\mathfrak{m}}/{\mathfrak{m}}^2\)) is equal to that of the Picard group.

\hypertarget{fga-3-ii-section-C.2}{%
\subsection{\texorpdfstring{(C.2) The schemes \(\underline{\operatorname{Hom}}_S(X,Y)\), \(\prod_{X/S}Z\), \(\underline{\operatorname{Aut}}(X)\), etc.}{(C.2) The schemes \textbackslash underline\{\textbackslash operatorname\{Hom\}\}\_S(X,Y), \textbackslash prod\_\{X/S\}Z, \textbackslash underline\{\textbackslash operatorname\{Aut\}\}(X), etc.}}\label{fga-3-ii-section-C.2}}

Let \(X\) and \(Y\) be preschemes over \(S\);
\oldpage{195-13}for every prescheme \(T\) over \(S\), let \(X_T=X\times_S T\) and \(Y_T=Y\times_S T\), and consider the set
\[
  F(T)
  = \operatorname{Hom}_T(X_T,Y_T)
  = \operatorname{Hom}_S(X_T,Y)
  = \operatorname{Hom}_S(X\times_S T,Y)
\]
as a contravariant functor in \(T\).
If it is representable, then we denote by \(\underline{\operatorname{Hom}}_S(X,Y)\) the prescheme over \(S\) that represents it, and we then have a functorial isomorphism
\[
  \operatorname{Hom}_S(T,\underline{\operatorname{Hom}}_S(X,Y)) \xrightarrow{\sim}\operatorname{Hom}_S(T\times_S X,Y).
\]
There are variants of this universal problem, the solutions to which can be summarised as follows: a prescheme of \emph{\(S\)-automorphisms of} an \(S\)-prescheme \(X\) (which will be a prescheme in \emph{groups}), a prescheme of \emph{\(S\)-homomorphisms} from an \(S\)-prescheme in \emph{groups} to another (which will be a prescheme in commutative groups if the latter scheme in groups is commutative), etc.
We can also generalise the definition of \(\underline{\operatorname{Hom}}_S(X,Y)\) by considering a prescheme \(Z\) over the prescheme \(X\) over \(S\), and the functor
\[
  F(T) = \operatorname{Hom}_{X_T}(X_T,Z_T)
\]
(the set of ``sections'' of \(Z_T\) over \(X_T\));
if this functor is representable, then the \(S\)-prescheme that represents it will be denoted by \(\Pi_{X/S}Z\), and we will thus have, by definition, a functorial isomorphism
\[
  \operatorname{Hom}_S(T,\Pi_{X/S}Z) = \operatorname{Hom}_{X_T}(X_T,Z_T).
\]
Setting \(Z=Y\times_S X\), we recover \(\underline{\operatorname{Hom}}_S(X,Y)\).
From these definitions follows a formula for the new preschemes thus introduced that is as trivial as it is useful, that we will not give here (given that it holds in every category where products and fibred products exist).
More serious is the question of \emph{existence} of schemes of the type \(\underline{\operatorname{Hom}}_S(X,Y)\).
We note first of all that, for fixed \(X\), \(\underline{\operatorname{Hom}}_S(X,Y)\) (resp. \(\Pi_{X/S}Z\)) can only exist for all \(Y\) over \(S\) (resp. for all \(Z\) over \(X\)) if \(X\) is \emph{flat} over \(S\).
Furthermore, we can convince ourselves that it is not reasonable to expect the existence of a solution, for general enough \(Y\), except in the case where \(X\) is further \emph{proper} over \(S\).
It seems, however, that these conditions are sufficient for the existence of \(\underline{\operatorname{Hom}}_S(X,Y)\) and \(\Pi_{X/S}Z\), with the condition that, if necessary, we make some sort of ``quasi-projective'' hypothesis on \(Y/S\) (resp. \(Z/X\));
this is what we can verify anyway in numerous cases (for example, when \(Y\) is \emph{affine} over \(S\), or, by direct elementary constructions, when \(X\) is \emph{finite} over \(S\)).
Then \protect\hyperlink{fga-3-ii-section-B-theorem-1}{Theorem 1} and \protect\hyperlink{fga-3-ii-section-B-theorem-2}{Theorem 2} give:

\leavevmode\vadjust pre{\hypertarget{fga-3-ii-section-C.2-proposition-2.1}{}}%
\begin{itenv}{Proposition 2.1}
\oldpage{195-14}Let \(\Lambda\) be a Noetherian ring, and \(X\) and \(Y\) arbitrary preschemes over \(\Lambda\).
Consider the functor
\[
  F(A) = \operatorname{Hom}_A(X_A,Y_A)
\]
on the category \({\mathcal{C}}_0\) of Artinian \(\Lambda\)-algebras.
If \(X\) is flat over \(\Lambda\), then this functor is pro-representable.

\end{itenv}

Furthermore, we can show that, for all \(A\in{\mathcal{C}}_0\) and all \(\xi\in F(A)\), we have a canonical isomorphism
\[
  F(I_A,\xi) = \operatorname{H}^1\Big(X_A,\underline{\operatorname{Hom}}_{{\mathscr{O}}_{X_A}}\big(\xi^*(\Omega_{Y_A/A}^1),{\mathscr{O}}_{X_A}\big)\Big)
\]
where \(\Omega_{Y_A/A}^1\) is the sheaf of Kähler \(1\)-differentials of \(Y_A\) with respect to \(A\).
Taking \(A\) to be a field, we find, using \protect\hyperlink{fga-3-ii-section-A.5-proposition-5.1}{§A, Proposition 5.1} and the finiteness theorem from {[}\protect\hyperlink{ref-Gro1958a}{7}{]}, the following corollary:

\begin{itenv}{Corollary}
Suppose that \(X\) is flat and proper over \(S\), and that \(Y\) is of finite type over \(S\).
Then \(F\) is pro-representable, and the local components of the corresponding topological \(\Lambda\)-algebra are \emph{Noetherian} rings.

\end{itenv}

\begin{rmenv}{Remarks}
The problems considered in this section, and many others, having been generally studied, in the framework of classical algebraic geometry, via the ``Chow coordinates'' of cycles in projective space, allow us to consider these cycles as points of suitable projective varieties.
This procedure, and, more generally, the use of Chow coordinates, seems irredeemably insufficient from the point of view of schemes, since it destroys the nilpotent elements in the parameterised varieties, and, in particular, do not lend themselves to a satisfying study of infinitesimal variations of cycles (without taking its non-intrinsic nature, linked to the projective space, into account).
The language of Chow coordinates has sadly been the only one used by many algebraic geometers for the study of families of varieties or families of cycles, which seems to have been a serious obstacle to the understanding of these notions, despite its certain technical interest (probably temporary).
If we wish to obtain the analogue of Chow varieties in the theory of schemes, we are led to the following universal problem:
let \(X\) be a prescheme over \(S\), and, for every prescheme \(T\) over \(S\), consider the set \(F(T)\) of closed sub-preschemes of \(X_T=X\times_S T\) that are \emph{flat} over \(T\); we want to represent this functor in \(T\) via some prescheme over \(S\).
\oldpage{195-15}More generally, we can start with a quasi-coherent sheaf \({\mathscr{G}}\) on \(X\), and take \(F(T)\) to be the set of quotient sheaves of \({\mathscr{G}}_T\) that are flat over \(T\).
It seems that there exists a solution to this problem, with a scheme \(C\) that is locally of finite type over \(S\), if \(X\) is proper over \(S\), if \(S\) is locally Noetherian, and if \(F\) is furthermore coherent.
In any case, supposing only that \(S\) is locally Noetherian, the restriction of \(F\) to ``Artinian \(S\)-algebras'' is pro-representable, and, if, furthermore, \(X\) is proper over \(S\), and \(F\) is coherent, then the local components of the corresponding topological ring \({\mathscr{O}}\) are Noetherian.
Of course, even after having proven the existence of the ``Chow scheme'' of \(X\) over \(S\), it remains to find a decomposition of it into disjoint open subsets \(C_i\) (corresponding to fixing continuous invariants, such as degree and dimension of the cycles that we vary) over \(S\), as well as to make precise the relations between this scheme with the classical Chow varieties, and to make precise when a \(C_i\) is \emph{projective} (or at least quasi-projective) over \(S\).

\end{rmenv}

\begin{rmenv}{Remark}
\emph{{[}Comp.{]}}
The problems described here are completely resolved in the projective case by ``Hilbert schemes'' (cf.~\href{FGA-3-IV.html}{FGA 3.IV}).
Examples by Nagata and Hironaka show, however, that the functors described are not necessarily representable if we do not make the projective hypothesis, even if we restrict to the classification of subvarieties of dimension \(0\) of a complete non-singular variety of dimension \(3\);
this is linked to the fact that the symmetric square of such a variety does not necessarily exist.

\end{rmenv}

\hypertarget{fga-3-ii-section-C.3}{%
\subsection{(C.3) Picard schemes}\label{fga-3-ii-section-C.3}}

\emph{{[}Comp.{]}} For a more complete study, see \href{FGA-3-V.html}{FGA 3.V}.

Let \(f\colon X\to S\) be an \(S\)-prescheme, and consider the multiplicative sheaf \({\mathscr{O}}_X^\times\) of units of the structure sheaf of \(X\), along with the group
\[
  P(X/S) = \operatorname{H}^0(S,\operatorname{R}^1f_*({\mathscr{O}}_X^\times)),
\]
called the \emph{relative Picard group} of \(X/S\).
An element of this group is thus defined by giving an open cover \((U_i)\) of \(S\), along with an invertible sheaf \({\mathscr{L}}_i\) on each \(f^{-1}(U_i)\), such that \({\mathscr{L}}_i|f^{-1}(U_i\cap U_j)\) is isomorphic to \({\mathscr{L}}_j|f^{-1}(U_i\cap U_j)\) for all \(i,j\), or, at least locally over \(U_i\cap U_j\) (i.e.~these two sheaves are ``equivalent'' in the sense of \protect\hyperlink{fga-3-i-section-B.4}{FGA 3.I, B.4}).
If \(X/S\) admits a section, then \(P(X/S)\) is exactly the set of classes of invertible sheaves on \(X/S\) up to ``equivalence'' (\emph{loc. cit.}).
We now set, for all \(T\) over \(S\),
\[
  F(T) = P(X_T/T)
\]
which gives a covariant functor in \(T\), that we call the \emph{Picard functor} of \(X/S\);
if this functor is representable, then the prescheme over \(S\) that represents it is called the \emph{Picard prescheme} of \(X/S\), and denoted by \({\mathscr{P}}(X/S)\).
In this case, we then have an isomorphism of functors:
\[
  \operatorname{Hom}_S(T,{\mathscr{P}}(X/S)) \xrightarrow{\sim}P(X_T/T).
\]
Taking the Picard prescheme is compatible with extension of the base, and, in particular, the Picard preschemes of the fibres of \(X\) over \(S\) (which are preschemes over the residue fields \(K(s)\) of the \(s\in S\)) are the fibres of \({\mathscr{P}}(X/S)\).
\oldpage{195-16}Of course, since \(P(X_T/T)=F(T)\) is a commutative group, the Picard preschemes are preschemes in groups.
Note as well that the \emph{generalised Jacobians} of Rosenlicht are exactly the connected components of the identity in the Picard schemes of complete curves (possibly with singularities), which should make most of their properties clear (once their existence has been proven).

\begin{rmenv}{Remark}
The definition adopted here is only reasonable when every point of \(Y\) admits an open neighbourhood \(U\) over which \(X\) admits a section.
In the general case, it is necessary to slightly modify the definition of the Picard functor in order to still obtain an existence theorem.

Here, the plausible existence conditions for a Picard prescheme are the following: \(X\) is \emph{proper} and \emph{flat} over \(S\); \(f_*({\mathscr{O}}_X)={\mathscr{O}}_S\); and \(X\) locally admits a section over \(S\).
This condition naturally arises in the application of the technique of descent, \emph{in eliminating the automorphisms of an invertible sheaf \({\mathscr{L}}\) on \(X\) by endowing them with a marked section over the section \(s\)} (\protect\hyperlink{fga-3-i-section-B.4}{FGA 3.I, B.4}).
Notably, we find the following:

\end{rmenv}

\leavevmode\vadjust pre{\hypertarget{fga-3-ii-section-C.3-proposition-3.1}{}}%
\begin{itenv}{Proposition 3.1}
Suppose that \(X\) is flat over \(S=\operatorname{Spec}(\Lambda)\), where \(\Lambda\) is Noetherian, and suppose that, for all \(T\) of finite type over \(S\), we have \({f_T}_*({\mathscr{O}}_{X_T})={\mathscr{O}}_T\) (if \(f\) is proper and separable and has separable fibres, or if \(S\) is the spectrum of a field, then it follows from Künneth that the latter condition is equivalent to \(f_*({\mathscr{O}}_X)={\mathscr{O}}_S\)).
Then the Picard functor of \(X/S\) on the category of Artinian \(\Lambda\)-algebras is pro-representable.

Furthermore, we then have
\[
  F(I_A,\xi) = \operatorname{H}^1(X_A,{\mathscr{O}}_{X_A}),
\]
and, in particular:

\end{itenv}

\begin{itenv}{Corollary}
If \(X\) is proper over \(S\), then the local components of the topological \(\Lambda\)-algebra corresponding to the Picard functor are Noetherian.

\end{itenv}

\begin{rmenv}{Remarks}
We can generalise the definitions and results from this section to the classification of principal bundles on \(X\), with structure group \(G\) being a scheme in groups over \(S\) that is \emph{affine} and \emph{flat over \(S\)}, and also \emph{commutative}.
In the case where \(G\) would not be commutative, and thus where the adjoint bundle in groups of a principal bundle (whose sections of the automorphisms of the principal bundle) would no longer be trivial, \protect\hyperlink{fga-3-ii-section-C.3-proposition-3.1}{Proposition 3.1} no longer holds true as it is stated.
We can, however, modify the universal problem in such a way that we again obtain a solution (at least, for now, in formal geometry).
\oldpage{195-17}\emph{The golden rule to remember, in the context of the current}
\emph{section and in the following, and every time we are looking for ``schemes of modules'' for classes of objects that are only defined up to isomorphism, is always the following:}
\emph{eliminate the possible automorphisms of the objects that we want to classify, by introducing, if necessary, auxiliary structures} (points or elements of marked sections, fixing differential forms, etc.) that we take to be insignificant enough that we do not substantially modify the initial problem.

\end{rmenv}

\begin{rmenv}{Remarks}
\emph{{[}Comp.{]}}
I have recently shown that the formal scheme of modules for an abelian variety over a field is indeed simple over the Witt ring, or, in other words, that every abelian scheme \(X\) over a local Artinian ring that is the quotient of another such scheme \(Y\) comes, by reduction, from an abelian scheme over \(Y\).
The proof simply uses the variance properties of the obstruction class of the covering, introduced in \protect\hyperlink{fga-2-section-6}{FGA II, p.12}.
Recall also that the schemes of modules for curves of genus \(g\) or for polarised abelian schemes have been constructed by Mumford (cf.~\emph{Séminaire Mumford--Tate}, Harvard University (1961--62)).

\end{rmenv}

\hypertarget{fga-3-ii-section-C.4}{%
\subsection{(C.4) Formal modules of a variety}\label{fga-3-ii-section-C.4}}

Let \(\Lambda\) be a \emph{local} Noetherian ring of residue field \(k\) (more often than not, \(\Lambda\) will be equal to \(k\), or to a Cohen \(p\)-ring), and let \(X_0\) be a prescheme over \(k\).
For every local Artinian \(\Lambda\)-algebra \(A\), consider the set \(F(A)\) of isomorphism classes of \(A\)-preschemes \(X\) that are \emph{flat} over \(A\), endowed with an isomorphism

\leavevmode\vadjust pre{\hypertarget{fga-3-ii-section-C.4-equation-asterisk}{}}%
\[
  X\otimes_A k(A) \xleftarrow{\sim}X_0\otimes_k k(A)
\tag{$*$}
\]

where \(k(A)\) is the residue field of \(A\);
of course, the isomorphisms between such flat \(A\)-preschemes should respect the above isomorphism given in the structure.
If \(A\) is a (not necessary local) Artinian \(\Lambda\)-algebra, with local components \(A_i\), then we take \(F(A)\) to be the product of the \(F(A_i)\).
Then \(F\) becomes a multiplicative functor in \(A\), and we call it the \emph{functor of modules} for \(X_0\) (and \(\Lambda\)).
If this functor is representable, then it has a corresponding local topological \(\Lambda\)-algebra \(O\), of residue field \(k\), and the formal spectrum of \(O\) is called the \emph{formal scheme of modules} for \(X_0\) (and \(\Lambda\)) (cf. {[}\protect\hyperlink{ref-Gro1958a}{7}{]} for some details on this).

Here, if we wish to apply the technique of descent, the ``finite'' automorphisms of \(X_0\) are inoffensive, since they have no influence on the existence of automorphisms (in the precise sense above) of \(A\)-preschemes \(X\);
the necessary and sufficient condition, if \(A\) is not simply a field, for \(X\) to not have any non-trivial \(A\)-automorphisms is that
\[
  \operatorname{H}^0(X_0,{\mathfrak{G}}_{X_0/k}) = 0
\]
where \({\mathfrak{G}}_{X_0/k}\) denotes the sheaf of \(k\)-derivations (i.e.~the tangent sheaf) of \(X_0\).
We can easily show (at least, if \(X_0\) is simple over \(k\)) that
\[
  F(I_A,\xi) = \operatorname{H}^1(X_A,{\mathfrak{G}}_{X_A/A}).
\]
We thus conclude, as per usual:

\leavevmode\vadjust pre{\hypertarget{fga-3-ii-section-C.4-proposition-4.1}{}}%
\begin{itenv}{Proposition 4.1}
Suppose that \(\operatorname{H}^0(X_0,{\mathfrak{G}}_{X_0/k})=0\).
\oldpage{195-18}Then the formal scheme of modules for \(X_0\) exists.
If, furthermore, \(X_0\) is proper over \(k\), then the formal scheme of modules is Noetherian.

\end{itenv}

\begin{rmenv}{Remarks}

---

\begin{enumerate}
\def\labelenumi{\arabic{enumi}.}
\item
  If \(X_0\) is not assumed to be simple over \(k\), then \(F(I_A,\xi)\) can be identified with a sub-\(A\)-module of
  \[
   \operatorname{Ext}_{{\mathscr{O}}_{P_A}}^1(P_A;{\mathscr{I}}_{X_A},{\mathscr{O}}_{X_A})
    \]
  where we set \(P_A=X_A\times_A X_A\), where \({\mathscr{O}}_{X_A}\) is considered as a coherent sheaf on \(P_A\) via the diagonal morphism \(X_A\to P_A\), and where \({\mathscr{I}}_{X_A}\) denotes the coherent sheaf of ideals on \(P_A\) defined by the diagonal morphism.
  More precisely, an easy globalisation of Hochschild theory shows that the \(\operatorname{Ext}^1\) above can be identified with the set of classes, up to isomorphism, of sheaves of \(I_A\)-algebras \({\mathscr{O}}\) that are flat over \(X_A\), and endowed with an augmentation isomorphism \({\mathscr{O}}\otimes_{I_A}A\to{\mathscr{O}}_{X_A}\) (recall that we set \(I_A=At/(t^2)\)).
  The submodule \(F(I_A,\xi)\) is that which corresponds to the sheaves of \emph{commutative} algebras.
  The simplicity hypotheses are thus not essential in the theory of modules, as {[}\protect\hyperlink{ref-Gro1958a}{7}{]} implies.
\item
  Recall (\emph{loc. cit.}) that, in particular, every \emph{simple} and \emph{proper algebraic curve} \(X_0\) over \(k\) admits a formal scheme of modules that is simple over \(\Lambda\), and of relative dimension equal to \(3g-3\) if the genus \(g\) is \(\geqslant 2\), and to \(g\) if \(g=0,1\).
  These two latter cases no longer figure directly in \protect\hyperlink{fga-3-ii-section-C.4-proposition-4}{Proposition 4.1}.
  We can, however, recover them in the case of elliptic curves (\(g=1\)) thanks to the remarks that will follow.
\end{enumerate}

\end{rmenv}

We can, of course, vary \protect\hyperlink{fga-3-ii-section-C.4-proposition-4}{Proposition 4.1} \emph{ad libitum} by considering systems of schemes over \(k\) endowed with various structures.
Suppose, for example, that \(X_0\) is an \emph{abelian scheme} over \(k\), with a marked origin (i.e.~\(X_0\) is considered as a scheme in \emph{groups} over \(k\)), and let \(F(A)\) be the set of isomorphism classes of \emph{abelian} schemes over \(A\) (i.e.~of schemes in groups that are proper and simple over \(A\)) endowed with an isomorphism \protect\hyperlink{fga-3-ii-section-C.4-equation-asterisk}{(\(*\))} of abelian schemes.
We can show that imposing a multiplicative structure (or even only a ``unit section'') eliminates the infinitesimal automorphisms, and that there thus exists a formal scheme of modules that corresponds to a complete local Noetherian ring \(O\).
We can also show that, if \(X\) is a proper and simple scheme with ``absolutely connected'' fibres over a locally Noetherian prescheme \(S\), then every multiplicative structure on \(X\) that admits a unit section is necessarily associative and commutative (provided that it is associative and commutative on \emph{one} fibre, and provided that \(S\) is connected), and is furthermore uniquely determined by the knowledge of the unit section.
\oldpage{195-19}Further, supposing that \(S\) is the spectrum of a local Artinian ring \(A\) of residue field \(k\), that \(X\) is proper over \(A\) and endowed with a section \(s\), and finally that \(X\otimes_A k\) is endowed with the structure of an abelian scheme over \(k\), admitting the point of \(X\otimes_A k\) corresponding to \(s\) as the zero element, an easy calculation of obstructions, combined with an argument due to Serre, allows us to prove that there exists on \(X\) a multiplicative structure admitting the section \(s\) as the unit section.
(From here, using the ``existence theorem'' of {[}\protect\hyperlink{ref-Gro1958a}{7}{]} to pass to the case where \(A\) is complete local Noetherian, and then the technique of descent from \href{FGA-3-I.html}{FGA 3.I} for the general case, we can prove the analogous claim for all locally Noetherian connected \(S\)).
This proves that the functor \(F(A)\) considered here is isomorphic to the analogous functor defined at the start of this section by abstracting the multiplicative structure on \(X_0\).
It then follows that, in particular, if \({\mathfrak{m}}\) is the maximal ideal of \(O\), then \({\mathfrak{m}}/{\mathfrak{m}}^2\) is canonically isomorphic to the dual of \(\operatorname{H}^1(X_0,{\mathfrak{G}}_{X_0/k})\), and is thus of dimension \(n^2\), where \(n=\dim X_0\).
It would be very interesting to determine if \(O\) is indeed \emph{simple} over \(\Lambda\), i.e.~isomorphic to an algebra of formal series in \(n^2\) variables over \(\Lambda\).
Now \protect\hyperlink{fga-3-ii-section-A.5-proposition-5.2}{§A, Proposition 5.2} allows us to give an equivalent formulation of this problem as an \emph{existence problem of abelian schemes that are reducible along a given abelian scheme}.
In any case, we see, by a transcendental way, that the answer is affirmative if \(k\) is of characteristic \(0\).
In characteristic \(p\neq0\), it evidently suffices to restrict to the case where \(\Lambda\) is the ring of Witt vectors constructed over an algebraically closed field \(k\).
This could be the moment for the ``\emph{Greenberg functor}'' to prove its worth\ldots{}

\hypertarget{fga-3-ii-section-C.5}{%
\subsection{(C.5) Extension of coverings}\label{fga-3-ii-section-C.5}}

Let \({\mathfrak{X}}\) be a \emph{formal} Noetherian prescheme {[}\protect\hyperlink{ref-Gro1958a}{7}{]}, \(U\) an open subset of \({\mathfrak{X}}\) defined locally by the ``non-vanishing'' of a section of \({\mathscr{O}}_{\mathfrak{X}}\) that is not a zero divisor (and thus large enough that every section of \({\mathscr{O}}_{\mathfrak{X}}\) over an open subset \(V\) that is zero on \(U\cap V\) is zero).
(\emph{{[}Comp.{]}} It is also necessary to assume that the section defining \(U\) is not a zero divisor not only on \({\mathfrak{X}}\), but also on every \(X_n\).)
Let \({\mathfrak{J}}\) be an ``ideal of definition'' for \({\mathfrak{X}}\), and let \(X_n=({\mathfrak{X}},{\mathscr{O}}_{\mathfrak{X}}/{\mathfrak{J}}^{n+1})\), which is thus a ordinary Noetherian prescheme.
Then, if \({\mathfrak{X}}'\) and \({\mathfrak{X}}''\) are \emph{flat coverings} of \({\mathfrak{X}}\) (i.e.~preschemes over \({\mathfrak{X}}\) defined by sheaves of algebras that are coherent and locally free as sheaves of modules) that are \emph{unramified over \(U\)}, the evident map
\[
  \operatorname{Hom}_{\mathfrak{X}}({\mathfrak{X}}',{\mathfrak{X}}'') \to \operatorname{Hom}_{X_0}(X'_0,X''_0)
\]
is \emph{injective};
in particular, \emph{an automorphism of \({\mathfrak{X}}'\) that induces the identity on \(X'_0\) is the identity}.
This allows us to apply the technique of descent to the situation.
\oldpage{195-20}We start, in particular, with a flat covering \(X'_0\) of \(X_0\), unramified over \(U_0\), and let \(G({\mathfrak{X}})\) be the set of classes, up to isomorphism (inducing the identity on \(X'_0\)), of flat coverings \({\mathfrak{X}}'\) of \({\mathfrak{X}}\) that induce \(X'_0\) on \(X_0\) (and that are thus necessarily unramified over \(U\)).
We similarly define \(G(V)\) for every open subset \(V\) of \({\mathfrak{X}}\), and, more generally, \(G({\mathfrak{Y}})\) for every formal prescheme \({\mathfrak{Y}}\) over \({\mathfrak{X}}\).
With this, the results of {[}\protect\hyperlink{ref-Gro1958a}{7}{]} and \href{FGA-3-I.html}{FGA 3.I} imply, first of all, the following results:

\begin{enumerate}
\def\labelenumi{\alph{enumi}.}
\tightlist
\item
  If \(V\) varies amongst open subset of \({\mathfrak{X}}\), then the \(G(V)\) form a \emph{sheaf} on \({\mathfrak{X}}\), say \({\mathscr{G}}_{\mathfrak{X}}={\mathscr{G}}\).
  The restriction of this sheaf to \(U\) is the \emph{constant} sheaf whose fibres consist of a single element.
\end{enumerate}

More generally, describing the fibres of \({\mathscr{G}}_{\mathfrak{X}}\) is a question of complete local rings, in a precise way:

\begin{enumerate}
\def\labelenumi{\alph{enumi}.}
\setcounter{enumi}{1}
\item
  For all \(x\in{\mathfrak{X}}\), we have
  \[
   {\mathscr{G}}_x = G(\operatorname{Spec}({\mathscr{O}}_{{\mathfrak{X}},x})) \subset G(\operatorname{Spec}(\widehat{{\mathscr{O}}}_{{\mathfrak{X}},x}))
    \]
  (i.e.~isomorphism classes of finite free algebras \(B\) over \(\widehat{{\mathscr{O}}}_{{\mathfrak{X}},x}\) endowed with an isomorphism from \(B\otimes_{\widehat{{\mathscr{O}}}_{{\mathfrak{X}},x}}{\mathscr{O}}_{X_0,x}\) to \((\widehat{{\mathscr{O}}}'_0)_x\), where \({\mathscr{O}}'_0\) is the sheaf of algebras on \(X_0\) that defines \(X'_0\)).
\item
  We have a canonical isomorphism \({\mathscr{G}}_{\mathfrak{X}}=\varprojlim{\mathscr{G}}_{{\mathfrak{X}}_n}\); in other words, for every open subset \(V\) of \({\mathfrak{X}}\), we have \(G(V)=\varprojlim G(V_n)\).
\item
  Suppose that \({\mathfrak{X}}\) comes from an ordinary \emph{proper} scheme \(X\) over a complete local Noetherian ring \(\Lambda\) that has ideal of definition \({\mathfrak{m}}\) by taking the \({\mathscr{J}}\)-adic completion of \({\mathscr{O}}_X\), where \({\mathscr{J}}={\mathfrak{m}}\cdot{\mathscr{O}}_X\).
  Then \(G({\mathfrak{X}})\) is canonically isomorphic to the set of classes of flat coverings of the \emph{ordinary} scheme \(X\) that are ``reducible along \(X'_0\)''.
\end{enumerate}

Figuratively speaking, we can say that (a) and (b) establish the fundamental relations between the local and global aspects of the problem; (c) gives the relations between the ``finite'' and ``infinitesimal'' aspects; and finally (d) remembers (under precise conditions) the identity between the ``formal'' and ``algebraic'' aspects.

Now suppose that \({\mathfrak{X}}\) is defined by a local complete Noetherian ring \(\Lambda\), with \({\mathscr{J}}={\mathfrak{m}}\cdot{\mathscr{O}}_X\) (and so \(X_0\) is a prescheme over \(\Lambda/{\mathfrak{m}}\)).
For every algebra \(A\) that is finite over \(\Lambda\), we set
\[
  F(A) = G({\mathfrak{X}}\times_\Lambda A).
\]
This is a covariant functor in \(A\), with values in the category of sets, and, by (c), this functor is completely determined by how it acts on Artinian algebras \(A\);
it is equivalent to say either that this functor is pro-representable, i.e.~of the form
\[
  F(A) = \operatorname{Hom}_{\text{top. }\Lambda\text{-algebras}}({\mathscr{O}},A)
\]
\oldpage{195-21}where \({\mathscr{O}}\) is a topological \(\Lambda\) algebra of the type described in \protect\hyperlink{fga-3-ii-section-A.5}{§A.5}, or that this is true when we restrict to only Artinian algebras \(A\).
The combination of \protect\hyperlink{fga-3-ii-section-B-theorem-1}{Theorem 1} and \protect\hyperlink{fga-3-ii-section-B-theorem-2}{Theorem 2} then effectively implies:

\leavevmode\vadjust pre{\hypertarget{fga-3-ii-section-C.5-proposition-5.1}{}}%
\begin{itenv}{Proposition 5.1}
The above functor is pro-representable.

\end{itenv}

Of course, by (a), if \(U={\mathfrak{X}}\), then \(G({\mathfrak{Y}})\) consists of a single element for all \({\mathfrak{Y}}\) over \({\mathfrak{X}}\), and the functor \(F\) is then not very interesting (we will have \({\mathscr{O}}=\Lambda\)).
It seems that, in practically every other case, the topological local ring \({\mathscr{O}}\) \emph{is not Noetherian}.
Its existence, however, shows, in a striking manner, the \emph{``continuous'' nature} of the set \(G({\mathfrak{X}})\) of solutions (corresponding intuitively to the fact that there is a ``continuous'' choice in the way in which the ramification spreads when we take an extension of \(X'_0\)).
We will compare this result with the point of view of J.-P. Serre {[}\protect\hyperlink{ref-Ser1958}{21}{]} via local class field theory, drawing attention as well to the \emph{continuous} character of the topological Galois group of the maximal abelian extension of a ``geometric'' local field, with the dual group (in the sense of Pontrjagin) appearing as an inductive limit of algebraic (or at least quasi-algebraic) groups;
here as well, the classification of extensions is given by infinite-dimensional ``varieties''.
We can also take, in the above, \({\mathfrak{X}}\) to be the formal spectrum of a complete local ring (of which \(\Lambda\) will be, for example, a Cohen subring), and we might hope that the results of this section can be used in the study of extensions of a local complete ring of dimension \(>1\).
Just as much in the local case as in the global case, they might allow us to formulate precise relations between the phenomena of higher ramification and phenomena in characteristic \(0\) (approachable via a transcendental way).
In any case, it is the preliminary analysis of \protect\hyperlink{fga-3-ii-section-C.5-proposition-5.1}{Proposition 5.1} that allows us to extend the methods described in {[}\protect\hyperlink{ref-Gro1958a}{7}{]} for the study of the fundamental group to the ``tamely ramified'' case, and to resolve, by a transcendental way, the ``problem of three points''.

To finish, we note that the situation simplifies if \(X_0\) is of dimension \(1\);
then, by (a) and (b), \(G({\mathfrak{X}})\) can be identified with \(\prod_i G(\operatorname{Spec}({\mathscr{O}}_{{\mathfrak{X}},x_i}))\), where the \(x_i\) are the points of \(X_0\setminus U\):
\emph{we can take arbitrary ``local'' extensions at the ramification points}.
Further, if \(X_0\) is normal, then we note that the formal scheme of modules guaranteed by \protect\hyperlink{fga-3-ii-section-C.5-proposition-5.1}{Proposition 5.1} is \emph{simple} over \(\operatorname{Spec}(\Lambda)\).

\hypertarget{fga-3.iii}{%
\section{Quotient preschemes}\label{fga-3.iii}}

\providecommand{\scr}[1]{{\mathscr{#1}}}
\renewcommand{\cal}[1]{{\mathcal{#1}}}
\renewcommand{\frak}[1]{{\mathfrak{#1}}}
\renewcommand{\geq}{\geqslant}
\renewcommand{\leq}{\leqslant}

\providecommand{\Set}{\mathtt{Set}}
\providecommand{\pr}{\mathrm{pr}}
\providecommand{\Gm}{\mathrm{G}_\mathrm{m}}
\providecommand{\id}{\operatorname{id}}
\providecommand{\Hom}{\operatorname{Hom}}
\providecommand{\Spec}{\operatorname{Spec}}
\providecommand{\Ker}{\operatorname{Ker}}
\providecommand{\GL}{\operatorname{GL}}
\providecommand{\sGL}{\operatorname{\mathscr{G}\!\!\mathscr{L}}}
\providecommand{\GP}{\operatorname{GP}}
\providecommand{\Aut}{\operatorname{Aut}}

{[}FGA 3.III{]}
Grothendieck, A.
``Technique de descente et théorèmes d'existence en géométrie algébrique, III: Préschemas quotients''.
\emph{Séminaire Bourbaki} \textbf{13} (1960--61), Talk no. 212.

\hypertarget{introduction}{%
\subsection*{Introduction}\label{introduction}}
\addcontentsline{toc}{subsection}{Introduction}

\begin{rmenv}{Remark}
\emph{{[}Comp.{]}}
We note that the application (of the theory developed here) in \href{FGA-3-V.html}{FGA 3.V} (``Picard schemes: existence theorems'') can equally be replaced by a suitable use of Hilbert schemes (cf.~\emph{Séminaire Mumford--Tate}, Harvard University (1961--62)).
As mentioned in \protect\hyperlink{fga-3-iii-section-8}{§8}, the most important gap in the theory presented here is the lack of existence criteria for quotients by a non-proper equivalence relation, such as the equivalence relations coming from certain actions of the projective group.
An important theorem in this direction has been obtained by Mumford {[}\protect\hyperlink{ref-Mum1961}{15}{]}.
For a refinement of his result, and various applications the the theory, see \emph{Séminaire Mumford--Tate}, Harvard University (1961--62).

\end{rmenv}

\oldpage{212-01}The problems discussed in the current talk differ from those discussed in the two previous ones, in that we try to represent certain covariant, no longer contravariant, functors of varying schemes.
The procedure of passing to the quotient is, however, essential in many questions of construction in algebraic geometry, including those from \href{FGA-3-I.html}{FGA 3.I} and \href{FGA-3-II.html}{FGA 3.II}.
Indeed, the question of \emph{effectiveness of a descent data} on a \(T\)-prescheme \(X\), with respect to a faithfully flat and quasi-compact morphism \(T\to S\), is equivalent to the question of existence of a quotient of \(X\) (satisfying reasonable properties that we examine below) by the flat equivalence relation on \(X\) defined by the descent data;
the questions raised in \protect\hyperlink{fga-3-i-section-A.2.c}{FGA 3.I, A.2.c} can probably be answered at the same time as the questions posed in \protect\hyperlink{fga-3-iii-section-2}{§2} of this current talk.
Similarly, the \emph{Picard scheme} (for the definition, see \protect\hyperlink{fga-3-ii-section-C.3}{FGA 3.II, C.3}) of an \(S\)-scheme \(X\) can be defined in many ways, such as as a quotient of certain other schemes (with positive divisors, or immersions into a projective) by flat equivalence relations, with the definition and construction of these auxiliary schemes being also more simple: they are basically schemes of the type \(\operatorname{Hom}_S(X,Y)\), and variants defined in \protect\hyperlink{fga-3-ii-section-C.2}{FGA 3.II, C.2}, and their construction will be the subject of the following talk (under suitable hypotheses of projectivity).
Thus, combining the results of the current talk with those of the following, we will obtain the construction of Picard schemes, under suitable hypotheses.

The problem of passing to the quotient in preschemes again offers unresolved questions.
The most important is mentioned in \protect\hyperlink{fga-3-iii-section-8}{§8}.
It currently remains as the only obstacle to the construction of \emph{schemes of modules over the integers for curves of arbitrary degree}, \emph{polarised abelian varieties}, etc.
That is to say, its solution deserves the efforts of specialists of algebraic groups.

\hypertarget{fga-3-iii-section-1}{%
\subsection{Equivalence relations, effective equivalence relations}\label{fga-3-iii-section-1}}

Let \({\mathcal{C}}\) be a category, and \(X\) an object of \({\mathcal{C}}\).
\oldpage{212-02}A pair of morphisms
\[
  p_1,p_2\colon R\rightrightarrows X,
\]
is said to be an ``\emph{equivalence pair}'' in \({\mathcal{C}}\), with \emph{target} \(X\) and \emph{source} \(R\), if, for every object \(T\) of \({\mathcal{C}}\), the corresponding maps
\[
  p_1(T),p_2(T)\colon R(T)\rightrightarrows X(T)
\]
(where we set \(Y(T)=\operatorname{Hom}(T,Y)\) for any object \(Y\) of \({\mathcal{C}}\)) define a map
\[
  R(T)\to X(T)\times X(T)
\]
that induces a bijection from \(R(T)\) to the graph of an equivalence relation on the set \(X(T)\).
We introduce an evident equivalence relation on equivalence pairs with target \(X\), and call an equivalence class an \emph{equivalence \({\mathcal{C}}\)-relation} on \(X\), or simply an equivalence relation if no confusion may arise.

If \(X\times X\) exists, then the data of an equivalence relation on \(X\) is equivalent to the data of a sub-object \(R\) of \(X\times X\) such that, for every object \(T\) of \({\mathcal{C}}\), the subset of \((X\times X)(T)=X(T)\times X(T)\) that corresponds to \(R(T)\) is the graph of an equivalence relation on \(X(T)\).
Denoting the morphisms from \(R\) to \(X\) induced by the projections \(\mathrm{pr}_1\) and \(\mathrm{pr}_2\) by \(p_1\) and \(p_2\) (respectively), the above condition says that \((p_1,p_2)\) is an equivalence pair.
We can also express the axioms of a set-theoretical equivalence relation for the \(R(T)\) in the \(X(T)\) diagrammatically in \({\mathcal{C}}\) (under the assumption that both \(X\times X\) and the fibre product \((R,p_2)\times_X(R,p_1)\) exist), following the general principle of \protect\hyperlink{fga-3-ii-section-A.1}{FGA 3.II, A.1}.
We will not need this.

Every time that we have a pair of morphisms \((p_1,p_2)\) with the same source \(R\) and the same target \(X\), we can define the \emph{cokernel} of the pair as an object \(Y\) of \({\mathcal{C}}\) that represents the contravariant (in \(Z\)) functor
\[
  \operatorname{Hom}_{p_1,p_2}(X,Z)
\]
which denotes the set of morphisms \(u\) from \(X\) to \(Z\) such that \(up_1=up_2\).
If \(Y\) exists, then it is determined up to unique isomorphism.
\oldpage{212-03}We will denote it by \(Y/(p_1,p_2)\), or, by an abuse of notation, \(Y/R\), with the latter mostly being used when \((p_1,p_2)\) is an equivalence pair: it is then common to identify, in notation, the equivalence relation defined by the pair with the one defined by \(R\).
Note that, if we consider \(Y\) as a quotient of \(X\), then it depends only on the equivalence relation defined by the pair \((p_1,p_2)\)

We now start with a morphism
\[
  f\colon X\to Y
\]
which allows us to consider \(X\) as an ``object over \(Y\)'', and we suppose that the fibre product
\[
  {\mathcal{R}}(f) = X\times_Y X
\]
exists.
Let \(p_1\) and \(p_2\) be its projections.
Then \((p_1,p_2)\) is an equivalence pair, and is said to be \emph{associated} with the morphism \(f\).
It thus defines an equivalence relation, and is said to be \emph{associated} with the morphism \(f\).

We say that a pair of morphisms \((p_1,p_2)\) with target \(X\), and source \(R\), is an \emph{effective equivalence pair} if

\begin{enumerate}
\def\labelenumi{\roman{enumi}.}
\tightlist
\item
  the cokernel \(Y=X/(p_1,p_2)\) exists ;
\item
  the fibre product \(X\times_Y X\) exists ; and
\item
  the morphism \(R\to X\times_Y X\) with components \(p_1\) and \(p_2\) is an isomorphism.
\end{enumerate}

Then the pair \((p_1,p_2)\) is indeed an equivalence pair.
We also say that the equivalence relation that it defines is an \emph{effective equivalence relation}.

We say that a morphism \(f\colon X\to Y\) is an \emph{effective epimorphism} if

\begin{enumerate}
\def\labelenumi{\roman{enumi}.}
\tightlist
\item
  the fibre product \(R=X\times_Y X\) exists ;
\item
  the quotient \(X/(p_1,p_2)\) exists, where \(p_1\) and \(p_2\) are the projections from \(R\) to \(X\) ; and
\item
  the morphism \(X/(p_1,p_2)\to Y\) induced by \(f\) is an isomorphism.
\end{enumerate}

Then \(f\) is indeed an epimorphism, and even a strict epimorphism (cf.\protect\hyperlink{fga-3-i-section-A.2.c}{FGA 3.I, A.2.c}), with the converse being true if the fibre product \(X\times_Y X\) exists.
We also say that the quotient object of \(X\) defined by the epimorphism \(f\) is an \emph{effective quotient} of \(X\).

The above definitions imply the following ``\emph{Galois correspondence}'':

\leavevmode\vadjust pre{\hypertarget{fga-3-iii-proposition-1.1}{}}%
\begin{itenv}{Proposition 1.1}
\oldpage{212-04}There is a bijective correspondence, respecting the natural orders, between the set of effective equivalence relations \(R\) on \(X\) and the set of effective quotients \(Y\) of \(X\), with such an \(R\) corresponding to the effective quotient \(X/R\), and such a \(Y\) corresponding to the effective equivalence relation defined by the canonical projection \(X\to Y\) (which is defined by the fibre product \(X\times_Y X\) endowed with its two projections).

\end{itenv}

In very nice categories (sets, sheaves of sets, etc.), every quotient is effective, and every equivalence relation is effective.
This is no longer true in categories such as the category of preschemes over a given prescheme \(S\), not even if \(S\) is the spectrum of field, nor even if we restrict to finite schemes over \(S\).
The question of effectiveness, and even (in the case of non-finite preschemes over \(S\)) the question of existence of quotients, more often than not turn out to be delicate.

\hypertarget{fga-3-iii-section-2}{%
\subsection{\texorpdfstring{Example: finite preschemes over \(S\)}{Example: finite preschemes over S}}\label{fga-3-iii-section-2}}

Let \({\mathcal{C}}\) be the category of finite preschemes over \(S\), which is assumed to be locally Noetherian.
Then \({\mathcal{C}}\) is equivalent to the opposite category of the category of coherent sheaves of commutative algebras on \(S\), or, if \(S\) is affine of ring \(A\), then it is equivalent to the opposite category of the category of finite \(A\)-algebras over \(A\) (i.e.~those that are modules of finite type over \(A\)).
We thus immediately conclude that, in \({\mathcal{C}}\), finite projective limits and finite inductive limits exist.
This is well known (without any finiteness hypotheses) for the former;
the fibre product of preschemes \(X\) and \(Y\) over \(S\) corresponds to the tensor product \(B\otimes_A C\) of corresponding algebras, and the kernel of two morphisms \(X\rightrightarrows Y\), defined by two \(A\)-algebra homomorphisms \(u,v\colon C\rightrightarrows B\), corresponds to the quotient of \(B\) by the ideal generated by the \(u(v)-v(c)\), etc.
For finite inductive limits, it suffices to consider, on one hand, finite sums, which correspond to the ordinary product of \(A\)-algebras, and, on the other hand, cokernels of pairs of morphisms \(X\rightrightarrows Y\), which correspond (as we can immediately see) to the sub-ring of \(C\) given by elements where the homomorphisms \(u,v\colon C\rightrightarrows B\) agree (with this sub-ring being finite over \(A\) thanks to the Noetherian hypothesis).
We also note that we can show, using the Noetherian hypothesis, that finite inductive limits, and, in particular, quotients, thus constructed in the category \({\mathcal{C}}\) of finite preschemes over \(S\) are, in fact, quotients in the category of \emph{all} preschemes.

\oldpage{212-05}As we mentioned in \href{FGA-3-I.html}{FGA 3.I}, \emph{there are non-effective epimorphisms in \({\mathcal{C}}\)} (or even non-strict, which is the same, since fibre products exist).
\emph{I do not know if equivalence relations are still effective} if we have no flatness hypothesis.
I have only obtained, in this direction, very partial, positive, results, that are vital for the proof of the fundamental theorem of the formal theory of modules (cf.~\protect\hyperlink{fga-3-ii-section-B-theorem-1}{FGA 3.II, B, Theorem 1}).
We note that it is easy, in the given problem, to reduce to the case where \(S\) is the spectrum of a local Artinian ring, with an algebraically closed residue field.
But even if \(A\) is a field, the answer is not known.

We can also consider the case of a prescheme \(X\) over \(S\) that is no longer assumed to be finite over \(S\), but by considering an equivalence relation \(R\) on \(X\) such that \(p_1\colon R\to X\) is a finite morphism.
We then say that \(R\) is a \emph{finite equivalence relation}.
Supposing, for simplicity, that \(S\) and \(X\) are affine (which implies that \(R\) is affine, so that the situation is reduced to one of pure commutative algebra), \emph{we do not know, even in this case, if there exists a quotient \(X/R=Y\), and if the canonical morphism \(X\to Y\) is finite}.
(The most simple case is that where we suppose that \(S\) is the spectrum of a field \(k\), and where \(X\) is the spectrum of \(k[t]\), i.e.~the affine line).
Of course, if the two problems above turn out to be true, then we can conclude that, in the situation described, \(R\) is effective.
Note that the problem of \emph{existence} of a quotient \(Y\) and of the \emph{finiteness} of \(f\colon X\to Y\) are stated in exactly the same terms if, instead of an equivalence graph in \(X\), we only have an equivalence pregraph in \(X\), in the sense of \protect\hyperlink{fga-3-iii-section-4}{§4}.

The question of passing to the quotient by a more or less arbitrary finite equivalence relation arises in the construction of preschemes by ``gluing'' given preschemes \(X_i\) along certain closed sub-preschemes;
the gluing law is expressed precisely by a finite equivalence relation on the prescheme \(X\) given by the sum of the \(X_i\).
We also expect that the solutions of the problems stated here, as well as of their many variations, will be a preliminary condition for the clarification of a general technique for non-projective constructions, in the direction introduced in \href{FGA-3-II.html}{FGA 3.II}.

The only general positive fact known to the author is the following:

\leavevmode\vadjust pre{\hypertarget{fga-3-iii-proposition-2.1}{}}%
\begin{itenv}{Proposition 2.1}
Let \(S\) be a locally Noetherian prescheme, \(s\) a point of \(S\), and \(\Omega\) an algebraically closed extension of \(k(s)\).
\oldpage{212-06}Consider the corresponding ``fibre functor'' \(F\), that associates, to any \(S\)-scheme \(X\) that is finite over \(S\), the set of points of \(X/S\) with values in \(\Omega\).
This functor (which is trivially left exact) is \emph{right exact}, i.e.~it commutes with finite inductive limits, and, in particular, with the cokernel of pairs of morphisms.

\end{itenv}

By using this result for all the ``geometric points'' of \(S\), we thus deduce that the ``quotient'' category \({\mathcal{C}}'\) of \({\mathcal{C}}\), given by arguing ``modulo surjective radicial morphisms'' (i.e.~by formally adjoining inverses for such morphisms), is a ``geometric'' category, i.e.~it satisfies the same ``finite nature'' properties as the category of sets.
In particular, every equivalence relation is effective.
This implies that, if \(R\) is an equivalence relation on \(X\), where \(X\) is finite over \(S\), then the canonical morphism \(R\to X\times_Y X\) (where \(Y=X/R\)) is \emph{radicial and surjective} (and, in fact, a surjective closed immersion, since it is a monomorphism).

\hypertarget{fga-3-iii-section-3}{%
\subsection{The case of a group with operators}\label{fga-3-iii-section-3}}

We now suppose that \({\mathcal{C}}\) is an arbitrary category.
Let \(G\) and \(X\) be objects of \({\mathcal{C}}\), and suppose that \(G\) is a \({\mathcal{C}}\)-group with operators on the object \(X\).
This implies (cf.~\protect\hyperlink{fga-3-ii-section-A.1}{FGA 3.II, A.1}) that, for every object \(T\) of \({\mathcal{C}}\), we have a group structure on \(G(T)\), and the structure of an operator domain on \(X(T)\) acting on \(G(T)\), such that, for variable \(T\), the structures in question ``vary functorially'' in \(T\).
If the products \(G\times G\) and \(G\times X\) exist in \({\mathcal{C}}\), then such a structure can also be defined as a pair of morphisms
\[
  \begin{gathered}
    G\times G\to G
  \\\pi\colon G\times X\to X
  \end{gathered}
\]
subject to the condition that, for every object \(T\) of \({\mathcal{C}}\), the corresponding composition laws for the sets \(G(T)\) and \(X(T)\) make \(G(T)\) into a group acting on \(X(T)\).
Translating this axiom into the commutativity of certain diagrams in \({\mathcal{C}}\) is easy, but tedious, and, in fact, perfectly useless in all cases known to me.

Suppose that \(G\times X\) exists, and consider the two morphisms
\[
  p_1,p_2\colon G\times X\rightrightarrows X
\]
\oldpage{212-07}with
\[
  \begin{aligned}
    p_1 &= \mathrm{pr}_1
  \\p_2 &= \pi.
  \end{aligned}
\]
We immediately note that the pair \((p_1,p_2)\) is an equivalence pair if, and only if, for every object \(T\) of \({\mathcal{C}}\), the map
\[
  G(T)\times X(T) \sim (G\times X)(T) \to X(T)\times X(T)
\]
defined by this pair is injective, i.e.~if the group \(G(T)\) acts \emph{freely} on the set \(X(T)\), i.e.~if \(g\in G(T)\), \(x\in X(T)\), and \(g\cdot x=x\), then \(g\) is the identity element of the group \(G(T)\).
We then say that \(G\) \emph{acts freely} on \(X\) (or that \(X\) is a \emph{principal \({\mathcal{C}}\)-space under \(G\)}).
The equivalence relation associated to the pair \((p_1,p_2)\) is then called the \emph{equivalence relation defined by the group \(G\)} acting freely on \(X\).
If \(X\times X\) also exists, and we consider the morphism
\[
  p\colon G\times X\to X\times X
\]
defined by the pair \((p_1,p_2)\), then the condition that \(G\) acts freely implies that \(p\) is a \emph{monomorphism}.

Of course, even if \(G\) dose not act freely on \(X\), we still wish to have existence criteria for a quotient of \(X\) by \(G\), i.e.~for the cokernel of the above pair \((p_1,p_2)\).

The cokernel in question will often be denoted by \(X/G\), or by \(X\backslash G\) if \(G\) acts on the left (with the previous notation being reserved for when \(G\) acts on the right).
We note that, even if the ``image'' of \(G\times X\) under \(p\) exists (this image being defined, for example, as the smallest sub-object of \(X\times X\) through which we can factor \(p\)), say, \(R\), then this is usually not an equivalence relation on \(X\).
If we then try to pass directly to the quotient under \(R\) (or, more precisely, under the pair of morphisms from \(R\) to \(X\) induced by the two projections \(\mathrm{pr}_i\)), then we lose the particular characteristics of the original pair \((p_1,p_2)\).
It is thus important to find a generalisation of the notion of equivalence relations, appealing directly to the pair defined by a \({\mathcal{C}}\)-group with operators.

\hypertarget{fga-3-iii-section-4}{%
\subsection{Equivalence pre-relations}\label{fga-3-iii-section-4}}

Recall that a \emph{groupoid} is defined to be a category where all the morphisms are isomorphisms.
\oldpage{212-08}A category should be defined as consisting of two base sets, \(X\) and \(R\), with the former being the set of \emph{objects} and the latter the set of \emph{arrows}, endowed with the following structures:

\begin{enumerate}
\def\labelenumi{\roman{enumi}.}
\tightlist
\item
  a pair of maps
  \[
   p_1,p_2\colon R\rightrightarrows X
    \]
  called the \emph{source map} and the \emph{target map} ;
\item
  a map
  \[
  \pi\colon(R,p_2)\times_X(R,p_1) \to R
    \]
  called the \emph{composition map}.
\end{enumerate}

These data should satisfy well-known axioms, which we will not repeat here, and which can be expressed in terms of the commutativity of certain diagrams along with the existence of a (necessarily unique) map \(D\colon X\to R\) that makes two other diagrams commute, where \(D\) corresponds to passing from an object to the corresponding identity map, and satisfies
\[
  p_1\circ D = p_2\circ D = \operatorname{id}_X.
\]
To say that a category is a groupoid then, implies the existence of a (necessarily unique) map
\[
  s\colon R\to R
\]
called the \emph{symmetry} of \(R\), that sends every arrow to an inverse arrow, which can be expressed in terms of the commutativity of four other diagrams, built from \(s\), \(\Delta\), and the above data, and of which the first two can be written as
\[
  \begin{aligned}
    p_1\circ s &= p_2
  \\p_2\circ s &= p_1.
  \end{aligned}
\]

Having recalled these notions, the general definitions in \protect\hyperlink{fga-3-ii-section-A.1}{FGA 3.II, A.1} show, in particular, what we should mean by ``the structure of a \emph{\({\mathcal{C}}\)-category}'' (resp. \emph{\({\mathcal{C}}\)-groupoid}) on a pair of objects \((X,R)\) of an arbitrary category \({\mathcal{C}}\):
it is, by definition, the data, for every object \(T\) in \({\mathcal{C}}\), of the structure of a category (resp. groupoid) in the set-theoretic sense, whose set of objects is \(X(T)\), and set of arrows is \(R(T)\), with these structures ``varying functorially'' in \(T\).
This thus implies the definition of two morphisms
\[
  p_1,p_2\colon R\rightrightarrows X
\]
\oldpage{212-09}called the \emph{source morphism} and the \emph{target morphism}, and, if the fibre product in question exists, a morphism
\[
  \pi\colon (R,p_2)\times_X(R,p_1) \to R
\]
called the \emph{composition morphism};
these three morphisms then suffice to determine the structure of a category (resp. groupoid) on \((X,R)\), with the condition to place on them being the following: for every \(T\), the three corresponding morphisms for \(X(T)\) and \(R(T)\) define the structure of a category (resp. groupoid) on the pair of sets \((X(T),R(T))\).
If necessary, this can be expressed in terms of the commutativity of certain diagrams, implying a well-determined morphism
\[
  D\colon X\to R
\]
and, in the case of groupoids, a well-determined morphism
\[
  s\colon R\to R
\]
where the diagrams are as in the ``set-theoretic'' case.
This tedious interpretation of the axioms is thankfully useless in practice, with the only theoretical interest in the possibility of being able to express the data and the axioms using morphisms and equalities of morphisms between certain fibre products being the following: if we have a left-exact functor \(F\colon{\mathcal{C}}\to{\mathcal{C}}'\) (i.e.~a functor that commutes with finite products and fibre products), then it sends \({\mathcal{C}}\)-categories (resp. \({\mathcal{C}}\)-groupoids) to a \({\mathcal{C}}'\)-categories (resp. \({\mathcal{C}}'\)-groupoids) (under the condition that finite products and fibre products exist in \({\mathcal{C}}\)).

It is important, in practice, to know how to understand the morphisms \(p_1\), \(p_2\), \(\pi\), \(D\), and \(s\) as \emph{simplicial operations} in a suitable semi-simplicial or simplicial objects of \({\mathcal{C}}\) (or, at least when fibre products exist in \({\mathcal{C}}\)).
To fix terminology, we introduce the category \({\mathcal{S}}\) of \emph{simplex types} as the category whose objects are finite sets of the form
\[
  \Delta_n = [0,n]
\]
for \(n\in\mathbb{Z}\), where \([0,n]\) denotes the interval of integers from \(0\) to \(n\) (inclusive), and whose morphisms are \emph{arbitrary maps} between these finite sets.
We note that the category \({\mathcal{S}}\) is equivalent to the category of finite \emph{non-empty} sets, where we take the morphisms to be maps between finite sets.
In \({\mathcal{S}}\), the sum of a finite \emph{non-empty} family of objects clearly exists, as does the amalgamated sum of two objects over a third (the dual operation to the fibre product).
We denote by \({\mathcal{S}}'\) the subcategory of \({\mathcal{S}}\) that has the same objects, but where the morphisms are \emph{increasing maps} between the \(\Delta_n\).
This category is equivalent to the category of finite non-empty totally ordered sets.
\oldpage{212-10}In this category, the sum of two objects never exists, and the amalgamated sum of two objects \(A\) and \(B\) over a third \(C\) does not exist in general (take, for example, \(C=\Delta_0\), and \(A=B=\Delta_1\), with the two structure maps \(u\colon C\to A\) and \(v\colon C\to B\) being the equal).
However, in certain cases, the amalgamated sum \emph{does} exist;
consider
\[
  \begin{gathered}
    A = \Delta_m
    \qquad B = \Delta_n
    \qquad C = \Delta_0
  \\u(0) = m
    \qquad v(0) = 0
  \end{gathered}
\]
which is such that
\[
  A\coprod_C B = \Delta_{m+n}.
\]

A \emph{simplicial object} (resp. \emph{semi-simplicial object}) in a category \({\mathcal{C}}\) is defined to be a contravariant functor \(K\) from \({\mathcal{S}}\) (resp. \({\mathcal{S}}'\)) to \({\mathcal{C}}\).
A simplicial object thus defines a semi-simplicial object by restriction, but the former differs from the latter essentially by the presence of \emph{symmetry operations} in the \(K_n=K(\Delta_n)\), which correspond to the images under the functor \(K\) of the elements of the symmetric group on \(n+1\) elements (considered as the automorphism group of \(\Delta_n\) in \({\mathcal{S}}\)).

With the above, for all \(n\), let \(\Delta'_n\) (resp. \(\Delta''_n\)) be the finite category whose set of objects is \(\Delta_n\), and whose set of arrows is defined by the ``chaotic order'' relation (resp. the natural total order relation) on \(\Delta_n\) (i.e.~the set of arrows is the graph of the order relation).
It is clear that \(\Delta'_n\) (resp. \(\Delta''_n\)) depends functorially on the object \(\Delta_n\) of \({\mathcal{S}}\) (resp. \({\mathcal{S}}'\)).
So if \(Z\) is a category, then \(\operatorname{Hom}(\Delta'_n,Z)\) (resp. \(\operatorname{Hom}(\Delta''_n,Z)\)) is, for varying \(\Delta_n\), a functor from the category \({\mathcal{S}}\) (resp. \({\mathcal{S}}'\)) to the category of sets, i.e.~a \emph{simplicial set} (resp. \emph{semi-simplicial set}), which is said to be \emph{associated to the category \(Z\)}, and denoted by \(Z'\) (resp. \(Z''\)).
We also have an obvious natural homomorphism from the semi-simplicial set associated to \(Z'\) to \(Z''\), and this is an isomorphism if and only if \(Z\) is a groupoid.
Then:

\leavevmode\vadjust pre{\hypertarget{fga-3-iii-proposition-4.1}{}}%
\begin{itenv}{Proposition 4.1}
The functor \(Z\mapsto Z''\) from the category of categories to the category of semi-simplicial sets is fully faithful, and defines an equivalence between the category of \emph{categories} and the category of semi-simplicial sets, i.e.~contravariant functors \(K\) from \({\mathcal{S}}'\) to \(\mathtt{Set}\) \emph{that send amalgamated sums \(A\coprod_C B\) (of the type described above) to fibre products of sets}.

\oldpage{212-11}Similarly, the functor \(Z\mapsto Z'\) from the category of groupoids to the category of simplicial sets is fully faithful, and defines an equivalence between the category of \emph{groupoids} and the category of simplicial sets, i.e.~contravariant functors \(K\) from \({\mathcal{S}}\) to \(\mathtt{Set}\) \emph{that send amalgamated sums to fibre products}.

\end{itenv}

We can thus consider categories as specific examples of semi-simplicial sets, and groupoids as specific examples of simplicial sets, with, of course, the condition that we argue ``up to isomorphism'', as is rigorous when we interpret certain structures in terms of others.
The usual procedure of reduction to the set-theoretic case then implies:

\leavevmode\vadjust pre{\hypertarget{fga-3-iii-corollary-4.2}{}}%
\begin{itenv}{Corollary 4.2}
The above claim remains true when we replace categories, groupoids, and simplicial sets with \({\mathcal{C}}\)-categories, \({\mathcal{C}}\)-groupoids, and \({\mathcal{C}}\)-simplicial objects (respectively), \emph{provided that} fibre products exist in \({\mathcal{C}}\).

\end{itenv}

The semi-simplicial object \(K\) in \({\mathcal{C}}\) associated to a category \((X,R,\ldots)\) in \({\mathcal{C}}\) can be made explicit by considering the component \(K_n=K(\Delta_n)\) of \(K\) as being the \((n+1)\)th fibre product of \((R,p_1)\) over \(X\), or, even better, by the inductive formula
\[
  \begin{aligned}
    K_0 &= R
  \\K_n &= (K_{n-1},p_n^{(n-1)})\times_X(R,p_1)
  \end{aligned}
\]
where the \(p_i^{(n-1)}\) (for \(0<i<n-1\)) are the natural projections from \(K_{n-1}\) to \(X\) (which can also be defined inductively).
In this way, \(p_1\), \(p_2\), \(\pi\), \(D\), and \(s\) can be understood as simplicial operations that correspond to morphisms in \({\mathcal{S}}\), namely: the \(0\) face of \(\Delta_1\), the \(1\) face of \(\Delta_1\), the \((0,2)\) face of \(\Delta_2\), the degeneracy \(\Delta_1\to\Delta_0\), and the symmetry of \(\Delta_1\) (respectively).
Every other semi-simplicial (resp. simplicial) operation can be formally obtained from the four (resp. five) aforementioned operations by composition and fibre products.

We now define an \emph{equivalence pre-relation} on an object \(X\) of a category to be the data of a groupoid whose object of objects is \(X\).
Such a data gives, amongst other things, an object \(R\) along with two morphisms
\[
  p_1,p_2\colon R\rightrightarrows X.
\]
But we note that only these data alone do not determine the structure in question, contrary to what happens for equivalence pairs.
In this talk, we are interested in this notion with the aim of obtaining criteria for the possibility of passing to the quotient, i.e.~for being able to form the cokernel of the pair \((p_1,p_2)\).
The statement of this problem thus makes no reference to the additional data inherent to a groupoid.
\oldpage{212-12}In the proof of the results that will follow, we will, however, make use of this additional data, and, in particular, of the simplicial operations (including the symmetry operations) up to dimension \(3\) (the fourth fibre power of \(R\) over \(X\) will appear).

An equivalence relation on an object \(X\) of \({\mathcal{C}}\) defines an equivalence pre-relation: it suffices to show this in the set-theoretic case, and we then associate, to an equivalence relation on a set \(X\), the groupoid whose set of objects is \(X\), and whose set of arrows is the graph set of the equivalence relation.

A \({\mathcal{C}}\)-monoid \(G\) acting on an object \(X\) of \({\mathcal{C}}\) defines a \({\mathcal{C}}\)-category whose basic objects are \(R=G\times X\) and \(X\) (under the condition that \(G\times X\) exists), and that is a \({\mathcal{C}}\)-groupoid if and only if \(G\) is a group.
It again suffices to prove this in the set-theoretic case.
We then define the composition of arrows \((g,a)\) and \((g',g\cdot a)\) as being
\[
  (g',g\cdot a) \circ (g,a) = (g'g,a)
\]
i.e.~if \(a,b\in X\) then \(\operatorname{Hom}(a,b)\) is, by definition, the transporter of \(a\) to \(b\), and morphisms compose thanks to the composition of elements of \(G\).

\begin{rmenv}{Remark}
We can avoid the logical difficulties that arise in a statement such as \protect\hyperlink{fga-3-iii-proposition-4.1}{Proposition 4.1} by implicitly assuming that all the objects in question can be found in a fixed ``universe'' (that is itself a set).

\end{rmenv}

\hypertarget{fga-3-iii-section-5}{%
\subsection{Quotient by a finite and flat equivalence relation}\label{fga-3-iii-section-5}}

\hypertarget{fga-3-iii-theorem-5.1}{}
\begin{itenv}{Theorem 5.1}

Let \(X=\operatorname{Spec}(B)\) be an affine scheme, \({\mathcal{R}}\) an equivalence pre-relation on \(X\), whose component \(R_1\) is affine: say, \(R_1=\operatorname{Spec}(C)\).
We suppose that the first projection \(p_1\colon R_1\to X\) is a finite and locally free morphism, i.e.~that the corresponding homomorphism of rings \(p'_1\colon B\to C\) makes \(C\) a projective \(B\)-module of finite type.
Let \(A\) be the subring of \(B\) given by the kernel of the pair of homomorphisms \(p'_1,p'_2\colon B\rightrightarrows C\) (i.e.~the set of elements \(b\) such that \(p'_1(b)=p'_2(b)\)).
Let \(Y=\operatorname{Spec}(A)\), and \(f\colon X\to Y\) the morphism defined by the embedding of \(A\) into \(B\).
\oldpage{212-13}Under these conditions:

\begin{enumerate}
\def\labelenumi{\roman{enumi}.}
\tightlist
\item
  \(B\) is integral over \(A\), i.e.~\(f\) is an integral morphism.
\item
  The morphism \(f\) is surjective, and its fibres are the set-theoretic equivalence classes
  \(p_2(p_1^{-1}(x))\) in \(X\) modulo \({\mathcal{R}}\), and the topology of \(Y\) is the quotient of that of \(X\).
\item
  \(Y\) is the quotient of \(X\) by \({\mathcal{R}}\) in the category of preschemes.
\item
  If \({\mathcal{R}}\) comes from an equivalence \emph{relation}, then the morphism \(f\colon X\to Y\) is finite and locally free (i.e.~\(B\) is a projective \(A\)-module of finite type), and the equivalence relation is effective, i.e.~\(R_1\to X\times_Y X\) is an isomorphism.
\end{enumerate}

\end{itenv}

This theorem generalises the well-known theorem concerning the case of a finite group \(G\) acting by automorphisms on the ring \(B\), and with ring \(A\) of invariants, and the proof is analogous to the known proof.
We can make (iii) more precise as follows:

\leavevmode\vadjust pre{\hypertarget{fga-3-iii-corollary-5.2}{}}%
\begin{itenv}{Corollary 5.2}
The canonical morphism \(R_1\to X\times_Y X\) is \emph{surjective}.

\end{itenv}

Let \({\mathcal{R}}\) continue to be a ``finite and locally free'' equivalence pre-relation on \(X\), but with \(X\) now being an arbitrary prescheme.
Suppose that we can find a prescheme \(Y\) and a morphism \(f\colon X\to Y\) such that \(fp_1=fp_2\), and further such that the sequence
\[
  {\mathcal{O}}_Y \to f_*({\mathcal{O}}_X) \rightrightarrows g_*({\mathcal{O}}_R)
\]
of homomorphisms of sheaves of rings on \(Y\) is exact (where \(g=fp_i\)).
It then follows from the theorem that we have conclusions (i) to (iv) analogous to those of the theorem, and, in particular, by (iii), \(Y\) is the quotient of \(X\) by \({\mathcal{R}}\), and thus determined up to unique isomorphism.
Under these conditions, we say that the equivalence pre-relation \({\mathcal{R}}\) on \(X\) is \emph{admissible}.
With this definition:

\leavevmode\vadjust pre{\hypertarget{fga-3-iii-theorem-5.3}{}}%
\begin{itenv}{Theorem 5.3}
Let \(X\) be a prescheme, and \({\mathcal{R}}\) an equivalence pre-relation on \(X\) such that \(p_1\colon R_1\to X\) is a finite and locally free morphism.
For \({\mathcal{R}}\) to be admissible, it is necessary and sufficient for every set-theoretic equivalence class \(p_2(p_1^{-1}(x))\) in \(X\) modulo \({\mathcal{R}}\) to be contained in an affine open subset (a condition that is always satisfied if every finite subset of \(X\) is contained in an affine open subset, for example if \(X\) is quasi-projective over an affine scheme).

\end{itenv}

We can in fact easily show that every equivalence class modulo \({\mathcal{R}}\) in \(X\) is then contained in an affine open subset that is \emph{stable} under \({\mathcal{R}}\), and we construct the quotient \(Y\) by gluing the pieces obtained by applying \protect\hyperlink{fga-3-iii-theorem-5.1}{Theorem 5.1}.

\leavevmode\vadjust pre{\hypertarget{fga-3-iii-corollary-5.4}{}}%
\begin{itenv}{Corollary 5.4}
\oldpage{212-14}Suppose that this condition is satisfied, and, further, that \({\mathcal{R}}\) comes from an equivalence relation.
Then the equivalence relation is question is effective, i.e.~\(R_1\to X\times_Y X\) is an isomorphism, and \(f\colon X\to Y\) is a finite and locally free morphism.

\end{itenv}

We then immediately conclude, by descent, the following:

\leavevmode\vadjust pre{\hypertarget{fga-3-iii-corollary-5.5}{}}%
\begin{itenv}{Corollary 5.5}
Under the conditions of \protect\hyperlink{fga-3-iii-corollary-5.4}{Corollary 5.4}, for \(X\) to be everywhere of rank \(n\) over \(Y\), it is necessary and sufficient that \((R_1,p_1)\) be everywhere of rank \(n\) over \(X\).
If \(X\) and \(R_1\) are \(Z\)-preschemes, and \(p_1\) and \(p_2\) are \(Z\)-morphisms (and thus \(Y\) a \(Z\)-prescheme), then \(X\) is flat over \(Z\) if and only if \(Y\) is flat over \(Z\).

\end{itenv}

In summary:

\begin{itenv}{Scholium}
The data of a finite, locally free, and surjective morphism \(f\colon X\to Y\) of preschemes is equivalent to the data of a prescheme \(X\) endowed with an equivalence relation \(R\) such that \(p_1\colon R\to X\) is finite and locally free, and such that every class \(p_2(p_1^{-1}(x))\) is contained in an affine open subset.

\end{itenv}

\hypertarget{fga-3-iii-remarks-5.6}{}
\begin{rmenv}{Remarks 5.6}

---

\begin{enumerate}
\def\labelenumi{\arabic{enumi}.}
\item
  We have not needed to make any Noetherian hypothesis.
\item
  This idea of passing to the quotient contains, as a particular case, the ``inseparable descent'' of Cartier, which corresponds to the determination of finite and locally free morphisms \(f\colon X\to Y\) such that \(f_*({\mathcal{O}}_X)\) admits a \(p\)-basis with respect to \({\mathcal{O}}_Y\) (where \(X\) is a given prescheme whose sheaf \({\mathcal{O}}_X\) is annihilated by the prime number \(p>0\)).
  We note that this result can also be easily expressed without any regularity hypothesis on the local rings, and without supposing that \(X\) is an algebraic scheme over a field.
  The theory of Jacobson--Bourbaki is obtained by taking \(X\) to be the spectrum of a field of characteristic \(p\).
\item
  Gabriel had already obtained a particular case of \protect\hyperlink{fga-3-iii-theorem-5.3}{Theorem 5.3} in the theory of passing to the quotient for finite commutative groups over a field \(k\).
  (Compare with \protect\hyperlink{fga-3-iii-corollary-7.4}{Corollary 7.4}).
\end{enumerate}

\end{rmenv}

\hypertarget{fga-3-iii-section-6}{%
\subsection{Quotient by a proper and flat equivalence relation}\label{fga-3-iii-section-6}}

\hypertarget{fga-3-iii-theorem-6.1}{}
\begin{itenv}{Theorem 6.1}

\oldpage{212-15}Let \(S\) be a \emph{locally Noetherian} prescheme, \(X\) a \emph{quasi-projective} \(S\)-scheme, and \({\mathcal{R}}\) an equivalence pre-relation on \(X\), such that:

\begin{enumerate}
\def\labelenumi{\alph{enumi}.}
\tightlist
\item
  \(p_1\colon R_1\to X\) is proper and flat; and
\item
  \(R_1\to X\times_S X\) is a finite morphism (or, equivalently, by (a), a morphism with finite fibres, which is a condition that is automatically satisfied if \({\mathcal{R}}\) comes from an equivalence relation).
\end{enumerate}

Under these conditions:

\begin{enumerate}
\def\labelenumi{\roman{enumi}.}
\item
  \(Y=X/{\mathcal{R}}\) exists, and (if \(S\) is Noetherian) is quasi-projective\footnote{\emph{{[}Comp.{]}} The fact that \(Y=X/{\mathcal{R}}\) is quasi-projective over \(S\) has only been proven, for now, in the case where \({\mathcal{R}}\) comes from an equivalence relation.} over \(S\).
\item
  The canonical morphism \(f\colon X\to Y\) is surjective, proper, and open, and its fibres are the equivalence classes \(p_2(p_1^{-1}(x))\) in \(X\) modulo \({\mathcal{R}}\), and so \(Y\) can be identified with the topological quotient space of \(X\) by the set-theoretic equivalence relation defined by \({\mathcal{R}}\).
  Finally, \(R_1\to X\times_Y X\) is surjective.
\item
  If \({\mathcal{R}}\) comes from an equivalence relation, then the equivalence relation in question is effective, i.e.~\(R_1\to X\times_Y X\) is an isomorphism, and, further, \(f\colon X\to Y\) is flat (and thus faithfully flat).
\end{enumerate}

\end{itenv}

For the proof, we can reduce to \protect\hyperlink{fga-3-iii-theorem-5.1}{Theorem 5.1} by considering suitable quasi-sections of \(X\) for \({\mathcal{R}}\), with the proof being analogous to the construction of algebraic quotient groups in the Séminaire Chevalley.

In summary:

\begin{itenv}{Scholium}
Let \(X\) be quasi-projective over \(S\), with \(S\) locally Noetherian.
Then the data of a proper, faithfully flat, and surjective morphism \(f\colon X\to Y\) from \(X\) to an \(S\)-prescheme \(Y\) is equivalent to the data of an equivalence relation \(R\) on \(X\) such that \(p_1\colon R\to X\) is proper and flat.

\end{itenv}

The same method gives the following result:

\hypertarget{fga-3-iii-theorem-6.2}{}
\begin{itenv}{Theorem 6.2}

Let \(S\) be a Noetherian prescheme, \(X\) a prescheme of finite type over \(S\), and \({\mathcal{R}}\) an equivalence pre-relation on the \(S\)-prescheme \(X\).
Suppose that

\begin{enumerate}
\def\labelenumi{\alph{enumi}.}
\tightlist
\item
  \(p_1\colon R_1\to X\) is flat and of finite type; and
\item
  the morphism \(R_1\to X\times_S X\) is quasi-finite (i.e.~has finite fibres).
\end{enumerate}

Then there exists a \emph{dense} open subset \(U\) of \(X\) that is \emph{saturated} for \({\mathcal{R}}\), such that:

\begin{enumerate}
\def\labelenumi{\roman{enumi}.}
\item
  If \({\mathcal{R}}_U\) is the equivalence pre-relation induced by \({\mathcal{R}}\) on \(U\), then \(U/{\mathcal{R}}_U\) exists, and is of finite type over \(S\).
\item
  The canonical morphism \(U\to U/{\mathcal{R}}_U\) is surjective and open, and its fibres are the set-theoretic equivalence classes for \({\mathcal{R}}_U\) (and thus \(U/{\mathcal{R}}_U\) is a topological quotient space of \(U\) by the set-theoretic equivalence relation defined by \({\mathcal{R}}_U\)).
  Finally, the morphism \(({\mathcal{R}}_U)_1\to U\times_{U/{\mathcal{R}}_U}U\) is surjective.
\item
  If \({\mathcal{R}}\) comes from an equivalence relation, then we can suppose that \(U\to U/{\mathcal{R}}_U\) is faithfully flat and that \({\mathcal{R}}_U\) is effective.
\end{enumerate}

\end{itenv}

\oldpage{212-16}This is a result of an essentially ``birational'' nature.

\hypertarget{fga-3-iii-remarks-6.3}{}
\begin{rmenv}{Remarks 6.3}

---

\begin{enumerate}
\def\labelenumi{\arabic{enumi}.}
\item
  I do not know if, in \protect\hyperlink{fga-3-iii-theorem-6.1}{Theorem 6.1} and \protect\hyperlink{fga-3-iii-theorem-6.2}{Theorem 6.2}, hypothesis (b) is useless.
  In practice, it obliges us, in the passage to the quotient by groups, to restrict to he case where the stabilisers are all finite groups.
\item
  We can ask if there are results analogous to \protect\hyperlink{fga-3-iii-theorem-6.1}{Theorem 6.1} and \protect\hyperlink{fga-3-iii-theorem-6.2}{Theorem 6.2} without any flatness hypothesis.
  I have no counter example in this direction.
  However, even keeping the flatness hypothesis, and restricting to equivalence relations such that \(p_1\colon R\to X\) is flat and quasi-finite (but not finite), and with \(X\) affine, it can still be the case that \(R\) is not effective: take the equivalence relations induced on the affine open subsets covering the Nagata variety (or a group with two elements acting in a ``non-admissible'' way).
\end{enumerate}

\end{rmenv}

\hypertarget{fga-3-iii-section-7}{%
\subsection{Applications}\label{fga-3-iii-section-7}}

As we said in the introduction, the most important application of \protect\hyperlink{fga-3-iii-theorem-6.1}{Theorem 6.1} is the construction of Picard schemes, as well as solutions to various other problems of ``modules'', to which we will later return.

We obtain a simple proof of the following result of Shimura:

\leavevmode\vadjust pre{\hypertarget{fga-3-iii-proposition-7.1}{}}%
\begin{itenv}{Proposition 7.1}
Let \(A\) be an abelian scheme defined over a discrete valuation ring \(V\) with field of fractions \(K\).
Then every abelian scheme \(B'\) over \(K\) that is isogenous to a quotient of \(A\otimes_V K\) ``simplifies well for \(V\)'', i.e.~is isomorphic to some \(B\otimes_V A\), where \(B\) is an abelian scheme over \(V\) (essentially unique, we recall).

\end{itenv}

\begin{proof}
We can suppose that \(B'\) is the quotient of \(A_K\) by a subscheme in groups \(C'\).
(N.B. \(C'\) will not, in general, be ``reduced'', i.e.~its local rings will have nilpotent elements).
Consider the closed subscheme \(C\) of \(A\) given by ``the closure'' of \(C'\), i.e.~the smallest closed subscheme of \(A\) such that \(C_K\) contains \(C'\).
Then \(C_K=C'\), and, since \(V\) is a discrete valuation ring, we easily deduce that \(C\) is a subscheme in groups of \(A\) over \(V\).
Since \(A\) is proper over \(\operatorname{Spec}(V)=S\), so too is \(C\).
Further, \(A\) is projective over \(S\).
\oldpage{212-17}We can thus apply \protect\hyperlink{fga-3-iii-theorem-6.1}{Theorem 6.1} in order to construct \(A/C=B\), which is the desired \(B\).
\end{proof}

Finally, essentially known arguments allow us to extract from \protect\hyperlink{fga-3-iii-theorem-6.2}{Theorem 6.2} the following result:

\leavevmode\vadjust pre{\hypertarget{fga-3-iii-theorem-7.2}{}}%
\begin{itenv}{Theorem 7.2}
Let \(S\) be the spectrum of an Artinian ring, \(F\) and \(G\) group schemes of finite type over \(S\), and \(u\colon F\to G\) a homomorphism of group schemes over \(S\).
Suppose that

\begin{enumerate}
\def\labelenumi{\roman{enumi}.}
\tightlist
\item
  \(F\) is flat over \(S\); and
\item
  the kernel of \(u\) is finite.
\end{enumerate}

Under these conditions, the quotient scheme \(G/F\) exists, and the canonical morphism \(G\to G/F\) is surjective and open, and its fibres are the set-theoretic equivalence classes defined by the right action of \(F\) on \(G\).
Finally, if \(u\) is a monomorphism, then the morphism \(G\to G/F\) is flat, and the morphism \(G\times F\to G\times_{(G/F)}G\) is an isomorphism, or, in other words, \(G\) is a principal homogeneous space over \(G/F\), with structure group \(F\) (acting on the right), or rather \(F\times_S(G/F)\) considered as a group scheme over \(G/F\) (cf.~\protect\hyperlink{fga-3-i-section-B.6}{FGA 3.I, B.6}).

\end{itenv}

\leavevmode\vadjust pre{\hypertarget{fga-3-iii-corollary-7.3}{}}%
\begin{itenv}{Corollary 7.3}
Under these conditions, for \(G\) to be flat over \(S\), it is necessary and sufficient that \(G/F\) be flat over \(S\).
If this condition is satisfied, then the passage to the quotient by \(F\) commutes with every extension of the base \(S\), and if \(F\) is an invariant subgroup of \(G\), then \(G/F\) can be endowed with the structure of a \emph{quotient group} of \(G\) by \(F\).

\end{itenv}

The situation is particularly simple if \(S\) is the spectrum of a field, since then every \(S\)-prescheme is automatically flat over \(S\).
We find:

\leavevmode\vadjust pre{\hypertarget{fga-3-iii-corollary-7.4}{}}%
\begin{itenv}{Corollary 7.4}
Let \(F\) and \(G\) be group schemes of finite type over a field \(k\), and let \(u\colon F\to G\) be a homomorphism of \(k\)-groups.
Then \(u\) factors as \(F\to F'\to G\), where \(F\to F'\) is a homomorphism given by passing to the quotient by the closed subgroup \(\operatorname{Ker}u\) of \(F\), and where \(F'\to G\) is a group homomorphism that is a closed immersion.
The quotient \(G/F=G/F'\) exists.
The usual formalism (as in the Noether theorems) holds amongst algebraic groups over \(k\).

\end{itenv}

This result allows us to treat the passage to the quotient in a uniform way for algebraic (in the classical sense, i.e.~irreducible over \(k\) and simple over \(k\)) groups, and the passage to the quotient by ``infinitesimal'' subgroups considered by Cartier.
\oldpage{212-18}It is advantageous to consider the ``hyperalgebras'' introduced by Cartier, following from the work of Dieudonné on formal groups, as groups in the category of formal schemes over \(k\), and, if necessary (if they correspond to hyperalgebras of finite rank over \(k\)), as algebraic groups that are finite over \(k\).

\hypertarget{fga-3-iii-section-8}{%
\subsection{A conjecture}\label{fga-3-iii-section-8}}

\begin{rmenv}{Remark}
\emph{{[}Comp.{]}}
It now appears that the conjectures in this section are false, even for non-singular varieties over a field of characteristic \(0\), both for the existence and the quasi-projectivity of the quotient, and even when \(G\) acts with a closed graph.

\end{rmenv}

The conjecture in question concerns the need of knowing how to pass to the quotient by the projective group acting on certain subschemes of ``Hilbert schemes'' (with these ``Hilbert schemes'' replacing, in the theory of schemes, Chow varieties).

Let \(S\) be a prescheme, and \(n\) an integer.
To every prescheme \(S'\) over \(S\), we associate the group \(\operatorname{GL}(n,\Gamma(S',{\mathcal{O}}_{S'}))\) of invertible \((n\times n)\) matrices with values in the ring of sections of \({\mathcal{O}}_{S'}\).
We thus obtain a contravariant functor in \(S'\), which can can easily show to be representable, and so the functor corresponds to a group scheme over \(S\) (which is further affine over \(S\)) which we denote by \(\operatorname{GL}(n)_s\).
Its construction is compatible with change of base, so that, in reality, everything comes from a group scheme over \(\mathbb{Z}\), denoted by \(\operatorname{GL}(n)\).
The group \(\operatorname{GL}(1)\), called the \emph{multiplicative group}, and often denoted by \(\operatorname{G_m}\), corresponds to the functor \(S\mapsto\Gamma(S,{\mathcal{O}}_S)^*\), with the latter being the group of ``units'' over \(S\).
We have an evident homomorphism \(\operatorname{GL}(1)\to\operatorname{GL}(n)\), and we can easily construct the quotient group, denoted by \(\operatorname{GP}(n-1)\), and called the \emph{projective group of degree $(n-1)$ over $\mathbb{Z}$}.
It represents the functor that sends \(S\) to the group of sections of the sheaf \(\operatorname{\mathscr{G}\!\!\mathscr{L}}(n)_S/\operatorname{\mathscr{G}\!\!\mathscr{L}}(1)_S\), where \(\operatorname{\mathscr{G}\!\!\mathscr{L}}(n)_S\) denotes the sheaf of germs of sections of \(\operatorname{GL}(n)_S\) over \(S\).
(Note that sections of \(\operatorname{GP}(n-1)_S\) over \(S\) do not, in general, come from sections of \(\operatorname{GL}(n)_S\) over \(S\)!)
Note that we can prove that \(\operatorname{GL}(n-1)\) equally represents the functor \(S\mapsto\operatorname{Aut}_S(\mathbb{P}_S^{n-1})\) (where \(\mathbb{P}_S^{n-1}\) is the projective-type scheme of relative dimension \((n-1)\) over \(S\)), at least when \(S\) is Noetherian.
It is in this way that it appears in the theory of modules.

Let \(S\) be a Noetherian scheme, which we can, if we want, suppose to be affine, and let \(X\) be a quasi-projective \(S\)-prescheme endowed with an invertible sheaf \({\mathscr{L}}\) that is very ample with respect to \(S\).
\oldpage{212-19}Suppose that the group \(G=\operatorname{GP}(n)_S\) acts on \(X\) and \({\mathscr{L}}\) simultaneously (in a way that is compatible with its action on \(X\)), and that it acts \emph{freely} on \(S\).

\hypertarget{fga-3-iii-conjecture-8.1}{}
\begin{itenv}{Conjecture 8.1}

Under the above conditions:

\begin{enumerate}
\def\labelenumi{\arabic{enumi}.}
\item
  The equivalence relation defined by \(G\) is effective, the quotient \(Y=X/G\) is of finite type over \(S\), and the canonical morphism \(f\colon X\to Y\) is flat and surjective (and thus \(X\) becomes a homogeneous principal bundle on \(Y\), with group \(G\times_S Y=\operatorname{GP}(n)_Y\)).
\item
  Let \({\mathscr{L}}'\) be the invertible sheaf on \(Y\) induced from \({\mathscr{L}}\) by ``faithfully flat descent'' under \(f\) (cf.~\protect\hyperlink{fga-3-i-section-B.1-theorem-1}{FGA 3.I, B, Theorem 1}).
  Then \({\mathscr{L}}'\) is ``pre-ample'' on \(Y\) with respect to \(S\), i.e.~there exists an integer \(m\) and a quasi-finite morphism from \(Y\) to some suitable projective-type scheme \(\mathbb{P}_S^N\) such that \(({\mathscr{L}}')^{\otimes m}\) is isomorphic to the inverse image of \({\mathcal{O}}_{\mathbb{P}_S^N}(1)\).
\end{enumerate}

\end{itenv}

We note that, even if \(X\) is separated over \(S\), then it can be the case that \(Y\) is not separated over \(S\) (a situation that arises in not-at-all-pathological ``module problems'').
If (1) is satisfied, then \(Y\) is separated if and only if the equivalence relation defined by \(G\) has a closed graph, i.e.~if \(G\times X\to X\times X\) has a closed image (and is thus a closed immersion).
If \(Y\) is separated, then \({\mathscr{L}}'\) is pre-ample on \(Y\) with respect to \(S\) if and only if it is ample, i.e.~if a suitable tensor power defines a projective immersion.
In the module problems mentioned in the introduction, we can show that the equivalence relation to which we arrive does indeed have a closed graph.

::: \{.rmenv \#fga-3-iii-remarks-8.2 title=``Remarks 8.2'' latex=``\{Remarks 8.2\}''\}f
We have assumed that \(G=\operatorname{GP}(n)_S\) to give a concrete example, and because it is currently the most important case in practice.
The reasonable hypothesis to make on \(G\) seems rather to be that \(G\) be one of the ``forms'' on \(S\) of a Tohokû group (whose construction over the integers has also been made by Chevalley).
The only positive fact that is known to me concerning the above conjecture is the following:
\emph{Let \(X\) be an affine scheme over a field \(k\) of characteristic \(0\), on which the group \(\operatorname{GL}(n)_k\) or \(\operatorname{GP}(n-1)_k\) acts freely. Then the equivalence relation defined by \(G\) is effective, the quotient \(X/G\) is affine, and the morphism \(X\to X/G\) is flat and surjective.}
The proof uses the following fact (that, for now, has only been proven in characteristic \(0\)):
if we let \(G\) act on the affine ring \(A\) of \(G\), considered as a vector space over \(k\), then the trivial representation of \(G\) only appears once (in a composition series of a vector subspace of finite dimension over \(k\) that is stable under \(G\)).
\oldpage{212-20}It seems possible that a systematic use of the theory of linear representations of \(G\) would give a proof of the conjecture, or at least when we work over a base field.
When we are no longer working over a base field, the author knows nothing.
:::

\begin{rmenv}{Remark}
\emph{{[}Comp.{]}}
As we note at the end of the next talk, the above conjecture is decidedly false.
The ``positive fact'' mentioned in the above remark seems to have been proven simultaneously by various authors (Nagata, Rosenlicht, Grothendieck, \ldots).

\end{rmenv}

\hypertarget{fga-3.iv}{%
\section{Hilbert schemes}\label{fga-3.iv}}

\providecommand{\scr}[1]{{\mathscr{#1}}}
\renewcommand{\cal}[1]{{\mathcal{#1}}}
\renewcommand{\frak}[1]{{\mathfrak{#1}}}
\renewcommand{\geq}{\geqslant}
\renewcommand{\leq}{\leqslant}

\providecommand{\Tor}{\operatorname{Tor}}
\providecommand{\shTor}{\mathscr{T}\kern -.5pt or}
\providecommand{\Ext}{\operatorname{Ext}}
\providecommand{\shExt}{\mathscr{E}\kern -.5pt xt}
\providecommand{\Hom}{\operatorname{Hom}}
\providecommand{\shHom}{\mathscr{H}\kern -.5pt om}
\providecommand{\repHom}{\underline{\Hom}}
\providecommand{\supp}{\operatorname{supp}}
\providecommand{\red}{{\mathrm{red}}}
\providecommand{\OO}{\scr{O}}
\providecommand{\PP}{\mathbb{P}}
\providecommand{\Quot}{\operatorname{Quot}}
\providecommand{\shQuot}{\mathscr{Q}\kern -.5pt out}
\providecommand{\repQuot}{\underline{\Quot}}
\providecommand{\Grass}{\operatorname{Grass}}
\providecommand{\shGrass}{\mathscr{G}\kern -.5pt rass}
\providecommand{\repGrass}{\underline{\Grass}}
\providecommand{\Hilb}{\operatorname{Hilb}}
\providecommand{\shHilb}{\mathscr{H}\kern -.5pt ilb}
\providecommand{\repHilb}{\underline{\Hilb}}
\providecommand{\Pic}{\operatorname{Pic}}
\providecommand{\repPic}{\underline{\Pic}}
\providecommand{\repIsom}{\underline{\operatorname{Isom}}}
\providecommand{\repImm}{\underline{\operatorname{Imm}}}
\providecommand{\FXS}{{\scr{F}/X/S}}
\providecommand{\FXSp}{{\scr{F}'/X'/S'}}
\providecommand{\RR}{\operatorname{R}}
\providecommand{\rank}{\operatorname{rank}}
\providecommand{\Spec}{\operatorname{Spec}}
\providecommand{\Symm}{\operatorname{Symm}}

{[}FGA 3.IV{]}
Grothendieck, A.
``Technique de descente et théorèmes d'existence en géométrie algébrique, IV: Les schémas de Hilbert''.
\emph{Séminaire Bourbaki} \textbf{13} (1960--61), Talk no. 221.

\hypertarget{introduction-1}{%
\subsection*{Introduction}\label{introduction-1}}
\addcontentsline{toc}{subsection}{Introduction}

\oldpage{221-01}The techniques described in \href{FGA-3-I.html}{FGA 3.I} and \href{FGA-3-II.html}{FGA 3.II} were, for the most part, independent of any projective hypotheses on the schemes in question.
Unfortunately, they have not as of yet allowed us to solve the existence problems posed in \href{FGA-3-II.html}{FGA 3.II}.
In the current article, and the following, we will solve these problems by imposing projective hypotheses.
The techniques used are typically projective, and practically make no use of any results from \href{FGA-3-I.html}{FGA 3.I} and \href{FGA-3-II.html}{FGA 3.II}.
Here we will construct ``Hilbert schemes'', which are meant to replace the use of Chow coordinates, as was mentioned in \protect\hyperlink{fga-3-ii-section-C.2}{FGA 3.II, C.2}.
In the next article, the theory of passing to the quotient in schemes, developed in \href{FGA-3-III.html}{FGA 3.III}, combined with the theory of Hilbert schemes, will allow us, for example, to construct Picard schemes (defined in \protect\hyperlink{fga-3-ii-section-C.3}{FGA 3.II, C.3}) under rather general conditions.

In summary, we can say that we now have a more or less satisfying technique of projective constructions, apart from the fact that we are still missing\footnote{See the addendum at the end of this article.} a theory of passing to the quotient by groups such as the projective group, acting ``without fixed points'' (cf.~\protect\hyperlink{fga-3-iii-section-8}{FGA 3.III, §8}).
The situation even seems slightly better in analytic geometry (if we restrict to the study of ``projective'' analytic spaces over a given analytic space), since, for analytic spaces, the difficulty of passing to the quotient by a group that acts nicely disappears.
Either way, in algebraic geometry, as well as in analytic geometry, it remains to develop a construction technique that works without any projective hypotheses.

\hypertarget{fga-3-iv-section-1}{%
\subsection{Bounded sets of sheaves: invariance properties}\label{fga-3-iv-section-1}}

Let \(k\) be a field, and \(X\) a \(k\)-prescheme (which we take to be of finite type, for simplicity).
For every extension \(K/k\), we obtain a \(K\)-prescheme \(X_K=X\otimes_k K\).
If \({\mathscr{F}}\) is a coherent sheaf on \(X_K\), and if \(K'\) is an extension of \(K\), then \({\mathscr{F}}\otimes_K K'={\mathscr{F}}_{K'}\) is a quasi-coherent sheaf on \(X_K\otimes_KK'=X_{K'}\).
\oldpage{221-02}So, if \(K\) and \(K'\) are arbitrary extensions of \(k\), and \({\mathscr{F}}\) a quasi-coherent sheaf on \(X_K\) and \({\mathscr{F}}'\) a quasi-coherent sheaf on \(X_{K'}\), then we say that \({\mathscr{F}}\) and \({\mathscr{F}}'\) are \emph{equivalent} if there exists an extension \(K''/k\) as well as \(k\)-homomorphisms \(K\to K''\) and \(K'\to K''\) such that \({\mathscr{F}}_{K''}\) and \({\mathscr{F}}'_{K''}\) are isomorphic over \(X_{K''}\).
This defines an equivalence relation, and we are interested in the equivalence classes of sheaves under this relation, and of sets of such equivalence classes.
Note that, if \(X_0\) is of finite type over \(k\), then every class of coherent sheaves can be defined by a coherent sheaf on \(X_K\), where \(K\) is some extension of \(k\) \emph{of finite type}.
We can thus, in the definition of classes of coherent sheaves, restrict ourselves to \emph{algebraically closed} extensions of \(k\), and we can also restrict ourselves to a fixed algebraically closed extension \(\Omega\) of \(k\), of infinite transcendence degree;
two coherent sheaves \({\mathscr{F}}\) and \({\mathscr{F}}'\) on \(X_\Omega\) are then equivalent if and only if there exists a \(K\)-automorphism \(\sigma\) of \(\Omega\) such that \({\mathscr{F}}\otimes_K(\Omega,\sigma)\) is isomorphic to \({\mathscr{F}}'\).
Note that there is a bijective correspondence between classes of coherent sheaves under the first definition and under the second.

Let \(E\) and \(E'\) be two sets of classes of coherent sheaves on \(X\).
Consider the classes of all sheaves of the form \({\mathscr{F}}\otimes{\mathscr{F}}'\), where \({\mathscr{F}}\) and \({\mathscr{F}}'\) are coherent sheaves on the \emph{same} \(X_K\), with the class of \({\mathscr{F}}\) being in \(E\) and the class of \({\mathscr{F}}'\) being in \(E'\).
We thus define a set of classes of coherent sheaves that we denote by \(E\otimes E'\).
We can similarly define \(\mathscr{T}\kern -.5pt or_i(E,E')\), etc.
Generally, to every function \({\mathscr{U}}\) that sends each sequence \({\mathscr{F}}_1,\ldots,scr{F}_n\) of \(n\) coherent sheaves on one single \(X_K\) to a set \({\mathscr{U}}({\mathscr{F}}_1,\ldots,{\mathscr{F}}_n)\) of coherent sheaves on \(X_K\), and that has the evident property of compatibility with isomorphisms of sheaves and inverse images under change of base, we associate a function, denoted by the same notation \({\mathscr{U}}\), that sends each sequence \(E_1,\ldots,E_n\) of \(n\) sets of classes of coherent sheaves to a set \({\mathscr{U}}(E_1,\ldots,E_n)\) of classes of coherent sheaves.

Our aim in this section is to give a definition of certain sets of classes of sheaves, said to be \emph{bounded}, and to show that the most standard operations \({\mathscr{U}}\), applied to bounded sets, give sets that are again bounded.

Let \(X\) be a prescheme of finite type over \(S\), with \(S\) \emph{Noetherian}.
For all \(s\in S\), the fibre \(X_s\) is a prescheme of finite type over \(k(s)\), and we will consider the classes of coherent sheaves on \(X_s\), in the above sense.
This gives meaning to the phrase ``class of coherent sheaves on a fibre of \(X/S\)'', as well as to analogous phrases.
Similarly, proceeding separately on each fibre, we can again consider operations such as \(E\otimes E'\) etc. that send systems of sets of classes of coherent sheaves on the fibres of \(X/S\) to sets of classes of coherent sheaves on the fibres of \(X/S\).

\leavevmode\vadjust pre{\hypertarget{fga-3-iv-definition-1.1}{}}%
\begin{rmenv}{Definition 1.1}
\oldpage{221-03}Let \(E\) be a set of classes of coherent sheaves on the fibres of \(X/S\).
We say that \(E\) is \emph{bounded} if there exists a prescheme \(S'\) of finite type over \(S\), along with a coherent sheaf \({\mathscr{F}}'\) on \(X'=X\times_S S'\), such that \(E\) is contained in the set of classes of sheaves on the fibres of \(X/S\) defined by \({\mathscr{F}}'\).

\end{rmenv}

This construction, by definition, sends \(s\in S\) to the classes of sheaves \({\mathscr{F}}'\otimes_{S'}k(s')\), where \(s'\) runs over the points of \(S'\) over \(s\) (so that \(k(s')\) is an extension of \(k(s)\), and \(X\otimes (X_s)_{k(s')}\) can be identified with the fibre \(X'\otimes_{S'}k(s')=X'_{s'}\) of \(X'\) at \(s'\)).
We can say that the bounded sets are those that are contained in an \emph{algebraic family} of coherent sheaves, parametrised by some \(S'\) of finite type over \(S\).

A finite union of bounded sets is bounded (take the prescheme given by the sum of the parametrising preschemes \(S_i\) defining the bounding algebraic families).
A base change \(T\to S\) sends a family which is bounded with respect to \(X/S\) to a family which is bounded with respect to \(X_T/T\), and the converse is true if \(T\to S\) is surjective (or, more generally, if its image contains the \(s\) which appear in the given family \(E\) for \(X/S\)).
This theoretically leads us to determine the bounded families only in the case where \(S\) is the spectrum of an algebra of finite type over the ring of integers \(\mathbb{Z}\).

If \(E\) and \(E'\) are bounded families of classes of sheaves with respect to \(X/S\), then \(E\otimes E'\) is also bounded:
indeed, if \(E\) (resp. \(E'\)) is bounded by the algebraic families defined by \(T\to S\) and \({\mathscr{F}}\) on \(X_T\) (resp. \(T'\to S\) and \({\mathscr{F}}'\) on \(X_{T'}\)), then \(E\otimes E'\) is bounded by the algebraic family defined by \(T''=T\times_S T'\to S\) and the sheaf \({\mathscr{F}}''\) on \(X_{T''}\) given by the tensor product of the inverse images of \({\mathscr{F}}\) and \({\mathscr{F}}'\) on \(X_{T''}\).
This argument is correct only because the functor \({\mathscr{F}}\otimes{\mathscr{F}}'\) is right exact in both \({\mathscr{F}}\) and \({\mathscr{F}}'\), and thus commutates with base extension (and, in particular, with passing to the fibres).
It is not applicable as-is to local operations, such as \(\mathscr{T}\kern -.5pt or_i(E,E')\), \(\mathscr{H}\kern -.5pt om(E,E')\), \(\mathscr{E}\kern -.5pt xt^i(E,E')\).
We can, however, show that these operations also send bounded sets to bounded sets, by proceeding as for \(E\otimes E'\), but by also using results of the following type (all contained in {[}\protect\hyperlink{ref-Gro1960b}{9}, IV 6.11{]}):
a bounded family \(E\) is always bounded by an algebraic family defined by a coherent sheaf \({\mathscr{F}}\) on some \(X_T\) (with \(T\) of finite type over \(S\)) that is \emph{flat} with respect to \(T\).
(We ``cut into bits'' the initial space of parameters).
\oldpage{221-04}Such flatness properties on suitable sheaves indeed ensure the commutativity of operations such as \(\mathscr{T}\kern -.5pt or_i({\mathscr{F}},{\mathscr{F}}')\) with arbitrary base change.
The same method applies to operations of a global nature:
direct images and derived direct images of coherent sheaves under proper morphisms, global \(\operatorname{Ext}\) with respect to proper morphisms (cf. {[}\protect\hyperlink{ref-GD1960}{10}, III §6{]}), etc.;
all these operations send bounded families of sheaves to bounded families of sheaves (N.B. here the preschemes over which we are taking the various sheaves can change under the operations in question).

The two claims that follow can be proven by \emph{essentially} the same flatness technique;
for the primary decomposition on the fibres of a morphisms of finite type, see, in particular, {[}\protect\hyperlink{ref-GD1960}{10}, IV{]}.

\hypertarget{fga-3-iv-proposition-1.2}{}
\begin{itenv}{Proposition 1.2}

Let \(E\) and \(E'\) be bounded sets of classes of sheaves on the fibres of \(X/S\), with \(X\) assumed to be proper over \(S\).
Then

\begin{enumerate}
\def\labelenumi{\roman{enumi}.}
\tightlist
\item
  the family of kernels, cokernels, and images of homomorphisms \({\mathscr{F}}\to{\mathscr{F}}'\) (where the class of \({\mathscr{F}}\) is in \(E\) and the class of \({\mathscr{F}}'\) is in \(E'\)) is bounded;
\item
  the family of extensions \({\mathscr{F}}''\) of \({\mathscr{F}}\) by \({\mathscr{F}}'\) (where the class of \({\mathscr{F}}\) is in \(E\) and the class of \({\mathscr{F}}'\) is in \(E'\)) is bounded.
\end{enumerate}

\end{itenv}

\begin{proof}
After potentially applying a suitable base change, we can suppose that \(E\) and \(E'\) are defined by coherent sheaves \({\mathscr{G}}\) and \({\mathscr{G}}'\) (respectively) on some \(X_T/T\), with \(T\) of finite type over \(S\).
Further, we can suppose that certain flatness hypotheses are satisfied, implying that constructing the sheaves \(f_T(\mathscr{H}\kern -.5pt om_{{\mathcal{O}}_{X_T}}({\mathscr{G}},{\mathscr{G}}'))\) and \(\mathscr{E}\kern -.5pt xt_{f_T}^1({\mathscr{G}},{\mathscr{G}}')\) commutes with base change by an arbitrary morphism \(T'\to T\).
Further, we can suppose that the coherent sheaves above are locally free on \(T\).
So let \(T_0\) and \(T_1\) be the vector bundles on \(T\) whose sheaves of germs of sections are (respectively) the above sheaves.
We can then canonically define a homomorphism \({\mathscr{G}}_{T_0}\to{\mathscr{G}}'_{T_0}\) of coherent sheaves on \(X_{T_0}\), along with an extension
\[
  0 \to {\mathscr{G}}_{T_1} \to {\mathscr{G}}'' \to {\mathscr{G}}'_{T_1} \to 0
\]
of coherent sheaves on \(X_{T_1}\) that has the evident universal property.
This second sheaf defines an algebraic family that bounds the family in question in (ii).
\oldpage{221-05}This is also true for the kernel, cokernel, and image of the above homomorphism, and the consideration of this proves (i) (provided that we assume the cokernel to be flat with respect to \(T_0\), in which case we can again reduce to cutting \(T_0\) into pieces\ldots).
\end{proof}

\leavevmode\vadjust pre{\hypertarget{fga-3-iv-proposition-1.3}{}}%
\begin{itenv}{Proposition 1.3}
Let \(E\) be a bounded family of classes of sheaves on the fibres of \(X/S\).
Then the classes of the structure sheaves of the \((\operatorname{supp}{\mathscr{F}})_\mathrm{red}\), where \({\mathscr{F}}\) is a coherent sheaf on some \(X_K\), with \(K\) algebraically closed, and whose class is in \(E\), form a bounded family.

\end{itenv}

Here, \((\operatorname{supp}{\mathscr{F}})_\mathrm{red}\) denotes the support of \({\mathscr{F}}\), endowed with the induced reduced structure, i.e.~its structure sheaf is the quotient of \({\mathcal{O}}_{X_K}\) by the largest sheaf of ideals that defines \(\operatorname{supp}{\mathscr{F}}\).
We can prove the analogous result to \protect\hyperlink{fga-3-iv-proposition-1.3}{(1.3)} for the sheaves canonically induced from \({\mathscr{F}}\) by the theory of primary decomposition;
for example, the \({\mathscr{F}}/{\mathscr{F}}_i\), where the \({\mathscr{F}}_i\) are the primary subsheaves of \({\mathscr{F}}\) for the components of the support of \({\mathscr{F}}\), and minimal with respect to this property;
or the \({\mathcal{O}}_{X_K}/{\mathfrak{p}}\), where \({\mathfrak{p}}\) is a prime sheaf of ideals associated to \({\mathscr{F}}\), or the \({\mathcal{O}}_{X_K}/{\mathfrak{q}}\), where the \({\mathscr{q}}\) is a primary sheaf of ideals associated to a component of the support of \({\mathscr{F}}\) (the reference field being algebraically closed).

\hypertarget{fga-3-iv-section-2}{%
\subsection{Bounded families and the Hilbert polynomial}\label{fga-3-iv-section-2}}

In the following, we assume that \(X\) is projective over \(S\), and endowed with a very ample sheaf, denoted by \({\mathcal{O}}_X(1)\).
For every extension \(K\) of a residue extension \(k(s)\) of a point \(s\) of \(S\), we consider the corresponding sheaf \({\mathcal{O}}_{X_K}(1)\) on \(X_K\), which will again be very ample.

To each coherent sheaf \({\mathscr{F}}\) on \(X_K\), we associate the function
\[
  P_{\mathscr{F}}(n) = \text{the Euler--Poincaré characteristic of }{\mathscr{F}}(n)\text{ on }X_k
\]
which is a polynomial in the integer \(n\), and called the \emph{Hilbert polynomial} of \({\mathscr{F}}\).
For large values of \(n\), \(P(n)\) is exactly the dimension of \(\operatorname{H}^0(X_k,{\mathscr{F}}(n))\) over \(K\), since the \(\operatorname{H}^i(X_k,{\mathscr{F}}(n))\) are zero for \(i>0\) and large enough \(n\).

Now, if \({\mathscr{F}}\) is a coherent sheaf on \(X\) which is \emph{flat} with respect to \(S\), then the Hilbert polynomials of the sheaves \({\mathscr{F}}_S\) induced on the fibres \(X_S\) (with respect to one single connected component of \(S\)) are all equal {[}\protect\hyperlink{ref-GD1960}{10}, III, §7{]}.
It thus follows (without any flatness hypothesis) that the set of Hilbert polynomials of the sheaves \({\mathscr{F}}_s\), for \(s\in S\), is finite for every coherent sheaf \({\mathscr{F}}\) on \(X\).

\oldpage{221-06}Recall also that, if \({\mathscr{F}}\) is a coherent sheaf on \(X\), then it is isomorphic to a quotient sheaf of a sheaf of the form \({\mathscr{O}}_X(-n)^N\), for some large enough \(n,N\).
So the sheaves \({\mathscr{F}}_s\) induced on the fibres are also quotients of the sheaf \({\mathscr{O}}(-n)\) on the fibre.

From these two remarks, we reduce the ``necessary'' part of the following theorem:

\hypertarget{fga-3-iv-theorem-2.1}{}
\begin{itenv}{Theorem 2.1}

Let \(X\) be projective over \(S\), with \(S\) Noetherian, and \({\mathscr{O}}_X(1)\) very ample over \(X\) with respect to \(S\).
Let \(E\) be a set of classes of sheaves on the fibres of \(X/S\).
For \(E\) to be bounded, it is necessary and sufficient that it satisfy the following conditions:

\begin{enumerate}
\def\labelenumi{\alph{enumi}.}
\tightlist
\item
  There exists a coherent sheaf \({\mathscr{L}}\) on \(X\) (which we can suppose to be of the form \({\mathscr{O}}_X(-n)^N\)) such that \(E\) is contained in the family of classes of coherent sheaves given by quotients of sheaves of the form \({\mathscr{L}}_K\);
\item
  The Hilbert polynomials \(P_{\mathscr{F}}\) of the sheaves \({\mathscr{F}}\) whose class is in \(E\) are elements of a single \emph{finite} set of polynomials.
\end{enumerate}

\end{itenv}

It remains to prove the ``sufficient'' part, which will be a particular case of a more precise result.
For every coherent module \({\mathscr{F}}\) on a prescheme of finite type over a field \(K\), and for every integer \(r\), let \(N_r\) be the submodule of \({\mathscr{F}}\) whose sections over an open subset are the sections of \({\mathscr{F}}\) over the same subset whose support is of dimension \(<r\).
We thus have that \(N_r={\mathscr{F}}\) for \(r>\dim\operatorname{supp}{\mathscr{F}}\), and \(N_r=0\) for \(r\leqslant 0\), and we thus obtain a finite increasing filtration of \({\mathscr{F}}\) whose factors \(N_r/N_{r+1}\) are such that their associated prime cycles are exactly the associated prime cycles of \({\mathscr{F}}\) that are of dimension \(r\).
We set
\[
  {\mathscr{F}}_{(r)} = {\mathscr{F}}/N_r
\]
so that the associated prime cycles of \({\mathscr{F}}_{(r)}\) are exactly the associated prime cycles of \({\mathscr{F}}\) that are of dimension \(\geqslant r\), and, in particular, \({\mathscr{F}}_{(r)}\) is equal to \({\mathscr{F}}\) if and only if the associated prime cycles of \({\mathscr{F}}\) are of dimension \(\geqslant r\).
With this, we have:

\leavevmode\vadjust pre{\hypertarget{fga-3-iv-theorem-2.2}{}}%
\begin{itenv}{Theorem 2.2}
Under the conditions of \protect\hyperlink{fga-3-iv-theorem-2.1}{Theorem 2.1}, let \(s\) be an integer, and suppose that \(E\) satisfies condition (a), as well as the following weakened form of (b):

\begin{itemize}
\tightlist
\item
  (bs) The Poincaré polynomials \(P_{\mathscr{F}}\) of the sheaves \({\mathscr{F}}\) whose class is in \(E\) have coefficients \emph{in degrees \(\leqslant(s-1)\)} that are bounded.
\end{itemize}

Under these conditions, the sheaves \({\mathscr{F}}_{(s)}\) (for the \({\mathscr{F}}\) whose class is in \(E\)) form a bounded family.
Furthermore, the coefficients in degree \((s-2)\) of the \(P_{\mathscr{F}}\) are bounded below.

\end{itenv}

\oldpage{221-07}Thus:

\leavevmode\vadjust pre{\hypertarget{fga-3-iv-corollary-2.3}{}}%
\begin{itenv}{Corollary 2.3}
Suppose that the sheaves \({\mathscr{F}}\) whose class is in \(E\) are such that all their associated prime cycles are of dimension \(d\), with \(s\leqslant d\leqslant r\).
Then, in condition (b) of \protect\hyperlink{fga-3-iv-theorem-2.1}{Theorem 2.1}, we can restrict to the coefficients of \(P_{\mathscr{F}}\) between degree \((s-1)\) and \(r\), inclusive.

\end{itenv}

The end of this section is dedicated to a sketch proof of \protect\hyperlink{fga-3-iv-theorem-2.2}{Theorem 2.2}.
The key lemmas are the two following lemmas, of which the first is well known (and summarises the useful mathematical content of Chow coordinates):

\leavevmode\vadjust pre{\hypertarget{fga-3-iv-lemma-2.4}{}}%
\begin{itenv}{Lemma 2.4 (Chow)}
Consider the structure sheaves of the subschemes \(Y\) with fibre \(X_K\) (where \(K\) is an algebraically closed extension of the residue field of \(S\)), where \(Y\) is reduced, and all its components are of the same dimension \(r\) (and with \({\mathscr{O}}_Y\) being thought of as a quotient sheaf of \({\mathscr{O}}_X\)).
If the degrees of \(Y\) are bounded, then the \(Y\) form a bounded family.

\end{itenv}

Here, the degree \(a\) of \(Y\) can be most conveniently defined as the coefficient of the dominant term of \(P_{{\mathscr{O}}_Y}=an^r/r!+\ldots\).

\leavevmode\vadjust pre{\hypertarget{fga-3-iv-lemma-2.5}{}}%
\begin{itenv}{Lemma 2.5}
Let \({\mathscr{L}}\) be a coherent sheaf on \(X\), and \(E\) a set of classes of the quotient sheaves \({\mathscr{F}}\) of the sheaf \({\mathscr{L}}_K\) (where \(K\) is a residue extension of \(S\)).
Suppose that the fibres of \(X\) over \(S\) are of dimension \(\leqslant r\), and set
\[
  P_{\mathscr{F}}(n) = a_{\mathscr{F}}n^r/r! + b_{\mathscr{F}}n^{r-1}/(r-1)! + \text{terms of degree }<r-1.
\]
Then the coefficient \(a_{\mathscr{F}}\) is bounded (above), and \(b_{\mathscr{F}}\) is bounded below.
If \(b_{\mathscr{F}}\) is bounded, then the family \({\mathscr{F}}_{(r)}\) is bounded.

\end{itenv}

\begin{proof}
By replacing \(S\) by a union of subschemes of \(S\) that cover \(S\), we can suppose that there exists a \emph{finite} morphism \(f\colon X\to\mathbb{P}_S^r\) such that \({\mathscr{O}}_X(1)\) is isomorphic to the inverse image of \({\mathscr{O}}_{\mathbb{P}_S^r}(1)\), and thus, for every coherent sheaf \({\mathscr{F}}\) on \(X\), we have that \(P_{\mathscr{F}}=P_{f_*({\mathscr{F}})}\).
We can also easily show (by the technique of the previous section) that a set of sheaves \({\mathscr{F}}\) on \(X\) is bounded if and only if the set of \(f_*({\mathscr{F}})\) is bounded.
Finally, we have that
\[
  f_*({\mathscr{F}})_{(r)} = f_*({\mathscr{F}}_{(r)}).
\]
\oldpage{221-08}This thus allows us to reduce to the case where \(X=\mathbb{P}_S^r\).
Furthermore, we can suppose that \({\mathscr{L}}={\mathscr{O}}_{\mathbb{P}_S^r}(k)^s\), for some suitable \(k\) and \(s\).
The coefficient \(a_{\mathscr{F}}\) satisfies
\[
  0\leqslant a_{\mathscr{F}} \leqslant s
\]
and is thus bounded.
With this in mind, saying that the \(n^{r-1}\) coefficient \(P_{\mathscr{F}}(n)\) is bounded below (resp. bounded) is equivalent to saying the same thing for the \(P_{\mathscr{F}}(n-k)=P_{{\mathscr{F}}(-k)}(n)\).
This leads us to the case where
\[
  {\mathscr{L}} = {\mathscr{O}}_{\mathbb{P}_S^r}^s.
\]

Consider the exact sequence
\[
  0 \to N_r \to {\mathscr{F}} \to {\mathscr{F}}_{(r)} \to 0
\]
whence
\[
  P_{\mathscr{F}} = P_{{\mathscr{F}}_{(r)}} + P_{N_r}
\]
and, since the \(n^{r-1}\) coefficient of \(P_{N_r}\) is positive (since \(\dim\operatorname{supp}N_r\leqslant r-1\)), we have that
\[
  b_{{\mathscr{F}}_{(r)}} \leqslant b_{\mathscr{F}}.
\]
This allows us, in proving the lemma, to replace \({\mathscr{F}}\) by \({\mathscr{F}}_{(r)}\), i.e.~to suppose that the quotients \({\mathscr{F}}\) of \({\mathscr{L}}\) in question are torsion free.

Since \(\mathbb{P}_K^r\) is normal, it follows that \({\mathscr{F}}\) is locally free of rank \(a=a_{\mathscr{F}}\) on an open \(U=\mathbb{P}_K^r\setminus Y\), where \(Y\) is of codimension \(\geqslant 2\).
Thus \(\bigwedge^a{\mathscr{F}}\) is a sheaf on \(\mathbb{P}_K^r\) whose restriction to \(U\) is invertible, and thus (since \(\mathbb{P}_K^r\) is regular, and \(Y\) is of codimension \(\geqslant 2\)) isomorphism to the restriction of an invertible sheaf on \(\mathbb{P}_K^r\), defined up to isomorphism.
This latter sheaf is of the form \({\mathscr{O}}_{\mathbb{P}_K^r}(d)\) for some well defined integer \(d\).
Since \(\bigwedge^a{\mathscr{F}}\) is a quotient of \(\bigwedge^a{\mathscr{O}}_{\mathbb{P}_K^r}^n\simeq{\mathscr{O}}_{\mathbb{P}_K^r}^N\) with \(N=\binom{n}{a}\), it admits \(N\) canonical sections, which thus define sections of \({\mathscr{O}}_{\mathbb{P}_K^r}(d)\) over \(U\), which are restrictions of sections \(s_i\) (for \(1\leqslant i\leqslant N\)) of \({\mathscr{O}}_{\mathbb{P}_K^r}(d)\) (since \(\mathbb{P}_K^r\) is normal, and \(Y\) is of codimension \(\geqslant 2\)).
\oldpage{221-09}These \(s_i\) generate \({\mathscr{O}}_{\mathbb{P}_K^r}(d)\) at the points of \(U\), and are thus not all zero, which implies that \(d\geqslant 0\).
An easy calculation also shows that
\[
  b_{\mathscr{F}} = a_{\mathscr{F}}(r+1)/2 + d.
\]
This shows, in particular, that \(b_{\mathscr{F}}\geqslant 0\), and so \(b_{\mathscr{F}}\) is bounded below.
It is bounded if and only if \(d\) is bounded;
we will show that \({\mathscr{F}}\) then remains in a bounded family.
We can fix \(a_{\mathscr{F}}\) and \(b_{\mathscr{F}}\), as well as \(a\) and \(b\) (and thus \(d\)).
The data of the \(N\) sections \(s_i\) of \({\mathscr{O}}_{\mathbb{P}_K^r}(d)\), i.e.~of a homomorphism \(s\colon\bigwedge^a{\mathscr{L}}_K\to{\mathscr{O}}_{\mathbb{P}_K^r}(d)\), allows us to recover \({\mathscr{F}}\) as the co-image of the corresponding composite homomorphism:
\[
  {\mathscr{L}}_K
  \to \mathscr{H}\kern -.5pt om\left( \bigwedge^{a-1}{\mathscr{L}}_{K'}, \bigwedge^a{\mathscr{L}}_K \right)
  \to \mathscr{H}\kern -.5pt om\left( \bigwedge^{a-1}{\mathscr{L}}_{K'}, {\mathscr{O}}_{\mathbb{P}_K^r}(d) \right)
\]
where the first arrow is the canonical homomorphism coming from the exterior product, and the second comes from \(s\).
We then conclude by part (i) of \protect\hyperlink{fga-3-iv-proposition-1.2}{(1.2)}.
\end{proof}

The combination of the two lemmas above allows us to prove the following:

\leavevmode\vadjust pre{\hypertarget{fga-3-iv-lemma-2.6}{}}%
\begin{itenv}{Lemma 2.6}
Suppose, under the preliminary conditions of \protect\hyperlink{fga-3-iv-theorem-2.1}{(2.1)}, that, for all \({\mathscr{F}}\), we have
\[
  P_{\mathscr{F}}(n) = a_{\mathscr{F}} n^r/r! + b_{\mathscr{F}} n^{r-1}/(r-1)! + \text{terms of degree }<r-1
\]
and that the coefficients \(a_{\mathscr{F}}\) are bounded.
Then the coefficients \(b_{\mathscr{F}}\) are bounded below..
Furthermore, if the \(b_{\mathscr{F}}\) are bounded, then the \({\mathscr{F}}_{(r)}\) are bounded.

\end{itenv}

\begin{proof}
We can suppose that the base field \(K\) of the sheaves \({\mathscr{F}}\) is algebraically closed.
We endow each \(\operatorname{supp}{\mathscr{F}}_{(r)}\)(the union of the components of degree \(r\)) with the induced reduced structure.
Then the degrees of the \(\operatorname{supp}{\mathscr{F}}_{(r)}\) are bounded above by \(a\), and so, by \protect\hyperlink{fga-3-iv-lemma-2.4}{(2.4)}, the \(\operatorname{supp}{\mathscr{F}}_{(r)}\) form a bounded set.
Furthermore, for each component of \(\operatorname{supp}{\mathscr{F}}_{(r)}\), the length of \({\mathscr{F}}_{(r)}\) for this component is \(\leqslant a\), and so, if \({\mathscr{I}}_{\mathscr{F}}\) is the ideal that defines \(\operatorname{supp}{\mathscr{F}}_{(r)}\), then \({\mathscr{F}}_{(r)}\) can be considered as a module on the subscheme \(Y_{\mathscr{F}}\) of \(X\) defined by \({\mathscr{I}}_{\mathscr{F}}^a\).
\oldpage{221-10}As in the previous lemma, we can also reduce to the case where \({\mathscr{F}}={\mathscr{F}}_{(r)}\), so that \({\mathscr{F}}\) comes from a module on \(Y_{\mathscr{F}}\).
The \(Y_{\mathscr{F}}\) correspond to a bounded family of quotient modules of the \({\mathscr{O}}_{X_K}\), and thus come from a closed subscheme \(Y\) of some scheme \(X\times_S T\).
We can then apply \protect\hyperlink{fga-3-iv-lemma-2.5}{(2.5)} to \(Y/T\) and \({\mathscr{L}}\otimes_X Y\), whence the conclusion.
\end{proof}

\begin{proof}
We can now prove \protect\hyperlink{fga-3-iv-theorem-2.2}{(2.2)} by induction on the upper bound \(r\) of the \(\dim\operatorname{supp}{\mathscr{F}}\).
The statement is trivial for \(r<0\), so suppose that \(r\geqslant 0\) and that the statement has been proven for \(r'<r\).
By \protect\hyperlink{fga-3-iv-lemma-2.6}{(2.6)}, the \({\mathscr{F}}_{(r)}\) form a bounded family, and so too, by part (i) of {[}(1.2){]}, do the kernels of the homomorphisms \({\mathscr{L}}_K\to{\mathscr{F}}_{(r)}\);
there thus exists a coherent module \({\mathscr{L}}'\) on \(X\) such that kernels in question, and thus also the \(N_r({\mathscr{F}})=\operatorname{Ker}({\mathscr{F}}\to{\mathscr{F}}_{(r)})\), are quotients of modules \({\mathscr{L}}'_K\).
Since the \({\mathscr{F}}_{(r)}\) are bounded, the \(P_{{\mathscr{F}}_{(r)}}\) are bounded, and the formula
\[
  P_{\mathscr{F}} = P_{{\mathscr{F}}_{(r)}} + P_{N_r}
\]
then shows that the \(P_{N_r}\) satisfy the same condition (bs) as the \(P_{\mathscr{F}}\).
Thus the \(N_r\) satisfy conditions (a) and (bs), and so, by the induction hypothesis, the \((N_r)_{(s)}\) are bounded.
But \({\mathscr{F}}_{(s)}\) is an extension of \({\mathscr{F}}_{(r)}\) by \((N_r)_{(s)}\), and so, by part (ii) of \protect\hyperlink{fga-3-iv-proposition-1.2}{(1.2)}, the \({\mathscr{F}}_{(s)}\) are bounded.
For the last claim of \protect\hyperlink{fga-3-iv-theorem-2.2}{(2.2)}, we note that the kernels \(N_s\) of \({\mathscr{F}}\to{\mathscr{F}}_{(s)}\) are bounded, by part (i) of \protect\hyperlink{fga-3-iv-proposition-1.2}{(1.2)}, and that the coefficient of the \(n^{s-1}\) term in \(P_{N_s}\) is bounded;
then \protect\hyperlink{fga-3-iv-lemma-2.6}{(2.6)} proves that the coefficient of the following term is bounded below.
This finishes the proof
\end{proof}

\hypertarget{fga-3-iv-section-3}{%
\subsection{Hilbert schemes: definition, existence theorem}\label{fga-3-iv-section-3}}

Let \(X\) be a prescheme over another prescheme \(S\), and \({\mathscr{F}}\) a quasi-coherent module on \(X\).
We denote by
\[
  \operatorname{Quot}({{\mathscr{F}}/X/S})
\]
the set of quasi-coherent modules given by quotients of \({\mathscr{F}}\) that are flat over \(S\).
Now let \(S'\to S\) be a base change morphisms, and set \(X'=X\times_S S'\), and \({\mathscr{F}}'={\mathscr{F}}\otimes_{{\mathscr{O}}_S}{\mathscr{O}}_{S'}\), so that \(X'\) is a prescheme over \(S'\) endowed with a quasi coherent module \({\mathscr{F}}'\), and we can consider \(\operatorname{Quot}({{\mathscr{F}}'/X'/S'})\).
\oldpage{221-11}We set
\[
  \mathscr{Q}\kern -.5pt out_{{{\mathscr{F}}/X/S}}(S') = \operatorname{Quot}({{\mathscr{F}}'/X'/S'})
\]
(where \(X'=X\times_S S'\), as above).

Now, if \(S''\to S'\) is an \(S\)-morphism, then \(X''=X\times_S S''\) is isomorphic to \(X'\times_{S'}S''\), and \({\mathscr{F}}''\) is isomorphic to \({\mathscr{F}}'\otimes_{{\mathscr{O}}_{S'}}{\mathscr{O}}_{S''}\), and, since the inverse image functor \({\mathscr{G}}'\mapsto{\mathscr{G}}'\otimes_{{\mathscr{O}}_{S'}}{\mathscr{O}}_{S''}\) from the category of quasi-coherent modules on \(X''\) is right exact, and sends \(S'\)-flat modules to \(S''\)-flat modules, we obtain a natural map
\[
  \operatorname{Quot}({{\mathscr{F}}'/X'/S'}) \to \operatorname{Quot}({\mathscr{F}}''/X''/S'')
\]
and so \(\mathscr{Q}\kern -.5pt out_{{{\mathscr{F}}/X/S}}(S')\) is a \emph{contravariant functor in \(S'\)} (where \(S'\) is a prescheme over \(S\)), with values in the categories of sets.
In what follows, we suppose that \(X\) is \emph{projective} over a \emph{Noetherian} \(S\), with \({\mathscr{F}}\) \emph{coherent};
for simplicity, we will limit ourselves to considering those \(S'\) which are \emph{locally Noetherian} over \(S\).

\leavevmode\vadjust pre{\hypertarget{fga-3-iv-theorem-3.1}{}}%
\begin{itenv}{Theorem 3.1}
Under these conditions, the contravariant functor \(\mathscr{Q}\kern -.5pt out_{{{\mathscr{F}}/X/S}}\) on the category of locally Noetherian \(S\)-preschemes is \emph{representable} by an \(S\)-prescheme \(\underline{\operatorname{Quot}}_{{{\mathscr{F}}/X/S}}\), given by the sum of a sequence of projective \(S\)-schemes (and a fortiori \(\underline{\operatorname{Quot}}_{{{\mathscr{F}}/X/S}}\) is locally of finite type over \(S\)).

\end{itenv}

We will obtain such a decomposition in the following way.
Let \({\mathscr{O}}_X(1)\) be an invertible sheaf on \(X\) that is very ample with respect to \(S\).
For every polynomial \(P(n)\) with rational coefficients, let \(\operatorname{Quot}^P({{\mathscr{F}}/X/S})\) be the subset of \(\operatorname{Quot}({{\mathscr{F}}/X/S})\) consisting of coherent quotients \({\mathscr{G}}\) of \({\mathscr{F}}\) that are flat over \(S\) and whose Hilbert polynomial at each \(s\in S\) is equal to \(P\).
We then set
\[
  \mathscr{Q}\kern -.5pt out_{{\mathscr{F}}/X/S}^P(S') = \operatorname{Quot}^P({{\mathscr{F}}'/X'/S'})
\]
and thus obtain a subfunctor of \(\mathscr{Q}\kern -.5pt out_{{\mathscr{F}}/X/S}\).
The invariance property of Hilbert polynomials (recalled in \protect\hyperlink{fga-3-iv-section-2}{§2}) implies the following:
\emph{For \(\mathscr{Q}\kern -.5pt out_{{\mathscr{F}}/X/S}\) to be representable, it is necessary and sufficient that the \(\mathscr{Q}\kern -.5pt out_{{\mathscr{F}}/X/S}^P\) be representable, and then the \(S\)-prescheme \(\underline{\operatorname{Quot}}_{{{\mathscr{F}}/X/S}}\) which represents it is isomorphic to the prescheme given by the sum of the \(\underline{\operatorname{Quot}}_{{{\mathscr{F}}/X/S}}^P\) that represent the functors \(\mathscr{Q}\kern -.5pt out_{{\mathscr{F}}/X/S}^P\)}.
With this, \protect\hyperlink{fga-3-iv-theorem-3.1}{Theorem 3.1} will be a consequence of the following theorem:

\leavevmode\vadjust pre{\hypertarget{fga-3-iv-theorem-3.2}{}}%
\begin{itenv}{Theorem 3.2}
\oldpage{221-12}With the above notation, the functor \(\mathscr{Q}\kern -.5pt out_{{\mathscr{F}}/X/S}^P\) is representable by a projective \(S\)-prescheme \(\operatorname{Quot}_{{\mathscr{F}}/X/S}^P\).

\end{itenv}

The rest of this section is dedicated to the proof of \protect\hyperlink{fga-3-iv-theorem-3.2}{Theorem 3.2}.

Let \(\nu\) be an integer.
For every \(S'\) over \(S\), we denote by \(A_\nu(S')\) the set of quotients \({\mathscr{G}}={\mathscr{F}}'/{\mathscr{H}}\) of \({\mathscr{F}}'={\mathscr{F}}\otimes_{{\mathscr{O}}_S}{\mathscr{O}}_{S'}\) that are coherent, flat over \(S'\), and satisfy the following conditions:

\begin{enumerate}
\def\labelenumi{\alph{enumi}.}
\tightlist
\item
  \(\operatorname{R}^i f'_*({\mathscr{G}}(n))=0\) for \(i>0\) and \(n\geqslant\nu\);
\item
  \(\operatorname{R}^i f'_*({\mathscr{H}}(n))=0\) for \(i>0\) and \(n\geqslant\nu\);
\item
  \(f'_*({\mathscr{H}}(\nu+k))={\mathcal{S}}'_k f'_*({\mathscr{H}}(\nu))\) for \(k\geqslant 0\).
\end{enumerate}

For this last condition, we suppose that \(X\) is written as the homogeneous prime spectrum of a quasi-coherent positively-graded algebra \({\mathcal{S}}_\bullet\) over \(S\) that is generated by \({\mathcal{S}}_1\), and we set \({\mathcal{S}}'={\mathcal{S}}\otimes_{{\mathscr{O}}_S}{\mathscr{O}}_{S'}\), so that \(X'\) is the homogeneous prime spectrum of \({\mathcal{S}}'\).
To prove \protect\hyperlink{fga-3-iv-theorem-3.2}{Theorem 3.2}, we can easily reduce to the case where \(X=\mathbb{P}_S^r\) (since \(S\) is a union of open subsets \(U\) such that \(X|U\) is a closed subscheme of \(\mathbb{P}_U^r\), and \({\mathscr{O}}_X(1)\) is induced by \({\mathscr{O}}_{\mathbb{P}_U^r}(1)\)), and where \({\mathscr{F}}\) is of the form \(({\mathscr{O}}_{\mathbb{P}_S^r})^N\), and thus flat over \(S\).
Then, in the above, the sheaves \({\mathscr{H}}\) are also flat over \(S'\).
It then follows from the Künneth relations {[}\protect\hyperlink{ref-GD1960}{10}, III, §7{]} and from (b) that the conditions (a) and (b) are stable under base change, and imply that, for \(n\geqslant\nu\), forming \(f'_*({\mathscr{G}}(n))\) and \(F'_*({\mathscr{H}}(n))\) commutes with extension of the base (\emph{loc. cit.}).
In other words, \emph{\(A_\nu(S')\) is a contravariant functor in \(S'\), and in a precise sense a subfunctor of \(A(S')=\mathscr{Q}\kern -.5pt out_{{{\mathscr{F}}/X/S}}^P(S')\)}.
For varying \(\nu\), we thus obtain an increasing sequence of subsets \(A_\nu(S')\) of \(A(S')\), whose union is \(A(S')\) by a well known theorem of Serre {[}\protect\hyperlink{ref-GD1960}{10}, III, §2{]}.
Note now that, if \({\mathscr{G}}\) is a coherent quotient of \({\mathscr{F}}'\) which is flat over \(S\), and \(s\) an element of \(S\) such that the base change \(\operatorname{Spec}(k(s))\to S\) gives rise to a quotient \({\mathscr{G}}_s\) of \({\mathscr{F}}_s\) satisfying conditions (a), (b), and (c), i.e.~is in \(A_\nu\operatorname{Spec}(k(s))\), then there exists an open neighbourhood \(U\) of \(s\) such that these same conditions are satisfied by \({\mathscr{G}}|(f')^{-1}(U)\), i.e.~this quotient is in \(A_\nu(U)\);
for (a) and (b), this follows in fact from the ``Theorem of holomorphic functions'' {[}\protect\hyperlink{ref-GD1960}{10}, III, §7{]}, and (c) follows from the Nakayama lemma and the fact that we know that \(f'_*({\mathscr{H}}(n+k))=S'_kf'_*({\mathscr{H}}(n))\) anyway for \(n\) large enough and \(k\geqslant 0\) {[}\protect\hyperlink{ref-GD1960}{10}, III,§2{]}.
\oldpage{221-13}From these remarks, we conclude the following (compare with {[}\protect\hyperlink{ref-Gro1960a}{8}, IV{]}).
\emph{For the functor \(A\) to be representable, it is necessary and sufficient that the functors \(A_\nu\) be representable, and then the \(S\)-prescheme \(Q\) that represents \(A\) is the increasing union of the opens \(Q_\nu\) that represent the \(A_\nu\).}

Let
\[
  M_\bullet
  = \sum_{n\geqslant 0} f_*({\mathscr{F}}(n))
  = {\mathcal{S}}_\bullet^N
\]
so that we have
\[
  M'_\bullet
  = M_\bullet\otimes_{{\mathscr{O}}_S}{\mathscr{O}}_{S'}
  = \sum_{n\geqslant 0} f'_*({\mathscr{F}}(n))
  = {{\mathcal{S}}'_\bullet}^N.
\]

It follows from (a) that we have

a'. \(f'_*({\mathscr{G}}(n))\) is locally free of rank \(P(n)\) for \(n\geqslant\nu\) {[}\protect\hyperlink{ref-GD1960}{10}, III, §7{]}

and it follows from (b),for \(i=1\), that we have

a'\,'. \(f'_*({\mathscr{G}}(n))\) is a quotient module of \(M'_n\).

Also, the knowledge of this quotient module, for \(n=\nu\), implies, by (c), that the knowledge of the submodules \(f'_*({\mathscr{H}}(n))\) of \(M_n\) for \(n\geqslant\nu\), and thus the knowledge of \({\mathscr{H}}\) and consequently of \({\mathscr{G}}\).
We thus obtain an \emph{injective map}
\[
  A_\nu(S') \to \mathscr{G}\kern -.5pt rass_{P(\nu)}(M'_\nu)
\]
from \(A_\nu(S')\) to the set of locally free quotient modules of \(M'\) of rank \(P(\nu)\), whence a \emph{functorial} homomorphism
\[
  i_\nu\colon A_\nu(S') \to \mathscr{G}\kern -.5pt rass_{P(\nu)}(M_\nu)(S')
\]
where the functor on the right hand side is representable by the Grassmannian scheme \(\operatorname{Grass}_{P(\nu)}(M_\nu)\) (compare with {[}\protect\hyperlink{ref-Gro1960a}{8}, V{]}), which is projective over \(S\).
Then

\leavevmode\vadjust pre{\hypertarget{fga-3-iv-lemma-3.3}{}}%
\begin{itenv}{Lemma 3.3}
\(A_\nu(S')\) is a representable functor, and the morphism \(Q_\nu\to\operatorname{Grass}_{P(\nu)}(M_\nu)\) that represents the homomorphism \(i_\nu\) is an immersion (which implies that \(Q_\nu\) is quasi-projective over \(S\)).

\end{itenv}

This claim is equivalent to the following (compare with {[}\protect\hyperlink{ref-Gro1960a}{8}, IV{]}):
If we have a quotient module \(N\) of \(M'_\nu\) that is locally free of rank \(P(\nu)\), then there exists a subprescheme \(Z\) of \(S'\) such that, for every locally Noetherian prescheme \(T'\) over \(S'\), the inverse image of \(N\) over \(T'\) is in \(\Im A_\nu(T')\) if and only if \(T'\to S'\) is bounded by the subprescheme \(Z\).
\oldpage{221-14}Changing notation, we can suppose that \(S'=S\), i.e.~we have a quotient \(N_\nu\) of \(M_\nu\) by a submodule \(R_\nu\).
For it to come from an element of \(A(s)\), it is necessary and sufficient that it satisfy the following two conditions:

\begin{enumerate}
\def\labelenumi{\roman{enumi}.}
\tightlist
\item
  \(M_{\nu+k}/{\mathcal{S}}_k R_\nu\) is locally free of rank \(P(\nu+k)\) for \(k\geqslant 0\).
\item
  Both the subsheaf \({\mathscr{H}}\) of \({\mathscr{F}}\) defined by the graded submodule \(R_\bullet=\sum_{k\geqslant 0}{\mathscr{S}}_kR_\nu\) of \(M_\bullet\) (cf. {[}\protect\hyperlink{ref-GD1960}{10}, II, §3{]}) and the quotient \({\mathscr{G}}={\mathscr{F}}/{\mathscr{H}}\) satisfy conditions (a) and (b) above.
\end{enumerate}

These conditions are clearly necessary, and if they are satisfied then the sheaf \({\mathscr{G}}\) defined in (ii) above, being isomorphic to the sheaf associated to the graded \({\mathscr{S}}_\bullet\)-module \(N_\bullet\) given by the sum of the \(M_{n+k}/{\mathscr{S}}_kR\) is \emph{flat} over \(S\) (since its fibres are direct factors of localisations of \(N\) for homogeneous prime ideals of \({\mathcal{S}}_\bullet\)), and correspond to the Hilbert polynomial \(P\) by virtue of (i).
Taking (ii) into account, we then see that (a') and (a'\,') are satisfied, and thus, for \(n\geqslant\nu\), the natural homomorphism \(N_n\to f_*({\mathscr{G}}(n))\) is a \emph{surjective} homomorphism of locally free modules of equal rank, and thus an isomorphism, and thus \(f_*({\mathscr{H}}(\nu+k))={\mathcal{S}}_kR_\nu\) for all \(k\geqslant 0\), which proves that \({\mathscr{G}}\in A_\nu(S)\) and that \(R_\nu\) is the element of \(\operatorname{Grass}_{P(\nu)}(M_\nu)\) defined by \({\mathscr{G}}\).

Criteria (i) and (ii) above apply equally to the situation obtained after a change of base \(S'\to S\).
We will prove first of all the fact that condition (i) is satisfied after the change of base \(S'\to S\) can be expressed by saying that \(S'\to S\) is bounded by a certain subprescheme \(Z\) of \(S\);
once we have shown this result, we are led (replacing \(S\) with \(Z\)) to the case where condition (i) is already satisfied on \(S\), and since it is stable under change of base, it remains to express condition (ii).
But then, if \(U\) denotes the set of \(s\in S\) such that the cohomology of the sheaves induced on the fibre \(X_s\) by \({\mathscr{G}}(n)\) and \({\mathscr{H}}(n)\) is zero in dimension \(>0\) for \(n\geqslant\nu\), then we have already shown that \(U\) is open, and condition (ii) will be satisfied after a change of base \(S'\to S\) if and only if \(S'\to S\) is bounded by \(U\), which proves \protect\hyperlink{fga-3-iv-lemma-3.3}{Lemma 3.3}.
It thus remains to prove the following lemma:

\leavevmode\vadjust pre{\hypertarget{fga-3-iv-lemma-3.4}{}}%
\begin{itenv}Lemma 3.4
Let \(S\) be a locally Noetherian prescheme, endowed with a quasi-coherent positively-graded algebra \({\mathcal{S}}_\bullet\) generated by \({\mathcal{S}}_1\), and let \(M_\bullet\) be a quasi-coherent graded \({\mathcal{S}}\)-module of finite type, \(P\) a polynomial with rational coefficients, and \(\nu\) and integer.
Then there exists a (clearly uniquet) subprescheme \(Z\) of \(S\) that has the following property:
for every prescheme \(S'\) over \(S\), for \(M_n\otimes_{{\mathscr{O}}_S}{\mathscr{O}}_{S'}\) to be locally free of rank \(P(n)\) for all \(n\geqslant\nu\), it is necessary and sufficient that \(S'\to S\) be bounded by \(Z\).

\end{itenv}

\oldpage{221-15}We can evidently suppose that \(S\) is affine, and thus Noetherian.
Then:

\leavevmode\vadjust pre{\hypertarget{fga-3-iv-lemma-3.5}{}}%
\begin{itenv}{Lemma 3.5}
For every integer \(N\geqslant\nu\), let \(U_N\) be the open subset of \(S\) consisting of \(s\in S\) such that \(\operatorname{rank}_{k(s)}M_{ns}\otimes_{{\mathscr{O}}_{S,s}}k(s)\leqslant P(n)\) for all \(\nu\leqslant n\leqslant N\).
Then the decreasing sequence of open subsets \(U_N\) stabilises.

\end{itenv}

\begin{proof}
We know {[}\protect\hyperlink{ref-Gro1960b}{9}, IV{]} that \(S\) admits a finite partition into reduced subschemes \(S_i\) such that each \(M\otimes_{{\mathscr{O}}_S}{\mathscr{O}}_{S_i}\) is \emph{flat} over \(S\).
We can thus suppose that \(M\) is flat, and thus that the \(M_n\) are flat.
Finally, we can evidently suppose that \(S\) is connected.
But then {[}\protect\hyperlink{ref-GD1960}{10}, III, §7{]} there exists an integer \(n_0\) and a polynomial \(Q\) such that
\[
  \operatorname{rank}_{k(s)}M_{ns}\otimes_{{\mathscr{O}}_{S,s}}k(s) = Q(n)
  \qquad\text{for }n\geqslant n_0.
\]
Suppose first of all that \(P<Q\), and so \(P(n)\neq Q(n)\) for large \(n\).
Then we evidently have \(U_N=\varnothing\) for large enough \(N\), and thus a fortiori the sequence of \(U_N\) stabilises.
In the contrary case, we have \(P(n)\geqslant Q(n)\) for large \(n\), and so \(U_N=U_{n_0}\) for \(N\geqslant n_0\), and the sequence of the \(U_N\) again stabilises.
\end{proof}

In particular, the set \(U_\infty\) of \(s\in S\) such that

\leavevmode\vadjust pre{\hypertarget{fga-3-iv-equation-asterisk}{}}%
\begin{eqenv}
\[
  \operatorname{rank}_{k(s)}M_{ns}\otimes_{{\mathscr{O}}_{S,s}}k(s) \leqslant P(n)
  \qquad\text{for all }n\geqslant\nu
\tag{$*$}
\]

\end{eqenv}

is open, since it is the intersection of the \(U_N\).
We can then, for the proof of \protect\hyperlink{fga-3-iv-lemma-3.4}{Lemma 3.4} replace \(S\) with the open subset \(U\), which leads us to the case where the inequality in \protect\hyperlink{fga-3-iv-equation-asterisk}{(\(*\))} is satisfied at all \(s\in S\).

\leavevmode\vadjust pre{\hypertarget{fga-3-iv-lemma-3.6}{}}%
\begin{itenv}{Lemma 3.6}
Let \(M\) be a module on a locally Noetherian prescheme \(S\), and \(r\) an integer.
Then there exists a (clearly unique) subprescheme \(Z\) of \(S\) that has the following property:
for all \(S'\) over \(S\), for \(M\otimes_{{\mathscr{O}}_S}{\mathscr{O}}_{S'}\) to be locally free of rank \(r\), it is necessary and sufficient that \(S'\to S\) be bounded by \(Z\).
If \(\operatorname{rank}_{k(s)}M_s\otimes_{{\mathscr{O}}_{S,s}}k(s)\leqslant r\) for all \(s\), then \(Z\) is a closed subprescheme of \(S\) (supposing that \(M\) is coherent).

\end{itenv}

\begin{proof}
\oldpage{221-16}Indeed, the above reasoning leads us to the case where we have the inequality (\(*\)) for all \(s\in S\) (by replacing, if necessary, \(S\) with the open subset consisting of the \(s\) where the inequality is satisfied).
We can then suppose that \(M\) fits into an exact sequence
\[
  {\mathscr{O}}_S^q \to {\mathscr{O}}_S^r \to M \to 0
\]
and the condition in question on the \(S'\) over \(S\) also implies that, in the corresponding exact sequence \({\mathscr{O}}_{S'}^q\to{\mathscr{O}}_{S'}^r\to M'\to 0\), the second arrow is an isomorphism, i.e.~the first is zero.
We then see that the closed subprescheme \(Z\) of \(S\) defined by the ideal generated by the coefficients of the matrix defining the homomorphism \({\mathscr{O}}_S^q\to{\mathscr{O}}_S^r\) satisfies the desired condition.
\end{proof}

\begin{proof}
Returning to the proof of \protect\hyperlink{fga-3-iv-lemma-3.4}{(3.4)} where we left off, we denote by \(Z_n\) the \emph{closed} subprescheme of \(S\) associated, by \protect\hyperlink{fga-3-iv-lemma-3.6}{(3.6)}, to the module \(M_n\) and the integer \(r=P(n)\), and by \(Z'_N\) the infimum of the \(Z_n\) for \(\nu\leqslant n\leqslant N\).
Then the \(Z_N\) form a decreasing sequence of closed subpreschemes of \(Z\), which is thus necessarily stationary.
Let \(Z\) be the constant value of the \(Z_W\) for large \(N\).
This is the desired \(Z\) in \protect\hyperlink{fga-3-iv-lemma-3.4}{(3.4)}.
This finishes the proof of \protect\hyperlink{fga-3-iv-lemma-3.4}{(3.4)}, and thus also of \protect\hyperlink{fga-3-iv-lemma-3.3}{(3.3)}
\end{proof}

We have thus proven that \emph{\(\mathscr{Q}\kern -.5pt out_{{{\mathscr{F}}/X/S}}^P\) is representable by an \(S\)-prescheme \(Q\) that is an increasing union of open quasi-projective subpreschemes \(Q_\nu\) over \(S\).}
To go further, we need to invoke \protect\hyperlink{fga-3-iv-theorem-2.1}{Theorem 2.1}, whence we easily conclude that \(Q\) is \emph{quasi-compact} (since it is the image of a prescheme \(S'\) of finite type over \(S\) that parametrises the family of quotient sheaves of the \({\mathscr{F}}_K\) whose Hilbert polynomial is \(P\)).
Thus \(Q\) is equal to one of the \(Q_\nu\), and thus quasi-projective over \(S\).
To prove that it is projective over \(S\), it thus remains to prove that it is \emph{proper} over \(S\), and for this it suffices to invoke the valuative criteria of properness in the form given in {[}\protect\hyperlink{ref-GD1960}{10}, II, 7.3.8{]}.
It suffices to verify the following:

\leavevmode\vadjust pre{\hypertarget{fga-3-iv-lemma-3.7}{}}%
\begin{itenv}{Lemma 3.7}
Let \(S\) be the spectrum of a discrete valuation ring, \(s\) its generic point, \(X\) a prescheme over \(S\), \({\mathscr{F}}\) a quasi-coherent module over \(X\), and \({\mathscr{G}}_s\) a quasi-coherent quotient module of \({\mathscr{F}}_s={\mathscr{F}}\otimes_{{\mathscr{O}}_S}k(s)\) over \(X_s\).
Then there exists a unique quasi-coherent quotient module \({\mathscr{G}}\) of \({\mathscr{F}}\) that is flat over \(S\) and whose restriction to \(X_s\) is \({\mathscr{G}}_s\).

\end{itenv}

\begin{proof}
\oldpage{221-17}Indeed, if \({\mathscr{G}}_s={\mathscr{F}}_s/{\mathscr{H}}_s\), it suffices to consider the largest subsheaf \({\mathscr{H}}\) of \({\mathscr{F}}\) that induces \({\mathscr{H}}_s\) ({[}\protect\hyperlink{ref-GD1960}{10}, I, 9.4.2{]}) and to take \({\mathscr{G}}={\mathscr{F}}/{\mathscr{H}}\).
We easily verify that this sheaf works.
\end{proof}

\protect\hyperlink{fga-3-iv-theorem-3.2}{Theorem 3.2}, and thus \protect\hyperlink{fga-3-iv-theorem-3.1}{Theorem 3.1}, is now completely proven.

The proof also shows, at the same time, the following:

\leavevmode\vadjust pre{\hypertarget{fga-3-iv-proposition-3.8}{}}%
\begin{itenv}{Proposition 3.8}
Under the conditions of \protect\hyperlink{fga-3-iv-theorem-3.2}{(3.2)}, let \(Q=\underline{\operatorname{Quot}}_{{{\mathscr{F}}/X/S}}^P\), \(X_Q=X\times_s Q\), \({\mathscr{F}}_Q={\mathscr{F}}\otimes_{{\mathscr{O}}_S}{\mathscr{O}}_Q\), and let \({\mathscr{G}}\) be the coherent quotient of \({\mathscr{F}}_Q\), which is flat over \(Q\), that has \(P\) as its relative Hilbert polynomial, so that \((Q,{\mathscr{G}})\) represents the functor \(\mathscr{Q}\kern -.5pt out_{{{\mathscr{F}}/X/S}}^P\).
Then there exists an integer \(\nu\) such that, for \(n\geqslant\nu\), \((f_Q)_*({\mathscr{G}}(n))\) is a locally free module over \(Q\) of rank \(P(n)\), and is very ample with respect to \(S\), i.e.~it defines an immersion of \(Q\) into a Grassmannian scheme \(\underline{\operatorname{Grass}}_{P(n)}(M)\) over \(S\).
A fortiori, for \(n\geqslant\nu\), the sheaf \(\bigwedge^{P(n)}(f_Q)_*({\mathscr{G}}(n))\) over \(Q\) is invertible and very ample with respect to \(S\).

\end{itenv}

\begin{proof}
Indeed, we can reduce, as in \protect\hyperlink{fga-3-iv-theorem-3.2}{(3.2)}, to the case where \({\mathscr{F}}\) is flat over \(S\), and then it suffices to take an integer \(\nu\) such that \(A_\nu=A\) (with the notation above).
\end{proof}

The most important application of \protect\hyperlink{fga-3-iv-theorem-3.2}{(3.2)} is in the case where \({\mathscr{F}}={\mathscr{O}}_X\).
We then write
\[
  \begin{aligned}
    \underline{\operatorname{Quot}}_{{\mathscr{O}}_X/X/S}
    &= \underline{\operatorname{Hilb}}_{X/S}
  \\\underline{\operatorname{Quot}}_{{\mathscr{O}}_X/X/S}^P
    &= \underline{\operatorname{Hilb}}_{X/S}^P
  \end{aligned}
\]
and so we have a decomposition
\[
  \underline{\operatorname{Hilb}}_{X/S} = \coprod_P \underline{\operatorname{Hilb}}_{X/S}^P.
\]
By definition, \(\underline{\operatorname{Hilb}}_{X/S}\) represents the functor \(\mathscr{H}\kern -.5pt ilb_{X/S}(S')\) which is given by the set of closed subpreschemes of \(X'=X\times_S S'\) that are flat over \(S\);
and \(\underline{\operatorname{Hilb}}_{X/S}^P\) represents the subfunctor corresponding to the closed subpreschemes that admit a given Hilbert polynomial \(P\).
These preschemes are also called the \emph{Hilbert prescheme} of \(X\) over \(S\) and the \emph{Hilbert prescheme of index \(P\)}, respectively.
The terminology is justified by the role played in the theory by the Hilbert polynomials.
Their difference in nature with the classical Chow varieties (meant to parametrise cycles, not varieties) is analogous to that between \emph{the Chow ring} of classes of cycles of a variety and \emph{the ring of classes of sheaves} of the variety (as is introduced in the Riemann--Roch theorem {[}\protect\hyperlink{ref-BS1958}{1}{]});
\oldpage{221-18}we note that, when \(X=\mathbb{P}_S^r\), with \(S\) the spectrum of a field, the knowledge of the Hilbert polynomial of a coherent module \({\mathscr{F}}\) over \(X\) is equivalent to that of the Chern classes of \({\mathscr{F}}\), or even of the class of \({\mathscr{F}}\) in the ring of classes of coherent sheaves on \(X\).

\leavevmode\vadjust pre{\hypertarget{fga-3-iv-remarks-3.9}{}}%
\begin{rmenv}{Remarks 3.9}
We also note that the construction of \(\underline{\operatorname{Quot}}_{{{\mathscr{F}}/X/S}}\) and \(\underline{\operatorname{Quot}}_{{{\mathscr{F}}/X/S}}^P\) was reduced to the case where \(X=\mathbb{P}_S^r\) and \({\mathscr{F}}={\mathscr{O}}_X^N\), with \({\mathscr{O}}_X(1)\) being the usual very ample sheaf;
more precisely, the general \(\underline{\operatorname{Quot}}_{{{\mathscr{F}}/X/S}}\) arise as closed subpreschemes of the above.
Since forming the \(\underline{\operatorname{Quot}}_{{{\mathscr{F}}/X/S}}^P\) is evidently compatible with base change \(S'\to S\), we see that we can reduce to the case where further \(S=\operatorname{Spec}(\mathbb{Z})\):
\[
  {\mathscr{Q}}_{r,N}^P = \underline{\operatorname{Quot}}_{({\mathscr{O}}_{\mathbb{P}_{\mathbb{Z}}^r})^N/\mathbb{P}_{\mathbb{Z}}^r/\operatorname{Spec}(\mathbb{Z})}
\]
and, more particularly, the absolute Hilbert schemes:
\[
  \underline{\operatorname{Hilb}}_r^P = {\mathscr{Q}}_{r,1}^P.
\]
A more detailed study of these schemes, starting with determining their connected components (are they connected?), and their irreducible components (by Serre {[}\protect\hyperlink{ref-Ser1961}{24}{]}, there can exist irreducible components that exist entirely over a prime number \(p\neq0\)), would be very interesting.
Recall the question of Weil, asking if the irreducible components of the fibres of \(\underline{\operatorname{Hilb}}_r^P\) over the \(s\in\operatorname{Spec}(\mathbb{Z})\) correspond to ``regular'' extensions of the prime field, i.e.~if they are ``relatively connected''.
It could be the case that these questions are more accessible for Hilbert schemes than for ``Chow varieties''.

\emph{{[}Comp.{]}}
The study of connected components of Hilbert schemes over an algebraically closed field was done by Hartshorne, who proves that the \(\underline{\operatorname{Hilb}}_r^P\) are connected, and determines the pairs \((r,P)\) for which \(\underline{\operatorname{Hilb}}_r^P\neq\varnothing\) {[}\protect\hyperlink{ref-Har1966}{11}{]}.

\end{rmenv}

\hypertarget{fga-3-iv-section-4}{%
\subsection{Variants}\label{fga-3-iv-section-4}}

\begin{enumerate}
\def\labelenumi{\alph{enumi}.}
\item
  Under the conditions of \protect\hyperlink{fga-3-iv-theorem-3.1}{(3.1)}, let \(U\) be open in \(X\), and denote by \(A'\) the subfunctor of \(A=\mathscr{Q}\kern -.5pt out_{{{\mathscr{F}}/X/S}}\) such that \(A'(S')\) is the set of quotient modules \({\mathscr{G}}\) of \({\mathscr{F}}'\) that are flat over \(S'\) and whose support is contained inside \(U'\).
  We immediately see that \(A'\) is representable by an open subset of the prescheme \(\underline{\operatorname{Quot}}_{{{\mathscr{F}}/X/S}}\) that represents \(A\).
  \oldpage{221-19}It follows that Theorems \protect\hyperlink{fga-3-iv-theorem-3.1}{(3.1)} and \protect\hyperlink{fga-3-iv-theorem-3.2}{(3.2)} remain true if we suppose that \(X\) is \emph{quasi-projective} over \(S\) instead of projective over \(S\), as long as we also replace in the conclusions the word ``projective'' by ``quasi-projective'', and use \(\mathscr{Q}\kern -.5pt out_{{{\mathscr{F}}/X/S}}(S')\) to mean the set of coherent quotients \({\mathscr{G}}\) of \({\mathscr{F}}'\) that are flat over \(S'\) and \emph{whose support is proper over \(S'\).}
\item
  Generally we can impose all sorts of supplementary natural conditions on the quotients \({\mathscr{G}}\) of \({\mathscr{F}}'={\mathscr{F}}\otimes_{{\mathscr{O}}_S}{\mathscr{O}}_{S'}\) that are flat over \(S'\) and stable under base change, thus obtaining as many subfunctors of \(\mathscr{Q}\kern -.5pt out_{{{\mathscr{F}}/X/S}}\) as we want to represent.
  The usual criteria allow us, in many cases, to prove that we again obtain functors that are representable by open subsets of \(\underline{\operatorname{Quot}}_{{{\mathscr{F}}/X/S}}\).
  This is, in particular, the case if we impose one of the following additional properties:
\end{enumerate}

\begin{enumerate}
\def\labelenumi{\arabic{enumi}.}
\tightlist
\item
  The dimensions of the prime cycles associated to the modules \({\mathscr{G}}_{s'}\) (for \(s'\in S'\)) that are induced on the fibres \(X'_{s'}\) belong to a given set of integers.
\item
  (In the case where \({\mathscr{F}}={\mathscr{O}}_X\), and thus \({\mathscr{G}}\) corresponds to a closed subprescheme \(Y\) of \(X'\)); \(Y\) is a \emph{simple} prescheme {[}\protect\hyperlink{ref-Gro1960b}{9}, IV{]} over \(S\), resp. \emph{normal} over \(S\) (i.e.~the fibres \(Y_{s'}\) are normal ``over \(k(s)\)'', i.e.~are normal under any extension of base field), resp. (if \(X\) is flat over \(S\)) are local complete \(k\)-intersections in \(X\) with respect to \(S\) (i.e.~the fibres \(Y_{s'}\) are local complete intersections in the \(X_{s'}\))
\end{enumerate}

Other conditions would involve properties of a cohomological nature on the modules \({\mathscr{G}}_{s'}\) induced on the \(X'_{s'}\), etc.
Of course, the conjunction of conditions where each is represented by an open \(U_i\) of \(\underline{\operatorname{Quot}}_{{{\mathscr{F}}/X/S}}\) is represented by the open intersection.
For example, considering, for all \(S'\) over \(S\), the set of closed subpreschemes \(Y\) of \(X'=X\times_S S'\) that are étale covers {[}\protect\hyperlink{ref-Gro1960b}{9}, I{]} of a given rank \(r\) over \(S'\), we obtain a representable contravariant functor in \(S'\).

\begin{enumerate}
\def\labelenumi{\alph{enumi}.}
\setcounter{enumi}{2}
\tightlist
\item
  The preschemes \(\underline{\operatorname{Hom}}_S(X,Y)\), \(\prod_{X/S}Z/S\), and \(\underline{\operatorname{Isom}}_S(X,Y)\), defined in \protect\hyperlink{fga-3-ii-section-C.2}{FGA 3.II, C.2} exist thanks to suitable projective hypotheses, and can be realised as opens in suitable Hilbert preschemes.
  Since we have \(\underline{\operatorname{Hom}}_S(X,Y)=\prod_{X/S}((X\times Y)/X)\), the case of \(\operatorname{Hom}_S(X,Y)\) reduces to that of \(\prod_{X/S}(Z/X)\).
  We then note that, for all \(S'\) over \(S\), the set of sections of \(Z'=Z\times_S S'\) over \(X'=X\times_S S'\) is in bijective correspondence with the set of subpreschemes \(\Gamma\) of \(Z\) (necessarily closed if \(Z\) is separated over \(X\)) such that the morphism \(\Gamma\to X'\) induced by \(Z'\to X'\) is an isomorphism.
  \oldpage{221-20}In this way, \emph{if \(X\) is flat and proper over \(S\), and \(Z\) quasi-projective over \(S\), then \(\prod_{X/S}(Z/X)\) exists and is realised as an open subprescheme of \(\underline{\operatorname{Hilb}}_{Z/S}\).}
  Thus \emph{if \(X\) is projective and flat over \(S\), and \(Y\) quasi-projective over \(S\), then \(\underline{\operatorname{Hom}}_S(X,Y)\) exists and is realised as an open subprescheme of \(\underline{\operatorname{Hilb}}_{(X\times_S Y)/S}\).}
  If \(X\) and \(Y\) are both projective over \(S\), then it immediately follows that \(\underline{\operatorname{Isom}}_S(X,Y)\) also exists, and is represented by an open subset of \(\underline{\operatorname{Hom}}_S(X,Y)\).
  Similarly, if \(X\) is flat and projective over \(S\), and \(Y\) quasi-projective over \(S\), then the \(S\)-prescheme \(\underline{\operatorname{Imm}}_S(X/Y)\) that corresponds to the subfunctor of the functor represented by \(\underline{\operatorname{Hom}}_S(X,Y)\) that corresponds to \(S'\)-homomorphisms \(X'\to Y'\) that are immersions is also representable by an open subset of \(\underline{\operatorname{Hom}}_S(X,Y)\).
\end{enumerate}

Let \({\mathscr{L}}\) (resp. \({\mathscr{M}}\)) be an invertible sheaf on \(X\) (resp. \(Y\)) that is very ample with respect to \(S\), whence we obtain a sheaf \({\mathscr{L}}\otimes_{{\mathscr{O}}_S}{\mathscr{M}}\) on \(X\times_S Y\) that is very ample with respect to \(S\).
Then, for any polynomial \(P\) with rational coefficients, \(\underline{\operatorname{Hilb}}_{(X\times_S Y)/S}^P\) is defined and is a quasi-projective prescheme over \(S\).
It thus induces, on \(\underline{\operatorname{Hom}}_S(X,Y)\), a subset that is both open and closed, and quasi-projective over \(S\), which we denote by \(\operatorname{Hom}_S(X,Y)^P\).
Thus the sections of \(\underline{\operatorname{Hom}}_S(X,Y)^P\) over \(S\) are the \(S\)-morphisms \(g\colon X\to Y\) such that, for any integer \(n\), we have
\[
    \chi\Big(({\mathscr{L}}\otimes_{{\mathscr{O}}_X}g^*({\mathscr{M}}))^{\otimes n}\Big)
    = P(n).
  \]
In this way we obtain generalisations of Matsusaka's theorem, affirming that the automorphisms of a ``polarised'' projective variety form an algebraic group, a claim that here has an evidently more precise meaning, since we have a definition of this group as the solution to a universal problem.
We note also that, over an algebraically closed field, the group of automorphisms considered in the past is that which is induced by the ``true'' one defined here, by dividing by the nilpotent elements;
this explains why there is little chance that the historical constructions could be done over a non-perfect base field, since the ideal of nilpotent elements that appears after an extension of the base field is not necessarily ``defined over \(k\)''.
This same remark applies equally to the majority of historical constructions.

\hypertarget{fga-3-iv-section-5}{%
\subsection{Differential study of Hilbert schemes}\label{fga-3-iv-section-5}}

\oldpage{221-21}We have the following result:

\leavevmode\vadjust pre{\hypertarget{fga-3-iv-proposition-5.1}{}}%
\begin{itenv}{Proposition 5.1}
Let \(S\) be a prescheme, \(S_0\) a subprescheme defined by a square-zero quasi-coherent ideal \({\mathscr{I}}\), \(X\) an \(S\)-prescheme, and \({\mathscr{F}}\) a quasi-coherent module on \(X\).
Let \(X_0={\mathscr{F}}\times_S S_0\) and \({\mathscr{F}}_0={\mathscr{F}}\otimes_{{\mathscr{O}}_S}{\mathscr{O}}_{S_0}\).
Finally, let \({\mathscr{G}}_0={\mathscr{F}}_0/{\mathscr{H}}_0\) be a quasi-coherent quotient module of \({\mathscr{F}}_0\) that is flat over \(S_0\).
For every open \(U\) of \(X\), let \({\mathscr{E}}(U)\) be the set of quasi-coherent quotient modules \({\mathscr{G}}\) of \({\mathscr{F}}|U\) that are flat over \(S\) and are such that \({\mathscr{G}}\otimes_{{\mathscr{O}}_S}{\mathscr{O}}_{S_0}={\mathscr{G}}_0\);
as \(U\) varies, the \({\mathscr{E}}(U)\) are the sections of a sheaf \({\mathscr{E}}\) on \(U\).
With this, the sheaf of groups
\[
  {\mathscr{A}} = \underline{\operatorname{Hom}}_{{\mathscr{O}}_{X_0}}({\mathscr{H}}_0,{\mathscr{G}}_0\otimes_{{\mathscr{O}}_{S_0}}{\mathscr{I}})
\]
acts naturally on \({\mathscr{E}}\), which thus becomes a ``formally \(\mathscr{A}\)-principal homogeneous'' sheaf (i.e.~for every open \(U\) in \(X\), \({\mathscr{E}}(U)\) is either empty or an \({\mathscr{A}}(U)\)-principal homogeneous set).

\end{itenv}

We thus conclude:

\leavevmode\vadjust pre{\hypertarget{fga-3-iv-corollary-5.2}{}}%
Suppose that there exists locally on \(X\) an extension \({\mathscr{G}}\) of \({\mathscr{G}}_0\) to a quotient of \({\mathscr{F}}\) that is flat over \(S\) (i.e.~that the fibres of the sheaf \({\mathscr{E}}\) are non-empty).
Then there exists a canonical obstruction class
\[
  c({\mathscr{G}}_0) \in \operatorname{H}^1(X,{\mathscr{A}})
\]
whose vanishing is necessary and sufficient for the existence of a global extension \({\mathscr{G}}\) of \({\mathscr{G}}_0\) to a quotient of \({\mathscr{F}}\) that is flat over \(S\).
If this class is zero, then the set \({\mathscr{E}}(X)\) of all possible extensions is a principal homogenous set for \({\mathscr{A}}(X)=\operatorname{Hom}_{{\mathscr{O}}_X}({\mathscr{H}}_0,{\mathscr{G}}_0\otimes_{{\mathscr{O}}_{S_0}}{\mathscr{I}})\).

The existence of a global extension is thus guaranteed, in particular, if \(\operatorname{H}^1(X,{\mathscr{A}})=0\).

\leavevmode\vadjust pre{\hypertarget{fga-3-iv-corollary-5.3}{}}%
Suppose that \(Q=\underline{\operatorname{Quot}}_{{\mathscr{F}}/X/S}\) exists (cf.~\protect\hyperlink{fga-3-iv-section-4}{§4.a}) --- for example, suppose that \(X\) is quasi-projective over some locally Noetherian \(S\), and \({\mathscr{F}}\) is coherent.
Let \(x\in Q\), corresponding to a residue extension \(K=k(x)\) of some \(k(s)\) (where \(s\in S\)).
\oldpage{221-22}Then \(x\) is defined by a coherent quotient module \({\mathscr{G}}_0={\mathscr{F}}_0/{\mathscr{H}}_0\) of the module \({\mathscr{F}}_0=F_K\) on the \(K\)-prescheme \(X_K\).
Let \({\mathscr{A}}\) be the coherent sheaf on \(X_K\) defined by
\[
  {\mathscr{A}} = \underline{\operatorname{Hom}}_{{\mathscr{O}}_{X_0}}({\mathscr{H}}_0,{\mathscr{G}}_0).
\]

Then the Zariski tangent space of the fibre \(Q_s\) at the point \(x\) (given by the dual over \(K\) of \({\mathfrak{m}}/{\mathfrak{m}}^2\), where \({\mathfrak{m}}\) is the maximal ideal of \({\mathscr{O}}_{Q_k,x}\)) is canonically isomorphic to \(\operatorname{H}^0(X_k,{\mathscr{A}})\).

The result giving the Zariski tangent space can be generalised, and gives a characterisation, for a given \(S\)-morphism \(g\colon S'\to Q\), i.e.~a section \(g'\) of \(Q'=Q\times_S S'\) over \(S'\), of the module
\[
  \Omega
  = g^*(\Omega_{Q/S}^1)
  = {g'}^*({\mathscr{J}}/{\mathscr{J}}^2)
\]
(where \({\mathscr{J}}\) is the ideal on \(Q'\) defined by the section \(g'\) of \(Q'\) over \(S'\)) by the formula
\[
  \operatorname{Hom}_{{\mathscr{O}}_{S'}}(\Omega,{\mathscr{M}})
  \simeq \operatorname{H}^0(X',{\mathscr{A}})
\]
which is functorial in the coherent module \({\mathscr{M}}\) over \(S'\), and where \({\mathscr{A}}\) is again the module on \(X'=X\times_S S'\) defined by
\[
  {\mathscr{A}} = \underline{\operatorname{Hom}}_{{\mathscr{O}}_{X'}}({\mathscr{H}},{\mathscr{G}}\otimes_{{\mathscr{O}}_S}{\mathscr{M}})
\]
(\({\mathscr{G}}={\mathscr{F}}'/{\mathscr{H}}\) being the quotient module of \({\mathscr{F}}'={\mathscr{F}}\otimes_{{\mathscr{O}}_S}{\mathscr{O}}_{S'}\) that corresponds to \(g\)).
It suffices to apply \protect\hyperlink{fga-3-iv-proposition-5.1}{(5.1)} by replacing \(S_0\) with \(S'\), and \(S\) with the prescheme \(D({\mathscr{M}})=(S',{\mathscr{O}}_{S'}+{\mathscr{M}})\), where \({\mathscr{M}}\) is considered as a square-zero ideal.

If, in \protect\hyperlink{fga-3-iv-proposition-5.1}{(5.1)}, we have \({\mathscr{F}}={\mathscr{O}}_X\), then the data of \({\mathscr{G}}_0\) corresponds to the data of a closed subprescheme \(Y_0\) of \(X_0\) that is flat over \(S_0\), defined by the ideal \({\mathscr{J}}_0={\mathscr{M}}_0\), and then \protect\hyperlink{fga-3-iv-equation-asterisk}{(\(*\))} gives
\[
  {\mathscr{A}} = \underline{\operatorname{Hom}}_{{\mathscr{O}}_{X_0}}({\mathscr{J}}_0/{\mathscr{J}}_0^2,{\mathscr{O}}_{Y_0}\otimes_{{\mathscr{O}}_{S_0}}{\mathscr{J}})
\]
where \({\mathscr{J}}/{\mathscr{J}}^2\) is thought of as the \emph{conormal sheaf} of \(Y_0\) in \(X_0\), which we also denote by \({\mathscr{N}}_{Y_0/X_0}\);
\oldpage{221-23}it is then interesting to consider \({\mathscr{A}}\) as a module over \(Y_0\), and to calculate \(\operatorname{H}^0\) and \(\operatorname{H}^1\) on \(Y\).
If \(Y_0\) is locally a complete intersection in \(X_0\), with \(X\) flat over \(S\), then, in \protect\hyperlink{fga-3-iv-proposition-5.1}{(5.1)}, the possibility of a local extension is guaranteed, and \({\mathscr{J}}/{\mathscr{J}}^2\) is locally free over \(Y_0\) and we can write
\[
  {\mathscr{A}} = \check{{\mathscr{N}}}_{X_0/Y_0}\otimes_{{\mathscr{O}}_{S_0}}{\mathscr{J}}
\]
where the first factor on the right-hand side is the normal cone of \(Y_0\) inside \(X_0\).
Using the fundamental criterion of simplicity {[}\protect\hyperlink{ref-Gro1960b}{9}, III, 3.1{]}, we find, for example:

\leavevmode\vadjust pre{\hypertarget{fga-3-iv-corollary-5.4}{}}%
Under the conditions of \protect\hyperlink{fga-3-iv-corollary-5.3}{(5.3)}, suppose that \({\mathscr{F}}={\mathscr{O}}_X\), with \(X\) flat over \(S\), and that the closed subprescheme \(Y_0\) of \(X_0\) that corresponds to \({\mathscr{G}}_0\) is locally a complete intersection.
Then the Zariski tangent space of \(Q_s\) at \(x\) is canonically isomorphic to \(\operatorname{H}^0(Y_0,\check{{\mathscr{N}}}_{X_0/Y_0})\).
If \(\operatorname{H}^1(Y_0,\check{{\mathscr{N}}_{X_0/Y_0}})=0\), then the Hilbert prescheme \(X\) is simple over \(S\) at the point \(x\) (where \(\check{{\mathscr{N}}}_{X_0/Y_0}\) is the normal sheaf of \(Y_0\) inside \(X_0\)).

\leavevmode\vadjust pre{\hypertarget{fga-3-iv-remark-5.5}{}}%
This statement applies in particular when \(Y_0\) is a complete intersection in \(X_0\) defined by \emph{one} single equation, i.e.~is a positive ``Cartier divisor''.
Then \(\check{{\mathscr{N}}}_{X_0/Y_0}\) is isomorphic to the sheaf on \(Y_0\) induced by the invertible sheaf \({\mathscr{J}}^{-1}\) on \(X_0\) defined by the divisor \(Y_0\).
This is the situation that we find in particular in the study of families of positive divisors on a non-singular projective variety \(X_0\).
The isomorphism between the Zariski tangent space at the point \(x\) of \(Q\) (or, if one prefers, of the open \(D\) of \(Q\) that corresponds to the divisors) and \({\mathscr{H}}^0(Y_0,\check{{\mathscr{N}}}_{X_0/Y_0})\) was known in classical algebraic geometry under the name of ``\emph{characteristic homomorphism}'' (from the former to the latter).
It was not defined when \(x\) was a simple point of the variety of parameters \(T\) of a ``complete continuous family'' of divisors, i.e.~from our point of view, of an irreducible component of the scheme \(D\), endowed with the induced \emph{reduced} structure.
The tangent space of \(T\) at \(x\) is then a \emph{subspace} of the tangent space of \(D\) at \(x\), and so the characteristic homomorphic of yore is indeed injective, but it is not surjective except for under supplementary conditions, for example if \(D\) is integral at \(x\).
\oldpage{221-24}In fact, Zappa {[}\protect\hyperlink{ref-Zap1945}{26}{]} constructed an example (with \(X\) a non-singular projective surface over the \emph{field of complex numbers}) where even at the generic point of \(T\) the characteristic homomorphism is not surjective.
\emph{This thus implies that \(D\) is not integral even at the generic point of the irreducible component in question.}
This shows in a particularly striking manner how varieties with nilpotent elements are necessary for understanding the most classical phenomena of the theory of surfaces.

\emph{{[}Comp.{]}}
Concerning the example of Zappa, we note that Mumford has even constructed an irreducible component of the Hilbert scheme for \(\mathbb{P}_\mathbb{C}^3\) (whose general points represent non-singular curves of degree \(14\) and genus \(24\)), which is non-reduced at its generic points.
Blowing up the curves obtained, he also obtains a regular projective scheme of dimension \(3\) over \(\mathbb{C}\), whose formal scheme of modules is non-reduced at its generic point, or, equivalently, such that its local variety of modules, in the sense of analytic geometry, over \(\mathbb{C}\) (see \emph{Séminaire Cartan \textbf{13}, 1960/61}) is non-reduced at all its points.

\leavevmode\vadjust pre{\hypertarget{fga-3-iv-remark-5.6}{}}%
We have given, in \protect\hyperlink{fga-3-iv-remark-5.4}{(5.4)}, a criterion for simplicity, which applies in particular to schemes of divisors.
Kodaira gave a different criterion in {[}\protect\hyperlink{ref-Kod1956}{12}{]}, given by the vanishing of \(\operatorname{H}^1(X_0,{\mathscr{L}})\), where \({\mathscr{L}}={\mathscr{J}}_0^{-1}\) is the invertible sheaf on \(X_0\) defined by the divisor \(Y_0\);
this criterion holds whenever \(S\) is the spectrum of a field of characteristic \(0\), and is proved in {[}\protect\hyperlink{ref-Kod1956}{12}{]} by transcendental methods in the case where the base field is \(\mathbb{C}\).
We note here that, in general, \(S\) now arbitrary, Kodaira's condition is a sufficient condition for the canonical morphism \(D\to\underline{\operatorname{Pic}}_{X/S}\) from the prescheme of divisors to the Picard prescheme of \(X/S\) to be simple at the point \(x\) in question (as we easily verify by the usual criterion for simplicity, once we have the existence of \(\underline{\operatorname{Pic}}_{X/S}\)).
Then if, further, \(\underline{\operatorname{Pic}}_{X/S}\) is simple over \(S\) at the point given by the image of \(x\) (for example if \(\underline{\operatorname{Pic}}_{X/S}\) is simple over \(S\)), then \(D\) is simple over \(S\) at \(x\).
On the other hand, Cartier has shown that every group prescheme locally of finite type over \emph{a field \(k\) of characteristic \(0\)} is simple over \(k\)\textgreater{}
By combining these two results, we recover the result of Kodaira.
Note that it follows from these remarks that, over a field \(K\) of characteristic \(p>0\), if \(\underline{\operatorname{Pic}}_{X/S}\) is not simple over \(k\) (which is the case whenever \(X\) is the Igusa surface), then the condition \(\operatorname{H}^1(X_0,{\mathscr{L}})=0\) implies to the contrary that \(D\) is not simple at \(x\), and even not reduced at \(x\) if \(K\) is algebraically closed.

To finish, we give the following result, which plays an important role in the differential study of fibred spaces:

\leavevmode\vadjust pre{\hypertarget{fga-3-iv-proposition-5.7}{}}%
Let \(X\) be a finite prescheme that is flat over \(S\) and locally Noetherian, and let \(Z\) be a prescheme over \(S\) such that \(\prod_{X/S}(Z/X)\) exists (which is the case if \(Z\) is quasi-projective over \(X\)).
If \(Z\) is simple over \(X\), then \(\prod_{X/S}(Z/X)\) is simple over \(S\).

\begin{proof}
This is an immediate consequence of the definition, and of the usual criterion of simplicity {[}\protect\hyperlink{ref-Gro1960b}{9}, III, §3.1{]}.
\end{proof}

Note that if \(X\) is finite and flat over \(S\), then the question of the existence of \(\prod_{X/S}(Z/X)\) can be dealt with in a very elementary manner, without using the theory of Hilbert schemes.
\oldpage{221-25}We find, for example, that if \(X\) is radicial over \(S\), then \(\prod_{X/S}(Z/X)\) exists without any restrictions on \(Z\).
For example, let \(T\) be an \(S\)-prescheme, and let \(T_n\) be ``the infinitesimal neighbourhood of order \(n\)'' of the diagonal of \(T\times_S T\) in \(T\times_S T\), endowed with the morphisms \(p_1,p_2\colon T_n\to T\) induced by the two projections.
We can consider \(T_n\) as a finite prescheme over \(T\) thanks to \(p_1\), and we suppose further that \(T_n\) is flat over \(T\) (which is the case if \(T\) is simple over \(S\)).
For every prescheme \(X\) over \(T\), set
\[
  (X/T/S)^{(n)} = \prod_{T_n/S}(p_2^*(X/T)/T_n)
\]
which is a prescheme over \(T\) called the \emph{bundle of germs of sections of order \(n\) of \(X\) over \(T\)} (with respect to \(S\)).
This depends functorially on \(X\), and is simple over \(T\) if \(X\) is.

\hypertarget{fga-3-iv-section-6}{%
\subsection{Relation to the notion of norm and symmetric products}\label{fga-3-iv-section-6}}

Let \(S\) be a prescheme, let \(X\) and \(Y\) be \(S\)-preschemes, and let
\[
  u\colon(X/S)^n\to Y
\]
be a \emph{symmetric} \(S\)-morphism from the \(n\)-th cartesian power of \(X/S\) to \(Y\).
Suppose, for simplicity, that \(S\) is locally Noetherian, and that \(X\) and \(Y\) are of finite type over \(S\).
We can then associate, to every coherent module \({\mathscr{F}}\) on \(X\) with finite support on \(S\) that is furthermore flat over \(S\) and of rank equal to \(n\) with respect to \(S\) (i.e.~such that \(f_*({\mathscr{F}})\) is a locally free module of rank \(n\) on \(S\)), in a natural way a section of \(Y\) over \(S\):
\[
  {\mathscr{N}}_{X/S}^u({\mathscr{F}}) \in \Gamma(Y/S).
\]
We will not give here the details of the definition, but instead content ourselves with noting that the formalism to which one arrives is a natural generalisation of the usual formalism of norms and traces.
When the symmetric \(n\)-th power of \(X\) over \(S\) exists (for example, if the orbits of the symmetric group \(\sigma_n\) acting on \((X/S)^n\) are contained inside affine opens), we can take \(Y\) to be this symmetric power \(\operatorname{Symm}_S^n(X)\), and we obtain a canonical element
\[
  {\mathscr{N}}_{X/S}({\mathscr{F}}) \in \Gamma(\operatorname{Symm}_S^n(X)/S)
\]
\oldpage{221-26}which allows us to recover the \({\mathscr{N}}_{X/S}^u({\mathscr{F}})\).
Another important case is that where \(X\) is a commutative monoid over \(S\), and \(X=Y\), and the morphism \(u\) comes from the composition law of \(X\).
We then simply write \({\mathscr{N}}({\mathscr{F}})\) for the section of \(X\) over \(S\) associated to the module \({\mathscr{F}}\) on \(X\).

Now suppose that we have a coherent module \({\mathscr{F}}\) on \(X\) such that \(\underline{\operatorname{Quot}}_{{{\mathscr{F}}/X/S}}\) exists, or at least such that the functor \(\mathscr{Q}\kern -.5pt out_{{{\mathscr{F}}/X/S}}^n\), which associates to each \(S'\) over \(S\) the set of coherent quotient sheaves \({\mathscr{M}}\) of \({\mathscr{F}}'={\mathscr{F}}\otimes_{{\mathscr{O}}_S}{\mathscr{O}}_{S'}\) that are flat over \(S\) and of relative rank \(n\), is representable by an \(S\)-prescheme \(\underline{\operatorname{Quot}}_{{{\mathscr{F}}/X/S}}^n\).
(If \(X\) is quasi-projective over \(S\), then \(\underline{\operatorname{Quot}}_{{{\mathscr{F}}/X/S}}^n\) indeed exists, and is exactly, with the notation of \protect\hyperlink{fga-3-iv-section-3}{§3}, \(\underline{\operatorname{Quot}}_{{{\mathscr{F}}/X/S}}^P\), where \(P\) is the polynomial consisting of the constant term \(n\)).
Since the formation of the \({\mathscr{N}}_{X/S}^u({\mathscr{M}})\) is compatible with base change, we thus obtain a canonical morphism
\[
  {\mathscr{N}}_{X/S}^u\colon \underline{\operatorname{Quot}}_{{{\mathscr{F}}/X/S}}^n \to Y
\]
and, in particular, if the \(n\)-th symmetric power of \(X\) over \(S\) exists,
\[
  {\mathscr{N}}_{X/S}\colon \underline{\operatorname{Quot}}_{{{\mathscr{F}}/X/S}}^n \to \operatorname{Symm}_S^n(X).
\]
The most important case is that where \({\mathscr{F}}={\mathscr{O}}_X\), which gives a morphism
\[
  {\mathscr{N}}_{X/S}\colon \underline{\operatorname{Hilb}}_{X/S}^n \to \operatorname{Symm}_S^n(X).
\]
This is evidently an isomorphism for \(n=0\) and \(n=1\).
But for \(n\geqslant 1\), even if \(S\) is the spectrum of a field \(k\), and \(X\) is simple over \(S\), it is not in general an isomorphism nor even an injective morphism, since a sub-scheme of dimension \(0\) of \(X\) (corresponding, for example, to a primary ideal \({\mathscr{I}}\) for the maximal ideal in a local ring \({\mathscr{O}}_{X,x}\), for a closed point \(x\) of \(X\)) is not known when we know only the cycle that it defines (to be precise, when we know the codimension over \(k\) of \({\mathscr{I}}\) in \({\mathscr{O}}_{X,x}\)).
We can only say the following (where \(S\) is once more arbitrary):

\begin{enumerate}
\def\labelenumi{\alph{enumi}.}
\tightlist
\item
  If \(X\) is simple over \(S\), then the norm morphism defines an isomorphism from the open of \(\underline{\operatorname{Hilb}}_{X/S}^n\) that corresponds to the classification of étale covers of rank \(n\) contained inside \(X\) (cf.~\protect\hyperlink{fga-3-iv-section-4}{§4.b}) to the open of \(\operatorname{Symm}_S^n(X)\) that corresponds to the \(n\)-cycles without multiple components.
\item
  If furthermore \(X\) is of relative dimension \(1\) over \(S\), then the norm morphism even defines an isomorphism from \(\underline{\operatorname{Hilb}}_{X/S}^n\) to \(\operatorname{Symm}_{X/S}^n\).
\end{enumerate}

\oldpage{221-27}This second fact is due to the fact that a subscheme of dimension \(0\) of a non-singular algebraic curve is known whenever we know the corresponding cycle.
The same remark also applies more generally to Cartier divisors that are positive over a non-singular algebraic scheme (and it is not excluded that, in this very particular case, the Chow variety gives the same thing as the Hilbert variety).

\hypertarget{fga-3-iv-section-7}{%
\subsection{Supplements and questions}\label{fga-3-iv-section-7}}

As remarked by J.-P. Serre, it follows from a well-known example of Nagata that we can find a scheme \(S\) that is the spectrum of a field \(k\), an \(S\)-scheme \(S'\) that is the spectrum of a quadratic extension \(k'\) of \(k\), and finally a simple and proper (but non-projective) \(S'\)-scheme \(X\) of dimension \(3\) such that \(\prod_{S'/S}(X/S)\) does not exist.
This implies a fortiori that the Hilbert scheme \(\underline{\operatorname{Hilb}}_{X/S}^2\) does not exist (nor even the \(k\)-scheme that would represent the étale covers of rank \(2\) of \(S\) contained inside \(X\), nor a fortiori the symmetric square of \(X\), cf.~\protect\hyperlink{fga-3-iv-section-6}{§6}).
This thus imposes serious limitations on the possibilities of non-projective constructions in algebraic geometry.
(It is, however, plausible that such limitations do not present themselves in analytic geometry, just as they do not present themselves in formal geometry (cf.~\href{FGA-3-II.html}{FGA 3.II})).
However, if \(X\) is a proper scheme over the spectrum \(S\) of a field \(k\), and if \(Z\) is quasi-projective over \(X\), then \(\prod_{X/S}(Z/X)\) exists, and is a scheme, given by the sum of a sequence of quasi-projective schemes over \(S\) (as in the projective case \protect\hyperlink{fga-3-iv-theorem-3.1}{(3.1)}).
To see this, we can reduce to the case where \(X\) is itself projective, by dominating \(X\) by a projective \(S\)-scheme \(X'\);
we will not give here the details of the proof, which also uses the result of factorisation of a finite morphism given in \protect\hyperlink{fga-3-i-section-A.2.b}{FGA 3.I, A.2.b}.
The success of the method is all in the fact that, with \(S\) the spectrum of a field, the \(X'\) that appears in Chow's lemma will automatically be flat over \(S\).
I do not know if the result remains true without any hypotheses on \(S\), supposing only that \(X\) is proper and flat over \(S\), and that \(Z\) is quasi-projective over \(X\).
An important case in the applications is that where \(Z\) is a closed subscheme of \(X\);
if then \(\prod_{X/S}(Z/X)\) exists, it is necessarily a closed subscheme of \(S\).
\oldpage{221-28}We can construct it directly in a relatively simple manner whenever \(X\) is projective over \(S\), without using the theory of Hilbert schemes, and the method used shows more generally that, if \(Z\) is affine over \(X\), then \(\prod_{X/S}(Z/X)\) exists and is affine over \(S\).
It equally shows that, if \(X\) is proper and flat over \(S\) (but not necessarily projective over \(S\)), then, for every vector bundle \(Z\) that is locally trivial on \(X\), \(\prod_{X/S}(Z/X)\) exists and is a vector bundle on \(S\).
It would be desirable for these results to be studied again and unified.

\hypertarget{fga-3.v}{%
\section{Picard schemes: Existence theorems}\label{fga-3.v}}

\providecommand{\scr}[1]{{\mathscr{#1}}}
\renewcommand{\cal}[1]{{\mathcal{#1}}}
\renewcommand{\frak}[1]{{\mathfrak{#1}}}
\renewcommand{\geq}{\geqslant}
\renewcommand{\leq}{\leqslant}

\providecommand{\Pic}{\operatorname{Pic}}
\providecommand{\shPic}{\mathscr{P}\kern -.5pt ic}
\providecommand{\repPic}{\underline{\Pic}}
\providecommand{\OO}{\scr{O}}
\providecommand{\RR}{\operatorname{R}}
\providecommand{\simto}{\xrightarrow{\sim}}
\providecommand{\HH}{\operatorname{H}}
\providecommand{\Sch}{\mathtt{Sch}}
\providecommand{\Hom}{\operatorname{Hom}}
\providecommand{\shHom}{\mathscr{H}\kern -.5pt om}
\providecommand{\from}{\leftarrow}
\providecommand{\pr}{\mathrm{pr}}
\providecommand{\div}{\operatorname{div}}
\providecommand{\Div}{\operatorname{Div}}
\providecommand{\shDiv}{\mathscr{D}\kern -.5pt iv}
\providecommand{\repDiv}{\underline{\Div}}
\providecommand{\Hilb}{\operatorname{Hilb}}
\providecommand{\shHilb}{\mathscr{H}\kern -.5pt ilb}
\providecommand{\repHilb}{\underline{\Hilb}}
\providecommand{\red}{\mathrm{red}}
\providecommand{\Spec}{\operatorname{Spec}}

{[}FGA 3.V{]}
Grothendieck, A.
``Technique de descente et théorèmes d'existence en géométrie algébrique, V: Les schémas de Picard: Théorèmes d'existence''.
\emph{Séminaire Bourbaki} \textbf{14} (1961--62), Talk no. 232.

\hypertarget{fga-3-v-section-1}{%
\subsection{Relative Picard groups and functors}\label{fga-3-v-section-1}}

\oldpage{232-01}For every prescheme (more generally, every ringed space) \(X\), we define the (\emph{absolute}) \emph{Picard group} of \(X\), denoted by \(\operatorname{Pic}(X)\), to be the group of isomorphism classes of invertible (i.e.~locally isomorphic to \({\mathscr{O}}_X\)) modules on \(X\).
We thus have a canonical isomorphism

\leavevmode\vadjust pre{\hypertarget{fga-3-v-equation-1.1}{}}%
\begin{eqenv}
\[
  \operatorname{Pic}(X)
  \xrightarrow{\sim}\operatorname{H}^1(X,{\mathscr{O}}_X^\times)
\tag{1.1}
\]

\end{eqenv}

where \({\mathscr{O}}_X^\times\) denotes the sheaf of units of \({\mathscr{O}}_X\) (which can be identified with the sheaf of automorphisms of the invertible module \({\mathscr{O}}_X\)).
Note that \(X\mapsto\operatorname{Pic}(X)\) is a contravariant functor in \(X\) in the evident way, and that the isomorphism \protect\hyperlink{fga-3-v-equation-1.1}{(1.1)} is functorial.

If \(X\) is a prescheme over a prescheme \(S\), then, for variable \(S'\) in the category \(\mathtt{Sch}_{/S}\) of preschemes over \(S\), we have a contravariant functor \(S'\mapsto\operatorname{Pic}(X\times_S S')\) thanks to the above.
This functor has no chance of being ``representable'' (\protect\hyperlink{fga-3-ii-section-A.1}{FGA 3.II, A}) since, as a consequence of the existence of \emph{non-trivial automorphisms} of invertible modules that we propose to classify, this functor is not of a \emph{``local nature''} ({[}\protect\hyperlink{ref-Gro1960a}{8}, IV, 5.4{]}).
There is thus an opportunity to ``make it local'', by introducing, for every relative prescheme \(X/S\), a group of a relative nature

\leavevmode\vadjust pre{\hypertarget{fga-3-v-equation-1.2}{}}%
\begin{eqenv}
\[
  \operatorname{Pic}'(X/S)
  = \operatorname{H}^0(S,\operatorname{R}^1f_*({\mathscr{O}}_X^\times))
\tag{1.2}
\]

\end{eqenv}

(where \(f\colon X\to S\) is the structure morphism) (cf.~\protect\hyperlink{fga-3-ii-section-C.3}{FGA 3.II, C.3}).
In \emph{loc. cit.} this group is called the relative Picard group, but it will be preferable to call it here the \emph{restricted relative Picard group} of \(X/S\), for reasons that will be made clear.
As \(S'\) varies over \(\mathtt{Sch}_{/S}\), \(S'\mapsto\operatorname{Pic}'(X\times_S S'/S')\) is a contravariant functor in \(S'\), denote also by \(\mathscr{P}\kern -.5pt ic'_{X/S}\), thus given essentially by the formula

\leavevmode\vadjust pre{\hypertarget{fga-3-v-equation-1.3}{}}%
\begin{eqenv}
\[
  \mathscr{P}\kern -.5pt ic'_{X/S}(S')
  = \operatorname{Pic}(X\times_S S'/S').
\tag{1.3}
\]

\end{eqenv}

\oldpage{232-02}This functor is now ``of local nature'', given that did what was necessary to make this happen.
Intuitively, the right-hand side of \protect\hyperlink{fga-3-v-equation-1.3}{(1.3)} can be understood as the set of ``algebraic families'' of classes of invertible sheaves on (the fibres of) \(X/S\), indexed by the parameter prescheme \(S'/S\).
When the functor \(\mathscr{P}\kern -.5pt ic'\) is representable, the prescheme over \(S\) that represents it is denoted by \(\underline{\operatorname{Pic}}_{X/S}\), and is called the \emph{Picard prescheme} of \(X\) over \(S\), and so we then have

\leavevmode\vadjust pre{\hypertarget{fga-3-v-equation-1.4}{}}%
\begin{eqenv}
\[
  \operatorname{Hom}_S(S',\underline{\operatorname{Pic}}_{X/S})
  \cong \mathscr{P}\kern -.5pt ic'_{X/S}(S')
  = \operatorname{Pic}'(X\times_S S'/S').
\tag{1.4}
\]

\end{eqenv}

There are, however, important cases where \(\mathscr{P}\kern -.5pt ic'_{X/S}\) is not representable (example: the ``Brauer--Severi'' variety over a field \(k\), without a rational point over \(k\)), but where there nevertheless exists a natural definition of a relative Picard prescheme.
This is due to the fact that, in the definition of the functor \(\mathscr{P}\kern -.5pt ic'\) from the absolute Picard groups \(\operatorname{Pic}(X\times_S S'/S')\), we have not localised enough;
more precisely, \(\mathscr{P}\kern -.5pt ic'\) is not in general ``compatible with faithfully flat descent''.
We now explain the details.

Let \(({\mathscr{M}})\) be the set of morphisms of preschemes that are \emph{faithfully flat and quasi-compact};
this set is stable under base change and composition.
Let \(P\) be a contravariant functor from \(\mathtt{Sch}_{/S}\) to the category of sets, and, for every \(S\)-morphism \(u\colon T'\to T\) with \(u\in({\mathscr{M}})\), consider the diagram

\leavevmode\vadjust pre{\hypertarget{fga-3-v-equation-1.5}{}}%
\begin{eqenv}
\[
  P(T)
  \to P(T')
  \rightrightarrows P(T'\times_T T')
\tag{1.5}
\]

\end{eqenv}

which is given by \(P\) applied to the diagram
\[
  T
  \leftarrow T'
  \underset{\mathrm{pr}_2}{\overset{\mathrm{pr}_1}{\leftleftarrows}} T'\times_T T'.
\]
If \(P\) is representable, it follows from the theory of descent (\protect\hyperlink{fga-3-i-section-B.1-theorem-2}{FGA 3.I, B, Theorem 2}) that the diagram \protect\hyperlink{fga-3-v-equation-1.5}{(1.5)} is exact for all \(u\in({\mathscr{M}})\).
We express this fact by saying that \(P\) is compatible with \(({\mathscr{M}})\), in the event that \(P\) is ``compatible with faithfully flat descent'', or that the \emph{``presheaf''} \(P\) on \(\mathtt{Sch}_{/S}\) is a \emph{``sheaf''} for the notion of localisation given by the set \(({\mathscr{M}})\).
If \(P\) is arbitrary, then a standard procedure, well known in the case of usual topological localisation, allows us to associate to it a ``sheaf'' \({\mathscr{P}}\) and a homomorphism of functors \(P\to{\mathscr{P}}\) that is universal in an obvious sense.
The construction of \({\mathscr{P}}\) can be made explicit in the following way: to define \({\mathscr{P}}(T)\), we denote, for all \(T'\) over \(T\) such that the morphism \(u\colon T'\to T\) is in \(({\mathscr{M}})\), by \(\overline{\operatorname{H}}^0(T'/T,P)\) the subset of \(P(T')\) consisting of the elements \(\xi\) such that their images \(\xi_1,x_2\) in \(P(T'\times_T T')\) are such that there exists a morphism \(v\colon T''\to T'\times_T T'\) in \(({\mathscr{M}})\) such that \(\xi_1\) and \(\xi_2\) have the same image in \(P(T'')\).

\oldpage{232-03}(N.B. The set \(\overline{\operatorname{H}}^0\) thus defined is larger than the set \(\operatorname{H}^0(T'/T,P)\) introduced in \protect\hyperlink{fga-3-i-section-A.4.a}{FGA 3.I, A.4.a}).
As \(T'\) varies over fixed \(T\) (always with \(u\in({\mathscr{M}})\)), the \(\overline{\operatorname{H}}^0(T'/T,P)\) form an inductive system (when the set of the \(T'\) is endowed with a preorder defined by domination), and we set

\leavevmode\vadjust pre{\hypertarget{fga-3-v-equation-1.6}{}}%
\begin{eqenv}
\[
  {\mathscr{P}}(T)
  = \varinjlim_{T'} \overline{\operatorname{H}}^0(T'/T,P).
\tag{1.6}
\]

\end{eqenv}

The functoriality in \(T\) of this expression is evident.

When
\[
  P(T)
  = \operatorname{Pic}(X\times_S T)
\]
the contravariant functor on \(\mathtt{Sch}_{/S}\) defined by \protect\hyperlink{fga-3-v-equation-1.6}{(1.6)} is called the \emph{relative Picard functor} of \(X\) over \(S\), and denoted by \(\mathscr{P}\kern -.5pt ic_{X/S}\), and we define the \emph{relative Picard group} of \(X\) over \(S\), denoted by \(\operatorname{Pic}(X/S)\), the group \(\mathscr{P}\kern -.5pt ic_{X/S}(S)\).
We then have an evident bijection

\leavevmode\vadjust pre{\hypertarget{fga-3-v-equation-1.7}{}}%
\begin{eqenv}
\[
  \mathscr{P}\kern -.5pt ic_{X/S}(T)
  \xrightarrow{\sim}\operatorname{Pic}(X\times_S T/T).
\tag{1.7}
\]

\end{eqenv}

An element of \(\operatorname{Pic}(X/S)\) is thus defined by means of an element \(\xi'\) of a group \(\operatorname{Pic}(X\times_S S')\) (where \(S'\to S\) is faithfully flat and quasi-compact) such that we can find a faithfully flat quasi-compact morphism \(S''\to S'\times_S S'\) such that the two inverse images of \(\xi'\) in \(\operatorname{Pic}(X\times_S S'')\) are the same.
An element \(\xi'\) of \(\operatorname{Pic}(X\times_S S')\) and an element \(\xi_1\) of \(\operatorname{Pic}(X\times_S S_1)\) (satisfying the conditions that we have just stated) define the same element of \(\operatorname{Pic}(X/S)\) if and only if there exists a faithfully flat quasi-compact morphism \(S'_1\to S'\times_S S_1\) such that the images of the two elements in question in \(\operatorname{Pic}(X\times_S S'_1)\) are equal.
It is often convenient to work instead with the functor \(P'=\mathscr{P}\kern -.5pt ic'_{X/S}\) introduced above, and we immediately note that the canonical morphism \(P\to P'\) defines an \emph{isomorphism}

\leavevmode\vadjust pre{\hypertarget{fga-3-v-equation-1.8}{}}%
\begin{eqenv}
\[
  {\mathscr{P}}
  \xrightarrow{\sim}{\mathscr{P}}'
\tag{1.8}
\]

\end{eqenv}

which gives a description of \(\mathscr{P}\kern -.5pt ic_{X/S}\) in terms of \(\mathscr{P}\kern -.5pt ic'_{X/S}=P'\) that is usually more convenient.
By \protect\hyperlink{fga-3-v-corollary-2.3}{(2.3)} below, if we replace \(P\) by \(P'\) in the description of \(\operatorname{Pic}(X/S)\) that we have just given then we can take \(S''=S'\times_S S'\) and \(S'_1=S'\times_S S_1\), at least under the conditions given in \emph{loc. cit.}.

If the functor \(\mathscr{P}\kern -.5pt ic_{X/S}\) is representable, we say that \(X/S\) admits a Picard prescheme, and the prescheme over \(S\) that represents the functor is called the \emph{Picard prescheme} of \(X\) over \(S\), and denoted by \(\underline{\operatorname{Pic}}_{X/S}\).
For this, it evidently suffices that \(P'=\mathscr{P}\kern -.5pt ic_{X/S}\) be representable, since then \(P'\) is already a ``sheaf'', and equation \protect\hyperlink{fga-3-v-equation-1.8}{(1.8)} proves that the morphism \(P'\to{\mathscr{P}}'\) can be identified with the canonical morphism

\leavevmode\vadjust pre{\hypertarget{fga-3-v-equation-1.9}{}}%
\begin{eqenv}
\[
  \mathscr{P}\kern -.5pt ic'_{X/S}
  \to \mathscr{P}\kern -.5pt ic_{X/S}
\tag{1.9}
\]

\end{eqenv}

which is then an isomorphism.
This means that our terminology is compatible with that introduced above with \protect\hyperlink{fga-3-v-equation-1.4}{(1.4)}.
In general, when \(\underline{\operatorname{Pic}}_{X/S}\) exists it is defined by the functorial isomorphism

\leavevmode\vadjust pre{\hypertarget{fga-3-v-equation-1.10}{}}%
\begin{eqenv}
\[
  \operatorname{Hom}_S(S',\underline{\operatorname{Pic}}_{X/S})
  \xrightarrow{\sim}\operatorname{Pic}(X\times_S S'/S').
\tag{1.10}
\]

\end{eqenv}

\hypertarget{fga-3-v-section-2}{%
\subsection{Relations between the various relative and absolute Picard groups}\label{fga-3-v-section-2}}

\leavevmode\vadjust pre{\hypertarget{fga-3-v-proposition-2.1}{}}%
\begin{itenv}{Proposition 2.1}
Let \(f\colon X\to S\) be a morphism such that \({\mathscr{O}}_S\xrightarrow{\sim}f_*({\mathscr{O}}_X)\).
Then we have an exact sequence
\[
  0
  \to \operatorname{Pic}(S)
  \to \operatorname{Pic}(X)
  \to \operatorname{Pic}'(X/S).
\]
If \(X\) admits a section over \(S\), then the last morphism is surjective, i.e.~we have an isomorphism
\[
  \operatorname{Pic}'(X/S)
  \xrightarrow{\sim}\operatorname{Pic}(X)/\operatorname{Pic}(S).
\]

\end{itenv}

\begin{proof}
The exact sequence can be considered as the low degrees of the exact sequence that corresponds to the Leray spectral sequence for \(f\) and \({\mathscr{O}}_X\).
The second claim is equally formal.
\end{proof}

\hypertarget{fga-3-v-proposition-2.2}{}
\begin{itenv}{Proposition 2.2}

Let \(f\colon X\to S\) be a quasi-compact separated morphism such that \({\mathscr{O}}_S\xrightarrow{\sim}f_*({\mathscr{O}}_X)\), and let \(S'\to S\) be a faithfully flat quasi-compact morphism.
Then

\begin{enumerate}
\def\labelenumi{\roman{enumi}.}
\tightlist
\item
  \(\operatorname{Pic}'(X/S)\to\operatorname{Pic}'(X\times_S S'/S')\) is injective;
\item
  If \(X\) locally admits a section over \(S\) (i.e.~every \(s\in S\) has an open neighbourhood \(U\) such that \(X|U\) has a section over \(U\)), then the diagram
  \[
  \operatorname{Pic}'(X/S)
  \to \operatorname{Pic}'(X\times_S S'/S')
  \rightrightarrows \operatorname{Pic}'(X\times_S S''/S'')
    \]
  (where \(S''=S'\times_S S'\)) is exact.
\end{enumerate}

\end{itenv}

\begin{proof}
The first claim follows, thanks to the elementary properties of faithfully flat descent, from the following general remark.
\oldpage{232-05}If \(f\colon X\to S\) is a morphism such that \({\mathscr{O}}_S\xrightarrow{\sim}f_*({\mathscr{O}}_X)\), then the functor \({\mathscr{F}}\mapsto f^*({\mathscr{F}})\), from the category of locally free modules of finite type on \(S\) to the category of locally free modules of finite type on \(X\), is fully faithful, and its essential image is given by the modules \({\mathscr{G}}\) on \(X\) such that \(f_*({\mathscr{G}})\) is locally free and such that the canonical homomorphism
\[
  f^*f_*({\mathscr{G}})
  \to {\mathscr{G}}
\]
is an isomorphism.
The second statement was proven by the theory of descent in \protect\hyperlink{fga-3-i-section-B.4}{FGA 3.I, B.4}.
\end{proof}

The results of \protect\hyperlink{fga-3-v-proposition-2.2}{(2.2)} can also be stated as follows:

\leavevmode\vadjust pre{\hypertarget{fga-3-v-corollary-2.3}{}}%
\begin{itenv}{Corollary 2.3}
Under the conditions of \protect\hyperlink{fga-3-v-proposition-2.2}{(2.2)}, the canonical homomorphism \protect\hyperlink{fga-3-v-equation-1.9}{(1.9)} \(\mathscr{P}\kern -.5pt ic'_{X/S}\to\mathscr{P}\kern -.5pt ic_{X/S}\) is injective, and even bijective if \(X\) locally admits a section over \(S\).
(In the latter case, the relative Picard group \(\operatorname{Pic}(X/S)\) is identified with the restricted relative Picard group \(\operatorname{Pic}'(X/S)\).)

\end{itenv}

Combining this with \protect\hyperlink{fga-3-v-proposition-2.1}{(2.1)}, we thus obtain:

\leavevmode\vadjust pre{\hypertarget{fga-3-v-corollary-2.4}{}}%
\begin{itenv}{Corollary 2.4}
Under the conditions of \protect\hyperlink{fga-3-v-proposition-2.2}{(2.2)}, we have an exact sequence
\[
  0
  \to \operatorname{Pic}(S)
  \to \operatorname{Pic}(X)
  \to \operatorname{Pic}(X/S).
\]
If \(X\) admits a section over \(S\), then the last homomorphism is surjective, i.e.~we have an isomorphism
\[
  \operatorname{Pic}(X/S)
  \xrightarrow{\sim}\operatorname{Pic}(X)/\operatorname{Pic}(S).
\]

\end{itenv}

\leavevmode\vadjust pre{\hypertarget{fga-3-v-remark-2.5}{}}%
\begin{rmenv}{Remark 2.5}
Let \(f\colon X\to S\) be a morphism such that \({\mathscr{O}}_S\xrightarrow{\sim}f_*({\mathscr{O}}_X)\), and let \(g\) be a section of \(X\) over \(S\).
Let \({\mathscr{L}}\) be an invertible module on \(X\).
We define the \emph{\(g\)-rigidification} of \({\mathscr{L}}\) to be an isomorphism \({\mathscr{O}}_S\xrightarrow{\sim}g^*({\mathscr{L}})\), and a \emph{\(g\)-rigidified invertible module} to be an invertible module \({\mathscr{L}}\) on \(X\) endowed with a \(g\)-rigidification.
Every automorphism of such a structure is trivial, and \(\operatorname{Pic}'(X/S)\) can be identified with the set of isomorphism classes of \(g\)-rigidified invertible modules on \(S\).
(It is this fact that allows us to use the theory of descent to prove \protect\hyperlink{fga-3-v-proposition-2.2}{(2.2), (ii)}.)
This gives a new interpretation of \(\operatorname{Pic}(X/S)\), at least when \(f\) is further quasi-compact and separated, so that \(\operatorname{Pic}(X/S)\xrightarrow{\sim}\operatorname{Pic}'(X/S)\) by \protect\hyperlink{fga-3-v-corollary-2.3}{(2.3)}.

\end{rmenv}

\leavevmode\vadjust pre{\hypertarget{fga-3-v-remark-2.6}{}}%
\begin{rmenv}{Remark 2.6}
Let \(f\colon X\to S\) be a morphism as in \protect\hyperlink{fga-3-v-proposition-2.2}{(2.2)}, and let \(S'\to S\) be a faithfully flat quasi-compact morphism such that there exists an \(S\)-morphism \(S'\to X\), i.e.~such that there exists a section of \(X'=X\times_S S'\) over \(S'\).
\oldpage{232-06}Let \(S''=S'\times_S S'\), and \(X''=X\times_S S''\), and consider the exact sequence
\[
  \operatorname{Pic}(X/S)
  \to \operatorname{Pic}(X'/S')
  \rightrightarrows \operatorname{Pic}(X''/S'').
\]
Applying \protect\hyperlink{fga-3-v-corollary-2.4}{(2.4)}, we obtain the exact sequence
\[
  \operatorname{Pic}(X/S)
  \to \operatorname{Pic}(X')/\operatorname{Pic}(S')
  \rightrightarrows \operatorname{Pic}(X'')/\operatorname{Pic}(S'').
\]
In particular, every element of the relative Picard group already ``comes from'' an element of \(\operatorname{Pic}(X')\).
This gives a substantial simplification of the description of the relative Picard group given in the previous section, and even of the Picard functor of \(X\) over \(S\), since, for all \(T\) over \(S\), we can apply the above to \(X\times_S T/T\) and to the morphism \(T'=S'\times_S T\to T\).
If, for example, \(f\) itself is faithfully flat, then we can take \(S'=X\), which allows us, whenever \(f\) is further of finite type (resp. simple, etc.), to restrict, in the description of the relative Picard functor \(T\mapsto\operatorname{Pic}(X\times_S T/T)\), to the base changes \(T'\to T\) that are of finite type (resp. simple, etc.).
If \(f\) is projective and flat, and \(S\) locally Noetherian, then we can prove that we can take in the above a \(S'\to S\) such that \(S'\) is the direct sum of \emph{flat covers} \(S'_i\) of opens \(S_i\) of \(S\) that cover \(S\);
if \(f\) is further separable, then we can take \(S'_i\) to be étale over \(S_i\).

\end{rmenv}

\hypertarget{fga-3-v-section-3}{%
\subsection{The principal existence theorem: statement}\label{fga-3-v-section-3}}

We do not have, not even conjecturally, an existence statement for Picard preschemes that encompasses all known cases.
A ``practically necessary'' condition, if we can say that, is that \(f\colon X\to S\) be \emph{proper} (ensuring essential finiteness properties) and \emph{flat}.
These conditions are not sufficient, even if \(S\) is the spectrum of the algebra of dual numbers \(k[t]/(t^2)\) over a field \(k\) (say, the field \(\mathbb{C}\) of complex numbers), and \(X\) is of dimension \(1\).
At the moment of writing this present talk, the most important existence theorems for the Picard prescheme follow from the following theorem:

\leavevmode\vadjust pre{\hypertarget{fga-3-v-theorem-3.1}{}}%
\begin{itenv}{Theorem 3.1}
Let \(f\colon X\to S\) be a morphism of locally Noetherian preschemes.
\oldpage{232-07}Suppose that

\begin{enumerate}
\def\labelenumi{\roman{enumi}.}
\tightlist
\item
  \(f\) is projective
\item
  \(f\) is flat
\item
  the geometric fibres of \(f\) are integral.
\end{enumerate}

Under these conditions, \(\underline{\operatorname{Pic}}_{X/S}\) exists.

\end{itenv}

The proof, which will be sketched in the following two sections, will at the same time show the following:
Let \(\xi\) be the section of \(\underline{\operatorname{Pic}}_{X/S}\) that corresponds to a very ample sheaf \({\mathscr{O}}_X(1)\) over \(X/S\) (i.e.~induced by a projective embedding \(X\to\mathbb{P}({\mathscr{E}})\));
then there exists an open subset \(U\) of \(\underline{\operatorname{Pic}}_{X/S}\), disjoint union of quasi-projective open subsets of \(S\), such that \(U\) is stable under translation by \(\xi\), and such that \(\underline{\operatorname{Pic}}_{X/S}\) is the increasing union of opens \(U\setminus n\xi\) (each isomorphic to \(U\)).
It thus follows, in particular, that, under the conditions of \protect\hyperlink{fga-3-v-theorem-3.1}{(3.1)}, that \(\underline{\operatorname{Pic}}_{X/S}\) is \emph{separated} over \(S\).

\leavevmode\vadjust pre{\hypertarget{remark-3.2}{}}%
\begin{rmenv}{Remark 3.2}
We see from examples (with \(S\) the spectrum of a discrete valuation ring, and \(X\) of relative dimension \(1\) over \(S\), for example), that if we omit hypothesis (iii) in \protect\hyperlink{fga-3-v-theorem-3.1}{(3.1)} and replace it with the weaker hypothesis that, for all \(s\in S\), the homomorphism \(k(s)\to\operatorname{H}^0(X_s,{\mathscr{O}}_{X_s})\) be an isomorphism, then \(\underline{\operatorname{Pic}}_{X/S}\) is not necessarily separated over \(S\);
both in the case where the geometric fibres of \(f\) are reduced, but where a generic integral geometric fibre ``blows up'' by specialisation into two irreducible components, and in the case where the geometric fibres of \(f\) are irreducible, but where a generic integral geometric fibre specialises to a ``multiple fibre''.
The first case happens, for example, with a conic that degenerates into two concurrent lines; an example of the second was shown to me by D. Mumford, with an elliptic curve that degenerates to a double elliptic curve.
These examples work in any characteristic.

\end{rmenv}

\leavevmode\vadjust pre{\hypertarget{fga-3-v-remark-3.3}{}}%
\begin{rmenv}{Remark 3.3}
Under the conditions of \protect\hyperlink{fga-3-v-theorem-3.1}{(3.1)}, I do not know if \(\underline{\operatorname{Pic}}_{X/S}\) is a disjoint union of opens that are of finite type, thus quasi-projective, over \(S\).
We note that the study of the Hilbert polynomials \(Q\in\mathbb{Q}[t]\) allows us, as in the case of Hilbert schemes (\href{FGA-3-IV.html}{FGA 3.IV}), to give a decomposition of \(\underline{\operatorname{Pic}}_{X/S}\) as a disjoint sum of opens \(\underline{\operatorname{Pic}}_{X/S}^Q\), and it seems plausible that these opens are quasi-projective over \(S\);
this is what we will see at least in the next talk when \(f\) is a simple morphism.
We draw attention to the fact that if we replace hypothesis (i) by the hypothesis ``\(X\) is \emph{locally} projective over \(S\)'' (which is sufficient to prove \protect\hyperlink{fga-3-v-theorem-3.1}{(3.1)}, since the question of existence of \(\underline{\operatorname{Pic}}_{X/S}\) is clearly local on \(S\)) however, then it is easy to give examples where \emph{\(\underline{\operatorname{Pic}}_{X/S}\) contains connected components that are not of finite type over \(S\)}.
\oldpage{232-08}For example, let \(X_0\) be a non-singular projective algebraic variety over an algebraically closed field \(k\), endowed with an automorphism \(u\) and an element \(\xi\) of the Néron--Severi group of \(X_0\) such that the \(u^n(\xi)\) are pairwise distinct.
We can, for example, take \(X_0\) to be the product of an elliptic curve \(E\) with itself, and \(u\) to be the automorphism \((x,y)\mapsto(x,y+x)\) of \(E\times E\).
Let \(S\) be the union of two non-singular irreducible curves that meet at two points \(a\) and \(b\).
There is a connected principal covering \(P\) on \(S\) of the group \(\mathbb{Z}\), and using the action of \(\mathbb{Z}\) on \(X_0\) defined by \(u\) we thus obtain an associated bundle on \(S\), with fibre \(X_0\) (trivial on \(S\setminus\{a\}\) and \(S\setminus\{b\}\)), which is in fact an \emph{abelian} scheme over \(S\) in the particular case in question.
We easily see that \(\underline{\operatorname{Pic}}_{X/S}\), which is also the bundle associated to \(P\) and to the action of \(\mathbb{Z}\) on \(\underline{\operatorname{Pic}}_{X_0/k}\) via \(u\), contains a connected component that is isomorphic to \(P\times\underline{\operatorname{Pic}}_{X_0/k}^0\) (where \(\underline{\operatorname{Pic}}^0\) denotes the connected component of the identity element in \(\underline{\operatorname{Pic}}\)), which is not of finite type over \(S\).
(One can equally produce analogous phenomena in various cases of non-separated Picard preschemes over \(S\), as described in \protect\hyperlink{fga-3-v-remark-3.3}{(3.3)}).

\emph{{[}Comp.{]}}
The question raised here has been answered in the positive by Mumford (see the next talk).

\end{rmenv}

\hypertarget{fga-3-v-section-4}{%
\subsection{Relative Cartier divisors and projective bundles}\label{fga-3-v-section-4}}

We will only need to use positive divisors, and we omit the qualification of ``positive'' in the rest of this section.

Let \(X\) be a prescheme.
A \emph{Cartier divisor}, or simply divisor, on \(X\) is a closed subprescheme \(D\) of \(X\) defined by an ideal \({\mathscr{J}}\) that is an \emph{invertible} module, i.e.~locally generated by a section that is a \emph{non-zero divisor} of \({\mathscr{O}}_X\).
To \(D\) we associate the invertible module
\[
  {\mathscr{L}}(D)
  = {\mathscr{J}}^{-1}
\]
and the canonical injection \({\mathscr{J}}\to{\mathscr{O}}_X\) gives a canonical homomorphism
\[
  s_D\colon {\mathscr{O}}_X
  \to {\mathscr{J}}^{-1}
  = {\mathscr{L}}(D)
\]
i.e.~\(s_D\in\Gamma(X,{\mathscr{L}}(D))\).
The data of a divisor is essentially \emph{equivalent} to the data of an invertible module \({\mathscr{L}}\) on \(X\) endowed with a section \(s\) that is nowhere a zero divisor, by associating to such a pair \(({\mathscr{L}},s)\) the ``divisor'' of \(s\), denoted by \(\operatorname{div}(s)\).
\oldpage{232-09}For a given invertible \({\mathscr{L}}\) on \(X\), the set of divisors \(D\) that define \({\mathscr{L}}\) is in bijective correspondence with the quotient set \(\Gamma(X,{\mathscr{L}})^\times/\Gamma(X,{\mathscr{O}}_X^\times)\), where \(\Gamma(X,{\mathscr{L}})^\times\) denotes the subset of \(\Gamma(X,{\mathscr{L}})\) consisting of sections that are nowhere zero divisors.

Now suppose that we have a morphism \(f\colon X\to S\) that is locally of finite type, and suppose, for simplicity, that \(S\) is locally Noetherian.
Let \({\mathscr{J}}\) be a coherent ideal on \(X\), with \(D\) the subscheme of \(X\) that it defines, and let \(x\in X\) and \(s=f(x)\).
We will show that the following conditions are equivalent:

\begin{enumerate}
\def\labelenumi{\roman{enumi}.}
\tightlist
\item
  \({\mathscr{J}}\) is invertible at \(x\) (i.e.~\({\mathscr{J}}_x\) is generated by a regular element of \({\mathscr{O}}_{X,x}\)) and \(D\) is flat over \(S\) at \(x\).
\item
  \(X\) and \(D\) are flat over \(S\) at \(x\), and \(D_s\) is a Cartier divisor on the fibre \(X_s\) at the point \(x\).
\item
  \(X\) is flat over \(S\) at \(x\), and \({\mathscr{J}}_x\) is generated by an element \(f_x\) that induces on \(X_s\) a non-zero divisor germ.
\end{enumerate}

We then say that \(D\) is a \emph{relative Cartier divisor}m or simply a relative divisor, on \(X/S\) at the point in question.
We note that, in (i), \(D\) is also a relative divisor at points in a neighbourhood of \(x\), so if \(X\) and \(D\) are flat over \(S\), with \(D\) proper over \(S\), then the set of \(s\in S\) such that \(D_s\) is a Cartier divisor in \(X_s\) (i.e.~such that \(D\) is a relative Cartier divisor at the points of \(X_s\)) is an open subset of \(S\).
We have also done what is necessary in the definition above in order to ensure that the notion of relative Cartier divisor be stable under arbitrary base change \(S'\to S\).
So consider the set \(\operatorname{Div}(X/S)\) of relative divisors on \(X/S\), and then the contravariant functor in \(S'\) (that varies over \(S\)) defined by
\[
  \mathscr{D}\kern -.5pt iv_{X/S}(S')
  = \operatorname{Div}(X\times_S S'/S').
\]
Suppose that \(X\) is flat and proper over \(S\).
Then by the characterisation (ii) of relative Cartier divisors, \(\mathscr{D}\kern -.5pt iv_{X/S}\) can be considered as a sub-functor of the functor \(\mathscr{H}\kern -.5pt ilb_{X/S}\) defined in \href{FGA-3-IV.html}{FGA 3.IV}, and the inclusion morphism
\[
  \mathscr{D}\kern -.5pt iv_{X/S}
  \to \mathscr{H}\kern -.5pt ilb_{X/S}
\]
is ``representable by open immersions'' (cf. {[}\protect\hyperlink{ref-Gro1960a}{8}, IV, 3.13{]}) by the above remarks.
Using the principal existence theorem of \href{FGA-3-IV.html}{FGA 3.IV}, we find:

\leavevmode\vadjust pre{\hypertarget{fga-3-v-proposition-4.1}{}}%
\begin{itenv}{Proposition 4.1}
Suppose that \(f\colon X\to S\) is projective and flat.
\oldpage{232-10}Then the functor \(\mathscr{D}\kern -.5pt iv_{X/S}\) is representable, and, more precisely, is represented by an open of \(\underline{\operatorname{Hilb}}_{X/S}\).

\end{itenv}

For a given very ample sheaf \({\mathscr{O}}_X(1)\) over \(X/S\), using the canonical decomposition of \(\underline{\operatorname{Hilb}}_{X/S}\) into a sum of opens \(\underline{\operatorname{Hilb}}_{X/S}^Q\) corresponding to Hilbert polynomials \(Q\in\mathbb{Q}[t]\), we obtain an analogous decomposition
\[
  \underline{\operatorname{Div}}_{X/S}
  = \sqcup_{Q\in\mathbb{Q}[t]} \underline{\operatorname{Div}}_{X/S}^Q
\]
into a sum of disjoint opens that are \emph{quasi-projective} over \(S\).

Using the map \(D\mapsto{\mathscr{L}}(D)\), we obtain a functorial homomorphism
\[
  \mathscr{D}\kern -.5pt iv_{X/S}
  \to \mathscr{P}\kern -.5pt ic_{X/S}
\tag{+}
\]
that we propose to study;
it appears to be relatively representable ({[}\protect\hyperlink{ref-Gro1960a}{8}, IV, 3{]}) under rather general conditions.
We thus start with an element \(\xi\) of \(\mathscr{P}\kern -.5pt ic_{X/S}(S')\), supposing, to simplify notation, that \(S'=S\);
we will show that the corresponding sub-functor of \(\mathscr{D}\kern -.5pt iv_{X/S}\) is representable.
Consider first of all the case where \(\xi\) is defined by an invertible module \({\mathscr{L}}\) on \(X\).
Suppose that \(X\) is proper and flat over \(S\), and that the geometric fibres of \(X\) over \(S\) are integral, which also implies ({[}\protect\hyperlink{ref-GD1960}{10}, III, 7{]}) that \({\mathscr{O}}_S\xrightarrow{\sim}f_*({\mathscr{O}}_X)\), and that this remains true after any base change \(S'\to S\).
Then the relative Cartier divisors \(D\) on \(X/S\) such that \({\mathscr{L}}(D)\) and \({\mathscr{L}}\) define the same element of \(\operatorname{Pic}(X/S)=\mathscr{P}\kern -.5pt ic_{X/S}(S)\), i.e.~by \protect\hyperlink{fga-3-v-corollary-2.4}{(2.4)} such that \({\mathscr{L}}(D)\) and \({\mathscr{L}}\) are locally isomorphic over \(S\), are in bijective correspondence with the sections of the quotient sheaf \(f_*({\mathscr{L}})^\times/{\mathscr{O}}_S^\times\).
This correspondence is compatible with base change.
General arguments of ``Künneth'' type from \emph{loc. cit.}\footnote{\emph{{[}Trans.{]}} The original also cites here a seminar on algebraic geometry at Harvard University by Mumford and Tate in the spring of 1962, but the notes from this were never completed and so there is no real existing citation. This citation appears again throughout both this present talk and the next.} show that the property of \(X/S\) and the flatness of \({\mathscr{L}}\) over \(S\) imply the existence of a coherent module \({\mathscr{Q}}\) on \(S\), defined up to unique isomorphism, and an isomorphism of sheaves
\[
  f_*({\mathscr{L}})
  \xrightarrow{\sim}\mathscr{H}\kern -.5pt om_{{\mathscr{O}}_X}({\mathscr{Q}},{\mathscr{O}}_S)
\]
and the formation of \({\mathscr{Q}}\) is furthermore compatible with base change.
Here \(f_*({\mathscr{L}})^\times\) denotes the subsheaf of sets of \(f_*({\mathscr{L}})\) whose sections over \(U\) are the sections of \({\mathscr{L}}\) over \(f^{-1}(U)\) that define relative Cartier divisors on \(f^{-1}(U)/U\), i.e.~that induces sections that are non-zero divisors on the \(X_s\) (for \(s\in U\)).
\oldpage{232-11}Using the hypothesis that the fibres \(X_s\) are integral, this simply implies that the induced sections on the fibres \(X_s\) are not identically zero, or, in terms of local homomorphisms \({\mathscr{Q}}\to{\mathscr{O}}_S\), that these homomorphisms are surjective (Nakayama).
This shows that the set of sections of \(f_*({\mathscr{L}})^\times/{\mathscr{O}}_S^\times\) is in bijective correspondence with the set of \emph{invertible quotient modules of \({\mathscr{Q}}\)}, or, by the definition of the projective bundle \(\mathbb{P}({\mathscr{Q}})\) associated to the coherent module \({\mathscr{Q}}\) (cf. {[}\protect\hyperlink{ref-Gro1960a}{8}, V, 2{]}), with the set of sections of \(\mathbb{P}({\mathscr{Q}})\) over \(S\).
This description is compatible with taking inverse images, and we thus obtain the theorem below.

\leavevmode\vadjust pre{\hypertarget{fga-3-v-theorem-4.3}{}}%
\begin{itenv}{Theorem 4.3}
Let \(f\colon X\to S\) be a flat proper morphism with integral geometric fibres, with \(S\) locally Noetherian, and let \({\mathscr{L}}\) be an invertible module on \(X\).
For every \(S'\) over \(S\), let \(T(S')\) be the set of relative divisors \(D\) on \(X\times_S S'/S'\) such that \({\mathscr{L}}(D)\) is locally isomorphic to \({\mathscr{L}}\otimes_{{\mathscr{O}}_S}{\mathscr{O}}_{S'}\) over \(S'\) (i.e.~such that \({\mathscr{L}}(D)\) and \({\mathscr{L}}\otimes_{{\mathscr{O}}_S}{\mathscr{O}}_{S'}\)) define the same element of \(\operatorname{Pic}(X\times_S S'/S')\).
Then there exists a coherent module \({\mathscr{Q}}\) on \(S\), determined up to unique isomorphism, such that the functor \(T\) is represented by the projective bundle \(\mathbb{P}({\mathscr{Q}})\).

\end{itenv}

\leavevmode\vadjust pre{\hypertarget{fga-3-v-corollary-4.4}{}}%
\begin{itenv}{Corollary 4.4}
If we suppose that \(f\) is projective, then the functorial homomorphism \(\mathscr{D}\kern -.5pt iv_{X/S}\to\mathscr{P}\kern -.5pt ic_{X/S}\) is representable by projective morphisms.

\end{itenv}

If \(X\) admits a section (resp. locally admits a section) over \(S\), then the above homomorphism is representable by projective bundles (resp. by local projective bundles) thanks to \protect\hyperlink{fga-3-v-theorem-4.3}{(4.3)} and \protect\hyperlink{fga-3-v-proposition-2.1}{(2.1)}.
In the case where \(f\) is quasi-projective, we can easily reduce to the previous case by a descent method, using the finite flat local quasi-sections of \(X\) over \(S\).

\leavevmode\vadjust pre{\hypertarget{fga-3-v-remark-4.5}{}}%
\begin{rmenv}{4.5}
Under the conditions of \protect\hyperlink{fga-3-v-theorem-4.3}{(4.3)}, the module \({\mathscr{Q}}\) on \(S\) is not in general locally free, as we can see by the fact that the dimension of the reduced fibres of \({\mathscr{Q}}\), i.e.~those of \(\operatorname{H}^0(X_s,{\mathscr{L}}_s)\) for varying \(s\in S\), can \emph{jump}.
Given a coherent module \({\mathscr{Q}}\) on the locally Noetherian prescheme \(S\), we can easily show that, for any given \(s\in S\), \({\mathscr{Q}}\) is free at \(s\) if and only if \(\mathbb{P}({\mathscr{Q}})\) is flat over \(S\) at the points over \(s\) (in which case it is even simple over \(S\) at the points over \(s\));
when this happens, with \({\mathscr{Q}}\) defined in terms of \({\mathscr{L}}\) as above, this also implies that forming the direct image \(f_*({\mathscr{L}})\) \emph{``commutes with base change on a neighbourhood of \(s\)''}, or that \(f_*({\mathscr{L}})_s\to\operatorname{H}^0(X_s,{\mathscr{L}}_s)\) is surjective.
\oldpage{232-12}This will be the case if, for example, \(\operatorname{H}^1(X_s,{\mathscr{L}}_s)=0\).
Subject to the existence of the preschemes in question, these criteria apply in particular to the universal situation \(\underline{\operatorname{Div}}_{X/S}\to\underline{\operatorname{Pic}}_{X/S}\), and give a necessary and sufficient condition (resp. sufficient) for this morphism to be simple at a given point of \(\underline{\operatorname{Div}}_{X/S}\).

\end{rmenv}

\hypertarget{fga-3-v-section-5}{%
\subsection{Proof of the principal existence theorem}\label{fga-3-v-section-5}}

Under the conditions of \protect\hyperlink{fga-3-v-theorem-3.1}{(3.1)}, choose some module \({\mathscr{O}}_X(1)\) that is very ample over \(X/S\), and let \(\xi\) be the corresponding element of \(\operatorname{Pic}(X/S)\).
For brevity, let \({\mathscr{P}}(S')=\operatorname{Pic}(X\times_S S'/S')\), and suppose, for simplicity, that \(X/S\) admits a section.
Let \({\mathscr{P}}^+(S')\) be the subset of \({\mathscr{P}}(S')\) consisting of classes of the \({\mathscr{L}}\) that are invertible on \(X\times_S S'\) such that
\[
  \begin{aligned}
    \operatorname{R}^i f'_*({\mathscr{L}}(n)) &= 0
    \qquad\text{for }i>0\text{ and all }n\geqslant 0
  \\f'_*({\mathscr{L}}(n)) &\neq0
    \qquad\text{for all }n\geqslant 0.
  \end{aligned}
\]
These conditions are stable under base change, and thus define a subfunctor \({\mathscr{P}}^+\) of \({\mathscr{P}}\) that is evidently stable under translation by \(\xi\).
Using Serre's ``Theorems A and B'' ({[}\protect\hyperlink{ref-GD1960}{10}, III, 2{]}) and generalities ({[}\protect\hyperlink{ref-Gro1960a}{8}, IV, 5{]}), we easily see that \({\mathscr{P}}\) is representable if and only if \({\mathscr{P}}^+\) is, and so \({\mathscr{P}}^+\) will be representable by an open \(U\) of the prescheme \(\underline{\operatorname{Pic}}_{X/S}\) that represents \({\mathscr{P}}\), and the latter will be an increasing union of opens \(U\setminus n\xi\).

For brevity, let \({\mathscr{D}}=\mathscr{D}\kern -.5pt iv_{X/S}\), and let \({\mathscr{D}}^+\) be the inverse image of \({\mathscr{P}}^+\) under the canonical morphism \({\mathscr{D}}\to{\mathscr{P}}\).
So we have a morphism
\[
  {\mathscr{D}}^+
  \to {\mathscr{P}}^+
\tag{+}
\]
and we already know that \({\mathscr{D}}^+\) is representable by an open \(D^+\) of the prescheme \(D=\underline{\operatorname{Div}}_{X/S}\) (and, more precisely, by \emph{projective bundles associated to locally free modules} that are everywhere non-zero);
this is due to the fact that, if \({\mathscr{L}}\) on \(X\times_S S'\) is, as at the start of this section, then \(f'_*({\mathscr{L}})\) is a \emph{locally free} non-zero module, whose formation commutes with base change;
with the notation of \protect\hyperlink{fga-3-v-theorem-4.3}{(4.3)}, \({\mathscr{Q}}\) is then isomorphic to the dual of \(f'_*({\mathscr{L}})\).
Using general criteria ({[}\protect\hyperlink{ref-Gro1960a}{8}, IV, 4.7{]}), we can thus conclude that \(P^+\) is representable.
In \emph{loc. cit.}, we take \({\mathcal{S}}\) to be the set of faithfully flat quasi-compact morphisms of preschemes (which are indeed effective epimorphisms, by \protect\hyperlink{fga-3-i-section-B.1}{FGA 3.I, B}).
\oldpage{232-13}Condition (a) of \emph{loc. cit.}, namely that \((+)\) is representable by morphisms that are elements of \({\mathcal{S}}\), is satisfied as we have just seen;
condition (b) says that the functor \({\mathscr{P}}^+\) is is compatible with faithfully flat quasi-compact descent, which is immediate.
It remains only to prove condition (c) of \emph{loc. cit.}, namely that the equivalence \(R\) in the prescheme \(D^+\) induced by the \({\mathcal{S}}\)-representable morphism (\(+\)) is \({\mathcal{S}}\)-effective, i.e.~is effective and such that \(D^+\to D^+/R\) is in \({\mathcal{S}}\).
For this, note first of all that the opens \({D^+}^Q\) of \(D^+\) that correspond to the virtual Hilbert polynomials \(Q\in\mathbb{Q}[t]\) are stable under \(R\) (since the fibres of \(R\) are connected), which reduces the problem to proving that, for all \(Q\), the induced equivalence relation \(R^Q\) on \({D^+}^Q\) is \({\mathcal{S}}\)-effective.
But now \({D^+}^Q\) is quasi-projective, and the equivalence relation \(R^Q\) is projective and flat.
We are thus under the hypothesis of \protect\hyperlink{fga-3-iii-theorem-6.1}{FGA 3.III, Theorem 6.1}, which implies the desired result.

In the general case where \(X/S\) does not necessarily admit a section, we can easily reduce to the above case by the technique of descent, where we can repeat the above proof with the modification that is imposed upon the definition of \({\mathscr{P}}^+\).

\leavevmode\vadjust pre{\hypertarget{fga-3-v-remarks-5.1}{}}%
\begin{rmenv}{Remarks 5.1}
The method that we followed is essentially that of Matsusaka for the projective construction of Picard varieties.
The result that we invoke from \href{FGA-3-III.html}{FGA 3.III} that allows us to pass to the effective quotient can also easily be deduced from the existence theorem for Hilbert schemes (cf.~for example {[}Mumford--Tate seminar, 1962{]}).
(Classically, these quotients are constructed by using Chow coordinates).
Note that the formation of the open \(\underline{\operatorname{Pic}}_{X/S}^+\) of \(\underline{\operatorname{Pic}}_{X/S}\) and its decomposition into opens \({\underline{\operatorname{Pic}}_{X/S}^+}^Q\) that are quasi-projcetive over \(S\) following the Hilbert polynomials for the \emph{divisors} that define the invertible modules in question, is compatible with base change (which allows us to apply the technique of descent).

\end{rmenv}

\leavevmode\vadjust pre{\hypertarget{fga-3-v-remarks-5.2}{}}%
\begin{rmenv}{Remarks 5.2}
It is not out of the question that \(\underline{\operatorname{Pic}}_{X/S}\) exists whenever \(f\colon X\to S\) is proper and flat and such that the homomorphisms \(k(s)\to\operatorname{H}^0(X_s,{\mathscr{O}}_{X_s})\) (for \(s\in S\)) are isomorphisms (this latter condition also then implying that \({\mathscr{O}}_S\xrightarrow{\sim}f_*({\mathscr{O}}_X)\), and that this remains true after any base change \(S'\to S\)).
This at least is what we can prove in the setting of analytic spaces, if \(f\) is further assumed to be projective, by a differential method (that of Chow, if I am not mistaken) explained in {[}\protect\hyperlink{ref-Gro1960a}{8}, IX, 3.1{]}.
\oldpage{232-14}In this method, the passage to the quotient by a \emph{proper} and flat equivalence relation in an open of the scheme of divisors is replaced by the passage to the quotient by the projective group in the scheme of immersions of \(X\) into \(\mathbb{P}_S^r\).
This method can probably be adapted to the case of schemes, using the results of Mumford on the passage to the quotient by the projective group {[}\protect\hyperlink{ref-Mum1961}{15}{]};
for now, there is no written proof, except when \(X\) has ``lots of local sections'' over \(S\), for example if \(X\) is separable over a complete local ring with algebraically closed residue field.
In principle, the method in question is of more general scope, since it also gives the existence of Picard preschemes in the case where these are not separated, and where the first method thus necessarily fails.
(Technically, the difficulty comes from the fact that, when the geometric fibres of \(f\) are not integral, then the functor envisaged in \protect\hyperlink{fga-3-v-theorem-4.3}{(4.3)} is no longer representable by the projective bundle \(\mathbb{P}({\mathscr{Q}})\) itself, but by an \emph{open} of this, which leads to the delicate question of the passage to the quotient by an equivalence relation that is flat but not proper).

\emph{{[}Comp.{]}}
As I point out at the start of the next talk, the existence conjecture suggested here is false, but Mumford has proven a slightly weaker theorem using his methods.

\end{rmenv}

\leavevmode\vadjust pre{\hypertarget{fga-3-v-remark-5.3}{}}%
\begin{rmenv}{Remark 5.3}
Note that the proof given here uses neither the preliminary construction of Jacobians of curves or families of curves nor the theory of abelian varieties or abelian schemes, and in this way it essentially distinguishes itself from traditional treatments, such as those in the book by Lang {[}\protect\hyperlink{ref-Lan1959}{13}{]} or the article by Chevalley {[}\protect\hyperlink{ref-Che1960}{3}{]}, which follow the path sketched by Weil.
Even in the case of Jacobians of non-singular curves over an algebraically closed field (the complex numbers, say), the construction given here for the Jacobian is the only one known that comes with the very strong properties that we took as definition in \protect\hyperlink{fga-3-v-section-1}{§1} (essentially those of Chevalley, but taking into account the ``variety of parameters'' with nilpotent elements).
The fact that the construction of Picard schemes should precede, not follow, the theory of abelian varieties is clear a priori, by the fact that, in general, Picard schemes are not, nor do they reduce to, abelian schemes, as we already see in the case of singular curves over an algebraically closed field, where we find the ``generalised Jacobians'' of Rosenlicht, which are not abelian varieties.
Furthermore, the theory of abelian varieties, and more generally of abelian schemes, becomes much simpler once we have a theory of Picard schemes in general.
In particular, the theory of duality for abelian schemes, and notably the results of Cartier type, then become slightly more formal (cf.~for example {[}Mumford--Tate seminar, 1962{]})

\end{rmenv}

\leavevmode\vadjust pre{\hypertarget{fga-3-v-remarks-5.4}{}}%
\begin{rmenv}{Remarks 5.4}
The ``compatibility principle'' of Igusa for the Jacobian of a curve degenerating to a singular curve can only be well understood as an existence theorem of the Picard scheme of a relative scheme in curves \(X/S\) that are not necessarily simple over \(S\).
This is thus a particular case of the principal existence theorem \protect\hyperlink{fga-3-v-theorem-3.1}{(3.1)} when the specialised curve is integral (i.e.~in classical terms, irreducible of multiplicity \(1\)).
We note that, for now, the case of a reducible special curve (even when the components are of multiplicity \(1\), i.e.~when the special curve is separable over the residue field) is not covered by the known existence theorems, except for in the case where we are over a \emph{complete} discrete valuation ring with algebraically closed residue field, cf.~\protect\hyperlink{fga-3-v-remarks-5.2}{(5.2)}.
This question of existence certainly arises in a geometric construction, in the theory of schemes, of Baily--Satake ``compactifications'' of modular schemes of curves of genus \(g\).
(This compactification is known for now only for \(g=1\), thanks to work of Igusa).

\end{rmenv}

\hypertarget{fga-3-v-section-6}{%
\subsection{Relative existence theorems}\label{fga-3-v-section-6}}

We will sketch here some useful cases where the existence of certain Picard schemes implies the existence of certain others, which allows us to deduce from the principal existence theorem \protect\hyperlink{fga-3-v-theorem-3.1}{(3.1)} various other existence theorems.

\leavevmode\vadjust pre{\hypertarget{fga-3-v-proposition-6.1}{}}%
\begin{itenv}{Proposition 6.1}
Let \(f\colon X\to S\) be a flat projective morphism such that, in the Stein factorisation \(f=f''f'\), the morphism \(f'\colon X\to S'\) is \emph{flat} and has integral geometric fibres (and thus satisfies the hypotheses of \protect\hyperlink{fga-3-v-theorem-3.1}{(3.1)}), and such that the finite morphism \(f''\colon S'\to\) is flat.
Then \(\underline{\operatorname{Pic}}_{X/S}\) exists and (with the notation introduced in \protect\hyperlink{fga-3-ii-section-C.2}{FGA 3.II, C.2}) we have a canonical isomorphism
\[
  \underline{\operatorname{Pic}}_{X/S}
  \xrightarrow{\sim}\prod_{S'/S} \underline{\operatorname{Pic}}_{X/S'}.
\]

\end{itenv}

\begin{proof}
To prove this, we first establish an isomorphism of functors
\[
  \mathscr{P}\kern -.5pt ic_{X/S}
  \xrightarrow{\sim}\prod_{S'/S} \mathscr{P}\kern -.5pt ic_{X/S'}
\]
and then use \protect\hyperlink{fga-3-v-theorem-3.1}{(3.1)}, which implies that \(\mathscr{P}\kern -.5pt ic_{X/S'}\) is representable;
we use the structure explained in \protect\hyperlink{fga-3-v-section-3}{§3} of \(\underline{\operatorname{Pic}}_{X/S'}\) (which implies that every finite subset of a fibre of \(\underline{\operatorname{Pic}}_{X/S'}\) over \(S\) is contained in an affine open) for the existence of \(\prod_{S'/S}\underline{\operatorname{Pic}}_{X/S'}\).
\end{proof}

For example, if \(X\) is a scheme given by a \emph{sum} of schemes \(X_i\) over \(S\) that satisfy the conditions of \protect\hyperlink{fga-3-v-theorem-3.1}{(3.1)}, then the statement of \protect\hyperlink{fga-3-v-proposition-6.1}{(6.1)} reduces to the trivial statement
\[
  \underline{\operatorname{Pic}}_{X/S}
  \xrightarrow{\sim}\prod_i \underline{\operatorname{Pic}}_{X_i/S}.
\]

\leavevmode\vadjust pre{\hypertarget{fga-3-v-corollary-6.2}{}}%
\begin{itenv}{Corollary 6.2}
Let \(f\colon X\to S\) be a projective flat morphism with locally integral geometric fibres (for example, a projective and normal morphism).
Then \(\underline{\operatorname{Pic}}_{X/S}\) exists.

\end{itenv}

\begin{proof}
In this case, \(S'\) is an étale covering of \(S\) (which is true once \(f\) is separable, i.e.~flat with reduced geometric fibres), and we see that the structure theorem stated in \protect\hyperlink{fga-3-v-section-3}{§3} for \(\underline{\operatorname{Pic}}_{X/S}\) still holds, thanks to the analogous structure of \(\underline{\operatorname{Pic}}_{X/S'}\).
\end{proof}

Applying a descent procedure gives a relative existence theorem, whose scope depends on the solution to questions about non-flat descent that were raised in \protect\hyperlink{fga-3-i-section-A.3.c}{FGA 3.I, A.3.c}, and of which we content ourselves here to explain only a particular case:

\leavevmode\vadjust pre{\hypertarget{fga-3-v-proposition-6.3}{}}%
\begin{itenv}{Proposition 6.3}
Let \(f\colon X\to S\) be a proper morphism, and let \(X_1\) and \(X_2\) be subpreschemes of \(X\) that are flat over \(S\), defined by coherent ideals \({\mathscr{J}}_1\) and \({\mathscr{J}}_2\) (respectively) such that \({\mathscr{J}}_1\cap{\mathscr{J}}_2=(0)\) and such that \({\mathscr{O}}_X/({\mathscr{J}}_1+{\mathscr{J}}_2)\) is flat over \(S\) (i.e.~the subprescheme of \(X\) that is the sup of \(X_1\) and \(X_2\) is \(X\) itself, whereas their inf \(Z\) is flat over \(X\)).
Suppose further that, for all \(s\in S\), the homomorphisms \(k(s)\to\operatorname{H}^0(X_{i_s},{\mathscr{O}}_{X_{i_s}})\) are bijective for \(i=1,2\).
Then the natural homomorphism of functors
\[
  \mathscr{P}\kern -.5pt ic_{X/S}
  \to \mathscr{P}\kern -.5pt ic_{X_1/S} \times \mathscr{P}\kern -.5pt ic_{X_2/S}
\]
is representable by affine morphisms, so if \(\underline{\operatorname{Pic}}_{X_1/S}\) and \(\underline{\operatorname{Pic}}_{X_2/s}\) exist, then so too does \(\underline{\operatorname{Pic}}_{X/S}\), and the canonical morphism
\[
  \underline{\operatorname{Pic}}_{X/S}
  \to \underline{\operatorname{Pic}}_{X_1/S} \times \underline{\operatorname{Pic}}_{X_2/S}
\]
is affine.

\end{itenv}

\begin{proof}
By faithfully flat descent, we can reduce to the case where \(Z\) admits a section over \(S\), thus defining sections of \(X\), \(X_1\), and \(X_2\) over \(S\), and allowing us to eliminate the automorphisms in the structures in question, as explained in \protect\hyperlink{fga-3-v-remark-2.5}{(2.5)}.
\oldpage{232-17}The proof then consists of noting that the data of a ``rigidified'' invertible module \({\mathscr{L}}\) on \(X\) is equivalent to the data of a triple \(({\mathscr{L}}_1,{\mathscr{L}}_2,u)\), where \({\mathscr{L}}_i\) is a ``rigidified'' module on \(X_i\), and \(u\) is an isomorphism from \({\mathscr{L}}_1|Z\) to \({\mathscr{L}}_2|Z\) that is compatible with the rigidifications.
It remains only to verify that, for \({\mathscr{L}}_1\) and \({\mathscr{L}}_2\) fixed, the data of \(u\) can be expressed as a section of a suitable scheme over \(S\) that is affine over \(S\), which is easy.
\end{proof}

From \protect\hyperlink{fga-3-v-proposition-6.3}{(6.3)} we easily conclude:

\leavevmode\vadjust pre{\hypertarget{fga-3-v-corollary-6.4}{}}%
\begin{itenv}{Corollary 6.4}
Let \(X\) be a proper and separable scheme over a field \(k\), and let \(X_i\) be the irreducible components of \(X\).
If the \(\underline{\operatorname{Pic}}_{X_i/k}\) exist, then so too does \(\underline{\operatorname{Pic}}_{X/k}\), and the canonical morphism
\[
  \underline{\operatorname{Pic}}_{X/k}
  \to \prod_i \underline{\operatorname{Pic}}_{X_i/k}
\]
is affine.

\end{itenv}

Combined with \protect\hyperlink{fga-3-v-corollary-6.2}{(6.2)}, this shows, for example, the existence of \(\underline{\operatorname{Pic}}_{X/k}\) whenever \(X\) is a projective scheme that is separable over a field \(k\).
If \(X\) is no longer separable over \(k\), then we equally have a reduction result, using the argument of Oort {[}\protect\hyperlink{ref-Oor1962}{17}{]}.
The method equally applies for a scheme with arbitrary base (a useful case, for example, in proving in the following talk the finiteness result stated in \protect\hyperlink{fga-3-v-remark-3.3}{(3.3)}).
To avoid an overly technical statement, we restrict ourselves to the case where we are over a base field:

\leavevmode\vadjust pre{\hypertarget{fga-3-v-proposition-6.5}{}}%
\begin{itenv}{Proposition 6.5}
Let \(X\) be a proper scheme over a field \(k\), and \(X_0\) a subscheme that has the same underlying set (thus defined by a nilpotent ideal on \(X\)).
Then the functorial morphism \(\mathscr{P}\kern -.5pt ic_{X/k}\to\mathscr{P}\kern -.5pt ic_{X_0/k}\) is representable by affine morphisms.
In particular, if \(\underline{\operatorname{Pic}}_{X_0/k}\) exists, then so too does \(\underline{\operatorname{Pic}}_{X/k}\), and the morphism \(\underline{\operatorname{Pic}}_{X/k}\to\underline{\operatorname{Pic}}_{X_0/k}\) is affine.

\end{itenv}

Combining this with \protect\hyperlink{fga-3-v-corollary-3.4}{(6.4)}, we easily conclude:

\leavevmode\vadjust pre{\hypertarget{fga-3-v-corollary-6.6}{}}%
\begin{itenv}{Corollary 6.6}
Let \(X\) be a projective prescheme over a field \(k\)\textgreater{}
Then \(\underline{\operatorname{Pic}}_{X/k}\) exists.

\end{itenv}

\leavevmode\vadjust pre{\hypertarget{fga-3-v-remark-6.6}{}}%
\begin{rmenv}{Remark 6.6}
It is extremely plausible that, for every \emph{proper} scheme \(X\) over a field \(k\), \(\underline{\operatorname{Pic}}_{X/k}\) exists.
The results above allow us to reduce, for this question, to the case where \(k\) is algebraically closed, and where \(X\) is integral.
We then know that there exists an integral scheme \(X'\) that is \emph{projective} over \(k\), and a dominant morphism \(g\colon X'\to X\) (Chow's lemma).
\oldpage{232-18}It would thus suffice to show that the corresponding functorial morphism \(\mathscr{P}\kern -.5pt ic_{X/k}\to\mathscr{P}\kern -.5pt ic_{X'/k}\) is representable (and, probably, representable by affine morphisms), since we already know that \(\mathscr{P}\kern -.5pt ic_{X'/k}\) is representable.
This raises questions about non-flat descent that are not answerable as of now.
Note that, if we restrict to considering the restriction of the functor \(\mathscr{P}\kern -.5pt ic_{X/k}\) to reduced preschemes (with \(X\) proper and integral over an algebraically closed field \(k\)), then we do indeed obtain a representable functor, as shown by Chevalley {[}\protect\hyperlink{ref-Che1960}{3}{]} in the case where \(X\) is normal, and by Seshadri {[}\protect\hyperlink{ref-Ses1962}{25}{]} by a descent method in the general case.
But with our notation, the scheme constructed by these authors is not \(\underline{\operatorname{Pic}}_{X/k}\), but instead \((\underline{\operatorname{Pic}}_{X/k})_\mathrm{red}\), i.e.~the reduced scheme corresponding to \(\underline{\operatorname{Pic}}_{X/k}\).

\emph{{[}Comp.{]}}
As we point out at the start of the next talk, the question raised here has just been answered in the affirmative by Murre.

\end{rmenv}

\leavevmode\vadjust pre{\hypertarget{fga-3-v-remark-6.7}{}}%
\begin{rmenv}{Remark 6.7}
More generally, let \(f\colon X'\to X\) be a surjective morphism of proper preschemes over \(k\).
Then considering non-flat descent leads us to conjecture that \(\underline{\operatorname{Pic}}_{X/k}\to\underline{\operatorname{Pic}}_{X'/k}\) is an \emph{affine} morphism, which would in particular imply (by dividing into the connected components of the identity elements) that the corresponding homomorphism on the Néron--Severi groups is injective modulo torsion.
We can verify this by the theory of intersections when \(X\) and \(X'\) are non-singular.
The answer does not seem to be known in any other case.

\emph{{[}Comp.{]}}
The question raised here has been answered in the affirmative (cf.~the last paragraph of the comments in the next talk).

\end{rmenv}

\leavevmode\vadjust pre{\hypertarget{fga-3-v-remark-6.8}{}}%
\begin{rmenv}{Remark 6.8}
Contrary to what we might think, the consideration of Picard schemes of algebraic schemes with nilpotent elements is useful, and even indispensable, for various questions.
If \(X\) is a projective scheme, simple over \(k\), say, and \(Y\) a hyperplane section, then we can consider the ``infinitesimal neighbourhoods'' \(X_n\) of \(Y\) of all orders, as well as the Picard schemes \(\underline{\operatorname{Pic}}_{X_n/k}\);
when \(X\) is irreducible of dimension \(\geqslant 4\) (resp. \(\geqslant 3\)), the canonical morphism
\[
  \underline{\operatorname{Pic}}_{X/k}
  \to\underline{\operatorname{Pic}}_{X_n/k}
  \qquad\text{for large }n
\]
is an isomorphism (resp. induces an isomorphism between the inverse images of the torsion subgroups of the Néron--Severi groups), and this result will be useful in the qualitative study of Picard schemes in the following talk.
Similarly, the consideration of Picard schemes of certain curves with nilpotent elements and the fundamental theorems of formal geometry {[}\protect\hyperlink{ref-Gro1958a}{7}{]} allow us to prove, in the case of equal characteristic, a conjecture of Mumford, namely that for every complete normal Noetherian local ring \(A\) of dimension \(2\), the group of classes of divisors of \(A\) can be considered as the set of rational points over \(k\) of an algebraic group \(G\) over the residue field \(k\) (with \(G\) being canonically determined once we have a field of representatives in \(A\)).
\oldpage{232-19}In the case where \(A\) is of arbitrary dimension, it is plausible that there exists an algebraic pro-group over \(k\) that plays the same role as \(G\) above, which is constructed, in the case where we can ``desingularise'' \(\operatorname{Spec}(A)\), as a projective limit of Picard schemes of suitable projective schemes (with nilpotent elements) over \(k\).

\end{rmenv}

\hypertarget{fga-3.vi}{%
\section{Picard schemes: General properties}\label{fga-3.vi}}

\providecommand{\scr}[1]{{\mathscr{#1}}}
\renewcommand{\cal}[1]{{\mathcal{#1}}}
\renewcommand{\frak}[1]{{\mathfrak{#1}}}
\renewcommand{\geq}{\geqslant}
\renewcommand{\leq}{\leqslant}

\providecommand{\Spec}{\operatorname{Spec}}
\providecommand{\red}{\mathrm{red}}
\providecommand{\Ga}{\operatorname{G_a}}
\providecommand{\Gm}{\operatorname{G_m}}
\providecommand{\HH}{\operatorname{HH}}
\providecommand{\Alb}{\operatorname{Alb}}

{[}FGA 3.VI{]}
Grothendieck, A.
``Technique de descente et théorèmes d'existence en géométrie algébrique, VI: Les schémas de Picard: Propriétés générales''.
\emph{Séminaire Bourbaki} \textbf{14} (1961--62), Talk no. 236.

\hypertarget{supplements-to-the-previous-talk-fga-3.v}{%
\subsection*{0. Supplements to the previous talk (FGA 3.V)}\label{supplements-to-the-previous-talk-fga-3.v}}
\addcontentsline{toc}{subsection}{0. Supplements to the previous talk (FGA 3.V)}

\oldpage{236-01}There has been some progress concerning the questions of existence of Picard preschemes raised in \href{FGA-3-V.html}{FGA 3.V}:

\begin{enumerate}
\def\labelenumi{\alph{enumi}.}
\item
  (Mumford).
  It is not true in general that, if \(f\colon X\to S\) is a projective and separable (i.e.~flat with separable fibres) morphism, then the Picard prescheme \(\underline{\operatorname{Pic}}_{X/S}\) exists, even if the fibres of \(f\) are of dimension \(1\) and \(S\) is the spectrum of a complete discrete valuation ring.
  A counterexample is given by taking \(S=\operatorname{Spec}\mathbb{R}[[t]]\), and taking \(X\) to be the subscheme of \(\mathbb{P}_S^2\) (with homogeneous variables \(x,y,z\)) defined by the equation \(x^2+y^2=tz^2\), which represents a conic degenerating to two geometrically concurrent lines, but the special fibre over the field \(\mathbb{R}\) is nevertheless irreducible (it is given by the equation \(x^2+y^2=0\) in \(\mathbb{R}\)).
  We easily see that, after the étale extension \(S'\to S\), with \(S'=\operatorname{Spec}\mathbb{C}[[t]]\), the Picard prescheme of \(X'/S'\) exists, and we thus obtain an explicit description of it as a sum of copies of \(\widetilde{S}\), where \(\widetilde{S}\) is induced by \(S\) by copying the origin an infinite number of times.
  We easily observe that the descent data on \(\underline{\operatorname{Pic}}_{X'/S'}\) for \(S'\to S\) (given here by the actions of the Galois group \(\mathbb{Z}/2\mathbb{Z}\) of \(S'\) over \(S\)) is not effective, since the group permutes certain doubled points (so that there are orbits that are not contained in an affine open).
  However, Mumford has shown that, if \(f\colon X\to S\) is a separable projective morphism such that, for all \(s\in S\), the irreducible components of \(X_s\) are geometrically irreducible with respect to \(k(s)\), then \(\underline{\operatorname{Pic}}_{X/S}\) exists;
  the proof relies on a refinement of his theorem of passage to the quotient, cf.~{[}Mumford--Tate seminar, 1962{]}.\footnote{\emph{{[}Trans.{]}} See translator footnotes in \protect\hyperlink{FGA-3-V.html}{FGA 3.V}.}
  Note also that it is still possible that, without any hypotheses on the irreducible components of the fibres \(X_s\), the scheme \(\underline{\operatorname{Pic}}_{X/S}^\tau\) (which will be introduced below) still exists.
\item
  (Murre).\oldpage{236-02}
  If \(X\) is a proper scheme over a field, then \(\underline{\operatorname{Pic}}_{X/k}\) exists.
  The proof partially uses the proof of Chevalley {[}\protect\hyperlink{ref-Che1960}{3}{]}, and is fundamentally based on the group structure of the Picard functor.
\end{enumerate}

For certain additional comments concerning the theory of Picard schemes, most notably in relation to abelian schemes, we recommend consulting {[}Mumford--Tate seminar, 1962{]}.
Finally, a notable shortcoming of the present talk is the absence of ``equivalence criteria'' that would allow us to compare the Picard scheme of a projective scheme and of its hyperplane sections;
the key theorems for developing such criteria can be found in {[}\protect\hyperlink{ref-Gro1960b}{9}{]}, with which one must combine the existence theorems for Picard schemes.

\hypertarget{fga-3-vi-section-1}{%
\subsection{Topological properties of preschemes of commutative groups}\label{fga-3-vi-section-1}}

Let \(k\) be a field, and \(G\) be a prescheme of groups over \(k\).
Since the identity element \(e\), being rational over \(k\), is necessarily closed, it immediately follows that the diagonal of \(G\times_k G\) is closed, and so \(G\) is \emph{separated}: \emph{every prescheme of groups over a field is separated}.
We denote by \(G^0\) the connected component of the identity element \(e\).
Since \(e\) is rational over \(k\), \(G^0\) is in fact \emph{geometrically connected}, i.e.~\(G\mapsto G^0\) is compatible with base change to another field.
It also follows that \(G^0\) is stable under multiplication (set-theoretically), and if \(G\) is locally Noetherian then \(G^0\) is open, and we can consider \(G^0\) as an \emph{open subgroup} of \(G\).
In what follows, we suppose \(G\) to be locally of finite type over \(k\);
then \(G^0\) is \emph{geometrically irreducible and of finite type over \(k\)}.
Indeed, we can suppose that \(k\) is algebraically closed, and thus that \(G\) is reduced (since \(G_\mathrm{red}\) is then a subgroup of \(G\), taking into account the fact that \(G_\mathrm{red}\times_k G_\mathrm{red}\) is also then reduced), and thus simple over \(k\) over a non-empty open, and thus everywhere, by translating this open subset.
But then \(G\) is locally irreducible, and so its irreducible components are identical to its connected components, and so \(G^0\) is irreducible.
So let \(U\) be an affine neighbourhood of \(e\) in \(G^0\);
using the fact that \(G^0\) is irreducible, we immediately see that \(U\cdot U=D^0\), which proves that \(G^0\) is quasi-compact, and thus of finite type over \(k\).

Suppose, for simplicity, that \(G\) is commutative.
For every integer \(n>0\), let \(G^{(n)}\) be the inverse image of \(G^0\) under the \(n\)-th power homomorphism \(\varphi_n\) to \(G\), so \(G^{(n)}\) is an open subgroup of \(G\).
We set
\[
  \begin{aligned}
    G^\tau
    &= \bigcup_{n>0} G^{(n)}
  \\G^\sigma
    &= \bigcup_{(n,p)=1} G^{(n)}
  \\G^\rho
    &= \bigcup_{h>0} G^{(p^h)}
  \end{aligned}
\]
where \(p\) is the characteristic order of the field \(k\).
\oldpage{236-03}These are open subgroups of \(G\) that satisfy
\[
  \begin{aligned}
    G^\sigma \cap G^\rho
    &= G^0
  \\G^\sigma \cdot G^\rho
    &= G^\tau.
  \end{aligned}
\]

\leavevmode\vadjust pre{\hypertarget{fga-3-vi-remark}{}}%
\begin{rmenv}{Remark}
We can construct the quotient group scheme \(G/G^0=N\) (cf.~\href{FGA-3-IV.html}{FGA 3.IV}) and then define \(G^\tau\), \(G^\sigma\), and \(G^\rho\) as the inverse images in \(G\) of the torsion subgroup of \(N\) (resp. of its \(p\)-primary component, resp. of the natural complement of its \(p\)-primary component, given by the sum of the \(q\)-primary components for \(q\) prime with \(q\neq p\)).
Note that \(N\) is a discrete group scheme that is separable over \(k\), thus (once we have chosen an algebraic closure \(\overline{k}\) of \(k\), giving rise to a Galois group \(\pi\)) can be identified with an ordinary discrete group on which \(\pi\) acts by automorphisms.
It is in this way that we can interpret, in an evident way, the construction of the torsion subgroup and the decomposition of this into its \(q\)-primary components.
If \(G\) is the Picard scheme of a proper scheme \(X\) over \(k\), then \(N\) could be called the (reduced) \emph{Néron--Severi scheme} of \(X\) over \(k\).
If \(G_\mathrm{red}^0\) is a group subscheme of \(G^0\), which is the case whenever, in particular, \(k\) is parfait, or \(G^0\) is proper over \(k\) (for example if \(X\) is geometrically normal), then we can equally introduce the quotient \(N'=G/G_\mathrm{red}^0\), which has the tendency to behave better than \(N\) from the specialisation point of view, i.e.~as \(X\) varies over a family of algebraic schemes.

\end{rmenv}

Now let \(S\) be a locally Noetherian prescheme, and \(G\) a group prescheme over \(S\) that is locally of finite type over \(S\).
We do not assume \(G\) to be of finite type over \(S\), nor separated over \(S\).
We then set
\[
  G^0
  = \bigcup_{s\in S} (G_s)^0
\]
and, if \(G\) is commutative,
\[
  \begin{aligned}
    G^\tau
    &= \bigcup_{s\in S} (G_s)^\tau
  \\G^\sigma
    &= \bigcup_{s\in S} (G_s)^\sigma
  \\G^\rho
    &= \bigcup_{s\in S} (G_s)^\rho.
  \end{aligned}
\]
\oldpage{236-04}These are subsets of \(G\), stable under the multiplication of \(G\), which does not obviously imply that they can be defined by means of sub-group preschemes of \(G\).
Notably, it seems that \emph{there does not exist in a general sub-group prescheme of \(G\) whose underlying set is \(G^0\)}.
Of course, if one of these sets is open, then, endowed with the induced structure, it is an open sub-group prescheme of \(G\).
We will see that this is always the case for \(G^\tau\);
in this way, from the point of view of representable functors, in particular \emph{from the ``specialisation'' point of view, numerical equivalence behaves in a more satisfying manner than algebraic equivalence}.
Here are the principle general properties of the sets that we have just defined:

\hypertarget{fga-3-vi-theorem-1.1}{}
\begin{itenv}{Theorem 1.1}

\(G^0\), \(G^\tau\), \(G^\sigma\), and \(G^\rho\) are \emph{locally constructible}.
Furthermore:

\begin{enumerate}
\def\labelenumi{\roman{enumi}.}
\tightlist
\item
  \(G^0\) is quasi-compact over \(S\).
  If the \(G_s^0\) are proper, and \(G\) is separated over \(S\), then \(G^0\) is proper over \(S\) and thus closed in \(G\).
\item
  \(G^\tau\) is open.
  If \(G^0\) is closed, then so too is \(G^\tau\).
\item
  If \(G^0\) is closed, then so too is \(G^\sigma\), provided that we are in equal characteristic, i.e.~that all the residue fields of \(S\) have the same characteristic.
  If \(G^0\) is closed and \(G\to S\) is universally open at the points of \(G^\sigma\) (cf.~\protect\hyperlink{fga-3-vi-corollary-1.5}{Corollary 1.5} below), then \(G^\sigma\to S\) is universally open.
\item
  If \(G^0\) is closed, then so too is \(G^\rho\).
  If we are in equal characteristic, and if, for every integer \(n>0\) such that \((n,p)=1\), the \(n\)-th power homomorphism to \(G\) is open, then \(G^\rho\) is open.
\end{enumerate}

\end{itenv}

We now give some hints towards the proof.
The fact that \(G^0\) is locally constructible, and quasi-compact over \(S\), is contained in the following lemma:

\leavevmode\vadjust pre{\hypertarget{fga-3-vi-lemma-1.2}{}}%
\begin{itenv}{Lemma 1.2}
If \(S\neq\varnothing\) then there exists a non-empty open \(U\) in \(S\), a group scheme \(H\) of finite type over \(U\) with connected fibres, and a homomorphism of group schemes \(H\to G|U\) that is an open immersion with image \(G^0|U\).

\end{itenv}

\begin{proof}
To prove this lemma, we can suppose that \(S\) is irreducible;
let \(\eta\) be its generic point.
If we make the base change \(S'=\operatorname{Spec}({\mathscr{O}}_{S,\eta})\to S\) then we find a group scheme \(G'\) over a local Artinian ring \({\mathscr{O}}_{S,s}=A\) inside which we have an \emph{open} group subscheme \({G'}^0\) \emph{of finite type over \(A\)}, as we said above.
This thus comes from a group scheme \(H\) of finite type over an open neighbourhood \(U\) of \(\eta\), and the canonical immersion \({G'}^0\to G'\) comes from an open immersion \(H\to G|U\), which will be a homomorphism of group schemes for \(U\) small enough.
\oldpage{236-05}Since the fibres of \(H\) are connected if we take \(U\) small enough, and since they are all of the same dimension, namely that of the fibres of \(G\), for \(U\) small enough, it follows that, for all \(s\in U\), the image of \(H_s\) in \(G_s\) is exactly \(G_s^0\) (for \(U\) small enough), which proves the lemma.
\end{proof}

The second claim in \protect\hyperlink{fga-3-vi-theorem-1.1}{(1.1), (i)} is contained in the following lemma (which we apply to an quasi-compact open neighbourhood of \(G^0\) in \(G\)):

\leavevmode\vadjust pre{\hypertarget{fga-3-vi-lemma-1.3}{}}%
\begin{itenv}{Lemma 1.3}
Let \(X\) be a separated prescheme of finite type over \(S\), with \(S\) locally Noetherian, and let \(g\) be a section of \(X\) over \(S\), and \(X^0\) the union of the connected components of the \(g(s)\) in the \(X_s\).
Let \(s\in S\) be such that \(X_s^0\) is proper over \(k(s)\).
Then there exists an open neighbourhood \(U\) of \(s\) such that \(X^0|U\) is proper over \(U\), and a fortiori closed in \(X|U\).

\end{itenv}

\begin{proof}
By faithfully flat descent of the base, we can reduce to the case where \(S\) is the spectrum of a complete local ring, and \(s\) is its closed point.
Applying {[}\protect\hyperlink{ref-GD1960}{10}, III, 5.5.1{]}, we see that \(X\) decomposes into a sum of two disjoint opens \(X'\) and \(X''\), with \(X'\) proper over \(S\), and such that \(X'_s=S_s^0\).
This allows us to reduce to the case where \(X=X'\), i.e.~where \(X\) is proper over \(S\).
In this case, we can apply a standard proof, using the valuative criterion of properness of a subset (forgotten in {[}\protect\hyperlink{ref-GD1960}{10}, II{]}).
\end{proof}

We now prove that \(G^\tau\) is open, or, equivalently, taking into account the fact that the formation of \(G^\tau\) (as that of \(G^0\), \(G^\sigma\), and \(G^\rho\)) commutes with base extension: for every sections \(g\) of \(G\) over \(S\), \(g^{-1}(G^\tau)\) is open.
This implies two things:

\begin{enumerate}
\def\labelenumi{\alph{enumi}.}
\tightlist
\item
  Let \(y\in S\) be such that \(g(y)\in G^\tau\).
  Then, for all neighbours \(y'\in\overline{y}\) of \(y\), we have \(g(y')\in G^\tau\).
\item
  Let \(y'\in S\) be such that \(g(y')\in G^\tau\).
  Then, for every generalisation \(y\) of \(y'\), we have \(g(y)\in G^\tau\).
\end{enumerate}

For (a), note that there exists an integer \(n>0\) such that \(g^n(y)\in G^0\); since \(G^0\) is constructible, \((g^n)^{-1}G^0\) is constructible;
it follows that we have \(g^n(y')\in G^0\) for all neighbours \(y'\in\overline{y}\) of \(y\).
For (b), note that the \(g(y')^n=g^n(y')\) remain in a quasi-compact open of \(G_{y'}\) (since they are contained in a finite number of classes modulo \(G_{y'}^0\)), and so there exists a quasi-compact open \(U\) in \(G\) that contains the \(g^n(y')\), and thus also their generalisations \(g^n(y)\), and so the powers of \(g(y)\) remain in a quasi-compact open of \(G_y\), which easily implies that \(g(y)\in G_y^\tau\).

\oldpage{236-06}Suppose that \(G^0\) is closed;
we will show that so too is \(G^\tau\).
As we already know that \(G^\tau\) is open, thus locally constructible, it remains only to show that it is stable under specialisation, which comes from the fact that it is a union of closed subsets, namely inverse images under the \(n\)-th power homomorphisms \(\varphi_n\) of the closed subset \(G^0\).

The same argument shows that \(G\sigma\) and \(G^\rho\) are closed if \(G^0\) is (under the additional hypothesis of equal characteristic in the case of the former), once we have shown that \(G^\sigma\) and \(G^\rho\) are locally constructible.
But, for \(x\in G^\tau\), let \(\nu(x)\) be the smallest integer \(n>0\) such that the \(n\)-th power homomorphism \(\varphi_n\) sends \(x\) to \(G\).
Then \(G^\sigma\) (resp. \(G^\rho\)) consists of the \(x\in X\) such that \(\nu(x)\) is coprime to \(p\) (resp. to a power of \(p\)), and our claim of constructibility then follows from the following, more precise lemma:

\leavevmode\vadjust pre{\hypertarget{fga-3-vi-lemma-1.4}{}}%
\begin{itenv}{Lemma 1.4}
The function \(\nu\) on \(G^\tau\) is locally constructible.

\end{itenv}

\begin{proof}
This means that, for every integer \(n>0\), the set of \(x\in G^\tau\) such that \(\nu(x)=n\) is locally constructible;
but this is the difference between \(\varphi_n^{-1}(G^0)\) and the union of the \(\varphi_d^{-1}(G^0)\) where \(d\) runs over the proper divisors of \(n\);
since \(G^0\) is locally constructible, so too are all the \(\varphi_i^{-1}(G^0)\), and thus also the aforementioned difference.
\end{proof}

Suppose that \(G^0\) is closed and that \(G\to S\) is universally open at the point of \(G^\sigma\);
we will show that \(G^\sigma\to S\) is open, i.e.~sends a neighbourhood in \(G^\sigma\) of any \(x\in G^\sigma\) to a neighbourhood of \(y=f(x)\).
Since \(G^\sigma\) is locally constructible, it suffices to show that, for every generalisation \(y'\) of \(y\), there exists a generalisation of \(x\)' of \(x\) in \(G\) in \(y'\).
By base change, this allows us to reduce to the case where \(S\) is the spectrum of a discrete valuation ring, and where \(y\) and \(y'\) are the closed point and the generic point (respectively).
Using the fact that \(G\to S\) is open at \(x\) (and thus there exists a generalisation \(x_1\) of \(x\) in \(G\) over \(y'\)), we can further suppose that there exists a section \(g\) of \(G\) over \(S\) such that \(x=g(y)\) (after performing another base change).
If \(k(y')\) is of characteristic \(0\), then it suffices to take any generalisation \(x'\) of \(x\) in \(G\) over \(y'\);
it is in \(G^\tau\) since \(G^\tau\) is open, and thus in \(G^\sigma\) since \(G_{y'}^\sigma=G_{y'}^\tau\).
If the characteristic of \(k(y')\) is \(p>0\), then let
\[
  \nu(g(y'))
  = p^hm
  \qquad\text{where }(m,p)=1
\]
\oldpage{236-07}and let \(a\) and \(b\) be integers such that \(ap^h+bm=1\), and set
\[
  \begin{aligned}
    g_1
    &= g^{ap^h}
  \\g_2
    &= g^{bm}
  \end{aligned}
\]
so that \(g=g_1g_2\).
By construction, we then have \(g_1(y')\in G^\sigma\) and \(g_2(y')\in G^\rho\).
Since \(G^0\) is closed, it follows that \(g_1(y)\in G^\sigma\) and \(g_2(y)\in G^\rho\), whence, since
\[
  g(y)
  = g_1(y)g_2(y) \in G^\sigma
\]
we also have that \(g_2(y)\in G^\sigma\), and so
\[
  g_2(y)\in G_y^\sigma\cap G_y^\rho
  = G_y^0.
\]
But from the hypothesis, and the fact that \(S\) is the spectrum of a discrete valuation ring, it follows that \(G\setminus(G_y\setminus G_y^0)\) is an open of \(G\) over which \(G\to S\) induces an open morphism, and thus, at every point of \(G_y^0\), gives a ``quasi-section'';
then, after possibly another extension of the base \(S\), we can suppose that there exists a section \(g'_2\) of \(G^0\) over \(S\) such that \(g'_2(y)=g_2(y)\).
Set \(g'=g_1g'_2\);
then, by construction, \(g'(y)=g(y)=x\), and \(g'(y')=g_1(y')g'_2(y')\in G^\sigma\), and so \(g(y')=x'\) is a generalisation of \(x\) in \(G^\sigma\) over \(y'\), which proves (iii).

Finally we prove the last claim (iv).
It suffices to prove that, if \(x\in G^\rho\), then every generalisation \(x'\) of \(x\) is in \(G^\rho\).
We can suppose (after taking the images under \(\varphi_{n^h}\) for suitable \(h\)) that we even have \(x\in G^0\).
Then, for every integer \(n>0\) coprime to the characteristic, \(x\) is in the image of \(\varphi_n\) (since the \(n\)-th power in a connected group of finite type over a field of characteristic coprime to \(n\) is surjective).
Since \(\varphi_n\) is open, it follows that \(x'\) is also in the image of \(\varphi_n\).
More precisely, let \(U\) be a quasi-compact open of \(G^\tau\) that contains \(G_y^0\), and then \(x'\in\varphi_n(U)\) for all \(n\) coprime to \(p\).
Taking \(n\) to be a common multiple of the factors of the \(\nu(z)\) that are coprime to \(p\), for \(z\in U\), we see that \(x'\in G^\rho\).

The claim \protect\hyperlink{fga-3-vi-theorem-1.1}{(1.1), (iii)} is completed like so:

\leavevmode\vadjust pre{\hypertarget{fga-3-vi-corollary-1.5}{}}%
\begin{itenv}{Corollary 1.5}
Let \(n>1\) be an integer such that the \(n\)-th power homomorphism \(\varphi_n\colon G\to G\) is universally open (for example, étale).
Let \(G^{(n)}=\varphi_n^{-1}(G^0)\), and suppose that the connected fibres \(G_s^0\) ``do not contain the additive component'' (i.e the group induced by field extension from \(k(s)\) to the algebraic closure does not contain a subgroup isomorphic to \(\operatorname{G_a}\)).
\oldpage{236-08}Then \(G\to S\) is universally open at the point of \(G^{(n^h)}\).
In particular, if the \(G_s^0\) do not contain the additive component, and if, for every integer \(n>1\), the homomorphism \(\varphi_n\) is universally open at the points of residual characteristic coprime to \(p\), then \(G\to S\) is universally open at the points of \(G^\sigma\), and thus \protect\hyperlink{fga-3-vi-theorem-1.1}{(1.1), (iii)} if \(G^0\) is closed then \(G^\sigma\to S\) is universally open.
In this case, \(s\mapsto\dim G_s\) is a locally constant function on \(S\).

\end{itenv}

\begin{proof}
It follows from the hypotheses that the kernel \({}_nG\) of \(\varphi_n\) is universally open over \(S\), and so \(G\to S\) is universally open at the points of \({}_nG\), and thus also (replacing \(n\) with \(n^h\)) at the points of \({}_{(n^h)}G\).
But the hypothesis on the fibres \(G_s^0\) tells us exactly that the points of order some power of \(n\) in \(G_s\) are dense in the union of the \(G^{(n^h)}\), whence it easily follows that \(G\to S\) is universally open at all points of this union, in particular along the identity section, whence it easily follows that \(s\mapsto\dim G_s\) is a locally constant function.
\end{proof}

\leavevmode\vadjust pre{\hypertarget{fga-3-vi-remark-1.6}{}}%
\begin{rmenv}{Remark 1.6}
Recall that a morphism \(X\to S\) is said to be universally open if it sends every open to an open, and retains this nice property after any base change \(S'\to S\).
This also implies (if \(S\) is locally Noetherian and \(X\to S\) is locally of finite type) that every irreducible component of \(X\) dominates \(S\), and that this property is retained after any base change \(S'\to S\).
In these two claims, it suffices (thanks to the finiteness hypotheses above) to verify only for the base changes \(S'\to S\) where \(S'\) is the spectrum of a discrete valuation ring (complete, with algebraically closed residue field, if one wants\ldots).
The definition extends in an evident way to the case of a subset \(Z\) of \(X\) (such as the subset \(G^\sigma\) of \(G\)).
It is a perfectly normal phenomenon, even if we start with a simple projective morphism \(X\to S\) with connected geometric fibres (for example, the fibre square of the modular family of elliptic curves over \(S=\operatorname{Spec}\mathbb{C}[j]\)), that \(\underline{\operatorname{Pic}}_{X/S}\) is not universally open over \(S\), i.e.~that there can be irreducible components of \(\underline{\operatorname{Pic}}_{X/S}\)that live entirely over a single point of \(S\);
this is linked to the fact that the rank of the Néron--Severi group of the fibres of \(X/S\) can jump up (``complex multiplication'' phenomena).
However, \protect\hyperlink{fga-3-vi-corollary-1.5}{(1.5)} assures us that, in good cases, \(\underline{\operatorname{Pic}}_{X/S}^\sigma\) (and usually, it seems, even \(\underline{\operatorname{Pic}}_{X/S}^\tau\)) is universally open over \(S\).

\end{rmenv}

\oldpage{236-09}Finally, here is a useful case where, exceptionally, \(G^0\) agrees to be open:

\leavevmode\vadjust pre{\hypertarget{fga-3-vi-corollary-1.7}{}}%
\begin{itenv}{Corollary 1.7}
Suppose that \(G\to S\) is universally open at the points of \(G^0\) (cf.~\protect\hyperlink{fga-3-vi-corollary-1.5}{(1.5)}) and that the fibres \(G_s\) are separable, thus simple over \(k(s)\) (this latter condition being automatically satisfied in residual characteristic zero, by a result of Cartier).
Then \(G^0\) is open in \(G\).
If, further, \(S\) is reduced, then \(G^0\) is simple (and in particular, flat) over \(S\).

\end{itenv}

\begin{proof}
The first claim can be made more precise by noting that if, for some given \(s\in S\), we have that \(G\to S\) is universally open at the points of \(G_s^0\), and if \(G_s^0\) is separable over \(k(s)\), then \(G^0\) is a neighbourhood of \(G_s^0\) (and it so happens that the hypotheses made on \(s\) remain satisfied by all neighbouring points).
A proof of this statement, which is independent of all the structure of the group, can be found in {[}\protect\hyperlink{ref-GD1960}{10}, IV, §7{]}.
The last claim in \protect\hyperlink{fga-3-vi-corollary-1.7}{(1.7)}, also independent of group structure and of any question of connected components, is a particular case of a flatness criteria given in {[}\protect\hyperlink{ref-Gro1960b}{9}{]}, which implies, more generally:
\end{proof}

\leavevmode\vadjust pre{\hypertarget{fga-3-vi-corollary-1.8}{}}%
\begin{itenv}{Corollary 1.8}
Let \(U\) be an open subset of a fibre \(G_s\) such that \(G\to S\) is universally open at the points of \(U\) (cf.~\protect\hyperlink{fga-3-vi-corollary-1.5}{(1.5)}).
If \(G_s\) is separable over \(k(s)\) and \(S\) is reduced at \(s\), then \(G\) is flat over \(S\) at the points of \(U\) (and then also simple over \(S\) at the points of \(U\)).

\end{itenv}

\leavevmode\vadjust pre{\hypertarget{fga-3-vi-remarks-1.9}{}}%
\begin{rmenv}{Remarks 1.9}
I do not know, if \(G\) is separated over \(S\), if \(G^0\) or \(G^\tau\) is always closed in \(G\);
this seems unlikely.
In any case, we find evident counter-examples if we drop the separation hypothesis, for example by taking the affine line with a countably infinite number of copies of the origin, thus obtaining a group scheme \(G\) over the affine line whose fibres are all the trivial group, apart from one, which is \(\mathbb{Z}\).
This group scheme is an open group subscheme, the closure of the identity section, of the Picard prescheme of the \(S\)-scheme \(X\) that corresponds to a family of conics degenerating to two concurrent lines.

Even starting with a flat and finite group scheme \(G\) over the spectrum \(S\) of a discrete valuation ring \(V\), if \(G^0\) is reduced at the identity section, and thus closed, then various conclusions become false if we drop certain hypotheses.
Suppose that \(V\) is of equal characteristic \(p>0\), and let \(G\) be the kernel of the homomorphism \(\operatorname{G_a}\to\operatorname{G_a}\) given by the homomorphism of functors \(f\mapsto f^p-tf\), where \(t\) is a uniformiser of \(V\).
\oldpage{236-10}(Recall that, by definition, the ``additive group'' \(\operatorname{G_a}\) over \(S\) represents the functor \(S'\mapsto\Gamma(S',{\mathscr{O}}_{S'})\)).
Then, as an \(S\)-scheme, \(G=\operatorname{Spec}V[x]/(x^p-tx)\) is the union of \(p\) concurrent ``lines'', giving a group \(\mathbb{Z}/p\mathbb{Z}\) that degenerates to an infinitesimal additive-type group.
In this example, \(G^0=G^\sigma\) (and this is one of the \(p\) lines), and so \(G^0\) is not open in \(G\).
In the case of unequal characteristic, with residual characteristic \(p>0\), we start with a group scheme \(H=\mu_p\) given by the kernel of the \(p\)-th power in \(\operatorname{G_m}\), so that \(H=\operatorname{Spec}V[x]/(x^p-1)\), which is again the union of \(p\) concurrent lines, giving a group \(\mathbb{Z}/p\mathbb{Z}\) (in characteristic \(0\)) that degenerates to an infinitesimal multiplicative-type group in characteristic \(p\).
Let \(H'\) be the ``constant group scheme'' defined by the ordinary finite group \(\mathbb{Z}/p\mathbb{Z}\), which is the disjoint sum of \(\mathbb{Z}/p\mathbb{Z}\) copies of \(S\), or even \(\operatorname{Spec}V^{\mathbb{Z}/p\mathbb{Z}}\).
Then \(G=H\times_S H'\) describes a group \((\mathbb{Z}/p\mathbb{Z})^2\) in characteristic \(0\) that degenerates to an infinitesimal group times \(\mathbb{Z}/p\mathbb{Z}\) in characteristic \(p\).
Here \(G^\rho\) is the union of the identity section and the special fibre, and so \(G^\rho\) is not open, even though \(G^\sigma\) is the union of the identity section and the general fibre, and is thus not closed, contrary to what is true in the case of equal characteristic in \protect\hyperlink{fga-3-vi-theorem-1.1}{(1.1), (iii) and (iv)}.
Of course, these are phenomena linked to characteristic \(p>0\).
The above results give:

\end{rmenv}

\leavevmode\vadjust pre{\hypertarget{fga-3-vi-corollary-1.10}{}}%
\begin{itenv}{Corollary 1.10}
Suppose that the residual characteristics of \(S\) are all \(0\), so that \(G^\tau=G^\sigma\) and \(G^0=G^\rho\).
Then \(G^\tau=G^\sigma\) is open, and even open and closed if \(G\) is separated over \(S\) and the \(G_s^0\) are proper;
under these same hypotheses, \(G^0=G^\rho\) is proper over \(S\) and thus closed in \(G\).
Suppose finally that, for every integer \(n>1\), the \(n\)-th power homomorphism in \(G\) is universally open, then \(G^0\) is open;
if furthermore the \(G_s^0\) do not have any additive component (for example the \(G_s^0\) are proper, as above), then \(G^\tau\to S\) is universally open, and even simple if \(S\) is reduced.

\end{itenv}

Finally we note the following easy result:

\leavevmode\vadjust pre{\hypertarget{fga-3-vi-proposition-1.11}{}}%
\begin{itenv}{Proposition 1.11}
There exists an open subset \(U\) of \(S\) such that the set \(G'\) of points of \(G\) at which \(G\) is simple (resp. flat) over \(S\) is the underlying set of an open group subscheme induced by \(G|U\).
Furthermore, every section of \(G\) over \(U\) is a section of \(G'\) over \(U\).

\end{itenv}

\leavevmode\vadjust pre{\hypertarget{fga-3-vi-corollary-1.12}{}}%
\begin{itenv}{Corollary 1.12}
\oldpage{236-11}If \(G\) is simple (resp. flat) over \(S\) at the points of the identity section, then it is simple (resp. flat) at the points of every section of \(G\) over \(S\), and at all points of \(G^0\).
If, further, for every integer \(n>0\) the \(n\)-th power homomorphism \(\varphi_n\colon G\to G\) is étale at the points of characteristic coprime to \(n\), then \(G\) is simple (resp. flat) over \(S\) at all points of \(G^\sigma\).

\end{itenv}

\hypertarget{fga-3-vi-section-2}{%
\subsection{Application to the local properties of Picard schemes}\label{fga-3-vi-section-2}}

\hypertarget{fga-3-vi-theorem-2.1}{}
\begin{itenv}{Theorem 2.1}

---

\begin{enumerate}
\def\labelenumi{\roman{enumi}.}
\tightlist
\item
  Let \(f\colon X\to S\) be a proper and simple morphism, and suppose that \(\underline{\operatorname{Pic}}_{X/S}\) exists (for example, if \(f\) is projective).
  Then \(\underline{\operatorname{Pic}}_{X/S}\) is separated over \(S\), and, for every closed subset \(Z\) of \(\underline{\operatorname{Pic}}_{X/S}\) that is of finite type over \(S\), we have that \(Z\) is proper over \(S\).
\item
  Let \(X\) be a prescheme over a field \(k\) that is proper and geometrically normal.
  Then \(\underline{\operatorname{Pic}}_{X/S}^0\) is proper over \(k\).
\end{enumerate}

\end{itenv}

\begin{proof}
For (i), with the valuative criteria of {[}\protect\hyperlink{ref-GD1960}{10}, II, §7{]} it suffices to prove the following: if \(S\) is the spectrum of a complete discrete valuation ring, and \(U\) the open consisting of the generic point of \(S\), then every rational section of \(\underline{\operatorname{Pic}}_{X/S}\) over \(S\), i.e.~every section over \(U\), extends uniquely to a section over \(S\).
Taking into account the definition of \(\underline{\operatorname{Pic}}_{X/S}\), this is equivalent to the following statement: for every invertible module \({\mathscr{L}}\) on \(V=f^{-1}(U)\), there exists an invertible module on \(X\) that extends \({\mathscr{L}}\), unique up to isomorphism.
But this follows easily from the description of invertible modules on \(V\) (resp. on \(X\)) in terms of the classes of ``Cartier'' divisors, taking into account the fact that the local rings of \(X\) are regular (since \(X\) is simple over \(S\), which is regular), and thus factorial, by Auslander, which implies that every divisor on \(S\) is a Cartier divisor.
Indeed, every divisor on \(V\) can be extended to a divisor on \(X\) by taking its ``closure''.

\oldpage{236-12}For (ii), using Chow's lemma we can reduce to the case where \(X\) is projective, and thus embedded into some \(\mathbb{P}_k^n\);
we can also assume that \(X\) is connected.
If \(\dim X=1\), then \(X\) is simple over \(k\), and we can apply (i).
If \(\dim X\geqslant 2\), then we can use a variant of the known ``equivalence criteria'', which implies that there exists a finite number of curves \(Y_i\), simple over \(X\) (obtained as intersections of \(X\) with suitable linear subspaces of \(\mathbb{P}_k^n\)), such that \(\underline{\operatorname{Pic}}_{X/k}^\tau\to\prod_i\underline{\operatorname{Pic}}_{Y_i/k}^\tau\) is a monomorphism, and induces a fortiori a monomorphism for the connected components.
Since the codomain is proper over \(k\) by the above, and since we are talking about a homomorphism of group schemes, which is necessarily a closed immersion, it follows that \(\underline{\operatorname{Pic}}_{X/k}^0\) is also proper over \(k\).
We can avoid recourse to delicate equivalence criteria by using the structure of commutative algebraic groups over an algebraically closed field (thanks to Chevalley--Borel);
we are then reduced to proving that every morphism from the affine line with the origin removed into \(\underline{\operatorname{Pic}}_{X/S}^\tau\) is constant, which is equivalent to saying that every invertible module on \(X[t,t^{-1}]\) comes from an invertible module on \(X\), which is a result that is well known and elementary and does not even use the fact that \(X\) is proper over \(k\) (the hypothesis that \(X\) is normal allowing us to immediately reduce to the case where \(X\) is regular).
\end{proof}

\leavevmode\vadjust pre{\hypertarget{fga-3-vi-remark-2.2}{}}%
\begin{rmenv}{Remark 2.2}
The above proof of (i) holds true even if we only suppose that \(f\) is flat and that its fibres \(X_s\) are locally complete intersections and simple over \(k(s)\) in codimension \(\leqslant 2\), taking into account the following fact which is proven in {[}\protect\hyperlink{ref-Gro1960b}{9}{]}: a Noetherian complete intersection local ring that is regular in codimension \(\leqslant 3\) is factorial (``Samuel's conjecture'').
We note that the result becomes false if we replace ``codimension \(\leqslant 2\)'' by ``codimension \(\leqslant 1\)'', i.e.~by the hypothesis ``normal'', as we can convince ourself by considering the example of a family of non-singular quadratics that degenerate to a quadratic cone.

\end{rmenv}

\leavevmode\vadjust pre{\hypertarget{fga-3-vi-corollary-2.3}{}}%
\begin{itenv}{Corollary 2.3}
Let \(f\colon X\to S\) be a proper and normal morphism (i.e.~flat with normal geometric fibres), and suppose that \(\underline{\operatorname{Pic}}_{X/S}\) exists.
Then \(\underline{\operatorname{Pic}}_{X/S}^0\) is proper over \(S\), thus closed in \(\underline{\operatorname{Pic}}_{X/S}\);
furthermore, \(\underline{\operatorname{Pic}}_{X/S}^\tau\) and \(\underline{\operatorname{Pic}}_{X/S}^0\) are both closed, as is \(\underline{\operatorname{Pic}}_{X/S}^\sigma\) in ``equal characteristic''.

\end{itenv}

\begin{proof}
We apply \protect\hyperlink{fga-3-vi-theorem-1.1}{(1.1)} and \protect\hyperlink{fga-3-vi-theorem-2.1}{(2.1), (ii)}.
\end{proof}

\leavevmode\vadjust pre{\hypertarget{fga-3-vi-corollary-2.4}{}}%
\begin{itenv}{Corollary 2.4}
Let \(f\colon X\to S\) be a proper and simple morphism such that \(\underline{\operatorname{Pic}}_{X/S}\) exists and is the sum of the schemes \(P^{(i)}\) of finite type over \(S\) (cf.~\protect\hyperlink{fga-3-v-proposition-4.1}{FGA 3.V, Proposition 4.1}).
Then each \(P^{(i)}\) is proper over \(S\).

\end{itenv}

\begin{proof}
This follows from \protect\hyperlink{fga-3-vi-theorem-2.1}{(2.1), (i)}.
\end{proof}

As we noted in \protect\hyperlink{fga-3-vi-remark-2.2}{(2.2)}, the above result can be generalised by making less restrictive hypotheses on the fibres of \(f\), but becomes false if we only suppose \(f\) to be normal.
In this case, I do not know if \(\underline{\operatorname{Pic}}_{X/S}^\tau\) is nevertheless proper over \(S\), even if assuming it to be of finite type over \(S\).

\leavevmode\vadjust pre{\hypertarget{fga-3-vi-theorem-2.5}{}}%
\begin{itenv}{Theorem 2.5}
\oldpage{236-13}Let \(f\colon X\to S\) be a proper and flat morphism such that \(\underline{\operatorname{Pic}}_{X/S}\) exists, and, for each integer \(n\), let \(\varphi_n\) be the \(n\)-th power homomorphism in this group prescheme.
Then \(\varphi_n\) is étale at all points \(x\in X\) of residual characteristic coprime to \(n\).

\end{itenv}

By the infinitesimal characterisation of étale morphisms, the above claim is equivalent to the following:

\leavevmode\vadjust pre{\hypertarget{fga-3-vi-lemma-2.6}{}}%
\begin{itenv}{Lemma 2.6}
Suppose that \(S\) is the spectrum of an Artinian local ring \(A\) whose maximal ideal \({\mathfrak{m}}\) is \((\nu+1)\)-th power null, and let \(A_{\nu-1}=A/{\mathfrak{m}}^\nu\) and \(X_{\nu-1}=X\otimes_A A_{\nu=1}\).
Let \({\mathscr{L}}\) be an invertible module on \(X\), and \({\mathscr{L}}'_{\nu-1}\) an invertible module on \(X_{\nu-1}\) whose \(n\)-th tensor power is isomorphic to \({\mathscr{L}}_{\nu-1}={\mathscr{L}}\otimes_A A_{\nu-1}\).
Then there exists an invertible module \({\mathscr{L}}'\) on \(X\) whose \(n\)-th tensor power is isomorphic to \({\mathscr{L}}\) (if \(n\) is coprime to the residual characteristic of \(k=k({\mathscr{L}})\)).

\end{itenv}

\begin{proof}
Set \(V={\mathfrak{m}}^\nu={\mathfrak{m}}^\nu/{\mathfrak{m}}^{\nu+1}\), which is a vector space over \(k=k(A)\).
We start by extending \({\mathscr{L}}'_{\nu-1}\) to an arbitrary invertible module over \({\mathscr{L}}'\) on \(X\).
The obstruction to doing this is found in \(\operatorname{H}^2(X_0,{\mathscr{O}}_{X_0})\otimes_k V\), but by the hypothesis on \({\mathscr{L}}'_{\nu-1}\) and the fact that \({\mathscr{L}}_{\nu-1}\) can be extended, we see that the product of this obstruction with \(n\) is zero, and so the obstruction itself must be zero since \(n\) is coprime to the characteristic.
The arbitrariness of the extension is found in \(\operatorname{H}^1(X_0,{\mathscr{O}}_{X_0})\otimes_k V\), and the deviation \(\xi\) of \({{\mathscr{L}}'}^{\otimes n}\) from \({\mathscr{L}}\) is found in the same module;
if we try to correct \({\mathscr{L}}'\) in such a way as to render this deviation zero, then we are led to finding some \(\eta\) in the aforementioned module such that \(n\eta=\xi\).
But this is again possible thanks to the fact that \(n\) is coprime to the characteristic.
\end{proof}

\leavevmode\vadjust pre{\hypertarget{fga-3-vi-corollary}{}}%
\begin{itenv}{Corollary}
Under the conditions of \protect\hyperlink{fga-3-vi-theorem-2.5}{(2.5)}, suppose further that the Picard schemes of the fibres \(X_s\) do not contain any additive component (for example, if the \(X_s\) are geometrically normal, cf.~\protect\hyperlink{fga-3-vi-theorem-2.1}{(2.1), (ii)}).
Then \(\underline{\operatorname{Pic}}_{X/S}\to S\) is universally open at the points of \(\underline{\operatorname{Pic}}_{X/S}^\sigma\).
If \(\underline{\operatorname{Pic}}_{X/S}^0\) is closed (for example, if the \(X_s\) are geometrically normal, cf.~\protect\hyperlink{fga-3-vi-corollary-2.3}{(2.3)}), then \(\underline{\operatorname{Pic}}_{X/S}^\sigma\) is itself universally open over \(S\).
Finally, in the case of equal characteristic, \(\underline{\operatorname{Pic}}_{X/S}^\rho\to S\) is universally open.

\end{itenv}

\begin{proof}
We apply \protect\hyperlink{fga-3-vi-corollary-1.5}{(1.5)} and \protect\hyperlink{fga-3-vi-theorem-1.1}{(1.1)}.
\end{proof}

\leavevmode\vadjust pre{\hypertarget{fga-3-vi-corollary-2.7}{}}%
\begin{itenv}{Corollary 2.7}
\oldpage{236-14}Let \(f\colon X\to Y\) be a proper and flat morphism such that \(\underline{\operatorname{Pic}}_{X/S}\) exists.
Then the function \(s\mapsto\dim\underline{\operatorname{Pic}}_{X_s/k(s)}\) on \(S\) is upper semi-continuous (i.e.~it can jump upwards, but not downwards), and it is even continuous (i.e.~locally constant) if the \(\underline{\operatorname{Pic}}_{X_s/k(s)}\) do not contain any additive component.

\end{itenv}

\begin{proof}
The first claim is trivially true, or almost so, for every group prescheme locally of finite type over a locally Noetherian base, since it suffices to look along the identity section.
The second claim follows from \protect\hyperlink{fga-3-vi-theorem-2.5}{(2.5)}.
\end{proof}

\leavevmode\vadjust pre{\hypertarget{fga-3-vi-remark-2.8}{}}%
\begin{rmenv}{Remark 2.8}
Let \(s,s'\in S\) be such that \(s\) is a specialisation of \(s'\).
Then \protect\hyperlink{fga-3-vi-corollary-2.7}{(2.7)} is equivalent to an inequality (resp. equality) between the dimensions of the Picard schemes of \(X_{s'}\) and of its ``specialisation'' \(X_s\).
Serre noted, even before the construction of Picard schemes, that the invariance of the dimensions of the Picard varieties of the \(X_s\) in the case of a \emph{simple} morphism \(f\colon X\to S\) was a formal consequence of the theory of specialisation of the fundamental group ({[}\protect\hyperlink{ref-Gro1960b}{9}, X{]}), classical relations à la Kummer between the points of finite order on the Picard variety, and the abelianisation of the fundamental group ({[}\protect\hyperlink{ref-Gro1960b}{9}, XI{]}).
If we denote by \(\alpha\), \(\mu\), and \(\lambda\) the dimensions of the abelian, multiplicative, and additive parts (respectively) of \(\underline{\operatorname{Pic}}_{X_s/k(s)}\), and we similarly define \(\alpha'\), \(\mu'\), and \(\lambda'\), then the known relations can be expressed as the following inequalities:
\[
  \alpha+\mu+\lambda
  \geqslant\alpha'+\mu'+\lambda'
\tag{$\ast$}
\]
(satisfied provided that \(\underline{\operatorname{Pic}}_{X/S}\) exists, and thus probably in all cases), and this inequality, for \(\lambda=\lambda'=0\), reduces to an equality, satisfied under the same existence hypotheses:
\[
  \alpha+\mu
  = \alpha'+\mu'.
\]
We also have
\[
  2\alpha+\mu
  \leqslant 2\alpha'+\mu'
\tag{$\ast\ast$}
\]
\oldpage{236-15}by the argument of Serre, if the \(X_s\) are separable (without even supposing the existence of \(\underline{\operatorname{Pic}}_{X/S}^\tau\)), or if the \({}_n\underline{\operatorname{Pic}}_{X/S}\) (kernels of the \(\varphi_n\) into the Picard functor) are \emph{separated} over \(S\) (taking into account the fact that they are étale over \(S\), thanks to \protect\hyperlink{fga-3-vi-theorem-2.5}{(2.5)}).
We are inclined to conjecture that (\(\ast\)) is an equality in all cases, or at least if the \(X_s\) are separable, and also that we have inequalities
\[
  \begin{aligned}
    \alpha &\leqslant\alpha'
  \\\lambda &\geqslant\lambda'
  \end{aligned}
\tag{$\ast\ast\ast$}
\]
which should be satisfied whenever we have a group prescheme that is locally of finite type over locally Noetherian \(S\), in which the dimension of the fibres is constant (see \protect\hyperlink{fga-3-vi-lemma-1.3}{(1.3)} for a positive result in this direction).

\end{rmenv}

\leavevmode\vadjust pre{\hypertarget{fga-3-vi-remark-2.9}{}}%
\begin{rmenv}{Remark 2.9}
In all known cases, \(\underline{\operatorname{Pic}}_{X/S}^\tau\) is universally open over \(S\), but we should probably not have excessive illusions, even if \(f\colon X\to S\) is simple;
in any case, Mumford has constructed an example (it is true with \(S\) non-reduced, in fact with \(S\) the spectrum of an Artinian ring) where \(\underline{\operatorname{Pic}}_{X/S}^\tau\) is not flat over \(S\), by infinitesimally varying the Igusa surface.
The point envisaged by Mumford can also be found in \(\underline{\operatorname{Pic}}_{X/S}^\rho\), and it remains possible (for \(f\colon X\to S\) simple) that \(\underline{\operatorname{Pic}}_{X/S}\) is flat over \(S\) at the points of \(\underline{\operatorname{Pic}}_{X/S}^\sigma\);
the speaker doubts, however, that this is always the case, even when restricting to the points of \(\underline{\operatorname{Pic}}_{X/S}^0\) and supposing \(S\) to be the spectrum of a discrete valuation ring.
The question is linked to the study of fixed points of an abelian scheme under a finite automorphism group, a situation for which we seem to lack examples.
It seems that even by restricting to simple and projective \(f\colon X\to S\), the results of local regularity on \(\underline{\operatorname{Pic}}_{X/S}\) stated in the present section, and the conjectures raised in \protect\hyperlink{fga-3-vi-remark-2.8}{(2.8)}, basically exhaust what can be said on this subject without more particular hypotheses on the nature of the fibres of \(f\).
We recall, however, that, if the geometric fibres of \(\underline{\operatorname{Pic}}_{X/S}\) are \emph{reduced} and have no additive component, then it follows from \protect\hyperlink{fga-3-vi-corollary-1.8}{(1.8)} and \protect\hyperlink{fga-3-vi-theorem-2.5}{(2.5)} that \(\underline{\operatorname{Pic}}_{X/S}\) is simple over \(S\) at the points of \(\underline{\operatorname{Pic}}_{X/S}^\sigma\) whenever \(S\) is reduced;
this result still holds even, if \(f\colon X\to S\) is normal, without any hypothesis on \(S\), and we will see in \protect\hyperlink{fga-3-vi-theorem-3.5}{(3.5)}.
On this note, we point out:

\end{rmenv}

\hypertarget{fga-3-vi-proposition-2.10}{}
\begin{itenv}{Proposition 2.10}

---

\begin{enumerate}
\def\labelenumi{\roman{enumi}.}
\tightlist
\item
  If \(\underline{\operatorname{Pic}}_{X/S}\) is simple (resp. flat) over \(S\) at the points of the identity section, then it is simple (resp. flat) at all points of \(\underline{\operatorname{Pic}}_{X/S}^\sigma\), and at the points of every section of \(\underline{\operatorname{Pic}}_{X/S}\) over \(S\).
\item
  \oldpage{236-16}Let \(s\in S\) be such that \(\operatorname{H}^2(X_s,{\mathscr{O}}_{X_s})=0\).
  Then there exists an open neighbourhood \(U\) of \(s\) such that \(\underline{\operatorname{Pic}}_{X/S}|U\) is simple over \(U\).
\item
  Let \(X\) be a proper scheme over a field.
  Then we have
  \[
      \dim\underline{\operatorname{Pic}}_{X/k}
      \leqslant\dim\operatorname{H}^1(X,{\mathscr{O}}_X)
    \]
  with equality if and only if \(\underline{\operatorname{Pic}}_{X/k}\) is simple over \(k\);
  this is always the case if \(k\) is of characteristic \(0\).
\end{enumerate}

\end{itenv}

\begin{proof}
Claim (i) follows from \protect\hyperlink{fga-3-vi-theorem-2.5}{(2.5)} and \protect\hyperlink{fga-3-vi-corollary-1.12}{(1.12)}, and (ii) follows from the infinitesimal criteria for simple morphisms and a well-known obstruction calculation, taking into account the fact that (by the ``semi-continuous theorem'') the hypothesis made on \(s\) will still hold true for neighbouring points.
Finally, (iii) follows from the fact that \(\operatorname{H}^1(X,{\mathscr{O}}_X)\) is isomorphic to the Zariski tangent space at the identity element of \(\underline{\operatorname{Pic}}_{X/k}\);
the last claim is a particular case of a theorem of Cartier, saying that a ``formal group'' in characteristic \(0\) is formally simple over \(k\).
\end{proof}

\hypertarget{fga-3-vi-section-3}{%
\subsection{\texorpdfstring{The canonical abelian subscheme of \(\underline{\operatorname{Pic}}_{X/S}\), and the Albanese scheme}{The canonical abelian subscheme of \textbackslash underline\{\textbackslash operatorname\{Pic\}\}\_\{X/S\}, and the Albanese scheme}}\label{fga-3-vi-section-3}}

\leavevmode\vadjust pre{\hypertarget{fga-3-vi-proposition-3.1}{}}%
\begin{itenv}{Proposition 3.1}
Let \(k\) be a field, and let \(G\) a group scheme of finite type over \(k\) that is commutative and ``without additive component''.
Then \(G_\mathrm{red}^0\) is separable over \(k\), and thus a simple group scheme over \(k\)\textgreater{}

\end{itenv}

\begin{proof}
Since the claim is trivial if \(k\) is perfect, and in particular for \(G_{\overline{k}}\), where \(\overline{k}\) is the algebraic closure of \(k\), it suffices to show that \((G_{\overline{k}}^0)_\mathrm{red}\) comes from a subscheme of \(G\).
But from the hypothesis that \(G_{\overline{k}}\) contains no additive component it easily follows that there exists an integer \(m\) such that \((G_{\overline{k}}^0)_\mathrm{red}\) is the ``scheme-theoretic'' image of the \(m\)-th power homomorphism in \(G_{\overline{k}}\).
Since the latter homomorphism comes from the analogous homomorphism for \(G\), the scheme-theoretic image of this provides the desired object.
\end{proof}

\leavevmode\vadjust pre{\hypertarget{fga-3-vi-corollary-3.2}{}}%
\begin{itenv}{Corollary 3.2}
Let \(X\) be a normal and proper scheme over \(k\) such that \(\underline{\operatorname{Pic}}_{X/k}\) exists.
Then there exists an abelian subscheme of \(\underline{\operatorname{Pic}}_{X/k}\) whose underlying set is \(\underline{\operatorname{Pic}}_{X/k}^0\).

\end{itenv}

\begin{proof}
By \protect\hyperlink{fga-3-vi-theorem-2.1}{(2.1), (ii)}, since \(\underline{\operatorname{Pic}}_{X/k}^0\) is proper over \(k\), it satisfies the conditions of \protect\hyperlink{fga-3-vi-proposition-3.1}{(3.1)}.
\end{proof}

\oldpage{236-17}The above result shows that, in certain cases, the classical ``Picard variety'' (which is \((\underline{\operatorname{Pic}}_{X/\overline{k}})_\mathrm{red}^0\) in the current theory) ``is defined over \(k\)'', without supposing the field \(k\) to be perfect.

Now let \(f\colon X\to S\) be a proper and flat relative scheme, with \({\mathscr{O}}_S\xrightarrow{\sim}f_*({\mathscr{O}}_X)\) for simplicity, such that \(\underline{\operatorname{Pic}}_{X/S}\) exists and that \(\underline{\operatorname{Pic}}_{X/S}^0\) is proper over \(S\).
Suppose further, for \protect\hyperlink{fga-3-vi-theorem-3.3}{(3.3), (ii)}, that there exists an open of \(\underline{\operatorname{Pic}}_{X/S}\) containing \(\underline{\operatorname{Pic}}_{X/S}^0\) that is quasi-projective over \(S\);
this condition is satisfied, as we have seen, if \(f\) is projective and with separable or irreducible geometric fibres.
Recall that an \emph{abelian scheme} over \(S\) is a group scheme over \(S\) that is proper and simple over \(S\) with connected geometric fibres.
We propose to examine whether or not there exists a group subscheme \(A\) of \(\underline{\operatorname{Pic}}_{X/S}\) that is an abelian scheme and whose underlying set is \(\underline{\operatorname{Pic}}_{X/S}^0\).
We have just seen that such an \(A\) always exists if \(S\) is the spectrum of a field.
Here is what we know how to say in the general case envisaged here:

\hypertarget{fga-3-vi-theorem-3.3}{}
\begin{itenv}{Theorem 3.3}

Under the above conditions:

\begin{enumerate}
\def\labelenumi{\roman{enumi}.}
\tightlist
\item
  If there exists an abelian subscheme of \(\underline{\operatorname{Pic}}_{X/S}\) whose underlying set is \(\underline{\operatorname{Pic}}_{X/S}^0\), then it is unique.
  Its formation is thus compatible with base change.
\item
  For there to exist such an abelian subscheme, it is necessary and sufficient that it exist after every base change \(S'\to S\), where \(S'\) is local Artinian;
  if \(S\) is the spectrum of a local ring, it even suffices to test with the \(S'=\operatorname{Spec}(A_n)\) where \(A_n=A/{\mathfrak{m}}^{n+1}\).
  If \(S\) is reduced, then it equally suffices to test with the \(S'\) that are the spectrum of a discrete valuation ring (complete, with algebraically closed residue field, if one desires).
\item
  Suppose that \(A\) exists, and let \(B=\operatorname{Alb}^0(X/S)\) be the dual abelian scheme (i.e.~\(B=\operatorname{Pic}_{A/S}^0\) {[}Mumford--Tate seminar, 1962{]}).
  Then we can canonically construct a principal homogeneous space \(P=\operatorname{Alb}^1(X/S)\) for \(B\), and an \(S\)-morphism \(X\to P\) that is universal for the \(S\)-morphisms from \(X\) to para-abelian schemes (i.e.~to principal homogeneous spaces for abelian schemes).
  The formation of \(\operatorname{Alb}^0(X/S)\), \(\operatorname{Alb}^1(X/S)\), and \(X\to\operatorname{Alb}^1(X/S)\) commutes with base change.
\end{enumerate}

\end{itenv}

\begin{proof}

We sketch the proof:

\begin{enumerate}
\def\labelenumi{\roman{enumi}.}
\tightlist
\item
  This is a general property of rigidity for abelian subschemes of commutative group schemes: if two such subschemes agree set-theoretically at a point \(s\in S\), then they agree over the entire connected component of \(s\) ({[}Mumford--Tate seminar, 1962{]}).
  (This result generalises a classical theorem of Chow).
\item
  Using Hilbert schemes, we see that the functor that, to every \(S'\) over \(S\) associates the set (consisting of either one or zero elements) of canonical abelian subschemes of \((\underline{\operatorname{Pic}}_{X/S})\times_S S'\) is representable by a scheme \(T\) of finite type over \(S\).
  By (i), \(T\to S\) is a monomorphism, and by \protect\hyperlink{fga-3-vi-corollary-3.2}{(3.2)}, it is surjective.
  To say that there exists a canonical abelian subscheme of \(\underline{\operatorname{Pic}}_{X/S}\) implies that \(T\) is a section over \(S\), or that \(T\to S\) is an isomorphism.
  This is equivalent to saying that \(T\to S\) is étale, or, in the case where \(S\) is reduced, that \(T\to S\) is proper.
  Whence immediately (ii).
\item
  Simply using the definition of \(\underline{\operatorname{Pic}}_{X/S}\), we note that, for every abelian scheme \(C\) over \(S\), the data of an \(S\)-morphism from \(X\) to a principal homogeneous space for \(C\) is equivalent to the data of a group homomorphism \(C'\to\underline{\operatorname{Pic}}_{X/S}\), where \(C'\) is the dual abelian scheme of \(C\).
  But if the canonical abelian subscheme \(A\) of \(\underline{\operatorname{Pic}}_{X/S}\) exists, then these homomorphisms necessarily factor through \(A\) (and we can see by again using the points of finite order).
  Whence immediately (iii).
\end{enumerate}

\end{proof}

\leavevmode\vadjust pre{\hypertarget{fga-3-vi-remarks-3.4}{}}%
\begin{rmenv}{Remarks 3.4}
We denote by \(\underline{\operatorname{Pic}}_{X/S}^{00}\) the canonical abelian subscheme of \(\underline{\operatorname{Pic}}_{X/S}\), if it exists.
This is, unfortunately, not always the case, as we can see by infinitesimally varying the Igusa surface (by first-order modular deformation).
It is however possible that \(\underline{\operatorname{Pic}}_{X/S}^{00}\) exists at least if \(S\) is reduced, or, equivalently, by (ii), if \(S\) is the spectrum of a discrete valuation ring.
So let \(X_0\) and \(X_1\) be the special and generic fibres of \(X\) (respectively), and let \(A_1=\underline{\operatorname{Pic}}_{X_1/K_1}\), where \(K\) is the field of fractions of the valuation ring \(V\).
By Koizumi, there exists an abelian scheme \(A\) over \(S\), essentially unique, whose general fibre is \(A_1\), and we easily see as in \protect\hyperlink{fga-3-vi-theorem-2.1}{(2.1), (i)} (supposing from now on that \(X\) is simple over \(S\)) that the identity morphism of \(A_1\) extends to a morphism
\[
  A \to \underline{\operatorname{Pic}}_{X/S}.
\]
From this, we obtain a homomorphism
\[
  A_0 \to \underline{\operatorname{Pic}}_{X_0/k}^{00}
\tag{$\ast$}
\]
which we can easily show to be a surjective homomorphism with kernel equal to a finite \(p\)-group, where \(p\) is the characteristic of the residue field \(k\) (still by using the points of finite order).
\oldpage{236-19}With this, the following conditions on \(X/S\) are equivalent:

\begin{enumerate}
\def\labelenumi{\alph{enumi}.}
\tightlist
\item
  The above homomorphism (\(\ast\)) is an isomorphism (which Shimura expresses by saying that the formation of the ``Picard variety'' is ``compatible with specialisations'').
\item
  \(\underline{\operatorname{Pic}}_{X/S}^{00}\) exists (and is then exactly \(A\)).
\item
  (For memory) The \(\underline{\operatorname{Pic}}_{X_n/S}^{00}\) exist.
\end{enumerate}

By the remark that we made concerning the kernel of (\(\ast\)), condition (a) is satisfied if the residual characteristic is zero, but this result will be notably generalised in \protect\hyperlink{fga-3-vi-theorem-3.5}{(3.5)}.

Of course, if \(\underline{\operatorname{Pic}}_{X/S}\) is simple over \(S\) at the points of \(\underline{\operatorname{Pic}}_{X/S}^0\), then the latter is open in \(\underline{\operatorname{Pic}}_{X/S}\) (cf.~\protect\hyperlink{fga-3-vi-corollary-1.7}{(1.7)}) and is thus, endowed with the induced structure, an abelian subscheme of \(\underline{\operatorname{Pic}}_{X/S}\), and thus equal to \(\underline{\operatorname{Pic}}_{X/S}^{00}\), which exists in this case.
But we have much better:

\end{rmenv}

\leavevmode\vadjust pre{\hypertarget{fga-3-vi-theorem-3.5}{}}%
\begin{itenv}{Theorem 3.5}
Under the conditions of \protect\hyperlink{fga-3-vi-theorem-3.3}{(3.3)}, let \(s\in S\) be such that \(\underline{\operatorname{Pic}}_{X_s/k(s)}\) is simple over \(k(s)\) (or, equivalently, such that \(\dim\underline{\operatorname{Pic}}_{X_s/k(s)}=\dim\operatorname{H}^1(X_s,{\mathscr{O}}_{X_s})\)).
Then there exists an open neighbourhood \(U\) of \(s\) such that \(\underline{\operatorname{Pic}}_{X/S}\) is simple over \(S\) at the points of \(\underline{\operatorname{Pic}}_{X/S}^0|U\), which is thus an open abelian subscheme in \(\underline{\operatorname{Pic}}_{X/S}|U\).
A fortiori, \(\underline{\operatorname{Pic}}_{X|U/U}^{00}\) exists.

\end{itenv}

\begin{proof}
We describe the principle of the proof.
The above allows us to reduce to the case where \(S\) is the spectrum of an Artinian local ring \(A\), and we argue by induction on the infinitesimal order of \(A\).
We can thus suppose that \(\underline{\operatorname{Pic}}_{X_n/A_n}^0\) is simple over \(A_n\), and reduce to proving that \(\underline{\operatorname{Pic}}_{X_{n+1}/A_{n+1}}^0\) is simple over \(A_{n+1}\).
Note that, for this, it suffices to construct an abelian scheme \(P_{n+1}\) over \(A_{n+1}\) that extends \(P_n=\underline{\operatorname{Pic}}_{X_n/A_n}^0\), along with an invertible module \({\mathscr{L}}_{n+1}\) on \(X_{n+1}\times_{A_{n+1}}P_{n+1}\) that extends the invertible module \({\mathscr{L}}_n\) on \(X_n\times_{A_n}P_n\) that arises in the definition of the Picard scheme \(\underline{\operatorname{Pic}}_{X_n/A_n}\) as the solution to a universal problem.
(N.B. We can suppose that \(X\) is endowed with a section over \(S\)\ldots).
For this construction, we must use the following key result: \emph{every abelian scheme defined over a quotient of an Artinian local ring can be extended} (in other words, the absolute ``formal scheme of modules'' (\href{FGA-3-II.html}{FGA 3.II}) for an abelian scheme over an algebraically closed field is simple over the ring of Witt vectors over \(k\));
\oldpage{236-20}this result can be obtained by using the general formal properties of the obstruction to lifting, and the group operations.
With this result, we start by extending \(P_n\) arbitrarily to \(P_{n+1}\);
we then find an obstruction to lifting \({\mathscr{L}}_n\), found in \(\operatorname{H}^2(X_0\times P_0,{\mathscr{O}}_{X_0\times P_0})\otimes_k V\) (where \(V={\mathfrak{m}}^{n+1}/{\mathfrak{m}}^{n+2}\)), and more precisely in the subspace \(\operatorname{H}^1(X_0,{\mathscr{O}}_{X_0})\otimes\operatorname{H}^1(P_0,{\mathscr{O}}_{P_0})\otimes_k V\) (taking into account the fact that the restriction of \({\mathscr{L}}_n\) to the two factors \(X_n\) and \(P_n\) is trivial).
But this latter space is exactly \(\operatorname{H}^1(P_0,{\mathscr{G}}_{P_0/k})\otimes V\), where \({\mathscr{G}}_{P_0/k}\) is the tangent sheaf to \(P_0/k\), and thus also the space that expresses the indeterminacy that there was in the lifting of \(P_n\) to \(P_{n+1}\) (\href{FGA-3-II.html}{FGA 3.II}).
So we can correct this lifting (in exactly one way, as should be the case) in such a way as to kill the obstruction to lifting \({\mathscr{L}}_n\).
\end{proof}

\leavevmode\vadjust pre{\hypertarget{fga-3-vi-corollary-3.6}{}}%
\begin{itenv}{Corollary 3.6}
Under the conditions of \protect\hyperlink{fga-3-vi-theorem-3.5}{(3.5)}, \(\operatorname{R}^1 f_*({\mathscr{O}}_X)\) is a locally free module on \(S\) in a neighbourhood of \(s\), and its formation commutes with base change.

\end{itenv}

\begin{proof}
This module is exactly the tangent module to \(\underline{\operatorname{Pic}}_{X/S}\) along the identity section.
\end{proof}

\leavevmode\vadjust pre{\hypertarget{fga-3-vi-remark-3.7}{}}%
\begin{rmenv}{Remark 3.7}
Using the same argument as for \protect\hyperlink{fga-3-vi-theorem-3.5}{(3.5)}, we can show that, if \(S'\) is a subscheme of \(S\) defined by a nilpotent coherent ideal, and if we suppose only that \(\underline{\operatorname{Pic}}_{X'/S'}^0\) exists and has simple fibres, then \(\underline{\operatorname{Pic}}_{X/S}^0\) necessarily exists and is an abelian scheme over \(S\).
This allows us to construct Picard schemes in certain cases, despite the absence of any projective hypothesis;
for example, the dual abelian scheme of an abelian scheme over an Artinian ring always exists.
Using \protect\hyperlink{fga-3-vi-corollary-3.6}{(3.6)} in the case where \(X\) is an abelian scheme over \(S\), and using the known structure of \(\operatorname{H}^\bullet(X_s,{\mathscr{O}}_{X_s})\) as an exterior algebra over \(\operatorname{H}^1(X_s,{\mathscr{O}}_{X_s})\) (Rosenlicht--Serre), we find that \(\operatorname{R}^if_*({\mathscr{O}}_X)\) is locally free for \emph{all} \(i\), and more precisely that it is isomorphic to the \(i\)-th exterior power of \(\operatorname{R}^1f_*({\mathscr{O}}_X)\).

\end{rmenv}

\leavevmode\vadjust pre{\hypertarget{fga-3-vi-remark-3.8}{}}%
\begin{rmenv}{Remark 3.8}
In the case of a simple projective morphism \(f\colon X\to S\), with \(S\) reduced and of residual characteristics zero, the result of \protect\hyperlink{fga-3-vi-corollary-3.6}{(3.6)} was already known, by transcendental methods, as a consequence of Hodge theory.
In fact, all the \(\operatorname{R}^pf_*(\Omega_{X/S}^q)\) are then locally free.
\oldpage{236-21}We have, however, counterexamples in the case of unequal characteristics for ``locally free \(\operatorname{R}^1f_*({\mathscr{O}}_X)\)'', by Serre varieties ({[}\protect\hyperlink{ref-Ser1958b}{23}{]}).
It seems that we do not have any counterexample in equal characteristic.

\end{rmenv}

\leavevmode\vadjust pre{\hypertarget{fga-3-vi-corollary-3.8}{}}%
\begin{itenv}{Corollary 3.8}
Under the conditions of \protect\hyperlink{fga-3-vi-theorem-3.5}{(3.5)}, \(\underline{\operatorname{Pic}}_{X/S}|U\) is simple over \(U\) at all points of \(\underline{\operatorname{Pic}}_{X/S}^\sigma|U\).

\end{itenv}

\begin{proof}
We apply \protect\hyperlink{fga-3-vi-proposition-2.10}{(2.10), (i)}.
\end{proof}

In particular, taking \protect\hyperlink{fga-3-vi-proposition-2.10}{(2.10), (iii)} into account:

\leavevmode\vadjust pre{\hypertarget{fga-3-vi-corollary-3.9}{}}%
\begin{itenv}{Corollary 3.9}
Suppose that \(S\) is of residual characteristics zero.
Then \(\underline{\operatorname{Pic}}_{X/S}^\tau\) is simple over \(S\).

\end{itenv}

\leavevmode\vadjust pre{\hypertarget{fga-3-vi-remark-3.10}{}}%
\begin{rmenv}{Remark 3.10}
We thus deduce, for example, that if \(\underline{\operatorname{Pic}}_{X/S}\) is also \emph{proper} over \(S\), then the Néron--Severi torsion group of geometric fibres of \(f\) is constant on every connected component of \(S\) (which is also evident by transcendental methods when \(f\) is simple and projective).
We note that the direct use of \protect\hyperlink{fga-3-vi-theorem-2.5}{(2.5)} allows us to show, more generally, that, if \(\underline{\operatorname{Pic}}_{X/S}\) is proper over \(S\) (for example, if \(f\colon X\to S\) is simple and projective), and if \(q\) is a prime number distinct from the residual characteristics of \(S\), then the \(q\)-primary component of the Néron--Severi torsion groups of geometric fibres of \(X/S\) is constant on every connected component of \(S\).
It is no longer, however, the case in characteristic \(p>0\) for the \(p\)-primary component of the torsion group.
However, it remains possible that the rank over the \(k(s)\) of \(\underline{\operatorname{Pic}}_{X_s/k(s)}/\underline{\operatorname{Pic}}_{X_s/k(s)}^{00}=T_{X_s/k(s)}\) is locally constant;
when \(S\) is reduced, we can show that this is equivalent to showing that \(\underline{\operatorname{Pic}}_{X/S}^{00}\) exists and that \(\underline{\operatorname{Pic}}_{X/S}\) is \emph{flat} over \(S\), and it suffices to test in the cases where \(S\) is the spectrum of a discrete valuation ring.
This is what I have verified in the few examples that I have looked at;
but since the corresponding statement with \(S\) Artinian is false (cf.~Remarks \protect\hyperlink{fga-3-vi-remark-2.9}{(2.9)} and \protect\hyperlink{fga-3-vi-remarks-3.4}{(3.4)}), we must not get carried away.

\end{rmenv}

\hypertarget{fga-3-vi-section-4}{%
\subsection{The finiteness theorem for the Picard scheme}\label{fga-3-vi-section-4}}

Let \(f\colon X\to S\) be a projective and flat morphism such that \(\underline{\operatorname{Pic}}_{X/S}\) exists.
Then the ``Hilbert polynomials'' \(Q\in\mathbb{Q}[t]\) allow us to decompose \(\underline{\operatorname{Pic}}_{X/S}\) into a sum of opens \(\underline{\operatorname{Pic}}_{X/S}^Q\).
If we do not make any further hypotheses, ensuring for example that \(\underline{\operatorname{Pic}}_{X/S}\) is separated over \(S\), then it will not be true in general that these opens are of finite type over \(S\);
we obtain a counterexample when \(X\) is ``a conic degenerating to two lines''.
\oldpage{236-22}It is possible, however, that this is the case if \(f\) is \emph{separable} and has \emph{irreducible geometric fibres}.
The question is linked to knowing if \(\underline{\operatorname{Pic}}_{X/S}^\tau\) is of finite type over \(S\), which can be true without any hypotheses on the fibres of \(X/S\).
When \(f\colon X\to S\) is simple, we note that \(\underline{\operatorname{Pic}}_{X/S}^\tau\) is contained in one of the \(\underline{\operatorname{Pic}}_{X/S}^Q\) (thanks to the fact that, on a non-singular projective variety, ``torsion'' equivalence is finer than (in fact, ``identical to'', thanks to Matsusaka) numerical equivalence for divisors), and is thus of finite type over \(S\) if the \(\underline{\operatorname{Pic}}_{X/S}^Q\) are.
Note that these finiteness questions that we have just highlighted still make sense even without supposing the existence of \(\underline{\operatorname{Pic}}_{X/S}\), since they can be expressed by saying that certain families of invertible modules are ``limited'', in the sense of \href{FGA-3-IV.html}{FGA 3.IV}:
Consider, for every algebraically closed field \(k\), the integral subschemes of \(\mathbb{P}_k^r\) of dimension \(n\) and degree \(d\), and the invertible modules on these preschemes that have a Hilbert polynomial \(Q\) (where \(r\), \(n\), \(d\), and \(Q\) are given), and show that the family of these modules (considered as coherent modules on the \(\mathbb{P}_k^r\)) is \emph{limited}, i.e.~can be parametrised by a scheme of finite type over \(\mathbb{Z}\).

Using the method of Matsusaka {[}\protect\hyperlink{ref-Mat1957}{14}{]}, a rather technical proof (using the ``equivalence criteria'' in a suitable form) allow us to answer in the affirmative when we restrict to the non-singular subvarieties of \(\mathbb{P}_k^r\).
More precisely, we obtain the following result:

\hypertarget{fga-3-vi-theorem-4.1}{}
\begin{itenv}{Theorem 4.1}

Let \(f\colon X\to S\) be a simple projective morphism with connected geometric fibres, with \(S\) Noetherian.
Let \({\mathscr{O}}_X(1)\) be a very ample module on \(X\) with respect to \(S\), and \(E\) a subset of \(\underline{\operatorname{Pic}}_{X/S}\) corresponding to a family \(({\mathscr{L}}_i)\) of invertible modules on the geometric fibres \(\overline{X}_{s_i}\) of \(X/S\), with \(D_i\) a (not-necessarily positive) divisor on \(\overline{X}_{s_i}\) that defines \({\mathscr{L}}_i\), and let \(a_n^{(i)}t^n+\ldots+a_0^{(i)}\) be the Hilbert polynomial of \({\mathscr{L}}_i\), and let \(\xi=\xi_i\) be a divisor that defines \({\mathscr{O}}_X(1)\), i.e.~a hyperplane section.

Then the following conditions are equivalent:

\begin{enumerate}
\def\labelenumi{\alph{enumi}.}
\tightlist
\item
  \(Q\) is quasi-compact, i.e.~contains an open of finite type over \(S\), i.e.. the family \(({\mathscr{L}}_i)\) is limited.
\item
  \(E\) is contained in the union of a finite number of the sets \(\underline{\operatorname{Pic}}_{X/S}^Q\), for \(Q\in\mathbb{Q}[t]\), i.e.~the set of Hilbert polynomials of the \({\mathscr{L}}_i\) is finite.
\item
  (If the fibres of \(X/S\) are all of the same dimension \(n\)).\footnote{\emph{{[}Trans.{]}} Conditions (c) and (d) were originally called (b') and (b'\,').}
  The coefficients \(a_{n-1}^{(i)}\) and \(a_{n-2}^{(i)}\) of the Hilbert polynomials of the \({\mathscr{L}}_i\) are contained in a finite set.
\item
  (If the fibres of \(X/S\) are all of the same dimension \(n\)).
  The integers \(\xi^{n-1}D_i\) and \(\xi^{n-2}D_i^2\) (the projective degrees of \(D_i\) and of \(D_i^2\)) are bounded above.
\end{enumerate}

\end{itenv}

\leavevmode\vadjust pre{\hypertarget{fga-3-vi-corollary-4.2}{}}%
\begin{itenv}{Corollary 4.2}
Let \(f\colon X\to S\) be a simple projective morphism, with connected geometric fibres.
Then the schemes \(\underline{\operatorname{Pic}}_{X/S}^Q\) (for \(Q\in\mathbb{Q}[t]\)) and \(\underline{\operatorname{Pic}}_{X/S}^\tau\) are projective over \(S\).

\end{itenv}

\begin{proof}
Since they are of finite type over \(S\), by \protect\hyperlink{fga-3-vi-theorem-4.1}{(4.1)}, we can apply \protect\hyperlink{fga-3-vi-corollary-2.4}{(2.4)}.
\end{proof}

\leavevmode\vadjust pre{\hypertarget{fga-3-vi-remark-comp}{}}%
\begin{rmenv}{Remark}
\emph{{[}Comp.{]}}
\oldpage{C-07}The questions of finiteness of the type highlighted in this section have been all but totally resolved since the editing of this present talk.
We note here the principal facts no known in this direction.
To simplify the statements, we implicitly assume that all the Picard preschemes that arise in the statements exist, even though an evident modification of these statements would allow us to get rid of any existence hypothesis.
In what follows, \(S\) denotes a Noetherian scheme, and \(X\) and \(Y\) are proper schemes over \(S\).

\begin{enumerate}
\def\labelenumi{\roman{enumi}.}
\item
  Let \(f\colon X\to Y\) be a surjective morphism.
  Then \(f^*\colon\underline{\operatorname{Pic}}_{Y/S}\to\underline{\operatorname{Pic}}_{X/S}\) is a morphism of finite type.
\item
  \oldpage{C-08}Suppose that \(Y\) is projective over \(S\), endowed with an invertible module that is ample with respect to \(S\), and let \(X\) be the scheme of zeros of an arbitrary section of this module.
  Let \(f\colon X\to Y\) be the canonical immersion.
  Finally, suppose that the irreducible components of the fibres of \(Y/S\) are of dimension \(\geqslant 3\).
  Then \(f^*\colon\underline{\operatorname{Pic}}_{Y/S}\to\underline{\operatorname{Pic}}_{X/S}\) is of finite type.
\item
  Suppose that \(X\) is projective over \(S\), and that all its geometric fibres are integral and of dimension \(n\).
  Let \({\mathscr{O}}_X(1)\) be an ample invertible module on \(X\), allowing us to define Hilbert polynomials.
  Let \(M\) be a subset of \(\underline{\operatorname{Pic}}_{X/S}\).
  Then \(M\) is quasi-compact if and only if, in the Hilbert polynomials \(a_0x^n+a_1x^{n-1}+a_2x^{n-2}+\ldots+a_n\) of the elements of \(M\), the coefficients \(a_1\) and \(a_2\) are bounded.
\item
  For every integer \(n\neq0\), the \(n\)-th power homomorphism in the group prescheme \(\underline{\operatorname{Pic}}_{X/S}\) is a morphism of finite type.
\end{enumerate}

Note that (i) and (ii) also imply (under the hypotheses given in their respective statements) that \emph{a subset \(M\) of \(\underline{\operatorname{Pic}}_{Y/S}\) is quasi-compact if and only if its image in \(\underline{\operatorname{Pic}}_{X/S}\) is quasi-compact}.
We thus conclude that an invertible module \({\mathscr{L}}\) on \(Y\) is \(\tau\)-equivalent to \(0\) if and only if its inverse image in \(X\) is;
in other words, \(\underline{\operatorname{Pic}}_{Y/S}^\tau\) is the inverse image of \(\underline{\operatorname{Pic}}_{X/S}^\tau\).
In particular, to show that the first prescheme is of finite type over \(S\), it suffices to prove this for the second, since \(\underline{\operatorname{Pic}}_{Y/S}\to\underline{\operatorname{Pic}}_{X/S}\) is of finite type.
Then using (i), Chow's lemma, and (iii), we find:

\begin{enumerate}
\def\labelenumi{\alph{enumi}.}
\setcounter{enumi}{21}
\tightlist
\item
  \(\underline{\operatorname{Pic}}_{X/S}^\tau\) is of finite type over \(S\).
\end{enumerate}

Generally, the conjunction of (i) for a finite morphism and (ii) allows us to reduce, for the majority of finiteness questions, to the case where \(X/S\) has integral and normal geometric fibres of dimension \(\leqslant 2\);
often, even, applying (i) for a surjective but not-necessarily finite morphism, along with the resolution of singularities of algebraic surfaces (due, in arbitrary characteristic, to Abhyankar), we can reduce to the case where \(X/S\) is even simple, and thus has non-singular geometric fibres of dimension \(2\).
This allows us, for example, taking into account (v) and the Picard--Igusa inequality bounding the rank of the Néron--Severi group of a non-singular projective surface, to prove the following generalisation of the Néron finiteness theorem:

\begin{enumerate}
\def\labelenumi{\roman{enumi}.}
\setcounter{enumi}{5}
\tightlist
\item
  Let \(X/S\) be proper over \(S\), but otherwise arbitrary.
  \oldpage{C-09}Then the Néron--Severi groups \(\underline{\operatorname{Pic}}_{X_i/k_i}/\underline{\operatorname{Pic}}_{X_i/k_i}^0\) of the geometric fibres \(X_i/k_i\) of \(X/S\) are of finite type, and their rank and the order of their torsion subgroups are bounded.
\end{enumerate}

The same method of reduction to the case of non-singular surfaces, and known theorems for this case (such as the Néron finiteness theorem, and the fact that the intersection form on the Néron--Severi group is non-degenerate) implies:

\begin{enumerate}
\def\labelenumi{\roman{enumi}.}
\setcounter{enumi}{6}
\tightlist
\item
  Let \(X\) be a proper scheme over an algebraically closed field.
  Then there exists a finite number of integral closed curves \(C_i\) (for \(1\leqslant i\leqslant r\)) in \(X\), such that the following property is satisfied:
  for every subset \(M\) of \(\underline{\operatorname{Pic}}_{X/k}\), \(M\) is quasi-compact if and only if the integers \(\deg{\mathscr{L}}_{C'_i}\) (for \({\mathscr{L}}\in M\)) are bounded (where \(C'_i\) denotes the normalisation of \(C_i\)).
\end{enumerate}

In the above we can take \(r\) to be the rank of the Néron--Severi group.
Once we know that this group is of finite type, (vii) reduces to the fact that the linear forms on the Néron--Severi group defined by the curves \(C\) in \(X\) do not simultaneously vanish except for on the torsion elements of the Néron--Severi group.
In the case where \(X\) is non-singular and projective, this result, as well as (v), was due to Matsusaka.
Using (vii), we easily obtain the following characterisation of invertible modules that are \(\tau\)-equivalent to \(0\) on \(X\):

\begin{enumerate}
\def\labelenumi{\roman{enumi}.}
\setcounter{enumi}{7}
\tightlist
\item
  Let \(X/k\) be a proper scheme over a field, and \({\mathscr{L}}\) an invertible module on \(X\).
  Then the following conditions are equivalent:
\end{enumerate}

\begin{enumerate}
\def\labelenumi{\alph{enumi}.}
\tightlist
\item
  \({\mathscr{L}}\) is \(\tau\)-equivalent to \(0\).
\item
  For every coherent module \(F\), we have \(\chi(F\otimes{\mathscr{L}})=\chi(F)\), where \(\chi\) denotes the Euler--Poincaré characteristic (or, condition (b'), we can simply take \(F={\mathscr{O}}_Y\), where \(Y\) is an integral closed subscheme of dimension \(1\) in \(X\)).
\item
  For every \(Y\) as above, writing \(Y'\) to mean the normalised curve, we have \(\deg{\mathscr{L}}_{Y'}=0\).
\end{enumerate}

If \(X/k\) is projective and endowed with an ample invertible module \({\mathscr{O}}_X(1)\), then the above conditions are also equivalent to the following:

\begin{enumerate}
\def\labelenumi{\alph{enumi}.}
\setcounter{enumi}{3}
\tightlist
\item
  For every integer \(m\), \({\mathscr{L}}^{\otimes m}(1)\) is ample.
\item
  (If \(X\) is integral).
  For every pair of integers \((m,n)\), we have
  \[
   \chi({\mathscr{L}}^{\otimes m}(n))
   = \chi({\mathscr{O}}(n))
      \]
  \oldpage{C-10}(i.e.~(b) is true for \(F={\mathscr{L}}^{\otimes m}(n)\)).
\end{enumerate}

For the sufficiency of this last condition, note that it implies that the Hilbert polynomials of the \({\mathscr{L}}^{\otimes m}\) are all equal, and thus, by the Mumford criteria (iii), the \({\mathscr{L}}^{\otimes m}\) remain in a quasi-compact subset of \(\underline{\operatorname{Pic}}_{X/k}\), i.e.~we have (a).
Conditions (b), (b'), and (c) should be considered as variants (for an arbitrary proper scheme) of the notion of \emph{numerical equivalence}, usually defined for non-singular projective varieties.
For such varieties, the equivalence of (a) and (c) was evidently well known (Matsusaka).

Criterion (e) from the above also implies the following result:

\begin{enumerate}
\def\labelenumi{\roman{enumi}.}
\setcounter{enumi}{8}
\tightlist
\item
  Let \(f\colon X\to S\) be a flat projective morphism whose geometric fibres are integral.
  Then \(\underline{\operatorname{Pic}}_{X/S}^\tau\) is open \emph{and closed} in \(\underline{\operatorname{Pic}}_{X/S}\).
\end{enumerate}

We restrict ourselves to some comments on the proofs of the key results (i), (ii), and (iii) (result (iv) is a little bit different from the others, and can be proven using only (i) for radicial surjective finite morphisms, or, more precisely, for a Frobenius morphism).
For (i), we use, in an essential way, the ideas of non-flat descent (see \protect\hyperlink{fga-3-i-section-A.2.c}{FGA 3.I, p.9}).
One finds that (thinking only of finiteness results) the lack of effectivity criteria for descent data is inoffensive.
Mumford has recently proven a slightly weaker form of (iii), where the criteria makes use of \emph{all} the coefficients of the Hilbert polynomial.
His argument is extremely simple, and is inspired by the proof of an amplitude criterion by Nakai (stated by the latter for non-singular surfaces, and generalised by Mumford to arbitrary projective morphisms).
It seems to me that this argument only works under a gentle additional restriction on the fibres of \(X/S\) (Serre's (\(\text{S}_2\)) property), which is satisfied if the geometric fibres are normal.
We then use this restricted criterion in the proof of (ii): criterion (i) allows us to reduce to the case where \(Y/S\) is flat with integral and normal geometric fibres, and applying the Mumford criteria we easily reduce to the case where \(X/S\) satisfies the same conditions.
From the dimension hypothesis it then follows that the geometric fibres of \(Y\) and of \(X\) are of depth \(\geqslant 2\) at their closed points, which allows us to apply the ``equivalence criteria'' under the form that is given in {[}\protect\hyperlink{ref-Gro1960b}{9}, XII{]}, and finishes the proof of (ii).
Once we have (i) and (ii), it is not difficult in the Mumford criteria to discard the hypothesis that the fibres be normal, and to prove it under the stronger form given in (iii).
\oldpage{C-11}We note finally that the proof of (i) and (ii) also shows that, in the case where \(S\) is the spectrum of a field \(k\), the morphism \(\underline{\operatorname{Pic}}_{Y/k}\to\underline{\operatorname{Pic}}_{X/k}\) is \emph{affine} (and not only of finite type).

\end{rmenv}

\hypertarget{bibliography}{%
\section*{Bibliography}\label{bibliography}}
\addcontentsline{toc}{section}{Bibliography}

\hypertarget{refs}{}
\begin{CSLReferences}{0}{0}
\leavevmode\vadjust pre{\hypertarget{ref-BS1958}{}}%
\CSLLeftMargin{{[}1{]} }%
\CSLRightInline{A. Borel, J.-P. Serre. {``Le théorème de {Riemann}\textendash{{Roch}}.''} \emph{Bull. Soc. Math. France}. \textbf{86} (1958), 97--136.}

\leavevmode\vadjust pre{\hypertarget{ref-Car1957}{}}%
\CSLLeftMargin{{[}2{]} }%
\CSLRightInline{P. Cartier. {``Des groups \(\mathrm{Ext}^s(A,B)\).''} \emph{Séminaire A. Grothendieck: Algèbre Homologique}. \textbf{1} (1957), Talk no. 3.}

\leavevmode\vadjust pre{\hypertarget{ref-Che1960}{}}%
\CSLLeftMargin{{[}3{]} }%
\CSLRightInline{C. Chevalley. {``Sur la théorie de picard.''} \emph{Amer. J. Of Math.} \textbf{82} (1960), 435--490.}

\leavevmode\vadjust pre{\hypertarget{ref-GR1958}{}}%
\CSLLeftMargin{{[}4{]} }%
\CSLRightInline{H. Grauert, R. Remmert. {``{Komplexe Räume}.''} \emph{Math. Annalen}. \textbf{136} (1958), 245--318.}

\leavevmode\vadjust pre{\hypertarget{ref-Gro1960}{}}%
\CSLLeftMargin{{[}5{]} }%
\CSLRightInline{A. Grothendieck. {``The cohomology theory of abstract algebraic varieties,''} in: \emph{Proceedings of the {Internation Congress} of {Mathematicians} 1958}. {Cambridge University Press}, 1960.}

\leavevmode\vadjust pre{\hypertarget{ref-Gro1957}{}}%
\CSLLeftMargin{{[}6{]} }%
\CSLRightInline{A. Grothendieck. {``Sur quelques points d'algèbre homologique.''} \emph{Tohoku Math. J.} \textbf{9} (1957), 119--221.}

\leavevmode\vadjust pre{\hypertarget{ref-Gro1958a}{}}%
\CSLLeftMargin{{[}7{]} }%
\CSLRightInline{A. Grothendieck. {``Géométrie formelle et géométrie algébrique.''} \emph{Séminaire Bourbaki}. \textbf{11} (1958/59), Talk no. 182.}

\leavevmode\vadjust pre{\hypertarget{ref-Gro1960a}{}}%
\CSLLeftMargin{{[}8{]} }%
\CSLRightInline{A. Grothendieck. {``Techniques de construction en géométrie analytique, {I}--{X}.''} \emph{Séminaire Cartan}. \textbf{13} (1960/61), Talks no. 7--17.}

\leavevmode\vadjust pre{\hypertarget{ref-Gro1960b}{}}%
\CSLLeftMargin{{[}9{]} }%
\CSLRightInline{A. Grothendieck. \emph{{Séminaire de Géométrie Algébrique}}. {Paris, Institut des Hautes Études Scientifiques}, 1960/61.}

\leavevmode\vadjust pre{\hypertarget{ref-GD1960}{}}%
\CSLLeftMargin{{[}10{]} }%
\CSLRightInline{A. Grothendieck, J. Dieudonné. {``{Eléments de Géométrie Algébrique}.''} \emph{Pub. Math. De l'IHÉS}. \textbf{4, 8, 11, 17, 20, 24, 28, 32} (1960/67).}

\leavevmode\vadjust pre{\hypertarget{ref-Har1966}{}}%
\CSLLeftMargin{{[}11{]} }%
\CSLRightInline{R. Hartshorne. {``Connectedness of the {Hilbert} scheme.''} \emph{Pub. Math. De l'IHÉS}. \textbf{29} (1966), 5--48.}

\leavevmode\vadjust pre{\hypertarget{ref-Kod1956}{}}%
\CSLLeftMargin{{[}12{]} }%
\CSLRightInline{K. Kodaira. {``Characteristic linear systems of complete continuous systems.''} \emph{Amer. J. Of Math.} \textbf{78} (1956), 716--744.}

\leavevmode\vadjust pre{\hypertarget{ref-Lan1959}{}}%
\CSLLeftMargin{{[}13{]} }%
\CSLRightInline{S. Lang. \emph{Abelian varieties}. Interscience Publishers, 1959.}

\leavevmode\vadjust pre{\hypertarget{ref-Mat1957}{}}%
\CSLLeftMargin{{[}14{]} }%
\CSLRightInline{T. Matsusaka. {``The criteria for algebraic equivalence and the torsion group.''} \emph{Amer. J. Math.} \textbf{79} (1957), 53--66.}

\leavevmode\vadjust pre{\hypertarget{ref-Mum1961}{}}%
\CSLLeftMargin{{[}15{]} }%
\CSLRightInline{D. Mumford. {``An elementary theorem in geometric invariant theory.''} \emph{Bull. Amer. Math. Soc.} \textbf{67} (1961), 483--486.}

\leavevmode\vadjust pre{\hypertarget{ref-Mur1958}{}}%
\CSLLeftMargin{{[}16{]} }%
\CSLRightInline{J.P. Murre. {``On a connectedness theorem for a birational transformation at a simple point.''} \emph{Amer. J. Of Math.} \textbf{80} (1958), 3--15.}

\leavevmode\vadjust pre{\hypertarget{ref-Oor1962}{}}%
\CSLLeftMargin{{[}17{]} }%
\CSLRightInline{F. Oort. {``Sur le schéma de picard.''} \emph{Bull. Soc. Math. France}. \textbf{90} (1962), 1--14.}

\leavevmode\vadjust pre{\hypertarget{ref-Ser1955}{}}%
\CSLLeftMargin{{[}18{]} }%
\CSLRightInline{J.-P. Serre. {``Faisceaux algébriques cohérents.''} \emph{Annals of Math.} \textbf{61} (1955), 197--278.}

\leavevmode\vadjust pre{\hypertarget{ref-Ser1956}{}}%
\CSLLeftMargin{{[}19{]} }%
\CSLRightInline{J.-P. Serre. {``Géométrie algébrique et géométrie analytique.''} \emph{Ann. Institut Fourier Grenoble}. \textbf{6} (1956), 1--42.}

\leavevmode\vadjust pre{\hypertarget{ref-Ser1956a}{}}%
\CSLLeftMargin{{[}20{]} }%
\CSLRightInline{J.-P. Serre. {``Sur la dimension homologique des anneaux et des modules {Noethériens},''} in: \emph{Proc. {Intern}. {Symp}. On Alg. Number Theory, {Tokyo} and {Nikko}, 1955}. {Science Council of Japan, Tokyo}, 1956: pp. 175--189.}

\leavevmode\vadjust pre{\hypertarget{ref-Ser1958}{}}%
\CSLLeftMargin{{[}21{]} }%
\CSLRightInline{J.-P. Serre. {``Corps locaux et isogénies.''} \emph{Séminaire Bourbaki}. \textbf{11} (1960), Talk no. 185.}

\leavevmode\vadjust pre{\hypertarget{ref-Ser1958a}{}}%
\CSLLeftMargin{{[}22{]} }%
\CSLRightInline{J.-P. Serre. {``Espaces fibrés algébriques.''} \emph{Séminaire Chevalley}. \textbf{3} (1958), Talk no. 1.}

\leavevmode\vadjust pre{\hypertarget{ref-Ser1958b}{}}%
\CSLLeftMargin{{[}23{]} }%
\CSLRightInline{J.-P. Serre. {``Sur la topologie des variétés algébriques en caractéristique \(p\),''} in: \emph{Symposium Internacional de Topologia Algebraica, {Mexico}, 1956}. {Universidad nacional autonoma, Mexico}, 1958: pp. 24--53.}

\leavevmode\vadjust pre{\hypertarget{ref-Ser1961}{}}%
\CSLLeftMargin{{[}24{]} }%
\CSLRightInline{J.-P. Serre. {``Exemples de variétés projectives en caractéristique \(p\) non relevables en caractéristique 0.''} \emph{Proc. Nat. Acad. Sc. U.S.A.} \textbf{47} (1961), 108--109.}

\leavevmode\vadjust pre{\hypertarget{ref-Ses1962}{}}%
\CSLLeftMargin{{[}25{]} }%
\CSLRightInline{C.S. Seshadri. {``La variété de picard d'une variété complète.''} \emph{Annali Di Mat. Pura Ed App.} \textbf{57} (1962), 117--42.}

\leavevmode\vadjust pre{\hypertarget{ref-Zap1945}{}}%
\CSLLeftMargin{{[}26{]} }%
\CSLRightInline{G. Zappa. {``Sull'esistenza, sopra la superficie algebriche, di sistemi continui completi infiniti, la cui curva e a serie caratteristica incompleta.''} \textbf{9} (1945), 91--93.}

\end{CSLReferences}

\end{document}
